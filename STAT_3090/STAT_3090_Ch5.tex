%% In the documentclass line, replace "noanswers" with "answers" to view the key.

\documentclass[noanswers]{exam}
\usepackage[utf8]{inputenc}

\title{Practice Problems}
\author{Chapter 5}
\date{STAT 3090}

\usepackage[bottom=2.2cm, left=2.2cm, right=2.2cm, top=2.2cm]{geometry}
%\usepackage[paperheight=11in, paperwidth=17in, margin=1in]{geometry}
\usepackage{dsfont}
\usepackage{amsmath}
\usepackage{amssymb}
\usepackage{amsthm}
\usepackage{array}
\usepackage{stmaryrd}
\usepackage{pgfplots}
\pgfplotsset{width=10cm,compat=1.9}
\usepackage{multicol}
\setlength{\columnsep}{1in}
\usepackage{nicefrac}

\usepackage{multirow}
\usepackage{enumitem}[shortlabels]
\usepackage{tabu}
\definecolor{purp}{RGB}{102,0,204}
\usepackage{tabularx}
\newcolumntype{C}{>{\centering\arraybackslash $}X<{$}}
\usepackage{wrapfig}
\usepackage[export]{adjustbox}


\makeatletter
\pagestyle{headandfoot}
\firstpageheader{\@date}{\@title}{\@author}
\firstpageheadrule
\runningfootrule
\runningfooter{}{\thepage\ / \numpages}{\@title}
\makeatother

\newcommand{\abs}[1]{\left|#1\right|}
\newcommand{\mat}[4]{\left( \begin{tabular}{>{$}c<{$} >{$}c<{$}} #1&#2 \\ #3&#4 \end{tabular} \right)}
\newcommand{\msc}[1]{\mathds{#1}}
\newcommand{\Z}{\mathds{Z}}
\newcommand{\R}{\mathds{R}}
\newcommand{\N}{\mathds{N}}
\newcommand{\Q}{\mathds{Q}}
\newcommand{\C}{\mathds{C}}
\newcommand{\so}{\implies}
\newcommand{\set}[2]{\left\{ #1 \:|\: #2 \right\}}
\newcommand{\bso}{\Longleftarrow}
\newcommand{\ra}{\rightarrow}
\newcommand{\gen}[1]{\left\langle #1 \right\rangle}
\newcommand{\olin}[1]{\overline{#1}}
\newcommand{\Img}[1]{\text{Im}\left(#1\right)}
\newcommand{\llra}{\longleftrightarrow}
\newcommand{\lra}{\longrightarrow}
\newcommand{\xra}[1]{\xrightarrow{#1}}
\newcommand{\wo}{\setminus}
\newcommand{\mcal}[1]{\mathcal{#1}}
\newcommand{\Aut}[1]{\text{Aut}\left(#1\right)}
\newcommand{\Inn}[1]{\text{Inn}\left(#1\right)}
\newcommand{\syl}[2]{\text{Syl}_{#1}(#2)}
\newcommand{\norm}[1]{\left\|#1\right\|}
\newcommand{\infnorm}[1]{\left\|#1\right\|_{\infty}}
\newcommand{\xn}{\{x_n\}}
\newcommand{\sig}{\sigma}
\newcommand{\id}{\text{id}}
\newcommand{\ep}{\epsilon}
\newcommand{\st}{\text{ s.t. }}
\newcommand{\ran}[1]{\text{Ran}(#1)}
\newcommand{\nCr}[2]{\binom{#1}{#2}}
\newcommand{\Exr}[1]{\paragraph{Exercise #1:}}
\newcommand{\pg}{\paragraph{}}
\newcommand{\ulin}[1]{\underline{#1}}
\newcommand{\tc}[1]{\textcolor{purp}{#1}}

% Solution Specs
\unframedsolutions
\renewcommand{\solutiontitle}{}
\SolutionEmphasis{\color{purp}}
\CorrectChoiceEmphasis{\color{purp}\bfseries}
\setlength\fillinlinelength{0in}
\renewcommand{\arraystretch}{2}


\begin{document}

%\noindent\begin{tabular}{@{}p{.3in}p{3in}@{}}
%Name: & \hrulefill
%\end{tabular}
%
%\vspace{2mm}
%
%\noindent\begin{tabular}{@{}p{1.05in}p{3.2in}@{}}
%Group Members: & \hrulefill
%\end{tabular}

\begin{questions} 
	
	\question Your favorite fast food joint has a specialty sandwich and side combo. For the sandwich you can order a hamburger (H) or a chicken sandwich (C), and for the side you can order fries (F), onion rings (R), or a \mbox{salad (S).}
	
\vspace{3mm}	
	
	\begin{parts}
	
	\part Describe the \textbf{sample space} $S$ of possible combos you could order. 
	\begin{solution}[\stretch{1}]
			\vspace{3mm}
		
			$S=\left\{\text{HF, HR, HS, CF, CR, CS}\right\}$ or $S=\left\{\text{(H,F), (H,R), (H,S), (C,F), (C,R), (C,S)}\right\}$

			\vspace{3mm}	
		\end{solution}
		
	\part Consider the event that someone orders fries as a side on their combo. Is this a \textbf{simple} or a \textbf{compound} event? How do you know?
	\begin{solution}[\stretch{1}]
			\vspace{3mm}
		
			Compound event --- more than one way to order a combo that includes fries

			\vspace{3mm}	
		\end{solution}
		
	\part The cashier is a little bit bored and decides to try and guess what sandwich and side combo customers will order. Let $A$ be the event that she guesses an order \textbf{correctly}. If all order combinations are equally likely, what is the probability of $A$? Use probability notation.
		
		\begin{solution}[\stretch{1}]
			\vspace{1mm}	
		
			P(correct) $=P(A)= \displaystyle \frac{1}{6}$
			
			\vspace{1mm}	
		\end{solution}	
	
\part Let $B$ be the event that a customer orders \textbf{onion rings} as their side. What is the probability of $B$?
		
		\begin{solution}[\stretch{1}]
			\vspace{1mm}		
			P(onion rings) $=P(B)= \displaystyle \frac{2}{6}=\frac{1}{3}$
			
			\vspace{3mm}	
		\end{solution}	
	
	\end{parts}
	
	\question Consider a probability experiment in which you toss a coin three times.
	
	\vspace{3mm}
	
	\begin{parts}
		\part Write out the \textbf{sample space} $S$ for the experiment. Let H denote heads and T denote tails.
		
		\begin{solution}[\stretch{1}]
		\vspace{3mm}
		
		$S=\left\{\text{HHH, HHT, HTH, THH, HTT, TTH, THT, TTT}\right\}$
		
		\vspace{3mm}
		
		\end{solution}
			
		\part Find the probability of $C=$ \textbf{two out of three} tosses result in heads. Use probability notation.
		
		\begin{solution}[\stretch{1}]
		\vspace{3mm}
		Three outcomes with 2 heads $\Rightarrow$ P(2 heads) $=P(C) =\displaystyle\frac{3}{8}$
		\vspace{3mm}
		\end{solution}
		
		\part Find the probability of $D=$ the experiment results in \textbf{at least one} tails. Use probability notation.
		
		\begin{solution}[\stretch{1}]
		\vspace{3mm}
		Seven outcomes with at least one tail $\Rightarrow$ P(at least one tails) $=P(D)=\displaystyle \frac{7}{8}$
		\vspace{3mm}
		\end{solution}
		
		\part Describe the \textbf{complement} of event $D$, denoted $D^C$, in terms of the experiment.
		
		\begin{solution}[\stretch{1}]
		\vspace{3mm}
		$D^C=$ the event that you do NOT get at least one tails (or the event that you get no tails)
		\vspace{3mm}
		\end{solution}
		
	\part Find the probability of $D^C$ using \textbf{the complement rule} and your result in Part (c).
		
		\begin{solution}[\stretch{1}]
		\vspace{3mm}
		$P(D^C)= 1-\text{P(D)}=\displaystyle 1-\frac{7}{8}=\frac{1}{8}$
		\vspace{3mm}
		\end{solution}
		
	\part Describe two \textbf{mutually exclusive} events in your sample space.
	
	\begin{solution}[\stretch{1}]
	\vspace{3mm}
	Student answers may vary. An example might be ``getting all heads'' and ``getting all tails.''
	\vspace{3mm}	
	\end{solution}
		
	\end{parts}
		
%	\question Consider the following table of data collected from a sample of 36 students in STAT 3090 last fall.
%	
%	\begin{center}
%\begin{tabular}{|c|c|c|c|c|}
%\hline
% & \hspace{3mm} Coffee (C) \hspace{3mm} & \hspace{4mm} Tea (T) \hspace{4mm} & Hot Cocoa (H) & TOTAL\\
% \hline
% Morning Person (M) & 3 & 3 & 2 & 8 \\
% \hline
% Night Person (N) & 17 & 3 & 8 & 28 \\
%\hline
%TOTAL & 20 & 6 & 10 & 36 \\
%\hline
%\end{tabular}
%\end{center}
%
%For each of the following, use proper probability \textbf{notation}, write the associated \textbf{fraction}, and express your final answer as a \textbf{decimal} number rounded to four places. Find the probability that a randomly selected student in the class\dots 
%
%\begin{parts}
%	
%	\part SEE STAT3090_FA19_LA7
%	
%\end{parts}

\newpage

\question Consider the following table of data collected from a sample of 36 students in STAT 3090 last fall. Students were asked whether they were a morning person or a night person, as well as what their hot beverage of choice is. Students could only choose one response for each variable.
	
	\begin{center}
\begin{tabular}{|c|c|c|c|c|}
\hline
 & \hspace{3mm} Coffee (C) \hspace{3mm} & \hspace{4mm} Tea (T) \hspace{4mm} & Hot Cocoa (H) & TOTAL\\
 \hline
 Morning Person (M) & 3 & 3 & 2 & 8 \\
 \hline
 Night Person (N) & 17 & 3 & 8 & 28 \\
\hline
TOTAL & 20 & 6 & 10 & 36 \\
\hline
\end{tabular}
\end{center}

For each of the following, use proper probability \textbf{notation}, write the associated \textbf{fraction}, and express your final answer as a \textbf{decimal} number rounded to four places. If you apply a probability rule (such as the Addition Rule or Complement Rule), \textbf{show the formula} in your work. 

\vspace{3mm}

Find the probability that a randomly selected student from the class\dots 

\begin{parts}
	
	\vspace{3mm}
	
	\part Prefers tea
	
	\begin{solution}[\stretch{1}]
	\vspace{3mm}
	$P(T)=\frac{6}{36}=0.1667$
	\vspace{3mm}
	\end{solution}
	
	\part Is a morning person \textbf{and} prefers coffee
	
	\begin{solution}[\stretch{1}]
	\vspace{3mm}
	$P(M\cap C)=\frac{3}{36}=0.0833$
	\vspace{3mm}
	\end{solution}
	
	\part Is a night person \textbf{and} prefers hot cocoa
	
	\begin{solution}[\stretch{1}]
	\vspace{3mm}
	$P(N\cap H)=\frac{8}{36}=0.2222$
	\vspace{3mm}
	\end{solution}
	
	\part Prefers tea \textbf{or} hot cocoa
	
	\begin{solution}[\stretch{1}]
	\vspace{3mm}
	$P(T\cup H)=\frac{6}{36}+\frac{10}{36}-0=\frac{16}{36}=0.4444$
	\vspace{3mm}
	\end{solution}
	
	\part Prefers coffee \textbf{or} is a morning person
	
	\begin{solution}[\stretch{1}]
	\vspace{3mm}
	$P(C\cup M)=\frac{20}{36}+\frac{8}{36}-\frac{3}{36}=\frac{25}{36}=0.6944$
	\vspace{3mm}
	\end{solution}
	
	\part Does \textbf{not} prefer coffee or tea
	
	\begin{solution}[\stretch{2}]
	\vspace{3mm}
	There are two ways to think about this problem. You could directly apply the Complement Rule: $P\left((C\cup T)^C\right)=1-P(C\cup T)=1-\left(\frac{20}{36}+\frac{6}{36}-0\right)=\frac{10}{36}=0.2778$
	
	\vspace{3mm}
	
	Or you could consider that the complement of $C\cup T$ is $(C\cup T)^C=H$:
	
	$P\left((C\cup T)^C\right)=P(H)=\frac{10}{36}=0.2778$
	\vspace{3mm}
	\end{solution}
	
\end{parts}

\newpage

\question Choose one of the probabilities that you calculated for Question 1 Parts (a)--(f). Write a sentence stating what the probability of a certain event is by stating the \textbf{sample} of interest, the \textbf{event} described, and the probability \textbf{value}.

\begin{solution}[\stretch{1}]

\vspace{3mm}

Example for Part (a): ``The probability that a randomly selected student in the class prefers tea is 0.1667.''

\vspace{3mm}
\end{solution} 

\question Identify two mutually exclusive events in this situation. What is the probability of their \textbf{intersection}, i.e., the probability that they both occur at the same time? Express your answer using probability notation.

\begin{solution}[\stretch{1}]

\vspace{3mm}

Example: ``Being a morning person and a night person are mutually exclusive. $P(M\cap N)=0$''

\vspace{3mm}
\end{solution} 

\question There were two variables used to record responses from students in the sample: whether a student is a morning or night person, and what their hot beverage of choice is. 

\vspace{3mm}

\begin{parts}
	\part What \textbf{type} of variables are these?
	
	\begin{solution}[\stretch{1}]

\vspace{3mm}

Qualitative

\vspace{3mm}
\end{solution} 
	
	\part What \textbf{level of measurement} do both of these variables have?
	\begin{solution}[\stretch{5}]

\vspace{3mm}

Nominal

\vspace{3mm}
\end{solution}

\end{parts}

\newpage

\question A survey of high school students indicated that 33\% are in a relationship, 25\% are involved in sports, and 11\% are involved in both. Use the information to answer the following questions.

\vspace{3mm}

\begin{parts}
		
	\part What is the probability that a student is involved in a relationship \textbf{given} that they're involved in sports? 
	
	\begin{solution}[\stretch{1}]
	\vspace{3mm}
	$P(R|S) = \frac{P(R \cap S)}{P(S)} = \frac{0.11}{0.25} = 0.44$
	\vspace{3mm}
	\end{solution}
	
	\part Is being in a relationship \textbf{independent} of being involved in sports? Justify your answer using probability (regardless of your own personal theories!).
	
	\begin{solution}[\stretch{1}]
	\vspace{3mm}
	$P(R|S)=0.44 \neq P(R)=0.33 \; \Rightarrow$ These are \textbf{dependent} events.
	\vspace{3mm}
	\end{solution}

\end{parts}

\question A company has two suppliers for electrical components. China ships 73\% of the electrical components used by the supplier. The probability that the component will be defective \textbf{given} that it was shipped from China is 0.06. What is the probability that a randomly selected component received by the supplier will ship from China \textbf{and} be defective?

\begin{solution}[\stretch{1}]
\vspace{1mm}
$P(C)=0.73$, $P(D|C)=0.06$

\vspace{3mm}
Using the Multiplication Rule: $P(C \cap D) = P(D|C)P(C) = 0.06(0.73) = 0.0438$
\vspace{1mm}
\end{solution}

\question You have a standard deck of 52 cards. Recall that a deck of cards has four suites (hearts, diamonds, spades, clubs), each with thirteen values (2-10, J, Q, K, A). Find the probability that you draw two aces \textbf{in a row} without replacing the first ace. 
	
	\begin{solution}[\stretch{1}]
	\vspace{1mm}
	$P(\text{1st A and 2nd A})=P(\text{1st A})\times P(\text{2nd A }|\text{ 1st A})=\frac{4}{52}\times\frac{3}{51}=0.0045$
	\vspace{1mm}
	\end{solution}
	
\question General Leia Organa can plan a campaign to fight one major intergalactic battle or three small galactic battles. She believes she has a probability of 0.77 of winning the large battle ($L$) and a probability of 0.89 of winning each of the small battles ($S$). Victories or defeats in the small battles are independent. Leia must win either the large battle or all three small battles to win the campaign. Which strategy should she choose?

\vspace{3mm}

\begin{parts}
\part First find the probability of winning the large battle.

\begin{solution}[\stretch{1}]

\vspace{3mm}
$P(L)=0.77$
\vspace{3mm}

\end{solution}

\part Find the probability of winning all three small battles.

\begin{solution}[\stretch{1}]

\vspace{3mm}
$P(S \text{ and } S \text{ and } S) = P(S)P(S)P(S)=(0.89)^3=0.7050$
\vspace{3mm}

\end{solution}

\part Which strategy should she choose if she wants to win the campaign? Justify your answer.

\begin{solution}[\stretch{1}]

\vspace{3mm}
She should choose to fight the large intergalactic battle since there is a higher probability of winning it than winning three small battles.
\vspace{3mm}

\end{solution}

\end{parts}

	
\end{questions}

%-----------------------------------------------------------------------------%

\end{document}
