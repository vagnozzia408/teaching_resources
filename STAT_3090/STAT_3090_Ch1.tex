%% In the documentclass line, replace "noanswers" with "answers" to view the key.

\documentclass[noanswers]{exam}
\usepackage[utf8]{inputenc}

\title{Practice Problems}
\author{Chapter 1}
\date{STAT 3090}

\usepackage[bottom=2.2cm, left=2.2cm, right=2.2cm, top=2.2cm]{geometry}
%\usepackage[paperheight=11in, paperwidth=17in, margin=1in]{geometry}
\usepackage{dsfont}
\usepackage{amsmath}
\usepackage{amssymb}
\usepackage{amsthm}
\usepackage{array}
\usepackage{stmaryrd}
\usepackage{pgfplots}
\pgfplotsset{width=10cm,compat=1.9}
\usepackage{multicol}
\setlength{\columnsep}{1in}
\usepackage{nicefrac}

\usepackage{multirow}
\usepackage{enumitem}[shortlabels]
\usepackage{tabu}
\definecolor{purp}{RGB}{102,0,204}
\usepackage{tabularx}
\newcolumntype{C}{>{\centering\arraybackslash $}X<{$}}
\usepackage{wrapfig}
\usepackage[export]{adjustbox}


\makeatletter
\pagestyle{headandfoot}
\firstpageheader{\@date}{\@title}{\@author}
\firstpageheadrule
\runningfootrule
\runningfooter{}{\thepage\ / \numpages}{\@title}
\makeatother

\newcommand{\abs}[1]{\left|#1\right|}
\newcommand{\mat}[4]{\left( \begin{tabular}{>{$}c<{$} >{$}c<{$}} #1&#2 \\ #3&#4 \end{tabular} \right)}
\newcommand{\msc}[1]{\mathds{#1}}
\newcommand{\Z}{\mathds{Z}}
\newcommand{\R}{\mathds{R}}
\newcommand{\N}{\mathds{N}}
\newcommand{\Q}{\mathds{Q}}
\newcommand{\C}{\mathds{C}}
\newcommand{\so}{\implies}
\newcommand{\set}[2]{\left\{ #1 \:|\: #2 \right\}}
\newcommand{\bso}{\Longleftarrow}
\newcommand{\ra}{\rightarrow}
\newcommand{\gen}[1]{\left\langle #1 \right\rangle}
\newcommand{\olin}[1]{\overline{#1}}
\newcommand{\Img}[1]{\text{Im}\left(#1\right)}
\newcommand{\llra}{\longleftrightarrow}
\newcommand{\lra}{\longrightarrow}
\newcommand{\xra}[1]{\xrightarrow{#1}}
\newcommand{\wo}{\setminus}
\newcommand{\mcal}[1]{\mathcal{#1}}
\newcommand{\Aut}[1]{\text{Aut}\left(#1\right)}
\newcommand{\Inn}[1]{\text{Inn}\left(#1\right)}
\newcommand{\syl}[2]{\text{Syl}_{#1}(#2)}
\newcommand{\norm}[1]{\left\|#1\right\|}
\newcommand{\infnorm}[1]{\left\|#1\right\|_{\infty}}
\newcommand{\xn}{\{x_n\}}
\newcommand{\sig}{\sigma}
\newcommand{\id}{\text{id}}
\newcommand{\ep}{\epsilon}
\newcommand{\st}{\text{ s.t. }}
\newcommand{\ran}[1]{\text{Ran}(#1)}
\newcommand{\nCr}[2]{\binom{#1}{#2}}
\newcommand{\Exr}[1]{\paragraph{Exercise #1:}}
\newcommand{\pg}{\paragraph{}}
\newcommand{\ulin}[1]{\underline{#1}}
\newcommand{\tc}[1]{\textcolor{purp}{#1}}

% Solution Specs
\unframedsolutions
\renewcommand{\solutiontitle}{}
\SolutionEmphasis{\color{purp}}
\CorrectChoiceEmphasis{\color{purp}\bfseries}

%\begin{solution}[\stretch{1}]
%	hurp durp flurp
%\end{solution}

%\pagestyle{empty}

\begin{document}

%\noindent\begin{tabular}{@{}p{.3in}p{3in}@{}}
%Name: & \hrulefill
%\end{tabular}

\begin{questions} 

	\question The Environmental Protection Agency (EPA) uses a few new automobiles of each model every year to collect data on gasoline mileage performance. For the 2017 Honda Accord model, they selected a sample of 16 cars and determined their average fuel efficiency to be 32.2 miles per gallon.
	
	\vspace{3mm}

	\begin{parts}
	
		\part What is the \textbf{population of interest} in this study?
	
		\begin{solution}[\stretch{1}]	
			\vspace{3mm}	
		
			All 2017 Honda Accords

			\vspace{3mm}
		\end{solution}
	
		\part Is the average fuel efficiency of 32.2 miles per gallon a statistic or a parameter? \textbf{Why}?
	
		\begin{solution}[\stretch{1}]
			\vspace{3mm}
				
			32.2 mpg is a statistic because it is calculated from the sample of 16 Accords.

			\vspace{3mm}	
		\end{solution}	
	\end{parts}

	\question A student engagement committee at Clemson decides to conduct a survey to learn about how students are involved in campus clubs and organizations. The committee randomly selects 125 students and finds that 73\% report being involved in at least one club or organization on campus.

\vspace{5mm}

	\begin{parts}
	
		\part Identify the \textbf{sample} in this situation.
		
		\begin{solution}[\stretch{1}]
			\vspace{3mm}
					
			125 Clemson students
			
			\vspace{3mm}
		\end{solution}
	
		\part Describe the \textbf{population} in this situation.
	
		\begin{solution}[\stretch{1}]		
			\vspace{3mm}				
			All Clemson students
			
			\vspace{3mm}
		\end{solution}
		
		\part Identify the \textbf{statistic} in this study.
	
		\begin{solution}[\stretch{1}]		
			\vspace{3mm}				
			The 73\% of students in the sample involved in at least one club or organization on campus
			
			\vspace{3mm}
		\end{solution}
		
		\part Describe the \textbf{parameter} in this study.
	
		\begin{solution}[\stretch{1}]		
			\vspace{3mm}				
			The true percentage of Clemson students who are involved in at least one club or organization on campus
			
			\vspace{3mm}
		\end{solution}

	\end{parts}
	
	\question Identify the goal of each of the following as either \textbf{descriptive} or \textbf{inferential} statistics. Describe \textbf{why} you give your answer.
	
	\begin{parts}
	\part A researcher wants to study the exercise habits of residents in Pickens County. She randomly selects a sample of residents using phone book listings and conducts short phone interviews with 203 residents who answer the phone, asking them about their weekly exercise routine. She finds that 62\% of residents in Pickens County exercise 2-3 times per week.
	
	\begin{solution}[\stretch{1}]		
			\vspace{3mm}				
			Inferential statistics --- the researcher wants to conclude something about the exercise habits of all residents in Pickens County and estimates her conclusion based on a sample.
			
			\vspace{3mm}
		\end{solution}
	
	\part The owner of a small start-up business with 19 employees asks each of them to submit a timesheet logging their hours worked per day. He finds that, in the last week, the employees at his business worked an average of 9.5 hours per day.
	
	\begin{solution}[\stretch{1}]		
			\vspace{3mm}				
			Descriptive statistics --- the small business owner only wants to summarize the data collected from the employees at his company and is not trying to generalize to a larger population.
			
			\vspace{3mm}
		\end{solution}
	
	\end{parts}
	

\end{questions}

%-----------------------------------------------------------------------------%

\end{document}
