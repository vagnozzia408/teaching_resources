%% In the documentclass line, replace "noanswers" with "answers" to view the key.

\documentclass[noanswers]{exam}
\usepackage[utf8]{inputenc}

\title{Learning Activity 16}
\author{Chapter 9}
\date{STAT 3090}

\usepackage[bottom=2.2cm, left=2.2cm, right=2.2cm, top=2.2cm]{geometry}
%\usepackage[paperheight=11in, paperwidth=17in, margin=1in]{geometry}
\usepackage{dsfont}
\usepackage{amsmath}
\usepackage{amssymb}
\usepackage{amsthm}
\usepackage{array}
\usepackage{stmaryrd}
\usepackage{pgfplots}
\pgfplotsset{width=10cm,compat=1.9}
\usepackage{multicol}
\setlength{\columnsep}{1in}
\usepackage{nicefrac}

\usepackage{multirow}
\usepackage{enumitem}[shortlabels]
\usepackage{tabu}
\definecolor{purp}{RGB}{102,0,204}
\usepackage{tabularx}
\newcolumntype{C}{>{\centering\arraybackslash $}X<{$}}
\usepackage{wrapfig}
\usepackage[export]{adjustbox}


\makeatletter
\pagestyle{headandfoot}
\firstpageheader{\@date}{\@title}{\@author}
\firstpageheadrule
\runningfootrule
\runningfooter{}{\thepage\ / \numpages}{\@title}
\makeatother

\newcommand{\abs}[1]{\left|#1\right|}
\newcommand{\mat}[4]{\left( \begin{tabular}{>{$}c<{$} >{$}c<{$}} #1&#2 \\ #3&#4 \end{tabular} \right)}
\newcommand{\msc}[1]{\mathds{#1}}
\newcommand{\Z}{\mathds{Z}}
\newcommand{\R}{\mathds{R}}
\newcommand{\N}{\mathds{N}}
\newcommand{\Q}{\mathds{Q}}
\newcommand{\C}{\mathds{C}}
\newcommand{\so}{\implies}
\newcommand{\set}[2]{\left\{ #1 \:|\: #2 \right\}}
\newcommand{\bso}{\Longleftarrow}
\newcommand{\ra}{\rightarrow}
\newcommand{\gen}[1]{\left\langle #1 \right\rangle}
\newcommand{\olin}[1]{\overline{#1}}
\newcommand{\Img}[1]{\text{Im}\left(#1\right)}
\newcommand{\llra}{\longleftrightarrow}
\newcommand{\lra}{\longrightarrow}
\newcommand{\xra}[1]{\xrightarrow{#1}}
\newcommand{\wo}{\setminus}
\newcommand{\mcal}[1]{\mathcal{#1}}
\newcommand{\Aut}[1]{\text{Aut}\left(#1\right)}
\newcommand{\Inn}[1]{\text{Inn}\left(#1\right)}
\newcommand{\syl}[2]{\text{Syl}_{#1}(#2)}
\newcommand{\norm}[1]{\left\|#1\right\|}
\newcommand{\infnorm}[1]{\left\|#1\right\|_{\infty}}
\newcommand{\xn}{\{x_n\}}
\newcommand{\sig}{\sigma}
\newcommand{\id}{\text{id}}
\newcommand{\ep}{\epsilon}
\newcommand{\st}{\text{ s.t. }}
\newcommand{\ran}[1]{\text{Ran}(#1)}
\newcommand{\nCr}[2]{\binom{#1}{#2}}
\newcommand{\Exr}[1]{\paragraph{Exercise #1:}}
\newcommand{\pg}{\paragraph{}}
\newcommand{\ulin}[1]{\underline{#1}}
\newcommand{\tc}[1]{\textcolor{purp}{#1}}

% Solution Specs
\unframedsolutions
\renewcommand{\solutiontitle}{}
\SolutionEmphasis{\color{purp}}
\CorrectChoiceEmphasis{\color{purp}\bfseries}
\setlength\fillinlinelength{0in}
\renewcommand{\arraystretch}{2}


\begin{document}
\noindent\begin{tabular}{@{}p{1.05in}p{5.5in}@{}}
Group Members: & \hrulefill
\end{tabular}

\vspace{3mm}

\begin{questions}
		
	\question Taylor wants to estimate the true proportion of undergraduate students at Clemson University who watch \textit{Game of Thrones}. She randomly selects 450 Clemson undergraduate students and finds that 157 of them are dedicated \textit{Game of Thrones} fans and watch the show regularly.
	
	\vspace{3mm}
	
	\begin{parts}
	
	\part Find a \textbf{point estimate} for the true proportion of Clemson students who watch \textit{Game of Thrones}. Label your value with the appropriate symbol and round your point estimate to four decimal places.
	
	\begin{solution}[\stretch{1}]
	
	\vspace{3mm}

	$\hat{p}=\frac{157}{450}=0.3489$
	
	\vspace{3mm}

	\end{solution}	
	
	\part Determine whether the \textbf{two conditions} for inference using confidence intervals are met.
		
	\begin{solution}[\stretch{1}]
	
	\vspace{3mm}

	(1) It is stated that a random sample of undergraduates is chosen from the population of Clemson students.
	
	\vspace{3mm}
	
	(2) The sampling distribution of $\hat{p}$ is normally distributed because $n\hat{p}=450(0.35)=157.5\geq5$ and $n(1-\hat{p})=450(0.65)=292.5\geq 5$.
	
	\vspace{3mm}

	\end{solution}	
	
	\part Find the \textbf{critical value} associated with a 94\% confidence level. You can do so either using the standard normal table or the \verb|invNorm| function in your calculator. (Drawing a sketch may be helpful.)
	
	\begin{solution}[\stretch{1}]
	
	\vspace{3mm}

	$\frac{\alpha}{2}=\frac{1-0.94}{2}=0.03$
	
	\vspace{3mm}
	
	Critical Value: $z_{.03}=\text{invNorm}(1-0.03,0,1)=1.88$
	
	\vspace{3mm}

	\end{solution}	
	
	\part Find a \textbf{94\% confidence interval} for the true proportion of Clemson students who watch \textit{Game of Thrones} based on the information gathered from Taylor's sample. \textbf{Show your work} by writing the confidence interval formula with the appropriate values plugged in. Round your final values to four decimal places and write your answer in interval notation.
	
	\begin{solution}[\stretch{1}]
	
	\vspace{3mm}

	$\hat{p}\pm z_{\nicefrac{\alpha}{2}}\sqrt{\frac{\hat{p}(1-\hat{p})}{n}}\;=\;0.3489\pm 1.88\sqrt{\frac{0.3489(1-0.3489)}{450}}\;=\;0.3489\pm0.0422\;=\;(0.3067,0.3911)$
	
	\vspace{3mm}

	\end{solution}	
	
	\part \textbf{Interpret} the confidence interval you found in Part (d).
	
	\begin{solution}[\stretch{1}]
	
	\vspace{3mm}
	
	CI Interpretation: We are 94\% confident that the true proportion of Clemson undergraduate students who watch \textit{Game of Thrones} is between 0.3067 and 0.3911.
	
	\vspace{3mm}
	
	\underline{\textbf{Note:}} If you downloaded LA 16 before noon on Friday, you may have downloaded a version that said to interpret the confidence \textbf{level}. This was my mistake --- I meant for you to interpret your \textbf{interval} from Part (d) --- but if you gave a correct interpretation of the confidence level, I did not take off points. 
	
	\vspace{3mm}
	
	Confidence Level Interpretation: If we took many samples of 450 randomly selected Clemson students, we would expect 94\% of the intervals to contain the true proportion who watch \textit{Game of Thrones}.
	\end{solution}
	
	\vspace{3mm}
	
	\end{parts}
	
	\question Taylor wants to expand her study. What is the \textbf{minimum} number of students that she would need to sample in order to generate a 99\% confidence interval with a 5\% margin of error? You can use your point estimate from Problem \#1 in your calculations. \textbf{Show your work }and include units in your answer.
	
	\begin{solution}[\stretch{1}]
	
	\vspace{3mm}
	
	$\displaystyle n=\frac{(2.576)^2(0.3489)(1-0.3489)}{(0.05)^2}=602.98\approx 603$ students
	
	\vspace{3mm}
	\end{solution}
	
	
\end{questions}
%-----------------------------------------------------------------------------%

\end{document}
