%% In the documentclass line, replace "noanswers" with "answers" to view the key.

\documentclass[noanswers]{exam}
\usepackage[utf8]{inputenc}

\title{Learning Activity 15}
\author{Chapter 9}
\date{STAT 3090}

\usepackage[bottom=2.2cm, left=2.2cm, right=2.2cm, top=2.2cm]{geometry}
%\usepackage[paperheight=11in, paperwidth=17in, margin=1in]{geometry}
\usepackage{dsfont}
\usepackage{amsmath}
\usepackage{amssymb}
\usepackage{amsthm}
\usepackage{array}
\usepackage{stmaryrd}
\usepackage{pgfplots}
\pgfplotsset{width=10cm,compat=1.9}
\usepackage{multicol}
\setlength{\columnsep}{1in}
\usepackage{nicefrac}

\usepackage{multirow}
\usepackage{enumitem}[shortlabels]
\usepackage{tabu}
\definecolor{purp}{RGB}{102,0,204}
\usepackage{tabularx}
\newcolumntype{C}{>{\centering\arraybackslash $}X<{$}}
\usepackage{wrapfig}
\usepackage[export]{adjustbox}


\makeatletter
\pagestyle{headandfoot}
\firstpageheader{\@date}{\@title}{\@author}
\firstpageheadrule
\runningfootrule
\runningfooter{}{\thepage\ / \numpages}{\@title}
\makeatother

\newcommand{\abs}[1]{\left|#1\right|}
\newcommand{\mat}[4]{\left( \begin{tabular}{>{$}c<{$} >{$}c<{$}} #1&#2 \\ #3&#4 \end{tabular} \right)}
\newcommand{\msc}[1]{\mathds{#1}}
\newcommand{\Z}{\mathds{Z}}
\newcommand{\R}{\mathds{R}}
\newcommand{\N}{\mathds{N}}
\newcommand{\Q}{\mathds{Q}}
\newcommand{\C}{\mathds{C}}
\newcommand{\so}{\implies}
\newcommand{\set}[2]{\left\{ #1 \:|\: #2 \right\}}
\newcommand{\bso}{\Longleftarrow}
\newcommand{\ra}{\rightarrow}
\newcommand{\gen}[1]{\left\langle #1 \right\rangle}
\newcommand{\olin}[1]{\overline{#1}}
\newcommand{\Img}[1]{\text{Im}\left(#1\right)}
\newcommand{\llra}{\longleftrightarrow}
\newcommand{\lra}{\longrightarrow}
\newcommand{\xra}[1]{\xrightarrow{#1}}
\newcommand{\wo}{\setminus}
\newcommand{\mcal}[1]{\mathcal{#1}}
\newcommand{\Aut}[1]{\text{Aut}\left(#1\right)}
\newcommand{\Inn}[1]{\text{Inn}\left(#1\right)}
\newcommand{\syl}[2]{\text{Syl}_{#1}(#2)}
\newcommand{\norm}[1]{\left\|#1\right\|}
\newcommand{\infnorm}[1]{\left\|#1\right\|_{\infty}}
\newcommand{\xn}{\{x_n\}}
\newcommand{\sig}{\sigma}
\newcommand{\id}{\text{id}}
\newcommand{\ep}{\epsilon}
\newcommand{\st}{\text{ s.t. }}
\newcommand{\ran}[1]{\text{Ran}(#1)}
\newcommand{\nCr}[2]{\binom{#1}{#2}}
\newcommand{\Exr}[1]{\paragraph{Exercise #1:}}
\newcommand{\pg}{\paragraph{}}
\newcommand{\ulin}[1]{\underline{#1}}
\newcommand{\tc}[1]{\textcolor{purp}{#1}}

% Solution Specs
\unframedsolutions
\renewcommand{\solutiontitle}{}
\SolutionEmphasis{\color{purp}}
\CorrectChoiceEmphasis{\color{purp}\bfseries}
\setlength\fillinlinelength{0in}
\renewcommand{\arraystretch}{2}


\begin{document}
\noindent\begin{tabular}{@{}p{1.05in}p{5.5in}@{}}
Group Members: & \hrulefill
\end{tabular}

\vspace{3mm}

\noindent \textbf{Introduction to Confidence Intervals. }Social networking sites have become fixtures in the social lives of many people around the world. A Pew Research poll surveyed U.S.\ residents to ask about their use of social media and study trends in usage. Of the 156 respondents aged 18 to 22 who use Facebook, 30.77\% stated that they updated their Facebook status at least once per day.

\vspace{3mm}

\begin{questions}
		
	\question Is 0.3077 a parameter or a statistic? Explain.
		
	\begin{solution}[\stretch{1}]
	
	\vspace{1mm}

	0.3077 is a \textbf{statistic} because it is a numerical summary of qualitative data from a \textbf{sample}.
	
	\vspace{1mm}

	\end{solution}	
	
	\question Does this finding necessarily \textbf{prove} that 30.77\% of all U.S.\ residents aged 18--22 years old update their Facebook status at least once per day? Explain why or why not.
	
	\begin{solution}[\stretch{1}]
	
	\vspace{1mm}

	No. This does not prove anything about the true percentage because it is \textbf{estimated} from a sample, which yields incomplete information about the population of all U.S.\ 18--22 year olds.
	
	\vspace{1mm}

	\end{solution}	
		
	\question We call the sample proportion $\hat{p}$ a \textbf{point estimate} for the population proportion $p$. The \textbf{\underline{estimated} \mbox{standard error}} of this sample statistic is defined as 

$$SE=\sqrt{\frac{\hat{p}(1-\hat{p})}{n}}.$$

\vspace{1mm}

	What is the difference between this estimate of standard error and the standard error $\sigma_{\hat{p}}$ we calculated in Chapter 8 for the sampling distribution of $\hat{p}$? (Hint: See page 124 of your Lecture Guide and compare.)
	
	\begin{solution}[\stretch{1}]
	
	\vspace{1mm}

	The standard error $\sigma_{\hat{p}}$ uses the population proportion $p$, and this estimate of standard error $SE$ uses the sample proportion $\hat{p}$.
	
	\vspace{1mm}

	\end{solution}	

\question Compute the estimated standard error of our sample statistic, $\hat{p}=0.3077$. Show your work and round to four decimal places.

	\begin{solution}[\stretch{1}]
	
	\vspace{1mm}

	$SE=\sqrt{\frac{0.3077(1-0.3077)}{156}}=0.0370$
	
	\vspace{1mm}

	\end{solution}
	
	\question Let's say we want to estimate \textbf{within two standard errors} of $\hat{p}$. We can do so by calculating $\hat{p}-2(SE)$ and $\hat{p}+2(SE)$. Find these values and write your result as an interval in the format (lower, upper). This is called an \textbf{interval estimate} of $p$.
	
	\begin{solution}[\stretch{1}]
	
	\vspace{1mm}

	$\hat{p}\pm 2(SE)=0.3077\pm 2(0.0370)=0.3077\pm0.0740=(0.2337,0.3817)$
	
	\vspace{1mm}

	\end{solution}
	
	\question Do you know for \textbf{certain} if $p$, the true proportion of \textbf{all} U.S.\ adults aged 18--22 who update their Facebook status at least one time per day, is contained in the interval? Why or why not?
	
	\begin{solution}[\stretch{1}]
	
	\vspace{1mm}

	We do not know for sure because this interval is based on data gathered from a sample, not the whole population. 
	
	\end{solution}
	
	Congratulations! You just constructed a \textbf{confidence interval}!
	
\end{questions}
%-----------------------------------------------------------------------------%

\end{document}
