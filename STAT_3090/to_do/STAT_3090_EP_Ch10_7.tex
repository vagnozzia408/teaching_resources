%% In the documentclass line, replace "noanswers" with "answers" to view the key.

\documentclass[noanswers]{exam}
\usepackage[utf8]{inputenc}

\title{Hypothesis Tests for Population Proportion}
\author{EP 10.7}
\date{STAT 3090}

\usepackage[bottom=2.2cm, left=2.2cm, right=2.2cm, top=2.2cm]{geometry}
%\usepackage[paperheight=11in, paperwidth=17in, margin=1in]{geometry}
\usepackage{dsfont}
\usepackage{amsmath}
\usepackage{amssymb}
\usepackage{amsthm}
\usepackage{array}
\usepackage{stmaryrd}
\usepackage{pgfplots}
\pgfplotsset{width=10cm,compat=1.9}
\usepackage{multicol}
\setlength{\columnsep}{1in}
\usepackage{nicefrac}

\usepackage{multirow}
\usepackage{enumitem}[shortlabels]
\usepackage{tabu}
\definecolor{purp}{RGB}{102,0,204}
\usepackage{tabularx}
\newcolumntype{C}{>{\centering\arraybackslash $}X<{$}}
\usepackage{wrapfig}
\usepackage[export]{adjustbox}


\makeatletter
\pagestyle{headandfoot}
\firstpageheader{\@date}{\@title}{\@author}
\firstpageheadrule
\runningfootrule
\runningfooter{}{\thepage\ / \numpages}{\@title}
\makeatother

\newcommand{\abs}[1]{\left|#1\right|}
\newcommand{\mat}[4]{\left( \begin{tabular}{>{$}c<{$} >{$}c<{$}} #1&#2 \\ #3&#4 \end{tabular} \right)}
\newcommand{\msc}[1]{\mathds{#1}}
\newcommand{\Z}{\mathds{Z}}
\newcommand{\R}{\mathds{R}}
\newcommand{\N}{\mathds{N}}
\newcommand{\Q}{\mathds{Q}}
\newcommand{\C}{\mathds{C}}
\newcommand{\so}{\implies}
\newcommand{\set}[2]{\left\{ #1 \:|\: #2 \right\}}
\newcommand{\bso}{\Longleftarrow}
\newcommand{\ra}{\rightarrow}
\newcommand{\gen}[1]{\left\langle #1 \right\rangle}
\newcommand{\olin}[1]{\overline{#1}}
\newcommand{\Img}[1]{\text{Im}\left(#1\right)}
\newcommand{\llra}{\longleftrightarrow}
\newcommand{\lra}{\longrightarrow}
\newcommand{\xra}[1]{\xrightarrow{#1}}
\newcommand{\wo}{\setminus}
\newcommand{\mcal}[1]{\mathcal{#1}}
\newcommand{\Aut}[1]{\text{Aut}\left(#1\right)}
\newcommand{\Inn}[1]{\text{Inn}\left(#1\right)}
\newcommand{\syl}[2]{\text{Syl}_{#1}(#2)}
\newcommand{\norm}[1]{\left\|#1\right\|}
\newcommand{\infnorm}[1]{\left\|#1\right\|_{\infty}}
\newcommand{\xn}{\{x_n\}}
\newcommand{\sig}{\sigma}
\newcommand{\id}{\text{id}}
\newcommand{\ep}{\epsilon}
\newcommand{\st}{\text{ s.t. }}
\newcommand{\ran}[1]{\text{Ran}(#1)}
\newcommand{\nCr}[2]{\binom{#1}{#2}}
\newcommand{\Exr}[1]{\paragraph{Exercise #1:}}
\newcommand{\pg}{\paragraph{}}
\newcommand{\ulin}[1]{\underline{#1}}
\newcommand{\tc}[1]{\textcolor{purp}{#1}}

% Solution Specs
\unframedsolutions
\renewcommand{\solutiontitle}{}
\SolutionEmphasis{\color{purp}}
\CorrectChoiceEmphasis{\color{purp}\bfseries}
\setlength\fillinlinelength{0in}
\renewcommand{\arraystretch}{2}
\usepackage{url}

\begin{document}

\noindent\begin{tabular}{@{}p{.5in}p{2.5in}@{}}
Name: & \hrulefill 
\end{tabular}

\vspace{3mm}

\begin{questions} 

\question Dunder Mifflin recently expanded its supply to include not just paper, but also copiers. David Wallace, the chief financial officer, claims that the expansion is a smashing success, with the copier production process producing defective parts only 0.7\% of the time.
    
    \vspace{3mm}
        
    Michael Scott is not superstitious\dots but he is a little stitious. He believes that David Wallace's claim is incorrect. He randomly selects 750 copier parts and finds 12 that are defective. Do these data provide evidence at the 10\% level that the proportion of defective copier parts is \textbf{not} 0.7\%?

\vspace{3mm}

\begin{parts}

\part Define the \textbf{parameter} of interest and state the \textbf{hypotheses}.

\begin{solution}[\stretch{1}]

\vspace{3mm}

Let $p=$ the true proportion of defective copier parts at Dunder Mifflin.

\vspace{3mm}

$H_0: p = 0.007$

$H_1: p \neq 0.007$

\vspace{3mm}

\end{solution}

\part Verify that the necessary \textbf{assumptions} hold.

\begin{solution}[\stretch{1}]

\vspace{3mm}

(1) Random sample --- stated that Michael randomly selects the copier parts.

\vspace{3mm}

(2) $\hat{p}$ must be approx.\ normal --- true because $np_0=750(0.007)\geq 5$ and $n(1-p_0)=750(0.993)\geq 5$.

\vspace{3mm}

\end{solution}

\part Calculate the appropriate \textbf{test} statistic. Show your work.

\begin{solution}[\stretch{1}]

\vspace{3mm}

$\displaystyle z_0=\frac{\hat{p}-p_0}{\sqrt{\frac{p_0(1-p_0)}{n}}}=\frac{\frac{12}{750}-0.007}{\sqrt{\frac{0.007(1-0.007)}{750}}} = 2.96$

\vspace{3mm}

\end{solution}

\part Find the \textbf{rejection region}. Include a sketch showing where your test statistic falls relative to the RR.

\begin{solution}[\stretch{1}]

\vspace{3mm}

$\alpha=0.1 \; \Rightarrow$ \underline{Critical Value}:  $z_{\nicefrac{\alpha}{2}}=z_{.05}=1.64$ \hspace{30mm} RR: $Z>1.64$ or $Z<-1.64$

\vspace{3mm}

\begin{tikzpicture}
        \def\normaltwo{\x,{2*1/exp(((\x-3)^2)/2)}}
        \def\y{1.5}
        \def\mu{3}
        \def\z{4.5}
        \def\fy{2*1/exp(((\y-3)^2)/2)}
        \def\fz{2*1/exp(((\z-3)^2)/2)}
        \fill [fill=purp!30] (-.5,-.1) -- plot[domain=-.5:\y] (\normaltwo) -- (\y,-.1) -- cycle;
        \fill [fill=purp!30] (\z,-.1) -- plot[domain=\z:6.5] (\normaltwo) -- (6.5,-.1) -- cycle;

        \draw[domain=-.5:6.5,samples=100] plot (\normaltwo) node[right] {};
        \draw[dashed] ({\y},{\fy}) -- ({\y},-.15) node[below] {\small{$-1.64$}};
        \draw[dashed] ({\z},{\fz}) -- ({\z},-.15) node[below] {\small{$1.64$}};

        \draw[] ({\mu},{0}) -- ({\mu},-.1) node[below] {\small{$0$}};
        \draw[-] (-.7,-.1) -- (6.7,-.1) node[right] {};
        \node[] at (7.0,-.1) {$Z$};
        \node[] at (0.7,1) {$0.05$};
        \draw[-] (0.7,0.8) -- (0.9,.1);
        \node[] at (5.3,1) {$0.05$};
        \draw[-] (5.3,0.8) -- (5.1,.1);
        
        \draw[] (5.5,{0}) -- (5.5,-.1) node[below] {\small{$2.96$}};
 \end{tikzpicture}

\vspace{3mm}

\end{solution} 

\part Use the results of the test to provide \textbf{support} for your decision about the null hypothesis.

\begin{solution}[\stretch{1}]

\vspace{3mm}

Because $z_0=2.96>1.64$ ($z_0$ is in the rejection region), we reject $H_0$.

\vspace{3mm}

\end{solution}

\part \textbf{Summarize} the results of your hypothesis test in context.

\begin{solution}[\stretch{1}]

\vspace{3mm}

At the $\alpha=0.1$ significance level, we have sufficient evidence that the true proportion of defective parts at Dunder Mifflin is not 0.7\% (i.e.\ we have sufficient evidence that David Wallace's claim is incorrect). 

\vspace{3mm}

\end{solution} 

\end{parts}

%\newpage 

\question As regional co-manager of Dunder Mifflin, Jim suggests using the $p$-value approach to double check behind Michael. Find the appropriate $p$-value (use appropriate probability notation) and provide support for a decision about $H_0$. Does your conclusion change at all?

\begin{solution}[\stretch{1}]

\vspace{3mm}

$p$-value $=2\times P(Z>2.96)=2\times \text{normalcdf}(2.96,1\text{\sc{e}}99,0,1)=2(0.001538)=0.0031$

\vspace{3mm}

Reject $H_0$ because $p$-value $=0.0031<\alpha=0.1$. The conclusion does not change.

\vspace{3mm}

\end{solution} 

\end{questions}




%-----------------------------------------------------------------------------%

\end{document}
