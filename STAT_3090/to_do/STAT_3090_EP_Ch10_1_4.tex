%% In the documentclass line, replace "noanswers" with "answers" to view the key.

\documentclass[noanswers]{exam}
\usepackage[utf8]{inputenc}

\title{Hypothesis Tests for Population Mean}
\author{EP 10.1 \& 10.4}
\date{STAT 3090}

\usepackage[bottom=2.2cm, left=2.2cm, right=2.2cm, top=2.2cm]{geometry}
%\usepackage[paperheight=11in, paperwidth=17in, margin=1in]{geometry}
\usepackage{dsfont}
\usepackage{amsmath}
\usepackage{amssymb}
\usepackage{amsthm}
\usepackage{array}
\usepackage{stmaryrd}
\usepackage{pgfplots}
\pgfplotsset{width=10cm,compat=1.9}
\usepackage{multicol}
\setlength{\columnsep}{1in}
\usepackage{nicefrac}

\usepackage{multirow}
\usepackage{enumitem}[shortlabels]
\usepackage{tabu}
\definecolor{purp}{RGB}{102,0,204}
\usepackage{tabularx}
\newcolumntype{C}{>{\centering\arraybackslash $}X<{$}}
\usepackage{wrapfig}
\usepackage[export]{adjustbox}


\makeatletter
\pagestyle{headandfoot}
\firstpageheader{\@date}{\@title}{\@author}
\firstpageheadrule
\runningfootrule
\runningfooter{}{\thepage\ / \numpages}{\@title}
\makeatother

\newcommand{\abs}[1]{\left|#1\right|}
\newcommand{\mat}[4]{\left( \begin{tabular}{>{$}c<{$} >{$}c<{$}} #1&#2 \\ #3&#4 \end{tabular} \right)}
\newcommand{\msc}[1]{\mathds{#1}}
\newcommand{\Z}{\mathds{Z}}
\newcommand{\R}{\mathds{R}}
\newcommand{\N}{\mathds{N}}
\newcommand{\Q}{\mathds{Q}}
\newcommand{\C}{\mathds{C}}
\newcommand{\so}{\implies}
\newcommand{\set}[2]{\left\{ #1 \:|\: #2 \right\}}
\newcommand{\bso}{\Longleftarrow}
\newcommand{\ra}{\rightarrow}
\newcommand{\gen}[1]{\left\langle #1 \right\rangle}
\newcommand{\olin}[1]{\overline{#1}}
\newcommand{\Img}[1]{\text{Im}\left(#1\right)}
\newcommand{\llra}{\longleftrightarrow}
\newcommand{\lra}{\longrightarrow}
\newcommand{\xra}[1]{\xrightarrow{#1}}
\newcommand{\wo}{\setminus}
\newcommand{\mcal}[1]{\mathcal{#1}}
\newcommand{\Aut}[1]{\text{Aut}\left(#1\right)}
\newcommand{\Inn}[1]{\text{Inn}\left(#1\right)}
\newcommand{\syl}[2]{\text{Syl}_{#1}(#2)}
\newcommand{\norm}[1]{\left\|#1\right\|}
\newcommand{\infnorm}[1]{\left\|#1\right\|_{\infty}}
\newcommand{\xn}{\{x_n\}}
\newcommand{\sig}{\sigma}
\newcommand{\id}{\text{id}}
\newcommand{\ep}{\epsilon}
\newcommand{\st}{\text{ s.t. }}
\newcommand{\ran}[1]{\text{Ran}(#1)}
\newcommand{\nCr}[2]{\binom{#1}{#2}}
\newcommand{\Exr}[1]{\paragraph{Exercise #1:}}
\newcommand{\pg}{\paragraph{}}
\newcommand{\ulin}[1]{\underline{#1}}
\newcommand{\tc}[1]{\textcolor{purp}{#1}}

% Solution Specs
\unframedsolutions
\renewcommand{\solutiontitle}{}
\SolutionEmphasis{\color{purp}}
\CorrectChoiceEmphasis{\color{purp}\bfseries}
\setlength\fillinlinelength{0in}
\renewcommand{\arraystretch}{2}
\usepackage{url}

\begin{document}

\noindent\begin{tabular}{@{}p{.5in}p{2.5in}@{}}
Name: & \hrulefill 
\end{tabular}

\vspace{3mm}

\begin{questions} 

\question Suppose that a media research group believes that the average age by which an individual under 25 has seen all 8 Harry Potter films is 14 years old. You want to see if this average is higher for students in your residence hall. You take a random sample of 65 fellow students from your dorm and find that the average age by which they had seen all 8 films is 14.8 years old with a standard deviation of 4.5 years old. Test your hypothesis at the $\alpha=0.01$ level using the \textbf{rejection region} approach.

\vspace{3mm}

\begin{parts}

\part Define the \textbf{parameter} of interest and state the \textbf{hypotheses}.

\begin{solution}[\stretch{1}]

\vspace{3mm}

Let $\mu=$ the true mean age by which students in your dorm have seen all eight Harry Potter films.

\vspace{3mm}

$H_0:\mu=14$

$H_1:\mu>14$

\vspace{3mm}

\end{solution}

\part Verify that the necessary \textbf{assumptions} hold.

\begin{solution}[\stretch{1}]

\vspace{3mm}

(1) The sample must be randomly selected from the target population --- it is stated in the problem that you took a random sample of 65 students from those in your dorm.

\vspace{3mm}

(2) The sampling distribution of $\overline{X}$ must be approximately normally distributed --- this is true by the Central Limit Theorem because $n=65>30$.

\vspace{3mm}

\end{solution}

\part Conduct your test by finding the \textbf{test statistic} and using the \textbf{rejection region} approach.

\begin{solution}[\stretch{1}]

\vspace{3mm}

\underline{Test Statistic}: $t_0=\displaystyle\frac{\overline{x}-\mu_0}{\nicefrac{s}{\sqrt{n}}}=\frac{14.8-14}{\nicefrac{4.5}{\sqrt{65}}}=1.43$

\vspace{3mm}

\underline{Critical Value}: $H_1$ indicates a \textit{right}-tailed test, so choose the $t$ critical value with an area of $\alpha=0.01$ to the \textit{right} and $df=65-1=64$.

\vspace{3mm}

$t_{.01,64}=\text{invT}(1-.01,64)=2.386$ \hspace{55mm} Rejection Region: $T>2.386$

\vspace{3mm}

\begin{tikzpicture}
        \def\normaltwo{\x,{2*1/exp(((\x-3)^2)/2)}}
        \def\y{5}
        \def\mu{3}
        \def\fy{2*1/exp(((\y-3)^2)/2)}
        \fill [fill=purp!30] (\y,-.1) -- plot[domain=\y:6.5] (\normaltwo) -- (6.5,-.1) -- cycle;
        \draw[domain=-.5:6.5,samples=100] plot (\normaltwo) node[right] {};
        \draw[dashed] ({\y},{\fy}) -- ({\y},-.15) node[below] {\small{$t_{.01,64}$}};
        \node at ({\y},-.8) {\small{$2.386$}};
        \draw[] ({\mu},{0}) -- ({\mu},-.1) node[below] {\small{$0$}};
        \draw[-] (-.7,-.1) -- (6.7,-.1) node[right] {};
        \node[] at (7.0,-.1) {$T$};
        \node[] at (5.5,1) {$0.01$};
        \draw[-] (5.5,0.8) -- (5.2,.1);
        \node at (3,1) {$0.99$};
  \end{tikzpicture}

\vspace{3mm}

\end{solution}

\part What is your decision about the null hypothesis $H_0$? Justify your answer with the appropriate \textbf{support} from your test results.

\begin{solution}[\stretch{1}]

\vspace{3mm}

We do not reject $H_0$ because the test statistic $t_0$ is not in the rejection region ($1.43<2.386$).

\vspace{3mm}

\end{solution} 

\part \textbf{Summarize} the results of your test in context.

\begin{solution}[\stretch{1}]

\vspace{3mm}

At the $\alpha=0.01$ significance level, we do not have sufficient evidence that the true mean age by which students in the dorm have seen all eight Harry Potter Films is higher than 14 years old (i.e.\ higher than the mean age for all individuals under 25 years old). 

\vspace{3mm}

\end{solution} 

\end{parts}

\newpage

\question Tony Stark is developing a new model of the Iron Man suit. His previous suit model uses an average of 222.4 kilowatts (kW), and he wishes to see if his new model is more efficient (in other words, if the mean energy consumption in kW is less than the previous model). He takes a sample of five randomly selected suits made under the new model and finds that the mean energy consumption is 210.2 kW with a standard deviation of 9.9 kW. He knows that the energy consumption for a given suit follows a normal distribution based on his previous tests. Test Tony's hypothesis at the $\alpha=0.05$ significance level using the \textbf{p-value} approach.

\vspace{3mm}

\begin{parts}

\part Define the \textbf{parameter} of interest and state the \textbf{hypotheses}.

\begin{solution}[\stretch{1}]

\vspace{3mm}

Let $\mu=$ true mean energy consumption in kW for the new Iron Man suit model.

\vspace{3mm}

$H_0:\mu=222.4$ kW

$H_1:\mu<222.4$ kW

\vspace{3mm}

\end{solution} 

\part State and verify the necessary \textbf{assumptions}.

\begin{solution}[\stretch{1}]

\vspace{3mm}

(1) Must have a random sample selected from the population --- the problem states that the five suits were randomly selected.

\vspace{3mm}

(2) $\overline{X}$ must be approximately normally distributed --- this is true since we know that the population energy consumption is normally distributed (stated).

\vspace{3mm}

\end{solution} 

\part Conduct your test by finding the \textbf{test statistic} and using the \textbf{p-value} approach.

\begin{solution}[\stretch{1}]

\vspace{3mm}

\underline{Test Statistic}: $t_0=\displaystyle\frac{\overline{x}-\mu_0}{\nicefrac{s}{\sqrt{n}}}=\frac{210.2-222.4}{\nicefrac{9.9}{\sqrt{5}}}=-2.76$

\vspace{3mm}

\underline{P-Value}: $p$-value $=P(T<-2.76)=\text{tcdf}(-\text{1\sc{e}99},-2.76,4)=0.0254$

\vspace{3mm}

\begin{tikzpicture}
        \def\normaltwo{\x,{2*1/exp(((\x-3)^2)/2)}}
        \def\y{1.2}
        \def\mu{3}
        \def\fy{2*1/exp(((\y-3)^2)/2)}
        \fill [fill=purp!30] (-.5,-.1) -- plot[domain=-.5:\y] (\normaltwo) -- (\y,-.1) -- cycle;
        \draw[domain=-.5:6.5,samples=100] plot (\normaltwo) node[right] {};
        \draw[dashed] ({\y},{\fy}) -- ({\y},-.15) node[below] {\small{$-2.76$}};
        \draw[] ({\mu},{0}) -- ({\mu},-.1) node[below] {\small{$0$}};
        \draw[-] (-.7,-.1) -- (6.7,-.1) node[right] {};
        \node[] at (7.0,-.1) {$T$};
        \node[] at (0.7,1) {$0.0254$};
        \draw[-] (0.7,0.8) -- (0.9,.1);
 \end{tikzpicture}

\vspace{3mm}

\end{solution} 

\part Provide \textbf{support} for your decision regarding the null hypothesis.

\begin{solution}[\stretch{1}]

\vspace{3mm}

We reject $H_0$ because the $p$-value is less than the significance level $\alpha$ ($0.0254<0.05$).

\vspace{3mm}

\end{solution}

\part \textbf{Summarize} the results of your test in context.

\begin{solution}[\stretch{1}]

\vspace{3mm}

At the $\alpha=0.05$ significance level, we have sufficient evidence that the true mean energy consumption for the new Iron Man suit is less than 222.4 kW (i.e.\ that the mean energy consumption for the new suit is lower than the previous model). 

\vspace{3mm}

\end{solution}

\end{parts}


\end{questions}




%-----------------------------------------------------------------------------%

\end{document}
