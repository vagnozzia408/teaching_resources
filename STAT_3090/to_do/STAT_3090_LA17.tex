%% In the documentclass line, replace "noanswers" with "answers" to view the key.

\documentclass[noanswers]{exam}
\usepackage[utf8]{inputenc}

\title{Learning Activity 17}
\author{Chapter 9}
\date{STAT 3090}

\usepackage[bottom=2.2cm, left=2.2cm, right=2.2cm, top=2.2cm]{geometry}
%\usepackage[paperheight=11in, paperwidth=17in, margin=1in]{geometry}
\usepackage{dsfont}
\usepackage{amsmath}
\usepackage{amssymb}
\usepackage{amsthm}
\usepackage{array}
\usepackage{stmaryrd}
\usepackage{pgfplots}
\pgfplotsset{width=10cm,compat=1.9}
\usepackage{multicol}
\setlength{\columnsep}{1in}
\usepackage{nicefrac}

\usepackage{multirow}
\usepackage{enumitem}[shortlabels]
\usepackage{tabu}
\definecolor{purp}{RGB}{102,0,204}
\usepackage{tabularx}
\newcolumntype{C}{>{\centering\arraybackslash $}X<{$}}
\usepackage{wrapfig}
\usepackage[export]{adjustbox}


\makeatletter
\pagestyle{headandfoot}
\firstpageheader{\@date}{\@title}{\@author}
\firstpageheadrule
\runningfootrule
\runningfooter{}{\thepage\ / \numpages}{\@title}
\makeatother

\newcommand{\abs}[1]{\left|#1\right|}
\newcommand{\mat}[4]{\left( \begin{tabular}{>{$}c<{$} >{$}c<{$}} #1&#2 \\ #3&#4 \end{tabular} \right)}
\newcommand{\msc}[1]{\mathds{#1}}
\newcommand{\Z}{\mathds{Z}}
\newcommand{\R}{\mathds{R}}
\newcommand{\N}{\mathds{N}}
\newcommand{\Q}{\mathds{Q}}
\newcommand{\C}{\mathds{C}}
\newcommand{\so}{\implies}
\newcommand{\set}[2]{\left\{ #1 \:|\: #2 \right\}}
\newcommand{\bso}{\Longleftarrow}
\newcommand{\ra}{\rightarrow}
\newcommand{\gen}[1]{\left\langle #1 \right\rangle}
\newcommand{\olin}[1]{\overline{#1}}
\newcommand{\Img}[1]{\text{Im}\left(#1\right)}
\newcommand{\llra}{\longleftrightarrow}
\newcommand{\lra}{\longrightarrow}
\newcommand{\xra}[1]{\xrightarrow{#1}}
\newcommand{\wo}{\setminus}
\newcommand{\mcal}[1]{\mathcal{#1}}
\newcommand{\Aut}[1]{\text{Aut}\left(#1\right)}
\newcommand{\Inn}[1]{\text{Inn}\left(#1\right)}
\newcommand{\syl}[2]{\text{Syl}_{#1}(#2)}
\newcommand{\norm}[1]{\left\|#1\right\|}
\newcommand{\infnorm}[1]{\left\|#1\right\|_{\infty}}
\newcommand{\xn}{\{x_n\}}
\newcommand{\sig}{\sigma}
\newcommand{\id}{\text{id}}
\newcommand{\ep}{\epsilon}
\newcommand{\st}{\text{ s.t. }}
\newcommand{\ran}[1]{\text{Ran}(#1)}
\newcommand{\nCr}[2]{\binom{#1}{#2}}
\newcommand{\Exr}[1]{\paragraph{Exercise #1:}}
\newcommand{\pg}{\paragraph{}}
\newcommand{\ulin}[1]{\underline{#1}}
\newcommand{\tc}[1]{\textcolor{purp}{#1}}

% Solution Specs
\unframedsolutions
\renewcommand{\solutiontitle}{}
\SolutionEmphasis{\color{purp}}
\CorrectChoiceEmphasis{\color{purp}\bfseries}
\setlength\fillinlinelength{1.5in}
\renewcommand{\arraystretch}{2}


\begin{document}
\noindent\begin{tabular}{@{}p{1.4in}p{5.2in}@{}}
Group Member Names: & \hrulefill
\end{tabular}

\vspace{1mm}
\noindent If you have group members who collaborated but were not logged into the Zoom session, please note in the submission comments how they collaborated on the learning activity.

\vspace{5mm}

\noindent Here are some problems to help you get comfortable working with the $t$-distribution! For each problem, \textbf{include a sketch} with the $t$ value on the horizontal axis and the corresponding shaded and labeled area. Use the \textbf{notation} on page 144 of the Lecture Guide to label your critical values.

\vspace{3mm}

\begin{questions}	
	
	\question What is the $t$ critical value for a 90\% confidence interval for $\mu$ for a sample of size 17?
		
	\begin{solution}[\stretch{1}]

\begin{multicols}{2}
[]

\begin{tikzpicture}
        \def\normaltwo{\x,{2*1/exp(((\x-3)^2)/2)}}
        \def\y{4}
        \def\mu{3}
        \def\fy{2*1/exp(((\y-3)^2)/2)}
        \fill [fill=purp!30] (\y,-.1) -- plot[domain=\y:6.5] (\normaltwo) -- (6.5,-.1) -- cycle;
        \draw[domain=-.5:6.5,samples=100] plot (\normaltwo) node[right] {};
        \draw[dashed] ({\y},{\fy}) -- ({\y},-.1) node[below] {\small{$t_{.05,16}$}};
        \draw[] ({\mu},{0}) -- ({\mu},-.1) node[below] {\small{$0$}};
        \draw[-] (-.7,-.1) -- (6.7,-.1) node[right] {};
        \node[] at (7.0,-.1) {$t$};
        \node[] at (5.5,1) {$0.05$};
        \draw[-] (5.5,0.8) -- (4.5,.3);
        \node at (3,1) {$0.95$};
    \end{tikzpicture}
    
    $\alpha=1-0.90=0.10 \Rightarrow \nicefrac{\alpha}{2}=0.050$

	$df=n-1=17-1=16$
	
	$t_{.05,16}=1.746$
    
\end{multicols}

	\end{solution}
	
	\question What is the $t$ critical value for a 99\% confidence interval for $\mu$ for a sample of size 17?
	
	\begin{solution}[\stretch{1}]
	
	\begin{multicols}{2}
	
	\begin{tikzpicture}
        \def\normaltwo{\x,{2*1/exp(((\x-3)^2)/2)}}
        \def\y{4.5}
        \def\mu{3}
        \def\fy{2*1/exp(((\y-3)^2)/2)}
        \fill [fill=purp!30] (\y,-.1) -- plot[domain=\y:6.5] (\normaltwo) -- (6.5,-.1) -- cycle;
        \draw[domain=-.5:6.5,samples=100] plot (\normaltwo) node[right] {};
        \draw[dashed] ({\y},{\fy}) -- ({\y},-.1) node[below] {\small{$t_{.005,16}$}};
        \draw[] ({\mu},{0}) -- ({\mu},-.1) node[below] {\small{$0$}};
        \draw[-] (-.7,-.1) -- (6.7,-.1) node[right] {};
        \node[] at (7.0,-.1) {$t$};
        \node[] at (5.5,1) {$0.005$};
        \draw[-] (5.5,0.8) -- (4.8,.2);
        \node at (3,1) {$0.995$};
    \end{tikzpicture}
    
    $\alpha=1-0.99=0.01 \Rightarrow \nicefrac{\alpha}{2}=0.005$
	
	$df=n-1=16$
	
	$t_{.005,16}=2.921$
	
	
	\end{multicols}

	\end{solution}	
	
	\question Answer the following based on your results for Problems \#1 and \#2: When the confidence level \textbf{increases}, we expect the $t$ critical value to be \fillin[larger] (larger/smaller). This means that the width of our confidence interval will \fillin[increase] (increase/decrease).
	
	\vspace{3mm}
	
	\question What is the $t$ critical value for a 99\% confidence interval for $\mu$ for a sample of size 31?
	
	\begin{solution}[\stretch{1}]
	
	\begin{multicols}{2}
	
	\begin{tikzpicture}
        \def\normaltwo{\x,{2*1/exp(((\x-3)^2)/2)}}
        \def\y{4.5}
        \def\mu{3}
        \def\fy{2*1/exp(((\y-3)^2)/2)}
        \fill [fill=purp!30] (\y,-.1) -- plot[domain=\y:6.5] (\normaltwo) -- (6.5,-.1) -- cycle;
        \draw[domain=-.5:6.5,samples=100] plot (\normaltwo) node[right] {};
        \draw[dashed] ({\y},{\fy}) -- ({\y},-.1) node[below] {\small{$t_{.005,30}$}};
        \draw[] ({\mu},{0}) -- ({\mu},-.1) node[below] {\small{$0$}};
        \draw[-] (-.7,-.1) -- (6.7,-.1) node[right] {};
        \node[] at (7.0,-.1) {$t$};
        \node[] at (5.5,1) {$0.005$};
        \draw[-] (5.5,0.8) -- (4.8,.2);
        \node at (3,1) {$0.995$};
    \end{tikzpicture}
	
	$\alpha=0.005, \; df=31-1=30$
	
	$t_{.005,30}=2.750$
	
	\end{multicols}
	
	\end{solution}
	
	\question Answer the following based on your results for Problems \#2 and \#4: When the confidence level stays the same and \textbf{sample size increases}, we expect the $t$ critical value to be \fillin[smaller] (larger/smaller). This means that the width of our confidence interval will \fillin[decrease] (increase/decrease).
	
	\vspace{3mm}
	
%\newpage
	\question Find the \textbf{approximate} $t$ critical value for a 95\% confidence interval for $\mu$ for a sample of size 84. Use the \textbf{t-table} (not a calculator). See Example 9.6 Part B.\ on page 145 of your lecture notes as a guide.
	
	\begin{solution}[\stretch{1}]
	
	\begin{multicols}{2}

	\begin{tikzpicture}
        \def\normaltwo{\x,{2*1/exp(((\x-3)^2)/2)}}
        \def\y{4.25}
        \def\mu{3}
        \def\fy{2*1/exp(((\y-3)^2)/2)}
        \fill [fill=purp!30] (\y,-.1) -- plot[domain=\y:6.5] (\normaltwo) -- (6.5,-.1) -- cycle;
        \draw[domain=-.5:6.5,samples=100] plot (\normaltwo) node[right] {};
        \draw[dashed] ({\y},{\fy}) -- ({\y},-.1) node[below] {\small{$t_{.025,83}$}};
        \draw[] ({\mu},{0}) -- ({\mu},-.1) node[below] {\small{$0$}};
        \draw[-] (-.7,-.1) -- (6.7,-.1) node[right] {};
        \node[] at (7.0,-.1) {$t$};
        \node[] at (5.5,1) {$0.025$};
        \draw[-] (5.5,0.8) -- (4.8,.2);
        \node at (3,1) {$0.975$};
    \end{tikzpicture}

	$\alpha=1-0.95=0.05 \Rightarrow \nicefrac{\alpha}{2}=0.025$
	
	$df=n-1=83$
	
	$t_{.025,83}\approx t_{.025,80}=1.990$
	
	\end{multicols}

	\end{solution}	
	
\end{questions}

\noindent Before you leave the Zoom session today, check to see if you have the \verb|invT| function in your TI calculator. If you do not and would like to have it, come chat with me before you leave class!
%-----------------------------------------------------------------------------%

\end{document}
