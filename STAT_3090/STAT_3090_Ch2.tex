%% In the documentclass line, replace "noanswers" with "answers" to view the key.

\documentclass[noanswers]{exam}
\usepackage[utf8]{inputenc}

\title{Practice Problems}
\author{Chapter 2}
\date{STAT 3090}

\usepackage[bottom=2.2cm, left=2.2cm, right=2.2cm, top=2.2cm]{geometry}
%\usepackage[paperheight=11in, paperwidth=17in, margin=1in]{geometry}
\usepackage{dsfont}
\usepackage{amsmath}
\usepackage{amssymb}
\usepackage{amsthm}
\usepackage{array}
\usepackage{stmaryrd}
\usepackage{pgfplots}
\pgfplotsset{width=10cm,compat=1.9}
\usepackage{multicol}
\setlength{\columnsep}{1in}
\usepackage{nicefrac}

\usepackage{multirow}
\usepackage{enumitem}[shortlabels]
\usepackage{tabu}
\definecolor{purp}{RGB}{102,0,204}
\usepackage{tabularx}
\newcolumntype{C}{>{\centering\arraybackslash $}X<{$}}
\usepackage{wrapfig}
\usepackage[export]{adjustbox}


\makeatletter
\pagestyle{headandfoot}
\firstpageheader{\@date}{\@title}{\@author}
\firstpageheadrule
\runningfootrule
\runningfooter{}{\thepage\ / \numpages}{\@title}
\makeatother

\newcommand{\abs}[1]{\left|#1\right|}
\newcommand{\mat}[4]{\left( \begin{tabular}{>{$}c<{$} >{$}c<{$}} #1&#2 \\ #3&#4 \end{tabular} \right)}
\newcommand{\msc}[1]{\mathds{#1}}
\newcommand{\Z}{\mathds{Z}}
\newcommand{\R}{\mathds{R}}
\newcommand{\N}{\mathds{N}}
\newcommand{\Q}{\mathds{Q}}
\newcommand{\C}{\mathds{C}}
\newcommand{\so}{\implies}
\newcommand{\set}[2]{\left\{ #1 \:|\: #2 \right\}}
\newcommand{\bso}{\Longleftarrow}
\newcommand{\ra}{\rightarrow}
\newcommand{\gen}[1]{\left\langle #1 \right\rangle}
\newcommand{\olin}[1]{\overline{#1}}
\newcommand{\Img}[1]{\text{Im}\left(#1\right)}
\newcommand{\llra}{\longleftrightarrow}
\newcommand{\lra}{\longrightarrow}
\newcommand{\xra}[1]{\xrightarrow{#1}}
\newcommand{\wo}{\setminus}
\newcommand{\mcal}[1]{\mathcal{#1}}
\newcommand{\Aut}[1]{\text{Aut}\left(#1\right)}
\newcommand{\Inn}[1]{\text{Inn}\left(#1\right)}
\newcommand{\syl}[2]{\text{Syl}_{#1}(#2)}
\newcommand{\norm}[1]{\left\|#1\right\|}
\newcommand{\infnorm}[1]{\left\|#1\right\|_{\infty}}
\newcommand{\xn}{\{x_n\}}
\newcommand{\sig}{\sigma}
\newcommand{\id}{\text{id}}
\newcommand{\ep}{\epsilon}
\newcommand{\st}{\text{ s.t. }}
\newcommand{\ran}[1]{\text{Ran}(#1)}
\newcommand{\nCr}[2]{\binom{#1}{#2}}
\newcommand{\Exr}[1]{\paragraph{Exercise #1:}}
\newcommand{\pg}{\paragraph{}}
\newcommand{\ulin}[1]{\underline{#1}}
\newcommand{\tc}[1]{\textcolor{purp}{#1}}

% Solution Specs
\unframedsolutions
\renewcommand{\solutiontitle}{}
\SolutionEmphasis{\color{purp}}
\CorrectChoiceEmphasis{\color{purp}\bfseries}

%\begin{solution}[\stretch{1}]
%	hurp durp flurp
%\end{solution}

%\pagestyle{empty}

\begin{document}

%\noindent\begin{tabular}{@{}p{.3in}p{3in}@{}}
%Name: & \hrulefill
%\end{tabular}

\begin{questions} 

	\question Identify whether each of the following is an \textbf{observational study} or a \textbf{designed experiment}. \textbf{Explain} how you know.
	
	\vspace{3mm}

	\begin{parts}
	
		\part 4-H is a youth development program that focuses on equipping young people with life skills they need to be successful. A university wishes to study the effects of being involved in 4-H growing up on college preparedness. Researchers identify 54 college students who are 4-H alumni and 52 college students who are not, then ask each of them whether they felt they had the skills they needed to succeed in college when they began as a freshman.
	
		\begin{solution}[\stretch{1}]	
			\vspace{2mm}	
		
			Observational study --- the researchers do not attempt to influence subjects or assign treatments. Rather, they simply observe characteristics of the individuals in the study.

			\vspace{3mm}
		\end{solution}
	
		\part A marine biologist wishes to study the effectiveness of two different dolphin training methods. She randomly selects three dolphins at the aquarium to be trained under the first method and three dolphins to be trained under the second. She works with each dolphin using the assigned method and records the length of time required for the dolphin to learn to toss a ball through a hoop at a certain signal.
	
		\begin{solution}[\stretch{1}]
			\vspace{2mm}
				
			Designed experiment --- the researcher assigns treatments (training methods) to subjects (the dolphins).

			\vspace{3mm}	
		\end{solution}	
	\end{parts}

	\question Identify each of the following \textbf{variables} described as (1) qualitative or quantitative, (2) discrete, continuous, or neither, and (3) nominal, ordinal, interval, or ratio.

\vspace{3mm}

	\begin{parts}
	
		\part A stressed-out college student records the amount of sleep he gets per night during a semester.
		
		\begin{solution}[\stretch{1}]
			\vspace{2mm}
					
			Quantitative, continuous, ratio
			
			\vspace{3mm}
		\end{solution}
	
		\part A chemist records the temperature (in $^\circ$C) of a heated solution every 30 seconds as it cools.
	
		\begin{solution}[\stretch{1}]		
			\vspace{2mm}				
			Quantitative, continuous, interval
			
			\vspace{3mm}
		\end{solution}
		
		\part A researcher records whether individuals in a sample of Clemson students purchase their textbooks online or from the campus bookstore.
	
		\begin{solution}[\stretch{1}]		
			\vspace{2mm}				
			Qualitative, neither, nominal
			
			\vspace{3mm}
		\end{solution}
		
		\part A coffee shop owner records the number of caramel macchiatos that are ordered each day by customers.
	
		\begin{solution}[\stretch{1}]		
			\vspace{2mm}				
			Quantitative, discrete, ratio
			
			\vspace{3mm}
		\end{solution}
		
		\part An instructor records the responses to a satisfaction evaluation (Very Satisfied, Satisfied, Neutral, Slightly Dissatisfied, Very Dissatisfied).
	
		\begin{solution}[\stretch{1}]		
			\vspace{2mm}				
			Qualitative, neither, ordinal
			
			\vspace{3mm}
		\end{solution}

		\part A movie enthusiast records the number of villains vanquished by the superheroes while watching \textit{Avengers Endgame.}
	
		\begin{solution}[\stretch{1}]		
			\vspace{2mm}				
			Quantitative, discrete, ratio
			
			\vspace{3mm}
		\end{solution}
		
		\part A bargain-hunter records the price of a pair of jeans at each thrift store in a city where she lives.
	
		\begin{solution}[\stretch{1}]		
			\vspace{2mm}				
			Quantitative, discrete, ratio
			
			\vspace{3mm}
		\end{solution}

	\end{parts}
	\newpage
	\question Now it's your turn! Think of an example of a variable with each of the following levels of measurement.
	\vspace{3mm}
	\begin{parts}
	\part Interval
	\begin{solution}[\stretch{1}]		
			\vspace{2mm}	
			Answers may vary.
			\vspace{3mm}
		\end{solution}
	\part Ratio
	\begin{solution}[\stretch{1}]		
			\vspace{2mm}	
			Answers may vary.
			\vspace{3mm}
		\end{solution}
	\part Nominal
	\begin{solution}[\stretch{1}]		
			\vspace{2mm}	
			Answers may vary.
			\vspace{3mm}
		\end{solution}
	\part Ordinal
	
	\begin{solution}[\stretch{1}]		
			\vspace{2mm}	
			Answers may vary.
			\vspace{3mm}
		\end{solution}

	\end{parts}
	
	\question Describe an example of each of the following types of variables.
	\vspace{3mm}
	\begin{parts}
	\part Qualitative
	\begin{solution}[\stretch{1}]		
			\vspace{2mm}	
			Answers may vary.
			\vspace{3mm}
		\end{solution}
	\part Quantitative -- Discrete
	\begin{solution}[\stretch{1}]		
			\vspace{2mm}	
			Answers may vary.
			\vspace{3mm}
		\end{solution}
	\part Quantitative -- Continuous
	\begin{solution}[\stretch{1}]		
			\vspace{2mm}	
			Answers may vary.
			\vspace{3mm}
		\end{solution}
	\end{parts}

\newpage

\question You've been assigned the task of studying people's opinions on the latest full-house showing of \textit{The Sound of Music} at your local theatre. Because you took Statistical Methods, you know that you should generate a random sample to be representative of the opinions of all 240 play attendees. Identify the \textbf{sampling method} used in each of the strategies described below.  
	
	\vspace{3mm}

	\begin{parts}
	
		\part The theatre has three equally sized seating sections: left, center, and right. You randomly choose one person from the left section, one person from the center section, and one person from the right section, then repeat until you have selected 24 people.
	
		\begin{solution}[\stretch{1}]
	
			\vspace{3mm}		
		
			Stratified sampling

			\vspace{3mm}		
			
		\end{solution}
	
		\part You place all of the play attendees' tickets into a box and shuffle them around, then draw 24 numbers from the hat at random.
	
		\begin{solution}[\stretch{1}]
		
			\vspace{3mm}		
				
			Simple random sampling

			\vspace{3mm}		
				
		\end{solution}
	
		\part You select the first 24 people to leave the theatre.
		
		\begin{solution}[\stretch{1}]

			\vspace{3mm}		
			
			Convenience sampling

			\vspace{3mm}		
			
		\end{solution}
	
		\part Starting with the first seat in the front row, you select every fifth person until you have selected 24 people.
	
		\begin{solution}[\stretch{1}]
	
			\vspace{3mm}
	
			Systematic sampling
	
			\vspace{3mm}
	
		\end{solution}
	
		\part The theatre has ten rows of 24 seats. You throw a ten-sided die and choose the people in the row indicated by the die to be in your sample.
	
		\begin{solution}[\stretch{1}]
		
			\vspace{3mm}

			Cluster sampling
		
			\vspace{3mm}	
	
		\end{solution}
		
		\part Of the above five sampling methods, which are appropriate statistical techniques for an inferential statistical study? Why?
	
		\begin{solution}[\stretch{1}]
		
			\vspace{3mm}

			All but convenience sampling are appropriate sampling methods. Convenience sampling is non-random and can introduce bias into the study.
		
			\vspace{3mm}	
	
		\end{solution}
	
	\end{parts}
	
	\question In your own words, describe the difference between \textbf{random sampling} and \textbf{random assignment}. Consider how each one is performed and what each one allows us to argue about the relationship between an explanatory and a response variable.
	
	\begin{solution}[\stretch{1}]
		
			\vspace{3mm}

			Random sampling refers to how we select subjects for the sample from the population. Random  assignment refers to how we assign subjects to different treatments. 
			
			\vspace{3mm}
			
			Random sampling allows us to make inference to the entire population, while random assignment allows us to argue for causation between the explanatory and response variables.
		
			\vspace{3mm}	
	
		\end{solution}
	
\newpage	
	
	\question Chantix is a prescription pill used to help smokers who want to quit. However, some doctors are concerned about the potential negative side effects of the drug, including nausea, abnormal dreams, insomnia, and headaches. A study of the safety and effectiveness of Chantix included 1,492 smokers wanting to quit who volunteered to participate in a clinical trial of the drug.

\vspace{2mm}

At the start of the study, smokers were randomly assigned to either take Chantix or a placebo over a 24-week period, resulting in 746 participants in each treatment group. Medical researchers provided participants with the pill they were assigned, and participants were then observed for the occurrence of side effects. At the conclusion of the study, the proportions of participants who experienced serious side effects were compared for the two groups.

\vspace{3mm}

\begin{parts}
	\part Define the \textbf{explanatory} and \textbf{response} variables.
	\begin{solution}[\stretch{1}]
			\vspace{3mm}

			Explanatory variable: Whether a patient received Chantix or a placebo, Response variable: Whether or not the patient experienced serious side effects
		
			\vspace{3mm}	
		\end{solution}
	
	\part Describe the \textbf{control} in this experiment.
	\begin{solution}[\stretch{1}]
			\vspace{3mm}

			A treatment group receiving a placebo is present as a baseline for comparison.
		
			\vspace{3mm}	
		\end{solution}
	
	\part Describe how \textbf{replication} was achieved in this experiment.
	\begin{solution}[\stretch{1}]
			\vspace{3mm}

			Each treatment group has 746 participants.
		
			\vspace{3mm}	
		\end{solution}
	
	\part Describe how \textbf{randomization} is used in this study.
	\begin{solution}[\stretch{1}]
			\vspace{3mm}

			Participants are randomly assigned to either the Chantix or the placebo treatment.
		
			\vspace{3mm}	
		\end{solution}
		
	\part How would this experiment be conducted as a \textbf{double blind}? 
	\begin{solution}[\stretch{1}]
			\vspace{3mm}

			Participants would not know whether they were receiving the placebo or the real Chantix pill, and the medical researchers giving them the pill would not know which pill they were administering.
		
			\vspace{3mm}	
		\end{solution}
	
	\part Describe two \textbf{confounding variables} that could be present in this study.
	\begin{solution}[\stretch{1}]
			\vspace{3mm}

			Answers could include (but are not limited to):
			\begin{itemize}
				\item previous health of the participant
				\item age of the participant
				\item level of addiction to smoking 
				\item how many packs of cigarettes the participant smoked per day
				\item other drugs being taken (maybe there is an interaction between them)
			\end{itemize}
		
			\vspace{3mm}	
		\end{solution}
\end{parts}

\end{questions}

%-----------------------------------------------------------------------------%

\end{document}
