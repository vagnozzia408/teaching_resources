\documentclass[12pt]{article}
%%% DOCUMENT FORMATTING %%%
\usepackage[margin=1in]{geometry}
\usepackage{enumitem}
\setlength{\parindent}{0pt}
\newcommand{\disp}{\displaystyle}
\usepackage{multicol}

%%% HEADER %%%
\usepackage{fancyhdr}
\pagestyle{fancy}
\fancyhf{}
\lhead{MATH 1080}
\rhead{Vagnozzi}
\cfoot{\thepage}

%%% MATH NOTATION & SYMBOLS %%%
\usepackage{amssymb}
\usepackage{amsmath}
\newcommand{\R}{\mathbb{R}}
\newcommand{\N}{\mathbb{N}}
\newcommand{\Z}{\mathbb{Z}}
\newcommand{\lp}{\left(}
\newcommand{\rp}{\right)}
\newcommand{\ls}{\left[}
\newcommand{\rs}{\right]}
\newcommand{\lb}{\left\{}
\newcommand{\rb}{\right\}}
\newcommand{\arccot}{\text{arccot}}
\newcommand{\arccsc}{\text{arccsc}}
\newcommand{\arcsec}{\text{arcsec}}
\DeclareSymbolFont{matha}{OML}{txmi}{m}{it}% txfonts
\DeclareMathSymbol{\varv}{\mathord}{matha}{118} 

%%% TABLES %%%
\usepackage{colortbl}

%%% GRAPHS %%%
\usepackage{tikz}
\usepackage{pgfplots}
\pgfplotsset{compat=1.15}
\usepgfplotslibrary{fillbetween}
\usetikzlibrary{angles,quotes}

%%% ENVIRONMENTS %%%
\newcommand{\Example}{\paragraph{\Writinghand \hspace{0.1mm} Example.}}
\newcommand{\Examples}{\paragraph{\Writinghand \hspace{0.1mm} Examples.}}
\newcommand{\ExampleCont}{\paragraph{\Writinghand \hspace{0.1mm} Example (continued).}}
\newcommand{\ExamplesCont}{\paragraph{\Writinghand \hspace{0.1mm} Examples (continued).}}

\newcommand{\boxenv}[2]{
	\fbox{
	\begin{minipage}{0.97\textwidth}
	\vspace{2mm}	
	\paragraph{#1} #2
	\vspace{2mm}
	\end{minipage}
	}}

%%% FUN THINGS %%%
\newcommand*\tc[1]{\tikz[baseline=(char.base)]{
            \node[shape=circle,draw,inner sep=2pt] (char) {#1};}}
\usepackage{marvosym}

%%% MISC %%%
\usepackage{hyperref}


\setcounter{page}{137}

\begin{document}
\section*{11.1: Approximating Functions with Polynomials}

\boxenv{Learning Objectives.}{Upon successful completion of Section 11.1, you will be able to\dots
		
	\begin{itemize}[leftmargin=6mm]
		\item Answer conceptual questions about Taylor polynomials.
		\item Use linear and quadratic polynomials to approximate functions.
		\item Find the Taylor polynomial for a function centered at a specified number.
		\item Use Taylor polynomials to approximate functions.
		\item Compare the graph of a function and its Taylor polynomials.
		\item Find the remainder of an $n^\text{th}$ order Taylor polynomial for a given function.
		\item Find the remainder term of a Taylor approximation and use it to estimate error.
	\end{itemize}
	\vspace{-4mm}
}

\vspace{5mm}

\subsection*{Power Series}

Chapter 11 focuses on a particular type of infinite series called a \textbf{power series}. 

\vspace{3mm}

\boxenv{Definition.}{A \textbf{power series} is an infinite series of the form
$$\disp\sum_{k=0}^\infty c_k x^k=c_0+c_1x+c_2x^2+\cdots+c_nx^n+c_{n+1}x^{n+1}\cdots,$$

or, more generally,
\vspace{-1mm}
$$\disp\sum_{k=0}^\infty c_k(x-a)^k=c_0+c_1(x-a)+c_2(x-a)^2+\cdots+c_n(x-a)^n+c_{n+1}(x-a)^{n+1}\cdots,$$

where $a$ and $c_k$ are constants. The values of $c_k$ are called the \textbf{coefficients}.
}

\vspace{3mm}

\paragraph*{Motivation.} Why are power series important?
\begin{itemize}
\item Functions can be represented by power series.
\item Functions can be defined as power series.
\item Power series are like infinitely long polynomials, and polynomials have a lot of nice properties that make them straightforward to work with.
\item Power series can be used to solve differential equations. (Differential equations were introduced in Section 4.9 in MATH 1060.)
\end{itemize}

\newpage

\textbf{Question:} What should the form of the coefficients $c_k$ be to provide a good approximation of a function? Let's explore! We will consider the function $f(x)=e^x$ centered at $x=0$.

\begin{enumerate}
	\item[\tc{1}] First, we will find a \textit{linear approximation} for $f(x)$ at $x=0$. In other words, we will find the equation $p_1(x)=c_0+c_1 x$ of the line tangent to $f(x)$ at $x=0$.
	
	\vfill
	
	\vfill

	\textbf{Note:} $p_1(x)$ and $f(x)$ have the same \underline{$y$-value} and same \underline{first\phantom{y}derivative} at $x=0$.
	
	\item[\tc{2}] To obtain a better approximation, let's consider a \textit{quadratic approximation} that adds a quadratic term to $p_1(x)$:
	$$p_2(x)=p_1(x)+c_2x^2=c_0+c_1x+c_2x^2$$
	
	We want a value for $c_2$ that results in a good approximation of $f(x)$ near \mbox{$x=0$.} $p_1(x)$~was a good linear approximation to $f(x)$ because the $y$-values and first derivatives matched. It would be reasonable to want a value $c_2$ so that the \underline{second\phantom{y}derivatives} match as well.

\begin{enumerate}
	\item[(a)] Calculate $p_2(0)$ to show that the \underline{$y$-values} match.
	
	\vfill
	
	\item[(b)] Calculate $p_2'(0)$ to show that the \underline{first\phantom{y}derivatives} match.
	
	\vfill
	
	\item[(c)] Calculate $p_2''(x)$. Then find a value for $c_2$ so that the \underline{second\phantom{y}derivatives} match.
	
		\vfill

\end{enumerate}
\end{enumerate}

\newpage

\begin{enumerate}
\item[\tc{3}] We can keep adding higher-order terms to get a better approximation to $f(x)$ at $x=0$. Let 
$$p_3(x)=c_0+c_1x+c_2x^2+c_3x^3.$$

Based on our work from the previous page, we know what $c_0$, $c_1$, and $c_2$ are. Let's find $c_3$ so that the \underline{third\phantom{y}derivatives} match.

\vfill

\item[\tc{4}] Let's look at one more approximation:
$$p_4(x)=c_0+c_1x+c_2x^2+c_3x^3+c_4x^4.$$
Find the value of $c_4$ so that the \underline{fourth\phantom{y}derivatives} are equal. 

\vfill
\end{enumerate}

\newpage

\subsection*{Taylor Polynomials}

The polynomial approximations we explored on the previous two pages are referred to as \textbf{Taylor polynomials}.

\vspace{5mm}

\boxenv{Definition.}{The \textbf{\textit{n}th-order Taylor polynomial} $p_n(x)$ for $f$ centered at $x=a$ is

\vspace{40mm}}

\vspace{10mm}

\Example Consider the function $f(x)=\sqrt{x}$.

\begin{enumerate}
\item[(a)] Find the Taylor polynomial $p_3$ at $a=1$ for $f(x)=\sqrt{x}$.

\vfill

\vfill

\item[(b)] Now use $p_3$ to approximate $\sqrt{1.06}$.

\vfill
\end{enumerate}

\newpage

\subsubsection*{Approximations with Taylor Polynomials}

Taylor polynomials provide a good approximation of $f$ near the center $a$, and the approximation gets better as the order of the Taylor polynomial increases. Similar to our work when approximating the sum of a series, we can determine \textit{how accurate} an approximation is through the idea of looking at remainders and error bounds.

\vspace{5mm}

\boxenv{Definition.}{Let $p_n$ be the \textit{n}th-order Taylor polynomial for $f$. The \textbf{remainder} in using $p_n$ to approximate $f$ at the point $x$ is

\vspace{20mm}

$\big|R_n(x)\big|$ is the \textbf{error} made in approximating $f$ by $p_n$.}

\vspace{5mm}

There are two theorems that will allow us to estimate bounds on the error in approximation with Taylor polynomials.

\vspace{5mm}

\boxenv{Taylor's Theorem (Remainder Theorem).}{Let $f$ have continuous derivatives up to $f^{(n+1)}$ on an open interval containing $a$. For all $x$-values in the interval,

\vspace{15mm}

where $p_n$ is the $n$th-order Taylor polynomial for $f$ centered at $a$ and the remainder is

$$R_n(x)=\dfrac{f^{(n+1)}(c)}{(n+1)!}(x-a)^{n+1},$$

for some point $x$ between $x$ and $a$.
}

\vspace{5mm}

\boxenv{Theorem: Estimate of the Remainder.}{Let $n$ be a fixed positive integer. Suppose there exists a number $M$ such that $\big|f^{(n+1)}(c)\big|\leq M$ for all $c$ between $a$ and $x$ inclusive. The remainder in the $n$th-order Taylor polynomial for $f$ centered at $a$ satisfies

$$\big| R_n(x)\big|=\big| f(x)-p_n(x)\big|\leq  M\dfrac{\big|x-a\big|^{n+1}}{(n+1)!}$$

\vspace{-2mm}
}

\vspace{5mm}

As a result of the two theorems above, we can get a \textbf{bound on the error} by\dots
\begin{itemize}
	\item finding $f^{(n+1)}(x)$,
	\item finding a number $M$ so that $\big|f^{(n+1)}(c)\big|\leq M$ for all $c\in\left[a,x\right]$, and then
	\item plugging $M$ into the formula from the second Theorem above.
\end{itemize}

\newpage

\boxenv{Finding Error Bounds.}{For a Taylor polynomial used to approximate $f$\dots
\begin{enumerate}
	\item[\tc{1}] Find $f^{(n+1)}(x)$.
	\item[\tc{2}] Find a number $M$ so that $\big|f^{(n+1)}(c)\big|\leq M$ for all $c\in\left[a,x\right]$.
	\item[\tc{3}] Plug $M$ into the estimation theorem to find the error bound:
	$$\big| R_n(x)\big| \leq  M\dfrac{\big|x-a\big|^{n+1}}{(n+1)!}$$
\end{enumerate}

\vspace{-3mm}}

\Example Consider the function $f(x)=\cos x$ centered at $a=0$. 

\begin{itemize}
\item[(a)] Find the Taylor polynomials of order $n=2$ and $n=3$ for $f$. 

\vfill

\item[(b)] Use the remainder term to find a bound on the error in the approximation $p_3(x)$ to $f(x)$ on the interval $\left[-\frac{\pi}{4},\frac{\pi}{4}\right]$.

\vfill
\end{itemize}

\newpage

\Example Let $\sqrt{x}\approx 2+\dfrac{1}{4}(x-4)-\dfrac{1}{64}(x-4)^2$ on $[4,4.2]$. Use the remainder term to find a bound on the error in the approximation on the given interval.

\vfill

\Example What is the minimum order of Taylor polynomial required to approximate $e^{-0.5}$ with an absolute error no greater than $10^{-3}$?

\vfill



\end{document} 