\documentclass[12pt]{article}
%%% DOCUMENT FORMATTING %%%
\usepackage[margin=1in]{geometry}
\usepackage{enumitem}
\setlength{\parindent}{0pt}
\newcommand{\disp}{\displaystyle}

%%% HEADER %%%
\usepackage{fancyhdr}
\pagestyle{fancy}
\fancyhf{}
\lhead{MATH 1080}
\rhead{Vagnozzi}
\cfoot{\thepage}

%%% MATH NOTATION & SYMBOLS %%%
\usepackage{amssymb}
\usepackage{amsmath}
\newcommand{\R}{\mathbb{R}}
\newcommand{\N}{\mathbb{N}}
\newcommand{\Z}{\mathbb{Z}}
\newcommand{\lp}{\left(}
\newcommand{\rp}{\right)}
\newcommand{\ls}{\left[}
\newcommand{\rs}{\right]}
\newcommand{\lb}{\left\{}
\newcommand{\rb}{\right\}}
\newcommand{\arccot}{\text{arccot}}
\newcommand{\arccsc}{\text{arccsc}}
\newcommand{\arcsec}{\text{arcsec}} 

%%% TABLES %%%
\usepackage{colortbl}

%%% GRAPHS %%%
\usepackage{tikz}
\usepackage{pgfplots}
\pgfplotsset{compat=1.15}
\usepgfplotslibrary{fillbetween}
\usetikzlibrary{angles,quotes}

%%% ENVIRONMENTS %%%
\newcommand{\Example}{\paragraph{\Writinghand \hspace{0.1mm} Example.}}
\newcommand{\ExampleCont}{\paragraph{\Writinghand \hspace{0.1mm} Example (continued).}}
\newcommand{\boxenv}[2]{
	\fbox{
	\begin{minipage}{0.97\textwidth}
	\vspace{2mm}	
	\paragraph{#1} #2
	\vspace{2mm}
	\end{minipage}
	}}

%%% FUN THINGS %%%
\newcommand*\tc[1]{\tikz[baseline=(char.base)]{
            \node[shape=circle,draw,inner sep=2pt] (char) {#1};}}
\usepackage{marvosym}

%%% MISC %%%
\usepackage{hyperref}


\setcounter{page}{104}

\begin{document}
\section*{10.3: Infinite Series}

\boxenv{Learning Objectives.}{Upon successful completion of Section 10.3, you will be able to\dots
		
	\begin{itemize}[leftmargin=6mm]
		\item Answer conceptual questions involving infinite series.
		\item Evaluate geometric series.
		\item Solve applications involving series.
		\item Write repeating decimals as series and fractions.
		\item Evaluate telescoping series.
		\item Evaluate an infinite series.
	\end{itemize}
	\vspace{-4mm}
}

\vspace{5mm}

\subsection*{Introduction}

In Section 10.1, an infinite series was defined as a sum of infinitely many terms. More formally, given a sequence $\left\{a_1,a_2,a_3,\dots\right\}$, the sum of its terms
$$a_1+a_2+a_3+\cdots=\sum_{k=1}^\infty a_k$$
is called an \textbf{infinite series}.

\vspace{5mm}

The \textbf{sequence of partial sums} $\left\{S_n\right\}$ associated with this series has the terms\dots

\begin{align*}
	S_1 &= a_1\\
	S_2 &= a_1+a_2\\
	S_3 &= a_1+a_2+a_3\\
	\vdots & \\
	S_n &= a_1+a_2+a_3+\cdots + a_n=\sum_{k=1}^n a_k,\text{ for }n=1,2,3,\dots
\end{align*}

If the sequence of partial sums $\left\{S_n\right\}$ has a limit $L$, the infinite series \textbf{converges} to that limit, and we write
$$\disp\lim_{k=1}^\infty a_k=\lim_{n\to\infty}\sum_{k=1}^n a_k=\lim_{n\to\infty}S_n=L.$$
If the sequence of partial sums diverges, the infinite series also \textbf{diverges}.

\newpage

\Example Determine if the series $\disp\sum_{i=2}^\infty\frac{2}{i^2-1}$ converges or diverges. \\
 
If it converges, determine the value to which it converges. (Note: This is a special type of series called a \textbf{telescoping series}.)

\vfill
\vfill

\subsection*{Geometric Series Test}

The geometric series  $\disp\sum_{n=1}^\infty ar^{n-1}=a+ar+ar^2+ar^3+\cdots, a\neq 0$\dots

\vspace{4mm}

\hspace{5mm} converges if \underline{\hspace{50mm}}.

\vspace{6mm}

\hspace{5mm} diverges if \underline{\hspace{50mm}}.

\Examples For each of the following, determine if the series converges or diverges. If it converges, find its sum.

\vspace{5mm}

$\disp\sum_{n=1}^\infty \frac{1}{2^n}$

\vfill

\newpage

\ExamplesCont For each of the following, determine if the series converges or diverges. If it converges, find its sum.

\vspace{5mm}

$\disp 0.3+0.03+0.003+0.0003+\cdots$

\vfill

$3-4+\disp\frac{16}{3}-\frac{64}{9}+\dots$

\vfill

$\disp\sum_{n=1}^\infty \frac{(-3)^{n-1}}{4^n}$

\vfill

\newpage

\subsection*{Properties of Convergent Series}
\begin{itemize}
	\item[\tc{1}] Suppose $\sum a_k$ converges to $A$ and $c$ is a real number. Then the series $\sum c a_k$ converges, and $\sum ca_k=c\sum a_k=cA$.
	\item[\tc{2}] Suppose $\sum a_k$ diverges. Then $\sum ca_k$ also diverges, for any real number $c\neq 0$.
	\item[\tc{3}] Suppose $\sum a_k$ converges to $A$ and $\sum b_k$ converges to $B$. Then the series $\sum(a_k\pm b_k)$ converges, and $\sum(a_k\pm b_k)=\sum a_k\pm \sum b_k=A\pm B$.
	\item[\tc{4}] Suppose $\sum a_k$ diverges and $\sum b_k$ converges. Then $\sum(a_k\pm b_k)$ diverges.
	\item[\tc{5}] If $M$ is a positive integer, then for $\disp\sum_{k=1}^\infty a_k$ and $\disp\sum_{k=M}^\infty a_k$, either both converge or both diverge.
	\begin{itemize}
		\item In general, \textit{whether} a series converges or diverges does not depend on a finite number of terms added to or removed from the series.
		\item The \textit{value} of a convergent series, however, does change if nonzero terms are added or removed.
	\end{itemize}
\end{itemize}

\Example Evaluate the following series or state that it diverges.

\vspace{5mm}

$\disp\sum_{k=0}^\infty \left(3\left(\frac{2}{5}\right)^k-2\left(\frac{5}{7}\right)^k\right)$

\vfill
\end{document}