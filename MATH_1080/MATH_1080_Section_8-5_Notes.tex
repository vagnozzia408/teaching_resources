\documentclass[12pt]{article}
%%% DOCUMENT FORMATTING %%%
\usepackage[margin=1in]{geometry}
\usepackage{enumitem}
\setlength{\parindent}{0pt}
\newcommand{\disp}{\displaystyle}

%%% HEADER %%%
\usepackage{fancyhdr}
\pagestyle{fancy}
\fancyhf{}
\lhead{MATH 1080}
\rhead{Vagnozzi}
\cfoot{\thepage}

%%% MATH NOTATION & SYMBOLS %%%
\usepackage{amssymb}
\usepackage{amsmath}
\newcommand{\R}{\mathbb{R}}
\newcommand{\N}{\mathbb{N}}
\newcommand{\Z}{\mathbb{Z}}
\newcommand{\lp}{\left(}
\newcommand{\rp}{\right)}
\newcommand{\ls}{\left[}
\newcommand{\rs}{\right]}
\newcommand{\lb}{\left\{}
\newcommand{\rb}{\right\}}
\newcommand{\arccot}{\text{arccot}}
\newcommand{\arccsc}{\text{arccsc}}
\newcommand{\arcsec}{\text{arcsec}} 

%%% TABLES %%%
\usepackage{colortbl}

%%% GRAPHS %%%
\usepackage{tikz}
\usepackage{pgfplots}
\pgfplotsset{compat=1.15}
\usepgfplotslibrary{fillbetween}
\usetikzlibrary{angles,quotes}

%%% ENVIRONMENTS %%%
\newcommand{\Example}{\paragraph{\Writinghand \hspace{0.1mm} Example.}}
\newcommand{\ExampleCont}{\paragraph{\Writinghand \hspace{0.1mm} Example (continued).}}
\newcommand{\boxenv}[2]{
	\fbox{
	\begin{minipage}{0.97\textwidth}
	\vspace{2mm}	
	\paragraph{#1} #2
	\vspace{2mm}
	\end{minipage}
	}}

%%% FUN THINGS %%%
\newcommand*\tc[1]{\tikz[baseline=(char.base)]{
            \node[shape=circle,draw,inner sep=2pt] (char) {#1};}}
\usepackage{marvosym}

%%% MISC %%%
\usepackage{hyperref}


\setcounter{page}{68}

\begin{document}
\section*{8.5: Partial Fractions}

\boxenv{Learning Objectives.}{Upon successful completion of Section 8.5, you will be able to\dots
		
	\begin{itemize}[leftmargin=6mm]
		\item Answer conceptual questions involving partial fractions.
		\item Set up or find a partial fraction decomposition.
		\item Evaluate integrals involving partial fractions with only simple linear factors.
		\item Evaluate integrals involving partial fractions with repeated linear factors.
		\item Evaluate integrals involving partial fractions with irreducible quadratic factors (simple or repeated).
		\item Evaluate integrals involving improper rational functions (where long division is needed first).
		\item Use partial fractions to find areas, volumes, or arc lengths.
		\item Evaluate integrals involving partial fractions that require a preliminary step (such as a change of variables).
	\end{itemize}
	\vspace{-4mm}
}

\vspace{5mm}

\subsection*{Introduction}

Consider the integral $\disp\int\frac{x+5}{x^2+x-2}\,dx$. Will any of our previous integration techniques work?

\vspace{2mm}

\begin{itemize}
	\item Antiderivative Rules
	\item U-Substitution
	\item Algebraic Manipulation
	\item Integration by Parts
	\item Trig Integral
	\item Trig Substitution
\end{itemize}

\vspace{4mm}

We can factor the denominator and write $\disp\frac{x+5}{(x+2)(x-1)}=\frac{A}{x+2}+\frac{B}{x-1}$, for some $A,B\in\R$.\\

This will result in two ``simpler'' integrals that we can evaluate.

\newpage

\subsection*{The Idea of Partial Fraction Decomposition}

The technique of \textbf{partial fraction decomposition} allows us to rewrite a rational function as a sum of two simpler rational functions. Recall that a rational function is a function of the form $f(x)=\frac{P(x)}{Q(x)}$, where $P$ and $Q$ are polynomials.\\

When can we use partial fraction decomposition (PFD) for a rational function? To determine this, we look at the degree of the numerator $(n)$ and degree of the denominator $(k)$.
\begin{itemize}
	\item PFD can be used when we have a \textbf{proper} rational function ($n<k$).
	\item If the rational function is \textbf{improper} ($n\geq k$), perform long division before PFD.
	
\end{itemize}

\Example Consider the improper rational function $\disp\frac{x^4+x+5}{x^3+3}$.

\vfill

\subsection*{Steps for Partial Fraction Decomposition}
Given a rational function\dots
\begin{enumerate}
	\item[\tc{1}] Determine if the function is proper or improper. If improper, do long division.
	\item[\tc{2}] If proper, factor the denominator as much as possible.
	\item[\tc{3}] Determine the \textit{form} of the proper rational function based on the denominator.
	\begin{itemize}
		\item Simple linear form
		\item Repeated linear form
		\item Irreducible quadratic form
		\item Repeated irreducible quadratic
	\end{itemize}
	\item[\tc{4}] Solve for the constants needed.
\end{enumerate}

\newpage

\subsection*{Forms in Partial Fraction Decomposition}

\subsubsection*{\tc{1} Simple Linear}
\Example $f(x)=\disp\frac{3}{x^3-x^2-12x}$

\vfill

\subsubsection*{\tc{2} Repeated Linear}
\Example $f(x)=\disp\frac{x^2-5x+16}{(2x+1)(x-2)^2}$

\vfill

\Example $f(x)=\disp\frac{x+2}{x^2(x+1)^3(2x+3)}$

\vfill

\subsubsection*{\tc{3} Irreducible Quadratic}
\Example $f(x)=\disp\frac{10}{(x-1)(x^2+9)}$

\vfill

\newpage

\subsubsection*{\tc{3} Irreducible Quadratic (continued)}

\Example $f(x)=\disp\frac{3+x}{x^2(2x^2-x-1)}$

\vfill

\subsubsection*{\tc{4} Repeated Irreducible Quadratic}
\Example $f(x)=\disp\frac{1}{x(x^2+4)^2}$

\vfill

\Example Write out the form of the partial fraction decomposition of the function:
\begin{enumerate}
\item $f(x)=\disp\frac{x^4}{(x+3)^3(x^2-x+1)}$

\vspace{20mm}

\item $f(x)=\disp\frac{3x}{(x-1)(x^4+4x^2+4)}$

\vspace{20mm}

\end{enumerate}

\newpage

\subsection*{Integrating Rational Functions Using Partial Fractions}

\Example $\disp\int\frac{3}{x^3-x^2-12x}\,dx$

\newpage

\Example $\disp\int\frac{x^2-5x+16}{(2x+1)(x-2)^2}\,dx$

\newpage

\Example $\disp\int\frac{10}{(x-1)(x^2+9)}\,dx$

\newpage

\Example $\disp\int\frac{1}{x(x^2+4)^2}\,dx$

\newpage

How can \textbf{u-substitution} be used to convert an integrand into a rational function so we can apply partial fraction decomposition?\\

\Example $\disp\int\frac{e^{2x}}{e^{2x}+3e^x+2}\,dx$

\vfill

\Example $\disp\int\frac{\sqrt{1+\sqrt{x}}}{x}\,dx$

\vfill
\end{document}