\documentclass[12pt]{article}
%%% DOCUMENT FORMATTING %%%
\usepackage[margin=1in]{geometry}
\usepackage{enumitem}
\setlength{\parindent}{0pt}
\newcommand{\disp}{\displaystyle}
\usepackage{multicol}

%%% HEADER %%%
\usepackage{fancyhdr}
\pagestyle{fancy}
\fancyhf{}
\lhead{MATH 1080}
\rhead{Vagnozzi}
\cfoot{\thepage}

%%% MATH NOTATION & SYMBOLS %%%
\usepackage{amssymb}
\usepackage{amsmath}
\newcommand{\R}{\mathbb{R}}
\newcommand{\N}{\mathbb{N}}
\newcommand{\Z}{\mathbb{Z}}
\newcommand{\lp}{\left(}
\newcommand{\rp}{\right)}
\newcommand{\ls}{\left[}
\newcommand{\rs}{\right]}
\newcommand{\lb}{\left\{}
\newcommand{\rb}{\right\}}
\newcommand{\arccot}{\text{arccot}}
\newcommand{\arccsc}{\text{arccsc}}
\newcommand{\arcsec}{\text{arcsec}}
\DeclareSymbolFont{matha}{OML}{txmi}{m}{it}% txfonts
\DeclareMathSymbol{\varv}{\mathord}{matha}{118} 

%%% TABLES %%%
\usepackage{colortbl}

%%% GRAPHS %%%
\usepackage{tikz}
\usepackage{pgfplots}
\pgfplotsset{compat=1.15}
\usepgfplotslibrary{fillbetween}
\usetikzlibrary{angles,quotes}

%%% ENVIRONMENTS %%%
\newcommand{\Example}{\paragraph{\Writinghand \hspace{0.1mm} Example.}}
\newcommand{\Examples}{\paragraph{\Writinghand \hspace{0.1mm} Examples.}}
\newcommand{\ExampleCont}{\paragraph{\Writinghand \hspace{0.1mm} Example (continued).}}
\newcommand{\ExamplesCont}{\paragraph{\Writinghand \hspace{0.1mm} Examples (continued).}}

\newcommand{\boxenv}[2]{
	\fbox{
	\begin{minipage}{0.97\textwidth}
	\vspace{2mm}	
	\paragraph{#1} #2
	\vspace{2mm}
	\end{minipage}
	}}

%%% FUN THINGS %%%
\newcommand*\tc[1]{\tikz[baseline=(char.base)]{
            \node[shape=circle,draw,inner sep=2pt] (char) {#1};}}
\usepackage{marvosym}

%%% MISC %%%
\usepackage{hyperref}


\setcounter{page}{5}

\begin{document}
\section*{6.1: Velocity and Net Change}

\boxenv{Learning Objectives.}{Upon successful completion of Section 6.1, you will be able to\dots
		
	\begin{itemize}[leftmargin=6mm]
		\item Answer conceptual questions involving velocity and net change.
		\item Determine displacement, distance, and position from velocity.
		\item Determine position and velocity from acceleration.
		\item Solve additional applications involving velocity.
		\item Solve applications (other than velocity) involving net change.
	\end{itemize}
	\vspace{-4mm}
}

\vspace{5mm}

\subsection*{Introduction}

In this section, we will use the Second Fundamental Theorem of Calculus to work with the net change of a quantity. Recall this theorem from Calculus~I (Section~5.3 in MATH 1060).\\

\boxenv{Second Fundamental Theorem of Calculus.}{Let $f$ be continuous on $[a,b]$ and \\ suppose that $F$ is an antiderivative for $f$ on the same interval. Then
$$\int_a^b f(x)\,dx=\int_a^b F'(x)\,dx=F(b)-F(a).$$

\vspace{-5mm}}

\vspace{5mm}

\textbf{Main Idea:} Given the rate $Q'$ at which a quantity $Q$ changes over time, we can use integration to calculate the net change in the quantity $Q$ over a certain time interval and to find the value of $Q$ at some future time. 

\subsection*{Net Change and Future Value}

Suppose a quantity $Q$ changes over time $t$ at a known rate $Q'$.\\

\boxenv{Definition.}{The \textbf{net change} in $Q$ between $t=a$ and $t=b>a$ is
$$Q(b)-Q(a)=\phantom{\large{\int_a^b Q'(t)\,dt}}$$
\vspace{-5mm}
}

\vspace{5mm}

\boxenv{Definition.}{Given the initial value $Q(0)$, the \textbf{future value} of $Q$ at time $t\geq 0$ is
$$Q(t)=\phantom{\large{Q(0)+\int_0^t Q'(x)\,dx}}$$

}

\newpage

\subsection*{Velocity, Position, Displacement, Distance, and Acceleration}

Let $s(t)$ be the position (relative to the origin) of an object moving along a line at time $t$.\\

\boxenv{Definition.}{The \textbf{velocity} of the object at time $t$ is $v(t)=s'(t)$ and the \textbf{speed} of the object at time $t$ is $|v(t)|$.}

\vspace{5mm}

\boxenv{Definition.}{The \textbf{acceleration} of the object at time $t$ is $a(t)=v'(t)=s''(t)$.}

\vspace{5mm}

\boxenv{Definition.}{The \textbf{displacement} of the object between times $t=a$ and $t=b>a$ is

$$s(b)-s(a)=\phantom{\large{\int_a^b s'(t)\,dt=\int_a^b f(t)\,dt}}$$
\vspace{-2mm}}

\vspace{5mm}

\boxenv{Definition.}{The \textbf{distance traveled} by the object between $t=a$ and $t=b>a$ is

$$\phantom{\large{\int_a^b|v(t)|\,dt=\int_a^b|v(t)|\,dt=A_1+A_2}}$$}

\vspace{5mm}

The object's \textit{position} can be calculated from its velocity as

$$s(t)=\phantom{\large{s(0)+\int_0^t v(x)\,dx}}$$

\vspace{1mm}

for $t\geq 0$, given $v(t)$ and initial position $s(0)$.\\

The object's \textit{velocity} can be calculated from its acceleration as

$$v(t)=\phantom{\large{v(0)+\int_0^t a(x)\,dx}}$$

\vspace{1mm}

for $t\geq 0$, given $a(t)$ and initial velocity $v(0)$.

\newpage


\Example Consider an object moving along a line with velocity $v(t)=3t^2-6t$ on $[0,3]$, where time $t$ is measured in seconds and velocity has units of $m/s$.

\begin{enumerate}
\item[(a)] Determine when the motion is in the positive direction and when it is in the negative direction.

\vspace{35mm}

\item[(b)] Find the displacement over the interval $[0,3]$.

\vspace{35mm}

\item[(c)] Find the distance traveled over the interval $[0,3]$.

\vspace{35mm}
\end{enumerate}

\Example Consider an object moving along a line with velocity $v(t)=3\sin\left(\pi t\right)$ on $[0,4]$ with initial position $s(0)=1$. Determine the position function $s(t)$ for $t\geq 0$.

\newpage

\Example Find the position and velocity of an object moving along a straight line with acceleration $a(t)=\disp\frac{2t}{\left( t^2+1\right)^2}$, initial velocity $v(0)=0$, and initial position $s(0)=0$.

\vspace{80mm}

\Example Water flows from the bottom of a storage tank at a rate of $r(t)=200-4t$ liters per minute, where $0\leq t\leq 50$. Find the amount of water that flows from the tank during the first 10 minutes.

\newpage

\Example When records were first kept $(t=0)$, the population of a rural town was 250 people. During the following years, the population grew at a rate of $P'(t)=30\left(1+\sqrt{t}\right)$, where $t$ is measured in years.

\begin{enumerate}
\item[(a)] Find the population after 9 years.

\vspace{80mm}

\item[(b)] find the population $P(t)$ at any time $t\geq 0$.
\end{enumerate}

\newpage

\Example A data collection probe is dropped from a stationary balloon, and it falls with a velocity (in m/s) given by $v(t)=9.8t$, neglecting air resistance. After 10 seconds, a chute deploys and the probe immediately slows to a constant speed of 10 meters/second, which it maintains until it enters the ocean.

\begin{enumerate}
\item[(a)] Graph the velocity function.

\vspace{50mm}

\item[(b)] How far does the probe fall in the first 30 seconds after it is released?

\vspace{50mm}

\item[(c)] If the probe was released from an altitude of 3 kilometers, when does it enter the ocean?
\end{enumerate}



\end{document}