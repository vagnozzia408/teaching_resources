\documentclass[12pt]{article}
%%% DOCUMENT FORMATTING %%%
\usepackage[margin=1in]{geometry}
\usepackage{enumitem}
\setlength{\parindent}{0pt}
\newcommand{\disp}{\displaystyle}

%%% HEADER %%%
\usepackage{fancyhdr}
\pagestyle{fancy}
\fancyhf{}
\lhead{MATH 1080}
\rhead{Vagnozzi}
\cfoot{\thepage}

%%% MATH NOTATION & SYMBOLS %%%
\usepackage{amssymb}
\usepackage{amsmath}
\newcommand{\R}{\mathbb{R}}
\newcommand{\N}{\mathbb{N}}
\newcommand{\Z}{\mathbb{Z}}
\newcommand{\lp}{\left(}
\newcommand{\rp}{\right)}
\newcommand{\ls}{\left[}
\newcommand{\rs}{\right]}
\newcommand{\lb}{\left\{}
\newcommand{\rb}{\right\}}
\newcommand{\arccot}{\text{arccot}}
\newcommand{\arccsc}{\text{arccsc}}
\newcommand{\arcsec}{\text{arcsec}} 

%%% TABLES %%%
\usepackage{colortbl}

%%% GRAPHS %%%
\usepackage{tikz}
\usepackage{pgfplots}
\pgfplotsset{compat=1.15}
\usepgfplotslibrary{fillbetween}
\usetikzlibrary{angles,quotes}

%%% ENVIRONMENTS %%%
\newcommand{\Example}{\paragraph{\Writinghand \hspace{0.1mm} Example.}}
\newcommand{\ExampleCont}{\paragraph{\Writinghand \hspace{0.1mm} Example (continued).}}
\newcommand{\boxenv}[2]{
	\fbox{
	\begin{minipage}{0.97\textwidth}
	\vspace{2mm}	
	\paragraph{#1} #2
	\vspace{2mm}
	\end{minipage}
	}}

%%% FUN THINGS %%%
\newcommand*\tc[1]{\tikz[baseline=(char.base)]{
            \node[shape=circle,draw,inner sep=2pt] (char) {#1};}}
\usepackage{marvosym}

%%% MISC %%%
\usepackage{hyperref}


\setcounter{page}{130}

\begin{document}
\section*{10.7: The Ratio and Root Tests}

\boxenv{Learning Objectives.}{Upon successful completion of Section 10.7, you will be able to\dots
		
	\begin{itemize}[leftmargin=6mm]
		\item Answer conceptual questions involving the root and ratio tests.
		\item Apply the root and ratio tests.
		\item Determine if a series converges absolutely, converges conditionally, or diverges.
		\item Determine values for which a series converges.
	\end{itemize}
	\vspace{-4mm}
}

\vspace{5mm}

\subsection*{The Ratio Test}

\vspace{1mm}

\boxenv{Ratio Test.}{Consider the series $\disp\sum_{n=1}^\infty a_n$.
\begin{enumerate}
\item[\tc{1}] If $\disp\lim_{n\to\infty}\bigg|\dfrac{a_{n+1}}{a_n}\bigg|=L <1$, then $\sum a_n$ is absolutely convergent.
\item[\tc{2}] If $\disp\lim_{n\to\infty}\bigg|\dfrac{a_{n+1}}{a_n}\bigg|=L >1$, then $\sum a_n$ is divergent.
\item[\tc{3}] If $\disp\lim_{n\to\infty}\bigg|\dfrac{a_{n+1}}{a_n}\bigg|=L =1$, then the Ratio Test is inconclusive.
\end{enumerate}
\vspace{-2mm}
}

\paragraph*{Notes about the Ratio Test.}
\begin{itemize}
\item The Ratio test is useful for series whose terms contain exponential functions, factorials, or exponentials times powers of $n$.

\vfill

\item The Ratio Test is \textbf{not} useful for $p$-series type series (only powers of $n$ involved).

\vfill

\item A series cannot converge conditionally by the Ratio Test.
\end{itemize}

\newpage

\Examples Determine if each of the following series converges conditionally, absolutely, or diverges.

\begin{itemize}
\item[\tc{1}] $\disp\sum_{n=0}^\infty\dfrac{(-3)^n}{(2n+1)!}$

\vfill

\item[\tc{2}] $\disp\sum_{n=1}^\infty\dfrac{n^2 5^{n-1}}{(-2)^n}$

\vfill

\newpage

\end{itemize}

\Example Determine if the series converges conditionally, absolutely, or diverges.

\vspace{3mm}

$\disp\sum_{n=1}^\infty(-1)^n\dfrac{n}{n^2+4}$
\vfill

\paragraph*{Extra Examples.} Determine if the series converges conditionally, absolutely, or diverges.

\begin{multicols}{2}
\begin{enumerate}
\item[\tc{1}] $\disp\sum_{k=1}^\infty\dfrac{(-1)^{k+1}k^{2k}}{k!k!}$

\item[\tc{2}] $\disp\sum_{k=1}^\infty\dfrac{(-1)^k}{k^{0.99}}$
\end{enumerate}

\end{multicols}
\vspace{3mm}

\boxenv{Note.}{For the first example, recall that $\disp\lim_{n\to\infty}\left(1+\dfrac{a}{n}\right)^n=e^a$.}

\newpage

\subsection*{The Root Test}

\boxenv{Root Test.}{Consider the series $\disp\sum_{n=1}^\infty a_n$.
\begin{enumerate}
\item[\tc{1}] If $\disp\lim_{n\to\infty}\sqrt[n]{\big|a_n\big|} =L <1$, then $\sum a_n$ is absolutely convergent.
\item[\tc{2}] If $\disp\lim_{n\to\infty}\sqrt[n]{\big|a_n\big|} =L >1$, then $\sum a_n$ is divergent.
\item[\tc{3}] If $\disp\lim_{n\to\infty}\sqrt[n]{\big|a_n\big|} =L =1$, then the Root Test is inconclusive.
\end{enumerate}
\vspace{-2mm}
}

\paragraph*{Notes about the Root Test.}
\begin{itemize}
	\item If the Ratio Test is inconclusive, then the Root Test will be, too (and vice versa).
	\item A series cannot be conditionally convergent by the Root Test.
\end{itemize}

\Examples Determine if each of the following series converges conditionally, absolutely, or diverges.

\begin{enumerate}
\item[\tc{1}] $\disp\sum_{n=1}^\infty\left(\dfrac{n^2+1}{2n^2+1}\right)^{2n}$

\vfill 

\item[\tc{2}] $\disp\sum_{k=1}^\infty\left(1+\dfrac{3}{k}\right)^{k^2}$

\vfill

\end{enumerate}

\end{document}