\documentclass[12pt]{article}
%%% DOCUMENT FORMATTING %%%
\usepackage[margin=1in]{geometry}
\usepackage{enumitem}
\setlength{\parindent}{0pt}
\newcommand{\disp}{\displaystyle}

%%% HEADER %%%
\usepackage{fancyhdr}
\pagestyle{fancy}
\fancyhf{}
\lhead{MATH 1080}
\rhead{Vagnozzi}
\cfoot{\thepage}

%%% MATH NOTATION & SYMBOLS %%%
\usepackage{amssymb}
\usepackage{amsmath}
\newcommand{\R}{\mathbb{R}}
\newcommand{\N}{\mathbb{N}}
\newcommand{\Z}{\mathbb{Z}}
\newcommand{\lp}{\left(}
\newcommand{\rp}{\right)}
\newcommand{\ls}{\left[}
\newcommand{\rs}{\right]}
\newcommand{\lb}{\left\{}
\newcommand{\rb}{\right\}}
\newcommand{\arccot}{\text{arccot}}
\newcommand{\arccsc}{\text{arccsc}}
\newcommand{\arcsec}{\text{arcsec}} 

%%% TABLES %%%
\usepackage{colortbl}

%%% GRAPHS %%%
\usepackage{tikz}
\usepackage{pgfplots}
\pgfplotsset{compat=1.15}
\usepgfplotslibrary{fillbetween}
\usetikzlibrary{angles,quotes}

%%% ENVIRONMENTS %%%
\newcommand{\Example}{\paragraph{\Writinghand \hspace{0.1mm} Example.}}
\newcommand{\ExampleCont}{\paragraph{\Writinghand \hspace{0.1mm} Example (continued).}}
\newcommand{\boxenv}[2]{
	\fbox{
	\begin{minipage}{0.97\textwidth}
	\vspace{2mm}	
	\paragraph{#1} #2
	\vspace{2mm}
	\end{minipage}
	}}

%%% FUN THINGS %%%
\newcommand*\tc[1]{\tikz[baseline=(char.base)]{
            \node[shape=circle,draw,inner sep=2pt] (char) {#1};}}
\usepackage{marvosym}

%%% MISC %%%
\usepackage{hyperref}


\setcounter{page}{174}

\begin{document}
\section*{12.3: Calculus in Polar Coordinates}

\boxenv{Learning Objectives.}{Upon successful completion of Section 12.3, you will be able to\dots
		
	\begin{itemize}[leftmargin=6mm]
		\item Answer conceptual questions involving calculus in polar coordinates.
		\item Find the slope of the line tangent to a polar curve at a given point.
		\item Find the points at which a polar curve has horizontal or vertical tangent lines.
		\item Find intersection points for two polar curves.
		\item Find the area of a region bounded by polar curves.
		\item Find the lengths of polar curves.
	\end{itemize}
	\vspace{-4mm}
}

\vspace{5mm}

\subsection*{Tangents to Polar Curves}

Given a polar curve $r=f(\theta)$, how do we find $\dfrac{dy}{dx}$?

\vfill

\Example Let's consider the polar curve $r=1+\cos\theta$. Find the slope of the line tangent to the curve at $\theta=\dfrac{\pi}{2}$.

\vfill
\vfill

\newpage

\boxenv{Tangents to Polar Curves.}{The slope $\dfrac{dy}{dx}$ of the line tangent to a polar curve $r=f(\theta)$ is

$$\dfrac{dy}{dx}=\dfrac{dy/d\theta}{dx/d\theta}.$$

\vspace{5mm}

\textbf{Horizontal tangents} occur where \underline{\hspace{33mm}}, provided that \underline{\hspace{33mm}}.

\vspace{5mm}

\textbf{Vertical tangents} occur where \underline{\hspace{35mm}}, provided that \underline{\hspace{35mm}}.

 }
 
\vspace{15mm}

\Example Let's again consider the polar curve $r=1+\cos\theta$. Find the points $(r,\theta)$ where the graph has horizontal or vertical tangents.

\end{document}