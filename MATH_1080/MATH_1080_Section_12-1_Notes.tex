\documentclass[12pt]{article}
%%% DOCUMENT FORMATTING %%%
\usepackage[margin=1in]{geometry}
\usepackage{enumitem}
\setlength{\parindent}{0pt}
\newcommand{\disp}{\displaystyle}
\usepackage{multicol}

%%% HEADER %%%
\usepackage{fancyhdr}
\pagestyle{fancy}
\fancyhf{}
\lhead{MATH 1080}
\rhead{Vagnozzi}
\cfoot{\thepage}

%%% MATH NOTATION & SYMBOLS %%%
\usepackage{amssymb}
\usepackage{amsmath}
\newcommand{\R}{\mathbb{R}}
\newcommand{\N}{\mathbb{N}}
\newcommand{\Z}{\mathbb{Z}}
\newcommand{\lp}{\left(}
\newcommand{\rp}{\right)}
\newcommand{\ls}{\left[}
\newcommand{\rs}{\right]}
\newcommand{\lb}{\left\{}
\newcommand{\rb}{\right\}}
\newcommand{\arccot}{\text{arccot}}
\newcommand{\arccsc}{\text{arccsc}}
\newcommand{\arcsec}{\text{arcsec}}
\DeclareSymbolFont{matha}{OML}{txmi}{m}{it}% txfonts
\DeclareMathSymbol{\varv}{\mathord}{matha}{118} 

%%% TABLES %%%
\usepackage{colortbl}

%%% GRAPHS %%%
\usepackage{tikz}
\usepackage{pgfplots}
\pgfplotsset{compat=1.15}
\usepgfplotslibrary{fillbetween}
\usetikzlibrary{angles,quotes}

%%% ENVIRONMENTS %%%
\newcommand{\Example}{\paragraph{\Writinghand \hspace{0.1mm} Example.}}
\newcommand{\Examples}{\paragraph{\Writinghand \hspace{0.1mm} Examples.}}
\newcommand{\ExampleCont}{\paragraph{\Writinghand \hspace{0.1mm} Example (continued).}}
\newcommand{\ExamplesCont}{\paragraph{\Writinghand \hspace{0.1mm} Examples (continued).}}

\newcommand{\boxenv}[2]{
	\fbox{
	\begin{minipage}{0.97\textwidth}
	\vspace{2mm}	
	\paragraph{#1} #2
	\vspace{2mm}
	\end{minipage}
	}}

%%% FUN THINGS %%%
\newcommand*\tc[1]{\tikz[baseline=(char.base)]{
            \node[shape=circle,draw,inner sep=2pt] (char) {#1};}}
\usepackage{marvosym}

%%% MISC %%%
\usepackage{hyperref}


\setcounter{page}{160}

\begin{document}
\section*{12.1: Parametric Equations}

\boxenv{Learning Objectives.}{Upon successful completion of Section 12.1, you will be able to\dots
		
	\begin{itemize}[leftmargin=6mm]
		\item Graph and describe parametric equations and eliminate the parameter to find \\
		equations in $x$ and $y$.
		\item Find parametric equations for a given description of a curve.
		\item Solve applications involving parametric equations.
		\item Differentiate parametric equations.
		\item Find the slopes and equations of tangent lines to a parametric equation.
		\item Find the arc length of a parametric curve.
		\item Answer conceptual questions involving parametric equations.
		\item Find the area under a parametric curve.
		\item Find the area of a surface of revolution of a parametric curve.
	\end{itemize}
	\vspace{-4mm}
}

\vspace{5mm}

\subsection*{Introduction}

A \textbf{parametric curve} $C$ is a curve in the $xy$-plane described by three variables: $x$, $y$, and a \textit{parametric variable} (often $t$ or $\theta$). 

\begin{itemize}
	\item We call $x=f(t)$ and $y=g(t)$ \textbf{parametric equations}, where $t$ is the parameter.
	\item As $t$ varies, the point $(x,y)=\big(f(t),g(t)\big)$ varies and traces out the curve $C$.
\end{itemize}

\vfill

The curve $C$ is traced out in a specific direction. When sketching parametric curves, we use arrows to indicate this direction, called the \textbf{orientation} of the curve.
\newpage

\Example Consider the parametric equations $x=1-t^2$ and $y=t-2$ for $-2\leq t\leq 2$. 
\begin{enumerate}
	\item[(a)] Eliminate the parameter to obtain an equation in $x$ and $y$.
	
	\vfill
	
	\item[(b)] Sketch the parametric curve, indicating the positive orientation (increasing $t$). 
	
	\vfill
	\vfill
\end{enumerate}

\Example Consider the equations $x=\sin\left(\frac{\theta}{2}\right)$ and $y=\cos\left(\frac{\theta}{2}\right)$ for $-\pi\leq\theta\leq\pi$.
\begin{enumerate}
	\item[(a)] Eliminate the parameter to obtain an equation in $x$ and $y$.
	
	\vfill
	
	\item[(b)] Sketch the parametric curve, indicating the positive orientation. 
	
	\vfill
	\vfill
\end{enumerate}

\newpage

\subsection*{Finding Parametric Equations for a Curve}

\boxenv{Parametric Equations of a Circle.}{The parametric equations

$$x=x_0+a\cos(bt)\text{ and } y=y_0+a\sin(bt)$$

\vspace{2mm}

describe all or part of the \textbf{circle} $(x-x_0)^2+(y-y_0)^2=a^2$ centered at $(x_0,y_0)$ with radius $|a|$. If $b>0$, then the circle is generated in the counterclockwise direction.\\

if $x=x_0+a\sin(bt)$ and $y=y_0+a\cos(bt)$ with $b>0$, then the circle is generated in the clockwise direction.

}

\Example Find parametric equations for a circle centered at $(2,3)$ with a radius of $1$, generated counterclockwise. 

\vfill

What parametric equations correspond to only the lower half of the circle?

\vspace{30mm}

\newpage

\boxenv{Parametric Equations of a Line.}{The parametric equations

$$x=x_0+at\text{ and }y=y_0+bt\text{, for }-\infty<t<\infty,$$

where $x_0$, $y_0$, $a$, and $b$ are constants with $a\neq 0$, describe a \textbf{line} with a slope $\dfrac{b}{a}$ passing through the point $(x_0,y_0)$. If $a=0$ and $b\neq 0$, the line is vertical.
}

\Example Find parametric equations for the line segment starting at $P(-1,-3)$ and ending at $Q(6,-16)$.

\vfill

\vfill

\Example Find parametric equations for the complete parabola $y=2x^2-4$. 

\vspace{45mm}

What parametric equations correspond to the segment of the parabola where $-1\leq x\leq 5$?

\vspace{30mm}

\newpage

\Example The tip of the 15-inch second hand of a clock completes one revolution in 60 seconds. Find the parametric equations that describe the circular path of the tip of the second hand. Assume $(x,y)$ denotes the position relative to the origin at the center of the circle.

\vfill

\subsection*{Calculus in Parametric Equations}

Suppose we have a set of parametric equations $x=f(t)$ and $y=g(t)$ describing a curve $C$. If $f$ and $g$ are differentiable, how do we find the slope of the tangent line to a point on $C$?

\vspace{5mm}

\boxenv{Derivatives for Parametric Equations.}{
\hfill

\vspace{30mm}
}

\Example A curve is defined by the parametric equations $x=e^t$, $y=te^{-t}$. Find $\dfrac{dy}{dx}$.

\vspace{45mm}

\newpage

\Example A curve is defined by the parametric equations $x=\sin^3t$, $y=\cos^3t$, \mbox{$0\leq t\leq2\pi$.} Find an equation of the tangent line at the point where $t=\pi/6$.

\begin{flushright}
	\begin{tikzpicture}
    \begin{axis}[
       	axis x line=center,
       	xmax=1.45, xmin=-1.45,
       	xtick={-1,0,1},
       	axis y line=center,
       	ymax=1.25, ymin=-1.25,
       	ytick={-1,0,1},
       	xlabel=$x$,ylabel=$y$,
       	axis line style=<->
    ]
	\addplot [domain=0:2*pi,samples=50,color=blue,thick]({(sin(deg(x)))^3},{(cos(deg(x)))^3});
    \end{axis}
    \end{tikzpicture}
\end{flushright}

%    	\addplot[name path=f,smooth,domain=-5.25:5.25,color=blue,samples=100,<->,thick] {(2*x)/(x^2+1)^2};


\vfill

\Example Find all points at which the curve $x=2+\sqrt{t}$, $y=2-4t$ has the slope $-8$.

\begin{flushright}
	\begin{tikzpicture}
    \begin{axis}[
       	axis x line=center,
       	xmax=4.5, xmin=-0.5,
%       	xtick={-1,0,1},
       	axis y line=center,
       	ymax=2.25, ymin=-5.25,
%       	ytick={-1,0,1},
       	xlabel=$x$,ylabel=$y$,
       	axis line style=<->
    ]
	\addplot [domain=0:2*pi,samples=100,color=blue,thick]({2+sqrt(x)},2-4*x);
    \end{axis}
    \end{tikzpicture}
\end{flushright}

\vfill

\newpage

\boxenv{Arc Length for Parametric Equations.}{Suppose a curve $C$ is described by the parametric equations $x=f(t)$ and $y=g(t)$, $\alpha\leq t\leq \beta$, where $\dfrac{dx}{dt}>0$, $f'$ and $g'$ are continuous, and $C$ is transversed once as $t$ increases from $\alpha$ to $\beta$. \\

Then the \textbf{arc length \textit{L}} of the curve $C$ is 

\vspace{20mm}
 }

\Example Find the total length of the astroid curve $x=\cos^3\theta$, $y=\sin^3\theta$, which is traced out once for $0\leq\theta\leq 2\pi$.

\begin{flushright}
	\begin{tikzpicture}
    \begin{axis}[
       	axis x line=center,
       	xmax=1.45, xmin=-1.45,
       	xtick={-1,0,1},
       	axis y line=center,
       	ymax=1.25, ymin=-1.25,
       	ytick={-1,0,1},
       	xlabel=$x$,ylabel=$y$,
       	axis line style=<->
    ]
	\addplot [domain=0:2*pi,samples=50,color=blue,thick]({(cos(deg(x)))^3},{(sin(deg(x)))^3});
    \end{axis}
    \end{tikzpicture}
\end{flushright}

\vfill

\newpage


\boxenv{Area and Parametric Equations.}{Suppose $y=h(x)$ is nonnegative and continuous on $\left[a,b\right]$, implying that the area bounded by the graph of $h$ and the $x$-axis on $\left[a, b\right]$ equals $\disp\int_a^b h(x)\,dx=\int_a^b y\,dx$. \\

If the graph of $y=h(x)$ on $\left[a,b\right]$ is traced exactly once by the parametric equations $x=f(t)$, $y=g(t)$, for $\alpha\leq t\leq \beta$, then (by substitution) the area bounded by $h$ is

\vspace{30mm}}

\Example Use parametric requations of an ellipse centered at the origin $x=a\cos\theta$, $y=b\sin\theta$, $0\leq\theta\leq 2\pi$, to find the area that it encloses.

\begin{flushright}
	\begin{tikzpicture}
    \begin{axis}[
       	axis x line=center,
       	xmax=2.25, xmin=-2.25,
       	xtick=\empty,
       	axis y line=center,
       	ymax=2.25, ymin=-2.25,
       	ytick=\empty,
       	xlabel=$x$,ylabel=$y$,
       	axis line style=<->
    ]
	\addplot [domain=0:2*pi,samples=50,color=blue,thick]({1*(cos(deg(x)))},{2*(sin(deg(x)))});
    \end{axis}
    \end{tikzpicture}
\end{flushright}

\vfill

\newpage

\boxenv{Surface Area and Parametric Equations.}{Let $C$ be the curve $x=f(t)$, $y=g(t)$, for $\alpha\leq t\leq \beta$, where $f'$ and $g'$ are continuous, and $C$ does not intersect itself (except possibly at its endpoints). \\

If $g$ is nonnegative on $\left[\alpha,\beta\right]$, then the area of the surface obtained by revolving $C$ around the $x$-axis is

\vspace{15mm}

Likewise, if $f$ is nonnegative on $\left[\alpha,\beta\right]$, then the area of the surface obtained by revolving $C$ about the $y$-axis is

\vspace{15mm}


}

\Example Find the exact area of the surface obtained by rotating the curve given by $x=\cos^3\theta$, $y=\sin^3\theta$, $0\leq\theta\leq \pi/2$ about the $x$-axis.

\begin{flushright}
	\begin{tikzpicture}
    \begin{axis}[
       	axis x line=center,
       	xmax=1.45, xmin=-1.45,
       	xtick={-1,0,1},
       	axis y line=center,
       	ymax=1.25, ymin=-1.25,
       	ytick={-1,0,1},
       	xlabel=$x$,ylabel=$y$,
       	axis line style=<->
    ]
	\addplot [domain=0:2*pi,samples=50,color=blue,thick]({(cos(deg(x)))^3},{(sin(deg(x)))^3});
    \end{axis}
    \end{tikzpicture}
\end{flushright}

\vfill

\end{document}