\documentclass[12pt]{article}
%%% DOCUMENT FORMATTING %%%
\usepackage[margin=1in]{geometry}
\usepackage{enumitem}
\setlength{\parindent}{0pt}
\newcommand{\disp}{\displaystyle}
\usepackage{multicol}

%%% HEADER %%%
\usepackage{fancyhdr}
\pagestyle{fancy}
\fancyhf{}
\lhead{MATH 1080}
\rhead{Vagnozzi}
\cfoot{\thepage}

%%% MATH NOTATION & SYMBOLS %%%
\usepackage{amssymb}
\usepackage{amsmath}
\newcommand{\R}{\mathbb{R}}
\newcommand{\N}{\mathbb{N}}
\newcommand{\Z}{\mathbb{Z}}
\newcommand{\lp}{\left(}
\newcommand{\rp}{\right)}
\newcommand{\ls}{\left[}
\newcommand{\rs}{\right]}
\newcommand{\lb}{\left\{}
\newcommand{\rb}{\right\}}
\newcommand{\arccot}{\text{arccot}}
\newcommand{\arccsc}{\text{arccsc}}
\newcommand{\arcsec}{\text{arcsec}}
\DeclareSymbolFont{matha}{OML}{txmi}{m}{it}% txfonts
\DeclareMathSymbol{\varv}{\mathord}{matha}{118} 

%%% TABLES %%%
\usepackage{colortbl}

%%% GRAPHS %%%
\usepackage{tikz}
\usepackage{pgfplots}
\pgfplotsset{compat=1.15}
\usepgfplotslibrary{fillbetween}
\usetikzlibrary{angles,quotes}

%%% ENVIRONMENTS %%%
\newcommand{\Example}{\paragraph{\Writinghand \hspace{0.1mm} Example.}}
\newcommand{\Examples}{\paragraph{\Writinghand \hspace{0.1mm} Examples.}}
\newcommand{\ExampleCont}{\paragraph{\Writinghand \hspace{0.1mm} Example (continued).}}
\newcommand{\ExamplesCont}{\paragraph{\Writinghand \hspace{0.1mm} Examples (continued).}}

\newcommand{\boxenv}[2]{
	\fbox{
	\begin{minipage}{0.97\textwidth}
	\vspace{2mm}	
	\paragraph{#1} #2
	\vspace{2mm}
	\end{minipage}
	}}

%%% FUN THINGS %%%
\newcommand*\tc[1]{\tikz[baseline=(char.base)]{
            \node[shape=circle,draw,inner sep=2pt] (char) {#1};}}
\usepackage{marvosym}

%%% MISC %%%
\usepackage{hyperref}


\setcounter{page}{144}

\begin{document}
\section*{11.2: Properties of Power Series}

\boxenv{Learning Objectives.}{Upon successful completion of Section 11.2, you will be able to\dots
		
	\begin{itemize}[leftmargin=6mm]
		\item Answer conceptual questions involving power series.
		\item Find the interval and radius of convergence of power series.
		\item Combine power series.
		\item Find a power series by integrating or differentiating a known power series.
		\item Write a power series representation of a given function.
		\item Find a function represented by a given power series.
	\end{itemize}
	\vspace{-4mm}
}

\vspace{5mm}

\subsection*{Introduction}

Recall the definition of a power series from the previous section.

\vspace{5mm}


\boxenv{Definition.}{A \textbf{power series} has the general form
$$\disp\sum_{k=0}^\infty c_k(x-a)^k=c_0+c_1(x-a)+c_2(x-a)^2+c_3(x-a)^3+\cdots$$

where $a$ and $c_k$ are real numbers and $x$ is a variable. The $c_k$'s are the \textbf{coefficients} of the power series and $a$ is the \textbf{center} of the power series.}

\vspace{5mm}

One of our goals in this section is to answer the question: \textbf{When does a given power series converge?}

\begin{itemize}
\item For each fixed $x$, the power series is an infinite sum of \underline{\hspace{40mm}} that we can test for convergence or divergence.
\item The power series $\disp\sum_{k=0}^\infty c_k(x-a)^k$ will converge for some values of $x$ and diverge for other values. We want to find \underline{\hspace{54mm}} for which the series converges.
\item The power series always converges at \underline{\hspace{50mm}}, because for $x=a$, $\disp\sum_{k=0}^\infty c_k(x-a)^k=c_0$.
\item The sum of the power series (instead of being a number $S$) is \underline{\hspace{40mm}} whose domain is the set of all $x$-values for which the series converges.
\end{itemize}

\newpage

\boxenv{Definition.}{The set of $x$-values for which a power series converges is called its \textbf{interval of convergence}.}

\vspace{5mm}

\boxenv{Definition.}{The \textbf{radius of convergence} of a power series, denoted $R$, is the distance from the center of the series to the boundary of the interval of convergence.}

\vspace{5mm}

\Example For what values of $x$ does $\disp\sum_{n=0}^\infty x^n$ converge?

\vfill

\Example Find the interval of convergence and radius of convergence for $\disp\sum_{n=0}^\infty \dfrac{x^n}{n!}$.

\vfill

\newpage

\Example Find the interval of convergence and radius of convergence for $\disp\sum_{n=1}^\infty n!(2x-1)^n$.

\vfill

In these three examples, we have actually observed the only three possible types of sets of $x$-values for which a power series is convergent. These possibilities are summarized in the theorem below.

\vspace{5mm}

\boxenv{Theorem.}{For a given power series $\disp\sum_{n=0}^\infty c_n(x-a)^n$, there are three possibilities.


\begin{enumerate}
\item[\tc{1}] The series converges only when $x=a$.

\vspace{20mm}

\item[\tc{2}] The series converges for all $x$.

\vspace{20mm}

\item[\tc{3}] There is a positive number $R$ ($0<R<\infty$) such that the series converges if $|x-a|<R$ and diverges if $|x-a|>R$.

\vspace{20mm}
\end{enumerate}}

\newpage 

\Example Find the interval of convergence and radius of convergence for $\disp\sum_{n=1}^\infty\dfrac{3^n(x+4)^n}{\sqrt{n}}$.

\newpage

\subsection*{Combining Power Series}

\boxenv{Theorem.}{Suppose the power series $\sum c_kx^k$ and $\sum d_k x^k$ converge to $f(x)$ and $g(x)$, respectively, on an interval $I$.

\vspace{3mm}

\begin{enumerate}
\item[\tc{1}] \textbf{Sums/Differences:} The power series $\sum (c_k \pm d_k) x^k$ converges to $f(x)\pm g(x)$ on $I$.

\item[\tc{2}] \textbf{Multiplication by a Power:} Suppose $m$ is an integer such that $k+m\geq 0$, for all terms of the power series $x^m\sum c_k x^k=\sum c_k x^{k+m}$. This series converges to $x^m f(x)$, for all $x\neq 0$ in $I$. When $x=0$, the series converges to $\disp\lim_{x\to 0} x^m f(x)$.

\item[\tc{3}] \textbf{Composition:} If $h(x)=bx^m$, where $m$ is a positive integer and $b$ is a nonzero real number, the power series $\sum c_k \big( h(x)\big)^k$ converges to the composite function $f\big( h(x)\big)$, for all $x$ such that $h(x)$ is in $I$.

\vspace{-2mm}
\end{enumerate}}

\Example Given $\dfrac{1}{1=x}=\disp\sum_{k=0}^\infty x^k=1+x+x^2+x^3+\cdots$, for $|x|<1$, find the power series and interval of convergence for the function\dots

\begin{enumerate}
	\item[(a)] $\dfrac{1}{1+x^2}$
	
	\vfill
	
	\item[(b)] $\dfrac{x}{8-x^3}$
	
	\vfill
\end{enumerate}

\Example Find the function represented by the series $\disp\sum_{k=0}^\infty\dfrac{x^{2k}}{4^k}$, and find the interval of convergence of the series.

\vfill

\newpage


\subsection*{Differentiating and Integrating Power Series}

\boxenv{Theorem.}{Suppose that the power series $\sum c_k(x-a)^k$ converges for $|x-a|<R$ and defines a function $f$ on that interval.
\begin{enumerate}
\item[\tc{1}] Then $f$ is differentiable (which implies continuous) for $|x-a|<R$, and $f'$ is found by differentiating the power series $f$ term by term; that is, 

$$f'(x)=\sum k c_k(x-a)^{k-1},$$

for $|x-a|<R$.

\item[\tc{2}] The indefinite integral $f$ is found by integrating the power series for $f$ term by term; that is,
$$\int f(x)\,dx=\sum c_k\dfrac{(x-a)^{k+1}}{k+1}+C,$$

for $|x-a|<R$, where $C$ is an arbitrary constant.
\end{enumerate}
}

\Example Given $f(x)=\dfrac{1}{1-x}=\disp\sum_{k=0}^\infty x^k=1+x+x^2+x^3+\cdots$, for $|x|<1$\dots

\begin{enumerate}
\item[(a)] Find the power series for $f'(x)$ and its  interval of convergence. Also identify what function the power series represents.

\vfill

\item[(b)] Find the power series for $\int f(x)\,dx$ and its interval of convergence. Also identify what function the power series represents.

\vfill
\end{enumerate}


\newpage

\Example Find the power series representation for $g(x)=-\dfrac{1}{(1+x)^2}$ centered at $0$ by differentiating or integrating the power series for $f(x)=\dfrac{1}{1+x}$.

\vfill

\Example We found a power series representation for $\dfrac{1}{1+x^2}$ in a previous example.

\begin{enumerate}
\item[(a)] Use this power series to find a power series representation centered at $0$ for $\arctan(x)$. Give the interval of convergence for the resulting series.

\vfill

\item[(b)] Use the above series to find a power series representation centered at $0$ for $f(x)=\arctan(4x^2)$ and find the interval of convergence.

\vfill
\end{enumerate}

\end{document}