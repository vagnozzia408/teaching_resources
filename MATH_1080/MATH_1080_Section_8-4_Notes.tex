\documentclass[12pt]{article}
%%% DOCUMENT FORMATTING %%%
\usepackage[margin=1in]{geometry}
\usepackage{enumitem}
\setlength{\parindent}{0pt}
\newcommand{\disp}{\displaystyle}
\usepackage{multicol}

%%% HEADER %%%
\usepackage{fancyhdr}
\pagestyle{fancy}
\fancyhf{}
\lhead{MATH 1080}
\rhead{Vagnozzi}
\cfoot{\thepage}

%%% MATH NOTATION & SYMBOLS %%%
\usepackage{amssymb}
\usepackage{amsmath}
\newcommand{\R}{\mathbb{R}}
\newcommand{\N}{\mathbb{N}}
\newcommand{\Z}{\mathbb{Z}}
\newcommand{\lp}{\left(}
\newcommand{\rp}{\right)}
\newcommand{\ls}{\left[}
\newcommand{\rs}{\right]}
\newcommand{\lb}{\left\{}
\newcommand{\rb}{\right\}}
\newcommand{\arccot}{\text{arccot}}
\newcommand{\arccsc}{\text{arccsc}}
\newcommand{\arcsec}{\text{arcsec}}
\DeclareSymbolFont{matha}{OML}{txmi}{m}{it}% txfonts
\DeclareMathSymbol{\varv}{\mathord}{matha}{118} 

%%% TABLES %%%
\usepackage{colortbl}

%%% GRAPHS %%%
\usepackage{tikz}
\usepackage{pgfplots}
\pgfplotsset{compat=1.15}
\usepgfplotslibrary{fillbetween}
\usetikzlibrary{angles,quotes}

%%% ENVIRONMENTS %%%
\newcommand{\Example}{\paragraph{\Writinghand \hspace{0.1mm} Example.}}
\newcommand{\Examples}{\paragraph{\Writinghand \hspace{0.1mm} Examples.}}
\newcommand{\ExampleCont}{\paragraph{\Writinghand \hspace{0.1mm} Example (continued).}}
\newcommand{\ExamplesCont}{\paragraph{\Writinghand \hspace{0.1mm} Examples (continued).}}

\newcommand{\boxenv}[2]{
	\fbox{
	\begin{minipage}{0.97\textwidth}
	\vspace{2mm}	
	\paragraph{#1} #2
	\vspace{2mm}
	\end{minipage}
	}}

%%% FUN THINGS %%%
\newcommand*\tc[1]{\tikz[baseline=(char.base)]{
            \node[shape=circle,draw,inner sep=2pt] (char) {#1};}}
\usepackage{marvosym}

%%% MISC %%%
\usepackage{hyperref}


\setcounter{page}{62}

\begin{document}
\section*{8.4: Trigonometric Substitutions}

\boxenv{Learning Objectives.}{Upon successful completion of Section 8.4, you will be able to\dots
		
	\begin{itemize}[leftmargin=6mm]
		\item Answer conceptual questions involving trigonometric substitutions.
		\item Evaluate indefinite integrals involving trigonometric substitution.
		\item Evaluate definite integrals involving trigonometric substitution.
		\item Find area and volume for regions involving integrals requiring trigonometric \\
		 substitution.
		\item Complete the square to solve integrals involving trigonometric substitution.
	\end{itemize}
	\vspace{-4mm}
}

\vspace{5mm}

\subsection*{Introduction}

The integration technique we will learn in this section, \textbf{trigonometric substitution}, is useful for integrating functions containing expressions of the form $a^2-x^2$, $x^2+a^2$, and $x^2-a^2$, often under a root, where $a$ is a constant real number.\\

\textbf{Examples}\\

Expressions of the Form `` $a^2-x^2$ ''

\vfill

Expressions of the Form `` $x^2 + a^2$ ''

\vfill

Expressions of the Form `` $x^2-a^2$ ''

\vfill

In trig substitution, we introduce a trig function into our problem in order to use the Pythagorean identities. This will allow us to simplify the integral into something that we can evaluate using other techniques, often those from Section~8.3.

\newpage

\Example Consider $\disp\int x\sqrt{1-x^2}\,dx$ versus $\disp\int\sqrt{1-x^2}\,dx$. \\

How do our approaches to evaluating these two integrals differ?

\vfill

\subsection*{Key Ideas for Trig Substitution}

We can think about trig substitution as an ``inverse substitution.''

\vspace{15mm}

Note that, in the previous example, if we let $x=\sin\theta$, then $\theta=\arcsin(x)$. For our trig substitution to result in an equivalent integral, we need for our function to be \textbf{one-to-one}, which means restricting the domain of the trig function (i.e., the $\theta$-values).

\begin{center}
\renewcommand{\arraystretch}{1.7}
\begin{tabular}{|c|c|c|}
\hline
\textbf{Form} & \textbf{Substitution} & \textbf{Identity}\\
\hline
& & \\
`` $a^2-x^2$ '' & \hspace{60mm} & \hspace{60mm} \\
& & \\
\hline
& & \\
`` $x^2+a^2$ '' & & \\
& & \\
\hline
& & \\
`` $x^2-a^2$ '' & & \\
& & \\
\hline
\end{tabular}
\end{center}

\newpage

\Example $\disp\int\frac{dx}{x\sqrt{5-x^2}}$

\newpage

\Example $\disp\int\frac{dt}{\sqrt{t^2-6t+13}}$, $t\geq 1 $

\newpage

\Example $\disp\int\frac{x}{\sqrt{x^2+10x+16}}\,dx$, $x>2$

\newpage

\Example $\disp\int_0^{2/3}\frac{x^2}{\sqrt{4-9x^2}}\,dx$

\end{document}