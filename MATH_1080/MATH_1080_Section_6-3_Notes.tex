\documentclass[12pt]{article}
%%% DOCUMENT FORMATTING %%%
\usepackage[margin=1in]{geometry}
\usepackage{enumitem}
\setlength{\parindent}{0pt}
\newcommand{\disp}{\displaystyle}

%%% HEADER %%%
\usepackage{fancyhdr}
\pagestyle{fancy}
\fancyhf{}
\lhead{MATH 1080}
\rhead{Vagnozzi}
\cfoot{\thepage}

%%% MATH NOTATION & SYMBOLS %%%
\usepackage{amssymb}
\usepackage{amsmath}
\newcommand{\R}{\mathbb{R}}
\newcommand{\N}{\mathbb{N}}
\newcommand{\Z}{\mathbb{Z}}
\newcommand{\lp}{\left(}
\newcommand{\rp}{\right)}
\newcommand{\ls}{\left[}
\newcommand{\rs}{\right]}
\newcommand{\lb}{\left\{}
\newcommand{\rb}{\right\}}
\newcommand{\arccot}{\text{arccot}}
\newcommand{\arccsc}{\text{arccsc}}
\newcommand{\arcsec}{\text{arcsec}} 

%%% TABLES %%%
\usepackage{colortbl}

%%% GRAPHS %%%
\usepackage{tikz}
\usepackage{pgfplots}
\pgfplotsset{compat=1.15}
\usepgfplotslibrary{fillbetween}
\usetikzlibrary{angles,quotes}

%%% ENVIRONMENTS %%%
\newcommand{\Example}{\paragraph{\Writinghand \hspace{0.1mm} Example.}}
\newcommand{\ExampleCont}{\paragraph{\Writinghand \hspace{0.1mm} Example (continued).}}
\newcommand{\boxenv}[2]{
	\fbox{
	\begin{minipage}{0.97\textwidth}
	\vspace{2mm}	
	\paragraph{#1} #2
	\vspace{2mm}
	\end{minipage}
	}}

%%% FUN THINGS %%%
\newcommand*\tc[1]{\tikz[baseline=(char.base)]{
            \node[shape=circle,draw,inner sep=2pt] (char) {#1};}}
\usepackage{marvosym}

%%% MISC %%%
\usepackage{hyperref}


\setcounter{page}{16}

\begin{document}
\section*{6.3: Volume by Slicing}

\boxenv{Learning Objectives.}{Upon successful completion of Section 6.3, you will be able to\dots
		
	\begin{itemize}[leftmargin=6mm]
		\item Answer conceptual questions involving the general slicing, disk, and washer methods.
		\item Use the general slicing method to find volumes of solids.
		\item Use the disk and washer methods to find the volume of solids of revolution about the $x$-axis.
		\item Use the disk and washer methods to find the volume of solids of revolution about the $y$-axis.
		\item Use the disk and washer methods to find the volume of solids of revolution about horizontal lines ($y=a$) other than the $x$-axis and vertical lines ($x=b$) other than the $y$-axis.
		\item Describe the solid whose volume is given by an integral.
		\item Solve applications involving the general slicing, disk, and washer methods.
	\end{itemize}
	\vspace{-4mm}
}

\vspace{5mm}

\subsection*{General Slicing Method}

Consider a solid object, such as a loaf of bread, and imagine laying it along the $x$-axis.

\vspace{41mm}

\boxenv{Definition.}{Let $S$ be a solid that lies between $x=a$ and $x=b$. If the cross sections of $S$ are perpendicular to the $x$-axis and have area $A(x)$, where $A$ is a continuous function, then the \textbf{volume} of $S$ is

\vspace{-2mm}

$$V=\lim_{n\to\infty}\sum_{i=1}^n A\left(x_i^*\right)\Delta x=\int_a^b A(x)\,dx.$$

\vspace{-1mm}
}

\vspace{5mm}

\textbf{Note:} If $S$ is a solid lying between $y=c$ and $y=d$ with cross sections perpendicular to the $y$-axis that have area $A(y)$, where $A$ is a continuous function, then 

$$V=\int_c^d A(y)\,dy.$$

\newpage

\Example Find the volume of the solid with circular base of radius 5 whose cross sections perpendicular to the base and parallel to the $x$-axis are equilateral triangles.

\begin{enumerate}
\item[(a)] Draw a sketch of the solid object's base and associated cross section.
\vspace{40mm}

\item[(b)] Determine the area of the cross section.

\vspace{70mm}

\item[(c)] Use the general slicing method to find the volume of the solid described.
\end{enumerate}

\newpage

\ExampleCont What if, rather than equilateral triangles, the cross sections had been squares? How would this change the volume formula?

\vspace{70mm}

\subsection*{Solids of Revolution}

A specific type of solid we will work with is a \textbf{solid of revolution}. These solids are created by rotating a region $R$ on a coordinate plane around an \textit{axis of rotation}, which may be the $x$-axis, $y$-axis, or some other line. We'll learn two different methods known as the Disk and Washer Methods to find the volume of a solid of revolution.

\paragraph{Rotating about the \textbf{\textit{x}}-Axis.} Suppose $f$ is a continuous function on $[a,b]$ and $R$ is the region bounded by the graph of $f$, the $x$-axis, and the lines $x=a$ and $x=b$. Let's find the volume of the solid generated by revolving the region $R$ about the $x$-axis. We can do so by thinking about \textit{slices}!

\newpage 

\boxenv{Disk Method about the \textbf{\textit{x}}-Axis.}{Let $f$ be a continuous function $f(x)\geq 0$ on the interval $[a,b]$. If the region $R$ bounded by the graph of $f$, the $x$-axis, and the lines $x=a$ and $x=b$ is revolved about the $x$-axis, the volume of the resulting solid of revolution is
$$V=\int_a^b\pi \big(r(x)\big)^2\,dx,$$
where $r(x)$ is the radius of the disk cross section.}

%$$V=\int_a^b\pi \big(f(x)\big)^2\,dx,$$
%where $f(x)$ is the radius of the disk.

\Example Let $R$ be the region bounded by $y=\sqrt{x-1}$, $y=0$, and $x=5$. Find the volume of the solid generated when $R$ is rotated about the $x$-axis.

\newpage

\Example Let $R$ be the region bounded by $y=x^3$, $y=x$, and $x\geq 0$. Find the volume of the solid generated when $R$ is rotated about the $x$-axis.

\vfill

\boxenv{Washer Method about the \textbf{\textit{x}}-Axis.}{Let $f$ and $g$ be continuous functions such that $f(x)\geq g(x)\geq 0$ on $[a,b]$. Let $R$ be the region bounded by $y=f(x)$, $y=g(x)$, and the lines $x=a$ and $x=b$. When $R$ is revolved about the $x$-axis, the volume of the resulting solid of revolution is
$$V=\int_a^b\pi\big(R(x)^2-r(x)^2\big)\,dx,$$
where $R(x)$ is the outer radius of the washer cross section and $r(x)$ is the inner radius of the washer cross section.}

\newpage

\paragraph{Rotating about Other Horizontal Lines.} The Disk and Washer Methods can also be used to find  volumes of solids of revolution created by rotating a region about some horizontal line other than the $x$-axis.

\Example Let $R$ be the region bounded by $y=\sqrt{x-1}$, $y=0$, and $x=5$. Set up the integral(s) needed to find the volume of the solid generated when $R$ is rotated about the line $y=2$.

\newpage

\paragraph{Rotating about the \textbf{\textit{y}}-Axis and Other Vertical Lines.} Sometimes, it will be easier to create horizontal slices perpendicular to the $y$-axis rather than vertical slices perpendicular to the $x$-axis. Fortunately, the Disk and Washer Methods can be adapted to such situations.

\Example Let $R$ be the region bounded by $y=\ln x$, $y=1$, $y=2$, and $x=0$.

\begin{itemize}
\item[(a)] Find the volume of the solid generated when $R$ is rotated about the $y$-axis.

\vspace{90mm}

\item[(b)] Set up the integral(s) needed to find the volume of the solid generated when $R$ is rotated about the line $x=-2$.
\end{itemize}

\vfill

\boxenv{Helpful Tip.}{When working with solids of revolution, the slices you create will always be perpendicular to the axis of revolution. Drawing a sketch will be helpful in setting up these problems!}
\end{document}