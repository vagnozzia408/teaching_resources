\documentclass[12pt]{article}
%%% DOCUMENT FORMATTING %%%
\usepackage[margin=1in]{geometry}
\usepackage{enumitem}
\setlength{\parindent}{0pt}
\newcommand{\disp}{\displaystyle}

%%% HEADER %%%
\usepackage{fancyhdr}
\pagestyle{fancy}
\fancyhf{}
\lhead{MATH 1080}
\rhead{Vagnozzi}
\cfoot{\thepage}

%%% MATH NOTATION & SYMBOLS %%%
\usepackage{amssymb}
\usepackage{amsmath}
\newcommand{\R}{\mathbb{R}}
\newcommand{\N}{\mathbb{N}}
\newcommand{\Z}{\mathbb{Z}}
\newcommand{\lp}{\left(}
\newcommand{\rp}{\right)}
\newcommand{\ls}{\left[}
\newcommand{\rs}{\right]}
\newcommand{\lb}{\left\{}
\newcommand{\rb}{\right\}}
\newcommand{\arccot}{\text{arccot}}
\newcommand{\arccsc}{\text{arccsc}}
\newcommand{\arcsec}{\text{arcsec}} 

%%% TABLES %%%
\usepackage{colortbl}

%%% GRAPHS %%%
\usepackage{tikz}
\usepackage{pgfplots}
\pgfplotsset{compat=1.15}
\usepgfplotslibrary{fillbetween}
\usetikzlibrary{angles,quotes}

%%% ENVIRONMENTS %%%
\newcommand{\Example}{\paragraph{\Writinghand \hspace{0.1mm} Example.}}
\newcommand{\ExampleCont}{\paragraph{\Writinghand \hspace{0.1mm} Example (continued).}}
\newcommand{\boxenv}[2]{
	\fbox{
	\begin{minipage}{0.97\textwidth}
	\vspace{2mm}	
	\paragraph{#1} #2
	\vspace{2mm}
	\end{minipage}
	}}

%%% FUN THINGS %%%
\newcommand*\tc[1]{\tikz[baseline=(char.base)]{
            \node[shape=circle,draw,inner sep=2pt] (char) {#1};}}
\usepackage{marvosym}

%%% MISC %%%
\usepackage{hyperref}


\setcounter{page}{16}

\begin{document}
\section*{6.3: Volume by Slicing}

\boxenv{Learning Objectives.}{Upon successful completion of Section 6.3, you will be able to\dots
		
	\begin{itemize}[leftmargin=6mm]
		\item Answer conceptual questions involving the general slicing, disk, and washer methods.
		\item Use the general slicing method to find volumes of solids.
		\item Use the disk and washer methods to find the volume of solids of revolution about the $x$-axis.
		\item Use the disk and washer methods to find the volume of solids of revolution about the $y$-axis.
		\item Use the disk and washer methods to find the volume of solids of revolution about horizontal lines ($y=a$) other than the $x$-axis and vertical lines ($x=b$) other than the $y$-axis.
		\item Describe the solid whose volume is given by an integral.
		\item Solve applications involving the general slicing, disk, and washer methods.
	\end{itemize}
	\vspace{-4mm}
}

\vspace{5mm}

\subsection*{General Slicing Method}

\vspace{43mm}

\boxenv{Definition.}{Let $S$ be a solid that lies between $x=a$ and $x=b$. If the cross sections of $S$ are perpendicular to the $x$-axis and have area $A(x)$, where $A$ is a continuous function, then the \textbf{volume} of $S$ is

$$V=\lim_{n\to\infty}\sum_{i=1}^n A\left(x_i^*\right)\Delta x=\int_a^b A(x)\,dx.$$
}

\vspace{5mm}

\textbf{Note:} If $S$ is a solid lying between $y=c$ and $y=d$ with cross sections perpendicular to the $y$-axis that have area $A(y)$, where $A$ is a continuous function, then 

$$V=\int_c^d A(y)\,dy.$$

\newpage

\Example Find the volume of the solid with circular base of radius 5 whose cross sections perpendicular to the base and parallel to the $x$-axis are equilateral triangles.

\vspace{80mm}

A specific type of solid we will work with is a \textbf{solid of revolution}. Suppose $f$ is a continuous function on $[a,b]$ and $R$ is the region bounded by the graph of $f$, the $x$-axis, and the lines $x=a$ and $x=b$.

\vspace{50mm}

\textbf{Goal:} Find the volume of the solid generated by revolving the region $R$ about the $x$-axis.\\

\textbf{Idea:} Think about \underline{slices}.

\newpage 

\Example Let $R$ be the region bounded by $y=\sqrt{x-1}$, $y=0$, and $x=5$. Find the volume of the solid generated when $R$ is rotated about the $x$-axis.

\vspace{100mm}

\Example Let $R$ be the region bounded by $y=x^3$, $y=x$, and $x\geq 0$. Find the volume of the solid generated when $R$ is rotated about the $x$-axis.

\newpage

\Example Let $R$ be the region bounded by $y=\sqrt{x-1}$, $y=0$, and $x=5$. Set up the integral(s) needed to find the volume of the solid generated when $R$ is rotated about the line $y=2$.

\vspace{70mm}

\Example Let $R$ be the region bounded by $y=\ln x$, $y=1$, $y=2$, and $x=0$.

\begin{itemize}
\item[(a)] Find the volume of the solid generated when $R$ is rotated about the $y$-axis.

\vspace{70mm}

\item[(b)] Set up the integral(s) needed to find the volume of the solid generated when $R$ is rotated about the line $x=-2$.
\end{itemize}

\newpage



\end{document}