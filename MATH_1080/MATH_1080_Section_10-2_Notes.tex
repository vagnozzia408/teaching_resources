\documentclass[12pt]{article}
%%% DOCUMENT FORMATTING %%%
\usepackage[margin=1in]{geometry}
\usepackage{enumitem}
\setlength{\parindent}{0pt}
\newcommand{\disp}{\displaystyle}

%%% HEADER %%%
\usepackage{fancyhdr}
\pagestyle{fancy}
\fancyhf{}
\lhead{MATH 1080}
\rhead{Vagnozzi}
\cfoot{\thepage}

%%% MATH NOTATION & SYMBOLS %%%
\usepackage{amssymb}
\usepackage{amsmath}
\newcommand{\R}{\mathbb{R}}
\newcommand{\N}{\mathbb{N}}
\newcommand{\Z}{\mathbb{Z}}
\newcommand{\lp}{\left(}
\newcommand{\rp}{\right)}
\newcommand{\ls}{\left[}
\newcommand{\rs}{\right]}
\newcommand{\lb}{\left\{}
\newcommand{\rb}{\right\}}
\newcommand{\arccot}{\text{arccot}}
\newcommand{\arccsc}{\text{arccsc}}
\newcommand{\arcsec}{\text{arcsec}} 

%%% TABLES %%%
\usepackage{colortbl}

%%% GRAPHS %%%
\usepackage{tikz}
\usepackage{pgfplots}
\pgfplotsset{compat=1.15}
\usepgfplotslibrary{fillbetween}
\usetikzlibrary{angles,quotes}

%%% ENVIRONMENTS %%%
\newcommand{\Example}{\paragraph{\Writinghand \hspace{0.1mm} Example.}}
\newcommand{\ExampleCont}{\paragraph{\Writinghand \hspace{0.1mm} Example (continued).}}
\newcommand{\boxenv}[2]{
	\fbox{
	\begin{minipage}{0.97\textwidth}
	\vspace{2mm}	
	\paragraph{#1} #2
	\vspace{2mm}
	\end{minipage}
	}}

%%% FUN THINGS %%%
\newcommand*\tc[1]{\tikz[baseline=(char.base)]{
            \node[shape=circle,draw,inner sep=2pt] (char) {#1};}}
\usepackage{marvosym}

%%% MISC %%%
\usepackage{hyperref}


\setcounter{page}{98}

\begin{document}
\section*{10.2: Sequences}

\boxenv{Learning Objectives.}{Upon successful completion of Section 10.2, you will be able to\dots
		
	\begin{itemize}[leftmargin=6mm]
		\item Answer conceptual questions involving sequences.
		\item Find whether sequences are monotonic or whether they oscillate and give the limit if the sequence converges.
		\item Use properties and theorems to determine limits of sequences.
		\begin{itemize}
			\item[$\circ$] Note 1: It is useful to review L'H\^opital's Rule (Section 4.7).
			\item[$\circ$] Note 2: The fact that $\disp\lim_{x\to\infty}\left(1+\frac{a}{x}\right)^x=e^a$ may be used without proof.
		\end{itemize}
		\item Use the growth rate of sequences to determine limits of sequences that converge.
	\end{itemize}
	\vspace{-4mm}
}

\vspace{5mm}

\subsection*{Computing Limits of Sequences}

In Section 10.1, we introduced the general idea of what it means for a \textbf{sequence} to converge or diverge. We said that if the terms of a sequence $\left\{a_n\right\}$ approach some number $L$, then $\disp\lim_{n\to\infty}a_n=L$ exists and the sequence \textbf{converges} to $L$. \\

If the terms of the sequence do not approach a single number as $n$ increases, then the sequence has no limit and we say that it \textbf{diverges}.

\vspace{5mm}

\boxenv{Theorem: Limits of Sequences from Limits of Functions.}{Suppose that $f$ is a function such that $f(n)=a_n$, for positive integers $n$. If $\disp\lim_{x\to\infty}f(x)=L$, then the limit of the sequence $\left\{a_n\right\}$ is also $L$ $\left(\disp\lim_{n\to\infty}a_n=L\right)$, where $L$ may be $\pm\infty$.}

\paragraph{Limit Laws for Sequences.} Assume the sequences $\left\{ a_n\right\}$ and $\left\{ b_n \right\}$ have limits $A$ and $B$, respectively (that is, both sequences converge), and $c$ is a constant.

\vspace{2mm}
 
%\begin{multicols}{2}
\begin{itemize}
	\item[\tc{1}] $\disp\lim_{n\to\infty}\left(a_n\pm b_n\right)=A\pm B$ 
	\item[\tc{2}] $\disp\lim_{n\to\infty}c a_n=cA$, where $c\in\R$
	\item[\tc{3}] $\disp\lim_{n\to\infty}\left(a_n b_n\right)=AB$
	\item[\tc{4}] $\disp\lim_{n\to\infty}\frac{a_n}{b_n}=\frac{A}{B}$, provided $B\neq 0$
\end{itemize}
%\end{multicols}

\newpage

\Examples Determine if each of the following sequences converges or diverges. If the sequence converges, find the value to which it converges.

\vspace{5mm}

$\disp a_n=\frac{3+5n^2}{n+n^2}$

\vfill

$\disp\left\{\frac{n^3+2n}{n+1}\right\}$

\vfill

$\disp\left\{\tan\left(\frac{2n\pi}{1+8n}\right)\right\}$

\vfill

\boxenv{Definition.}{Let $r$ be a real number ($r\in\R$). Then $\left\{ r^n\right\}$ is a \textbf{geometric sequence}.}

\vspace{3mm}

For what value of $r$ does a geometric sequence converge?

\vfill
\vfill

\newpage
\boxenv{The Squeeze Theorem.}{If $a_n\leq b_n\leq c_n$ for all $n\geq N$ and $\disp\lim_{n\to\infty}a_n=\lim_{n\to\infty}c_n=L$, then $\disp\lim_{n\to\infty}b_n=L$.}

\Examples Determine if each of the following sequences converges or diverges. If the sequence converges, find the value to which it converges.

\vspace{5mm}

$\disp\left\{\frac{\cos^2n}{2^n}\right\}$

\vfill

$\left\{2^{n+1} 3^{-n}\right\}$

\vfill

$a_n=\disp\frac{(-1)^n}{2\sqrt{n}}$

\vfill

$\left\{0,1,0,0,1,0,0,0,1,\dots\right\}$

\vfill

\newpage
\Example $\disp\left\{\left(\frac{n}{n+5}\right)^n\right\}$ \hfill Hint: Recall from Section 4.7 that $\disp\lim_{x\to\infty}\left(1+\frac{a}{x}\right)^x=e^a$.
\vfill
\vfill
\vfill

\subsection*{Terminology for Sequences}
\begin{itemize}
	\item $\left\{a_n\right\}$ is \textbf{increasing} if
	
	\vfill
	
	\item $\left\{a_n\right\}$ is \textbf{nondecreasing} if
	\vfill

	\item $\left\{a_n\right\}$ is \textbf{decreasing} if
	\vfill

	\item $\left\{a_n\right\}$ is \textbf{nonincreasing} if
	\vfill

	\item $\left\{a_n\right\}$ is \textbf{monotonic} if it is either \textbf{nonincreasing} or \textbf{nondecreasing}.
	\vfill

	\item $\left\{a_n\right\}$ is \textbf{bounded above} if
	\vfill

	\item $\left\{a_n\right\}$ is \textbf{bounded below} if
	\vfill

	\item If $\left\{a_n\right\}$ is bounded above and below, then we say that $\left\{a_n\right\}$ is a \textbf{bounded} sequence.
	\vfill

\end{itemize}

\newpage

\Example Determine whether the sequence $\left\{ (-2)^{n+1} \right\}$ converges or diverges and state whether it is monotonic or whether it oscillates. Give the limit if the sequence converges.

\vfill

\boxenv{Monotonic Sequence Theorem.}{Every bounded monotonic sequence is convergent.}

\vspace{5mm}

\textbf{Notes on this theorem:}

\vfill

\newpage

\subsection*{Growth Rates of Sequences}

The relative growth rates of functions (established in Section 4.7: L'H\^opital's Rule) are now applied to sequences. A few notes:
\begin{itemize}
	\item To compare growth rates of two nondecreasing sequences of positive terms $\left\{a_n\right\}$ and $\left\{b_n\right\}$, evaluate $\disp\lim_{n\to\infty}\frac{a_n}{b_n}$.
	\begin{itemize}
		\item If $\disp\lim_{n\to\infty}\frac{a_n}{b_n}=0$, then $\left\{b_n\right\}$ grows faster than $\left\{a_n\right\}$.
		\item If $\disp\lim_{n\to\infty}\frac{a_n}{b_n}=\infty$, then $\left\{a_n\right\}$ grows faster than $\left\{b_n\right\}$.
	\end{itemize}
\item The notation $\left\{a_n\right\}<<\left\{b_n\right\}$ means that $\left\{b_n\right\}$ grows faster than $\left\{a_n\right\}$. 
\end{itemize}

\vspace{2mm}

\boxenv{Theorem: Growth Rates of Sequences.}{The following sequences are ordered according to increasing growth rates as $n\to\infty$; that is, if $\left\{a_n\right\}$ appears before $\left\{b_n\right\}$ in the list, then $\disp\lim_{n\to\infty}\frac{a_n}{b_n}=0$ and $\disp\lim_{n\to\infty}\frac{b_n}{a_n}=\infty$:

$$\left\{\ln^q n\right\}<<\left\{n^p\right\}<<\left\{n^p\ln^r n\right\}<<\left\{n^{p+s}\right\} << \left\{b^n\right\}<<\left\{n!\right\}<<\left\{n^n\right\}$$

The ordering applies for positive real numbers $p$, $q$, $r$, $s$, and $b>1$.}

\Examples Use the theorem on growth rates to find the limit of the following sequences or state that they diverge.

\vspace{5mm}

$\disp\left\{\frac{n^{10}}{\ln^{1000}n}\right\}$

\vfill

$a_n=\disp\frac{6^n+3^n}{6^n+n^{1000}}$

\vfill

\end{document}