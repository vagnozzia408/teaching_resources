\documentclass[12pt]{article}
%%% DOCUMENT FORMATTING %%%
\usepackage[margin=1in]{geometry}
\usepackage{enumitem}
\setlength{\parindent}{0pt}
\newcommand{\disp}{\displaystyle}
\usepackage{multicol}

%%% HEADER %%%
\usepackage{fancyhdr}
\pagestyle{fancy}
\fancyhf{}
\lhead{MATH 1080}
\rhead{Vagnozzi}
\cfoot{\thepage}

%%% MATH NOTATION & SYMBOLS %%%
\usepackage{amssymb}
\usepackage{amsmath}
\newcommand{\R}{\mathbb{R}}
\newcommand{\N}{\mathbb{N}}
\newcommand{\Z}{\mathbb{Z}}
\newcommand{\lp}{\left(}
\newcommand{\rp}{\right)}
\newcommand{\ls}{\left[}
\newcommand{\rs}{\right]}
\newcommand{\lb}{\left\{}
\newcommand{\rb}{\right\}}
\newcommand{\arccot}{\text{arccot}}
\newcommand{\arccsc}{\text{arccsc}}
\newcommand{\arcsec}{\text{arcsec}}
\DeclareSymbolFont{matha}{OML}{txmi}{m}{it}% txfonts
\DeclareMathSymbol{\varv}{\mathord}{matha}{118} 

%%% TABLES %%%
\usepackage{colortbl}

%%% GRAPHS %%%
\usepackage{tikz}
\usepackage{pgfplots}
\pgfplotsset{compat=1.15}
\usepgfplotslibrary{fillbetween}
\usetikzlibrary{angles,quotes}

%%% ENVIRONMENTS %%%
\newcommand{\Example}{\paragraph{\Writinghand \hspace{0.1mm} Example.}}
\newcommand{\Examples}{\paragraph{\Writinghand \hspace{0.1mm} Examples.}}
\newcommand{\ExampleCont}{\paragraph{\Writinghand \hspace{0.1mm} Example (continued).}}
\newcommand{\ExamplesCont}{\paragraph{\Writinghand \hspace{0.1mm} Examples (continued).}}

\newcommand{\boxenv}[2]{
	\fbox{
	\begin{minipage}{0.97\textwidth}
	\vspace{2mm}	
	\paragraph{#1} #2
	\vspace{2mm}
	\end{minipage}
	}}

%%% FUN THINGS %%%
\newcommand*\tc[1]{\tikz[baseline=(char.base)]{
            \node[shape=circle,draw,inner sep=2pt] (char) {#1};}}
\usepackage{marvosym}

%%% MISC %%%
\usepackage{hyperref}


\setcounter{page}{156}

\begin{document}
\section*{11.4: Working with Taylor Series}

\boxenv{Learning Objectives.}{Upon successful completion of Section 11.4, you will be able to\dots
		
	\begin{itemize}[leftmargin=6mm]
		\item Answer conceptual questions involving Taylor series.
		\item Evaluate limits using Taylor series.
		\item Differentiate Taylor series.
		\item Find power series solutions to differential equations.
		\item Approximate definite integrals using Taylor series.
		\item Approximate real numbers using Taylor series.
		\item Evaluate infinite series.
		\item Identify functions represented by power series.
	\end{itemize}
	\vspace{-4mm}
}

\vspace{5mm}

\subsection*{Working with Taylor Series}

So far, we've manipulated series by \textit{differentiation}, \textit{integration}, \textit{multiplication} by a value, and making a \textit{substitution} for $x$. In this section, we'll see a few more ways we can work with Taylor series.

\Example We can use series to evaluate limits. Use series to evaluate $\disp\lim_{x\to\infty}\dfrac{x-\ln(1+x)}{x^2}$.\\

Note: $\ln(1+x)=\disp\sum_{n=1}^\infty\left(-1\right)^{n+1}\dfrac{x^n}{n}$, for $-1<x\leq 1$.

\newpage

\subsubsection*{Common Maclaurin Series (Taylor Series Centered at \textit{a} = 0)}
We will often need to manipulate series using known power series. Some common Maclaurin series are included below for reference.
\begin{alignat*}{3}
\dfrac{1}{1-x} &= 1+x+x^2+\cdots + x^k+\cdots & &= \disp\sum_{k=0}^\infty x^k, & \text{ for }|x|<1\\
\dfrac{1}{1+x} &= 1-x+x^2-\cdots+(-1)^kx^k+\cdots & &= \disp\sum_{k=0}^\infty(-1)^kx^k, & \text{ for }|x|<1\\
e^x &= 1+x+\dfrac{x^2}{2!}+\cdots+\dfrac{x^k}{k!}+\cdots & &= \disp\sum_{k=0}^\infty\dfrac{x^k}{k!}, & \text{ for }|x|<\infty\\
\sin(x) &= x-\dfrac{x^3}{3!}+\dfrac{x^5}{5!}-\cdots+\dfrac{(-1)^k x^{2k+1}}{(2k+1)!}+\cdots  & &= \disp\sum_{k=0}^\infty\dfrac{(-1)^kx^{2k+1}}{(2k+1)!}, & \text{ for } |x|<\infty\\
\cos(x) &= 1-\dfrac{x^2}{2!}+\dfrac{x^4}{4!}-\cdots +\dfrac{(-1)^k x^{2k}}{(2k)!}+\cdots & &= \disp\sum_{k=0}^\infty\dfrac{(-1)^{k}x^{2k}}{(2k)!}, & \text{ for }|x|<\infty\\
\ln(1+x) &= x-\dfrac{x^2}{2}+\dfrac{x^3}{3}-\cdots+\dfrac{(-1)^{k+1}x^k}{k}+\cdots & &= \disp\sum_{k=1}^\infty\dfrac{(-1)^{k+1}x^k}{k}, & \text{ for }-1< x\leq 1\\
-\ln(1-x) &= x+\dfrac{x^2}{2}+\dfrac{x^3}{3}+\cdots+\dfrac{x^k}{k}+\cdots & &= \disp\sum_{k=1}^\infty\dfrac{x^k}{k}, & \text{ for }-1\leq x < 1\\
\arctan(x) &= x-\dfrac{x^3}{3}+\dfrac{x^5}{5}-\cdots +\dfrac{(-1)^kx^{2k+1}}{2k+1}+\cdots & &= \disp\sum_{k=0}^\infty\dfrac{(-1)^kx^{2k+1}}{2k+1}, & \text{ for }|x|\leq 1\\
\end{alignat*}

\Example Use a known power series to find the sum of the series $\disp\sum_{n=0}^\infty\dfrac{(-1)^n\pi^{2n+1}}{4^{2n+1}(2n+1)!}$.

\vfill

\Example Use a known power series to find the sum of the series $\disp\sum_{n=0}^\infty\dfrac{3^n}{5^n n!}$.

\vfill

\newpage

\Example Identify the function represented by the power series $\disp\sum_{k=0}^\infty 2^k x^{2k+1}$.

\vfill

\Example We can also use Taylor series to approximate integrals. Use a Taylor series to approximate the integral $\disp\int_{0}^{0.35}\arctan x\,dx$. Retain as many terms needed to ensure that the error is less than $1/10^4$.

\vfill
\vfill
\vfill

\newpage

\Example Lastly, we can use power series to solve \textit{differential equations}. Find a power series for the solution of the differential equation $y'(t)-3y=10$ with initial condition $y(0)=2$. Then identify the function represented by the power series.

\end{document}