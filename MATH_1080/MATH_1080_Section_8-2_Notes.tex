\documentclass[12pt]{article}
%%% DOCUMENT FORMATTING %%%
\usepackage[margin=1in]{geometry}
\usepackage{enumitem}
\setlength{\parindent}{0pt}
\newcommand{\disp}{\displaystyle}
\usepackage{multicol}

%%% HEADER %%%
\usepackage{fancyhdr}
\pagestyle{fancy}
\fancyhf{}
\lhead{MATH 1080}
\rhead{Vagnozzi}
\cfoot{\thepage}

%%% MATH NOTATION & SYMBOLS %%%
\usepackage{amssymb}
\usepackage{amsmath}
\newcommand{\R}{\mathbb{R}}
\newcommand{\N}{\mathbb{N}}
\newcommand{\Z}{\mathbb{Z}}
\newcommand{\lp}{\left(}
\newcommand{\rp}{\right)}
\newcommand{\ls}{\left[}
\newcommand{\rs}{\right]}
\newcommand{\lb}{\left\{}
\newcommand{\rb}{\right\}}
\newcommand{\arccot}{\text{arccot}}
\newcommand{\arccsc}{\text{arccsc}}
\newcommand{\arcsec}{\text{arcsec}}
\DeclareSymbolFont{matha}{OML}{txmi}{m}{it}% txfonts
\DeclareMathSymbol{\varv}{\mathord}{matha}{118} 

%%% TABLES %%%
\usepackage{colortbl}

%%% GRAPHS %%%
\usepackage{tikz}
\usepackage{pgfplots}
\pgfplotsset{compat=1.15}
\usepgfplotslibrary{fillbetween}
\usetikzlibrary{angles,quotes}

%%% ENVIRONMENTS %%%
\newcommand{\Example}{\paragraph{\Writinghand \hspace{0.1mm} Example.}}
\newcommand{\Examples}{\paragraph{\Writinghand \hspace{0.1mm} Examples.}}
\newcommand{\ExampleCont}{\paragraph{\Writinghand \hspace{0.1mm} Example (continued).}}
\newcommand{\ExamplesCont}{\paragraph{\Writinghand \hspace{0.1mm} Examples (continued).}}

\newcommand{\boxenv}[2]{
	\fbox{
	\begin{minipage}{0.97\textwidth}
	\vspace{2mm}	
	\paragraph{#1} #2
	\vspace{2mm}
	\end{minipage}
	}}

%%% FUN THINGS %%%
\newcommand*\tc[1]{\tikz[baseline=(char.base)]{
            \node[shape=circle,draw,inner sep=2pt] (char) {#1};}}
\usepackage{marvosym}

%%% MISC %%%
\usepackage{hyperref}


\DeclareSymbolFont{matha}{OML}{txmi}{m}{it}% txfonts
\DeclareMathSymbol{\varv}{\mathord}{matha}{118}

\setcounter{page}{50}

\begin{document}
\section*{8.2: Integration by Parts}

\boxenv{Learning Objectives.}{Upon successful completion of Section 8.2, you will be able to\dots
		
	\begin{itemize}[leftmargin=6mm]
		\item Answer conceptual questions involving integration by parts.
		\item Find indefinite integrals using integration by parts.
		\item Evaluate definite integrals using integration by parts.
		\item Find volumes of solids of revolution using integration by parts.
		\item Combine integration methods to evaluate integrals.
	\end{itemize}
	\vspace{-4mm}
}

\vspace{5mm}

\subsection*{Introduction}

We will spend the next several sections learning new integration techniques. As we study each technique, notice the types of functions that the technique is used to integrate. \textit{Pattern recognition} --- recognizing which types of integrals can be solved with which technique --- is key to choosing the appropriate technique required to solve a problem. 


\subsubsection*{Connections to Derivative Rules}

One of the ways that we can think about the integration technique \textbf{u-substitution} is as the \textbf{chain rule} in reverse. Recall that the chain rule is used to differentiate \textit{composite} functions.

$$\frac{d}{dx}\Big(f\big(\textcolor{blue}{g(x)}\big)\Big)=f'\big(\textcolor{blue}{g(x)}\big)\cdot \textcolor{red}{g'(x)}$$

$$\int f'\big(\textcolor{blue}{g(x)}\big)\cdot\textcolor{red}{g'(x)\,dx} = \int f'(\textcolor{blue}{u})\,\textcolor{red}{du}=f(\textcolor{blue}{u})+C=f\big(\textcolor{blue}{g(x)}\big)+C $$

\vspace{3mm}

Similarly, the integration technique introduced in this section, \textbf{integration by parts}, can be thought of as the \textbf{product rule} in reverse. To differentiate a \textit{product} of two functions $\textcolor{blue}{u=u(x)}$ and $\textcolor{red}{\varv=\varv(x)}$\dots

\newpage

\boxenv{Integration by Parts.}{Suppose $u$ and $v$ are differentiable functions. Then
$$\int u\,d\varv=u\varv-\int \varv\,du.$$
\vspace{-5mm}}

\vspace{5mm}

\textbf{Integration by parts }is useful for integrals where the integrand is a \textit{product} of two different types of functions. Whenever using this technique, we must choose one term of the product to be $\textcolor{blue}{u}$, the function we will \textit{differentiate} to find $\textcolor{blue}{du}$, and the other term to be $\textcolor{red}{\varv}$, which we will \textit{integrate} to find $\textcolor{red}{d\varv}$. The goal is to choose a $u$ that gets \textit{simpler} when differentiated. \\

The acronym ``ILATE'' can serve as a guideline for how to prioritize your choice of $u$, from greatest priority to lowest priority.
\begin{itemize}
	\item[\textbf{I:}] Inverse trig functions
	\item[\textbf{L:}] Logarithmic functions
	\item[\textbf{A:}] Algebraic functions
	\item[\textbf{T:}] Trigonometric functions
	\item[\textbf{E:}] Exponential functions
\end{itemize}

\vspace{3mm}

\Example Evaluate $\disp\int xe^x\,dx$.

\newpage

\Example Evaluate $\disp\int x^2\sin 2x\,dx$.

\vfill

\Example Evaluate $\disp\int\ln x\,dx$.

\vfill

\Example Evaluate $\disp\int\arctan y\,dy$.

\vfill

\newpage

\Example Evaluate $\disp\int e^{-x}\sin(4x)\,dx$.

\vfill

\Example Consider the integral $\disp\int\sec^2 x\ln(\tan x+2)\,dx$. \\

Evaluating this integral will involve integration by parts, but what substitution could be done first to simplify the integral?

\vspace{40mm}
\newpage

\Example Find the volume of the solid that is generated when the region bounded by $f(x)=x\ln x$ and the $x$-axis on $\left[1,e^2\right]$ is revolved around the $x$-axis.

\vspace{5mm}

\begin{tikzpicture}[scale=.9]
	\begin{axis}[grid=none,
    	axis x line=middle,
%    	xtick=\empty,
%    	ytick=\empty,
        xmax=8.5, xmin=-.5,
        axis y line=center,
        ymax=15.5, ymin=-1,
        axis line style=<->,
		xlabel=$x$,ylabel=$y$, axis on top                    	]

    \addplot[->,name path=g,smooth,domain=0:7.4,color=blue,samples=500,very thick] {x*ln(x)};

    \end{axis}
\end{tikzpicture}



\end{document}