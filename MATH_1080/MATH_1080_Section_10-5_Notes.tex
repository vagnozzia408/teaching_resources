\documentclass[12pt]{article}
%%% DOCUMENT FORMATTING %%%
\usepackage[margin=1in]{geometry}
\usepackage{enumitem}
\setlength{\parindent}{0pt}
\newcommand{\disp}{\displaystyle}

%%% HEADER %%%
\usepackage{fancyhdr}
\pagestyle{fancy}
\fancyhf{}
\lhead{MATH 1080}
\rhead{Vagnozzi}
\cfoot{\thepage}

%%% MATH NOTATION & SYMBOLS %%%
\usepackage{amssymb}
\usepackage{amsmath}
\newcommand{\R}{\mathbb{R}}
\newcommand{\N}{\mathbb{N}}
\newcommand{\Z}{\mathbb{Z}}
\newcommand{\lp}{\left(}
\newcommand{\rp}{\right)}
\newcommand{\ls}{\left[}
\newcommand{\rs}{\right]}
\newcommand{\lb}{\left\{}
\newcommand{\rb}{\right\}}
\newcommand{\arccot}{\text{arccot}}
\newcommand{\arccsc}{\text{arccsc}}
\newcommand{\arcsec}{\text{arcsec}} 

%%% TABLES %%%
\usepackage{colortbl}

%%% GRAPHS %%%
\usepackage{tikz}
\usepackage{pgfplots}
\pgfplotsset{compat=1.15}
\usepgfplotslibrary{fillbetween}
\usetikzlibrary{angles,quotes}

%%% ENVIRONMENTS %%%
\newcommand{\Example}{\paragraph{\Writinghand \hspace{0.1mm} Example.}}
\newcommand{\ExampleCont}{\paragraph{\Writinghand \hspace{0.1mm} Example (continued).}}
\newcommand{\boxenv}[2]{
	\fbox{
	\begin{minipage}{0.97\textwidth}
	\vspace{2mm}	
	\paragraph{#1} #2
	\vspace{2mm}
	\end{minipage}
	}}

%%% FUN THINGS %%%
\newcommand*\tc[1]{\tikz[baseline=(char.base)]{
            \node[shape=circle,draw,inner sep=2pt] (char) {#1};}}
\usepackage{marvosym}

%%% MISC %%%
\usepackage{hyperref}


\setcounter{page}{116}

\begin{document}
\section*{10.5: Comparison Tests}

\boxenv{Learning Objectives.}{Upon successful completion of Section 10.5, you will be able to\dots
		
	\begin{itemize}[leftmargin=6mm]
		\item Answer conceptual questions involving the Comparison Tests.
		\item Use the Comparison Test or Limit Comparison Tests to determine series convergence or divergence.
		\item Determine series convergence or divergence using a test of your choice.
	\end{itemize}
	\vspace{-4mm}
}

\vspace{5mm}

\subsection*{The (Direct) Comparison Test}

Similar to the Comparison Theorem for Improper Integrals that was introduced in Section~8.9, we can compare a given infinite series to a series that we know to be convergent or divergent. We will usually do comparisons to a $p$-series or a geometric series.

\vspace{5mm}

\boxenv{Comparison Test.}{Suppose that $\sum a_n$ and $\sum b_n$ are series with positive terms. Then\dots
\vspace{2mm}
\begin{itemize}
\item[(1)] If $\sum b_n$ converges and $a_n\leq b_n$ for all $n$, then $\sum a_n$ converges.
\vspace{2mm}
\item[(2)] If $\sum b_n$ diverges and $a_n\geq b_n$ for all $n$, then $\sum a_n$ diverges.
\end{itemize}
}

\Example $\disp\sum_{n=1}^\infty\dfrac{n}{2n^3+1}$

\newpage

\Example $\disp\sum_{k=2}^\infty \dfrac{\ln k}{k}$

\vfill

\subsubsection*{Important Notes on Using Comparison Tests}

\vspace{60mm}

\newpage

\subsection*{The Limit Comparison Test}

\boxenv{Limit Comparison Test.}{Suppose that $\sum a_n$ and $\sum b_n$ are series with positive terms.\\

If $\disp\lim_{n\to\infty}\dfrac{a_n}{b_n}=c$, where $0<c<\infty$,\\

then either $\sum a_n$ and $\sum b_n$ both converge or both diverge.}

\Example $\disp\sum_{n=1}^\infty\dfrac{n}{2n^3-1}$

\newpage

\Example $\disp\sum_{n=1}^\infty\dfrac{\sqrt{n^4+1}}{n^3+n^2}$

\vfill

\Example $\disp\sum_{n=1}^\infty\tan\left(\dfrac{1}{n}\right)$

\vfill

\newpage

\Example $\disp\sum_{n=1}^\infty\dfrac{n+4^n}{n+6^n}$

\end{document}