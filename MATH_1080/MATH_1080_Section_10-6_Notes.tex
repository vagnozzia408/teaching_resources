\documentclass[12pt]{article}
%%% DOCUMENT FORMATTING %%%
\usepackage[margin=1in]{geometry}
\usepackage{enumitem}
\setlength{\parindent}{0pt}
\newcommand{\disp}{\displaystyle}

%%% HEADER %%%
\usepackage{fancyhdr}
\pagestyle{fancy}
\fancyhf{}
\lhead{MATH 1080}
\rhead{Vagnozzi}
\cfoot{\thepage}

%%% MATH NOTATION & SYMBOLS %%%
\usepackage{amssymb}
\usepackage{amsmath}
\newcommand{\R}{\mathbb{R}}
\newcommand{\N}{\mathbb{N}}
\newcommand{\Z}{\mathbb{Z}}
\newcommand{\lp}{\left(}
\newcommand{\rp}{\right)}
\newcommand{\ls}{\left[}
\newcommand{\rs}{\right]}
\newcommand{\lb}{\left\{}
\newcommand{\rb}{\right\}}
\newcommand{\arccot}{\text{arccot}}
\newcommand{\arccsc}{\text{arccsc}}
\newcommand{\arcsec}{\text{arcsec}} 

%%% TABLES %%%
\usepackage{colortbl}

%%% GRAPHS %%%
\usepackage{tikz}
\usepackage{pgfplots}
\pgfplotsset{compat=1.15}
\usepgfplotslibrary{fillbetween}
\usetikzlibrary{angles,quotes}

%%% ENVIRONMENTS %%%
\newcommand{\Example}{\paragraph{\Writinghand \hspace{0.1mm} Example.}}
\newcommand{\ExampleCont}{\paragraph{\Writinghand \hspace{0.1mm} Example (continued).}}
\newcommand{\boxenv}[2]{
	\fbox{
	\begin{minipage}{0.97\textwidth}
	\vspace{2mm}	
	\paragraph{#1} #2
	\vspace{2mm}
	\end{minipage}
	}}

%%% FUN THINGS %%%
\newcommand*\tc[1]{\tikz[baseline=(char.base)]{
            \node[shape=circle,draw,inner sep=2pt] (char) {#1};}}
\usepackage{marvosym}

%%% MISC %%%
\usepackage{hyperref}


\setcounter{page}{121}

\begin{document}
\section*{10.6: Alternating Series}

\boxenv{Learning Objectives.}{Upon successful completion of Section 10.6, you will be able to\dots
		
	\begin{itemize}[leftmargin=6mm]
		\item Answer conceptual questions involving alternate series.
		\item Apply the alternating series test if possible. Otherwise, apply a different appropriate test.
		\item Use Theorem 10.18 to find an upper bound for the error in using the $n^\text{th}$ partial sum to estimate the value of the series.
		\item Determine the number of terms needed to ensure a given error.
		\item Estimate the value of an alternating series.
		\item Determine if a series converges absolutely, converges conditionally, or diverges.
	\end{itemize}
	\vspace{-4mm}
}

\vspace{5mm}

\subsection*{Introduction}

Both the Integral Test and the Comparison Tests apply only to series with \textit{positive terms}. An important type of series with positive and negative terms is called an \textbf{alternating} series.

\Examples Consider the terms of each of the following series.

\vspace{2mm}

$\disp\sum_{n=1}^\infty\left(-\dfrac{1}{2}\right)^n$

\vfill

$\disp\sum_{n=1}^\infty\left(-1\right)^{n+1}e^{2/n}$

\vfill

$\disp\sum_{n=1}^\infty\left(-1\right)^{n-1}\dfrac{1}{n}$

\vfill

\newpage

\boxenv{Definition.}{An \textbf{alternate series} is an infinite sum of the form

$$\disp\sum_{n=1}^\infty\left(-1\right)^{n-1}b_n\text{ or } \sum_{n=1}^\infty\left(-1\right)^{n}b_n\text{ with } b_n>0.$$

The terms of an alternating series will \textit{alternate} between positive and negative values.

}

\vspace{4mm}

\boxenv{Alternating Series Test.}{Consider an alternating series defined as above. If both of the following are true\dots

\vspace{2mm}

\begin{enumerate}
	\item $b_{n+1}\leq b_n$ for all $n\geq N$, and
	\item $\disp\lim_{n\to\infty}b_n=0$,
\end{enumerate}
then the series converges.
}

\vspace{5mm}

\paragraph{Thinking about the conditions of the Alternating Series Test.} Is it possible to have a sequence that decreases but does not converge to 0? Give an example or explain why not.

\vfill

Is it possible to have a sequence that converges to 0 but doesn't decrease? Give an example or explain why not.

\vfill

\newpage

\paragraph{Idea behind the proof of the Alternating Series Test.} Why do the conditions of the test imply convergence?

\vfill

\Examples Determine whether the following series converge or diverge.

\vspace{2mm}

\begin{itemize}
\item[\tc{1}] The \textbf{alternating harmonic series}: $\disp\sum_{n=1}^\infty(-1)^{n-1}\dfrac{1}{n}$

\end{itemize}

\vfill

\newpage

\ExamplesCont

\begin{itemize}
\item[\tc{2}] $\disp\sum_{n=1}^\infty(-1)^{n+1}e^{2/n}$

\vfill

\item[\tc{3}] $\disp\sum_{n=1}^\infty (-1)^{n+1}\dfrac{n^2}{n^3+4}$

\vfill

\end{itemize}

\newpage

\ExamplesCont

\begin{itemize}
\item[\tc{4}] $\disp\sum_{n=0}^\infty \dfrac{\sin\left(n+\frac{1}{2}\right)\pi}{1+\sqrt{n}}$
\end{itemize}

\vfill

\subsection*{Estimating Sums}

Recall that the error in estimating the sum $S$ of a convergent series by $S_n$ is the \textbf{remainder} $R_n=S-S_n$. The Integral Estimation Theorem allows us to find the bounds on the remainder,
$$\int_{n+1}^\infty f(x)\,dx\leq R_n\leq \int_n^\infty f(x)\,dx$$

where $f(x)$ is continuous, positive, and decreasing for $x\geq 1$ and $f(n)=a_n$. This theorem will not work for alternating series because the terms are, by definition, not always positive, so we'll need a different theorem.

\vspace{4mm}

\boxenv{Alternating Series Estimation Theorem.}{If $S=\disp\sum_{n=1}^\infty (-1)^{n-1}b_n$ is the sum of an alternating series that satisfies (1) $b_{n+1}\leq b_n$ and (2) $\disp\lim_{n\to\infty}b_n=0$, then\dots

\vspace{30mm}}

\vspace{30mm}

\newpage

\Example Consider the infinite series $\disp\sum_{n=1}^\infty\dfrac{(-1)^{n+1}}{n^6}$.

\begin{enumerate}
\item[(a)] If we use the sum of the first three terms to estimate the sum of the series, what is the bound on the error?

\vfill

\item[(b)] How many terms of the series do we need to add in order to estimate the sum so that $\big|$error$\big|<1/10^6$?

\vfill
\end{enumerate}

\newpage

\subsection*{Absolute Convergence and Conditional Convergence}

So far, we have only considered \textit{whether} a series converges or diverges, but now we will also consider the \textit{type} of convergence. The types of convergence we're interested in has to do with the relationship between the convergence/divergence of $\sum a_n$ and $\sum |a_n|$. 

\vspace{5mm}

\boxenv{Definition.}{A series $\disp\sum a_n$ is called \textbf{absolutely convergent} if the series $\disp\sum|a_n|$ is convergent.}

\vspace{5mm}

\textbf{Important Notes:}
\begin{itemize}
	\item If a series $\sum a_n$ is absolutely convergent, then it is convergent.
	\item This means that if $\sum |a_n|$ converges, then $\sum a_n$ converges.
	\item So if $\sum a_n$ is absolutely convergent, then both $\sum |a_n|$ and $\sum a_n$ converge.
	\item \textbf{\underline{Caution:}} If $\sum a_n$ converges, $\sum |a_n|$ may converge or diverge.
\end{itemize}

\vspace{5mm}

\boxenv{Definition.}{A series $\disp\sum a_n$ is called \textbf{conditionally convergent} if $\disp\sum a_n$ converges but $\disp\sum |a_n|$ diverges.}

\vspace{5mm}

\paragraph{Steps to Determine if a Series is Absolutely Convergent, Conditionally \\ Convergent, or Divergent.}
\begin{enumerate}
	\item[\tc{1}] Look at $\sum |a_n|$. Determine if this series converges or diverges.
	\begin{itemize}
		\item[(a)] If $\sum |a_n|$ converges, then $\sum a_n$ is absolutely convergent and we are done.
		\item[(b)] If $\sum|a_n|$ diverges, then go to Step 2.
	\end{itemize}
	\item[\tc{2}] Look at $a_n$. Determine if this series converges or diverges.
	\begin{itemize}
		\item[(a)] If $\sum a_n$ converges, then $\sum a_n$ is conditionally convergent.
		\item[(b)] If $\sum a_n$ diverges, then $\sum a_n$ is divergent.
	\end{itemize}
\end{enumerate}

\newpage

\Examples Determine if each of the following series is absolutely convergent, conditionally convergent, or divergent.

\begin{itemize}
\item[\tc{1}] $\disp\sum_{n=1}^\infty \dfrac{(-1)^n}{n^4}$

\vfill

\item[\tc{2}] $\disp\sum_{n=1}^\infty \dfrac{(-1)^n}{n}$

\vfill
\end{itemize}

\newpage

\ExamplesCont Determine if each of the following series is absolutely convergent, conditionally convergent, or divergent.

\begin{itemize}
\item[\tc{3}] $\disp\sum_{n=1}^\infty(-1)^n\arctan(n)$

\vfill

\item[\tc{4}] $\disp\sum_{n=1}^\infty \dfrac{\sin(4n)}{4^n}$

\vfill
\end{itemize}

\end{document}