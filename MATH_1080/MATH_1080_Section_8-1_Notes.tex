\documentclass[12pt]{article}
%%% DOCUMENT FORMATTING %%%
\usepackage[margin=1in]{geometry}
\usepackage{enumitem}
\setlength{\parindent}{0pt}
\newcommand{\disp}{\displaystyle}

%%% HEADER %%%
\usepackage{fancyhdr}
\pagestyle{fancy}
\fancyhf{}
\lhead{MATH 1080}
\rhead{Vagnozzi}
\cfoot{\thepage}

%%% MATH NOTATION & SYMBOLS %%%
\usepackage{amssymb}
\usepackage{amsmath}
\newcommand{\R}{\mathbb{R}}
\newcommand{\N}{\mathbb{N}}
\newcommand{\Z}{\mathbb{Z}}
\newcommand{\lp}{\left(}
\newcommand{\rp}{\right)}
\newcommand{\ls}{\left[}
\newcommand{\rs}{\right]}
\newcommand{\lb}{\left\{}
\newcommand{\rb}{\right\}}
\newcommand{\arccot}{\text{arccot}}
\newcommand{\arccsc}{\text{arccsc}}
\newcommand{\arcsec}{\text{arcsec}} 

%%% TABLES %%%
\usepackage{colortbl}

%%% GRAPHS %%%
\usepackage{tikz}
\usepackage{pgfplots}
\pgfplotsset{compat=1.15}
\usepgfplotslibrary{fillbetween}
\usetikzlibrary{angles,quotes}

%%% ENVIRONMENTS %%%
\newcommand{\Example}{\paragraph{\Writinghand \hspace{0.1mm} Example.}}
\newcommand{\ExampleCont}{\paragraph{\Writinghand \hspace{0.1mm} Example (continued).}}
\newcommand{\boxenv}[2]{
	\fbox{
	\begin{minipage}{0.97\textwidth}
	\vspace{2mm}	
	\paragraph{#1} #2
	\vspace{2mm}
	\end{minipage}
	}}

%%% FUN THINGS %%%
\newcommand*\tc[1]{\tikz[baseline=(char.base)]{
            \node[shape=circle,draw,inner sep=2pt] (char) {#1};}}
\usepackage{marvosym}

%%% MISC %%%
\usepackage{hyperref}


\setcounter{page}{44}

\begin{document}
\section*{8.1: Basic Approaches to Integration}

\boxenv{Learning Objectives.}{Upon successful completion of Section 8.1, you will be able to\dots
		
	\begin{itemize}[leftmargin=6mm]
		\item Answer conceptual questions involving basic approaches to integration.
		\item Find indefinite integrals using basic methods.
		\item Evaluate definite integrals using basic methods.
		\item Find the area of a region bounded by two curves using basic methods.
		\item Find the volume of a solid of revolution using basic methods.
	\end{itemize}
	\vspace{-4mm}
}

\vspace{5mm}

\subsection*{Review of Integration Techniques}

So far, we know a few different techniques for evaluating integrals\dots
\begin{itemize}
	\item Basic integration rules (e.g., recognizing antiderivatives, applying ``reverse power rule'')
	\item Using algebra to simplify the integrand
	\item The substitution method (u-substitution)
	\item Using trig identities to simplify the integrand
\end{itemize}

Some common \textbf{trig identities} that can be used to simplify integrals are included below.\\

\textbf{Pythagorean Identities}
$$\cos^2x+\sin^2x=1\hspace{30mm}1+\tan^2x=\sec^2x$$

\textbf{Half-Angle Formulas}
$$\cos^2x=\frac{1+\cos(2x)}{2}\hspace{30mm}\sin^2x=\frac{1-\cos(2x)}{2}$$

In this chapter, we'll be building on these techniques and introducing some new ones!

\newpage

\Example Evaluate $\disp\int e^x\left(1+e^x\right)^9\left(1-e^x\right)\,dx$.

\vfill

\Example Evaluate $\disp\int\frac{x+2}{x^2+4}\,dx$.

\vfill

\newpage 

\Example Evaluate $\disp\int\frac{dx}{\sec x-1}$.

\newpage

\Example Evaluate $\disp\int\frac{x^2+2}{x-1}\,dx$.

\newpage

\Example Evaluate $\disp\int\frac{dx}{\sqrt{27-6x-x^2}}$.

\newpage

\Example Evaluate $\disp\int_{-1}^0\frac{x}{x^2+2x+2}\,dx$.
\end{document}