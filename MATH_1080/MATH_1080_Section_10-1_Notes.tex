\documentclass[12pt]{article}
%%% DOCUMENT FORMATTING %%%
\usepackage[margin=1in]{geometry}
\usepackage{enumitem}
\setlength{\parindent}{0pt}
\newcommand{\disp}{\displaystyle}
\usepackage{multicol}

%%% HEADER %%%
\usepackage{fancyhdr}
\pagestyle{fancy}
\fancyhf{}
\lhead{MATH 1080}
\rhead{Vagnozzi}
\cfoot{\thepage}

%%% MATH NOTATION & SYMBOLS %%%
\usepackage{amssymb}
\usepackage{amsmath}
\newcommand{\R}{\mathbb{R}}
\newcommand{\N}{\mathbb{N}}
\newcommand{\Z}{\mathbb{Z}}
\newcommand{\lp}{\left(}
\newcommand{\rp}{\right)}
\newcommand{\ls}{\left[}
\newcommand{\rs}{\right]}
\newcommand{\lb}{\left\{}
\newcommand{\rb}{\right\}}
\newcommand{\arccot}{\text{arccot}}
\newcommand{\arccsc}{\text{arccsc}}
\newcommand{\arcsec}{\text{arcsec}}
\DeclareSymbolFont{matha}{OML}{txmi}{m}{it}% txfonts
\DeclareMathSymbol{\varv}{\mathord}{matha}{118} 

%%% TABLES %%%
\usepackage{colortbl}

%%% GRAPHS %%%
\usepackage{tikz}
\usepackage{pgfplots}
\pgfplotsset{compat=1.15}
\usepgfplotslibrary{fillbetween}
\usetikzlibrary{angles,quotes}

%%% ENVIRONMENTS %%%
\newcommand{\Example}{\paragraph{\Writinghand \hspace{0.1mm} Example.}}
\newcommand{\Examples}{\paragraph{\Writinghand \hspace{0.1mm} Examples.}}
\newcommand{\ExampleCont}{\paragraph{\Writinghand \hspace{0.1mm} Example (continued).}}
\newcommand{\ExamplesCont}{\paragraph{\Writinghand \hspace{0.1mm} Examples (continued).}}

\newcommand{\boxenv}[2]{
	\fbox{
	\begin{minipage}{0.97\textwidth}
	\vspace{2mm}	
	\paragraph{#1} #2
	\vspace{2mm}
	\end{minipage}
	}}

%%% FUN THINGS %%%
\newcommand*\tc[1]{\tikz[baseline=(char.base)]{
            \node[shape=circle,draw,inner sep=2pt] (char) {#1};}}
\usepackage{marvosym}

%%% MISC %%%
\usepackage{hyperref}


\setcounter{page}{93}

\begin{document}
\section*{10.1: An Overview of Sequences and Series}

\boxenv{Learning Objectives.}{Upon successful completion of Section 10.1, you will be able to\dots
		
	\begin{itemize}[leftmargin=6mm]
		\item Answer conceptual questions involving sequences, series, and recurrence relations.
		\item Find terms of sequences.
		\item Write recurrence relations and/or explicit formulas for the terms of sequences.
		\item Determine limits of sequences if the sequences converge.
		\item Solve applications involving sequences.
		\item Determine limits of sequences of partial sums if the sequences converge.
	\end{itemize}
	\vspace{-4mm}
}

\vspace{5mm}

\subsection*{Motivation} Chapter~10 is about \textit{infinite sequences and series}. One of the important applications of infinite series is Taylor series, which allow us to represent functions as sums of infinite series. Taylor series can allow us to\dots
\begin{itemize}
	\item work with exponential, trigonometric, and logarithmic functions using just addition, subtraction, multiplication, and division
	\item solve differential equations
	\item do approximations in physics and chemistry
	\item represent $\pi$, $e$, and repeating decimals as sums of infinite series
\end{itemize}

\vspace{2mm}

We'll first need to understand the difference between a \textbf{sequence} and a \textbf{series}.\\

A \textbf{sequence} is an infinite ordered list of numbers.

\vfill

A \textbf{series} is an infinite sum of numbers.

\vfill

What does it mean for a sequence or series to \textbf{converge}?
\begin{itemize}
	\item For sequences, we will use the same tools as for limits of functions at infinity.
	\item For series, we will need some new techniques!
\end{itemize}

\newpage

\subsection*{Sequences}

\boxenv{Definition.}{A \textbf{sequence} is an ordered list of numbers of the form
$$\left\{a_1,a_2,a_3,\dots,a_n,\dots\right\},$$

where each number in the sequence is called a \textbf{term}.

\begin{itemize}[leftmargin=6mm]
	\item A sequence may be generated by a \textbf{recurrence relation} of the form $a_{n+1}=f\left(a_n\right)$, for $n=1,2,3,\dots$, where $a_1$ is given.
	\item A sequence may also be defined with an \textbf{explicit formula} of the form $a_n=f(n)$, for $n=1,2,3,\dots$.
\end{itemize}

\vspace{-3mm}
}

\Examples Consider the following examples of sequences.\\

\hspace{5mm} $\left\{1, 4, 7, 10, 13, \dots\right\}$

\vfill

\hspace{5mm} $\disp\left\{\frac{1}{2^n}\right\}_{n=1}^\infty$

\vfill

\hspace{5mm} $a_n=(-1)^n$, $n=1,2,3,\dots$

\vfill

\hspace{5mm} $a_{n+1}=3a_n-12$; $a_1=10$

\vfill

\newpage

\boxenv{Limit of a Sequence.}{If the terms of a sequence $\left\{a_n\right\}$ approach a unique number $L$ as $n$ increases --- that is, if $a_n$ can be made arbitrarily close to $L$ by taking a sufficiently large $n$ --- then we say that $\disp\lim_{n\to\infty}a_n=L$ exists and the sequence \textbf{converges} to $L$. \\

If the terms of the sequence do not approach a single number as $n$ increases, the sequence has no limit and \textbf{diverges}.}

\vspace{20mm}

\Example For each of the following, write the first four terms of the sequence. If the sequence appears to converge, make a conjecture about its limit. If the sequence diverges, explain why.\\

(a) $a_n=1-10^{-n}$; $n=1,2,3,\dots$ \hspace{25mm} (b) $a_n=3+\cos\left(\pi n\right)$, $n=1,2,3,\dots$

\vspace{45mm}

\Example Jack took a 200-mg dose of a painkiller at midnight. every hour, 5\% of the drug is washed out of his bloodstream. Let $d_n$ be the amount of the drug in Jack's blood $n$ hours after the drug was taken, where $d_0=200$ mg. Write out the first 5 terms, and find an explicit formula for the $n^{th}$ term, find a recurrence relation that generates the sequence, and estimate the limit of the sequence.

\newpage

\subsection*{Infinite Series}

\boxenv{Definition.}{An \textbf{infinite series} is a sum of infinitely many terms and can be represented as follows.

$$\disp\sum_{n=1}^\infty a_n=a_1+a_2+a_3+\dots$$}

\vspace{5mm}

\Examples The following are examples of infinite series.\\

\hspace{5mm} $1+2+3+4+5+\dots$

\vfill

\hspace{5mm} $0.\overline{3}$

\vfill

\hspace{5mm} $\disp\frac{1}{2}+\frac{1}{4}+\frac{1}{8}+\frac{1}{16}+\dots$

\vfill

\newpage

A \textit{sequence} $\left\{a_n\right\}=\left\{a_1,a_2,a_3,\dots\right\}$ converges if $\disp\lim_{n\to\infty}a_n=L$, where $L\in\R$, and diverges otherwise.\\

What does it mean for a \textit{series} to converge?\\

\boxenv{Series Convergence/Divergence.}{Given a series $\disp\sum_{i=1}^{\infty}a_i=a_1+a_2+a_3+\cdots $, we say that the series \textbf{converges} if its sequence of partial sums $s_n=\disp\sum_{i=1}^n a_i=a_1+a_2+\cdots+a_n$ converges, meaning

\vspace{15mm}

We say that the series \textbf{diverges} otherwise.}

\Examples Use the definition of convergence/divergence of a series to determine if the series converges or diverges. If it converges, find its sum.\\

\hspace{5mm} $\disp\sum_{i=1}^\infty i$

\vfill

\hspace{5mm} $\disp\sum_{i=1}^\infty\frac{1}{2^i}$

\vfill




\end{document}