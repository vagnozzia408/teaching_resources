\documentclass[12pt]{article}
%%% DOCUMENT FORMATTING %%%
\usepackage[margin=1in]{geometry}
\usepackage{enumitem}
\setlength{\parindent}{0pt}
\newcommand{\disp}{\displaystyle}
\usepackage{multicol}

%%% HEADER %%%
\usepackage{fancyhdr}
\pagestyle{fancy}
\fancyhf{}
\lhead{MATH 1080}
\rhead{Vagnozzi}
\cfoot{\thepage}

%%% MATH NOTATION & SYMBOLS %%%
\usepackage{amssymb}
\usepackage{amsmath}
\newcommand{\R}{\mathbb{R}}
\newcommand{\N}{\mathbb{N}}
\newcommand{\Z}{\mathbb{Z}}
\newcommand{\lp}{\left(}
\newcommand{\rp}{\right)}
\newcommand{\ls}{\left[}
\newcommand{\rs}{\right]}
\newcommand{\lb}{\left\{}
\newcommand{\rb}{\right\}}
\newcommand{\arccot}{\text{arccot}}
\newcommand{\arccsc}{\text{arccsc}}
\newcommand{\arcsec}{\text{arcsec}}
\DeclareSymbolFont{matha}{OML}{txmi}{m}{it}% txfonts
\DeclareMathSymbol{\varv}{\mathord}{matha}{118} 

%%% TABLES %%%
\usepackage{colortbl}

%%% GRAPHS %%%
\usepackage{tikz}
\usepackage{pgfplots}
\pgfplotsset{compat=1.15}
\usepgfplotslibrary{fillbetween}
\usetikzlibrary{angles,quotes}

%%% ENVIRONMENTS %%%
\newcommand{\Example}{\paragraph{\Writinghand \hspace{0.1mm} Example.}}
\newcommand{\Examples}{\paragraph{\Writinghand \hspace{0.1mm} Examples.}}
\newcommand{\ExampleCont}{\paragraph{\Writinghand \hspace{0.1mm} Example (continued).}}
\newcommand{\ExamplesCont}{\paragraph{\Writinghand \hspace{0.1mm} Examples (continued).}}

\newcommand{\boxenv}[2]{
	\fbox{
	\begin{minipage}{0.97\textwidth}
	\vspace{2mm}	
	\paragraph{#1} #2
	\vspace{2mm}
	\end{minipage}
	}}

%%% FUN THINGS %%%
\newcommand*\tc[1]{\tikz[baseline=(char.base)]{
            \node[shape=circle,draw,inner sep=2pt] (char) {#1};}}
\usepackage{marvosym}

%%% MISC %%%
\usepackage{hyperref}


\usepackage{bm}

\setcounter{page}{55}

\begin{document}
\section*{8.3: Trigonometric Integrals}

\boxenv{Learning Objectives.}{Upon successful completion of Section 8.3, you will be able to\dots
		
	\begin{itemize}[leftmargin=6mm]
		\item Answer conceptual questions involving trigonometric integrals.
		\item Evaluate indefinite integrals involving powers of sine and cosine.
		\item Evaluate definite integrals involving powers of sine of cosine.
		\item Evaluate indefinite integrals involving powers of tangent and secant or cotangent and cosecant.
		\item Evaluate definite integrals involving powers of tangent and secant or cotangent and cosecant.
		\item Use trigonometric integrals to find areas, volumes, or arc length.
	\end{itemize}
	\vspace{-4mm}
}

\vspace{5mm}

\subsection*{Trig Integrals}
Recall the trig antiderivative rules that we know for sine and cosine.\\

$$\disp\int\sin x\,dx=\hspace{60mm}\int\cos x\,dx=$$

\vspace{5mm}

We can also identify trig antiderivative rules for the tangent and cotangent functions.\\

$\disp\int\tan x\,dx=$

\vfill

$\disp\int\cot x\,dx=$

\vspace{10mm}

\newpage

What about our other two trig functions?\\

$\disp\int\sec x\,dx=$

\vspace{50mm}

$\disp\int\csc x\,dx=$

\vspace{10mm}

\subsection*{A Review of Common Trig Identities}

\vspace{3mm}

\subsubsection*{Pythagorean Identities}

\hspace{5mm}$\cos^2x+\sin^2x=1$\\

\hspace{5mm}$1+\tan^2x = \sec^2 x$ \hfill (found by dividing each term of the first identity by $\cos^2x$)\\

\hspace{5mm}$\cot^2x+1=\csc^2x$\hfill (found by dividing each term of the first identity by $\sin^2x$)

\vspace{3mm}

\subsubsection*{Half-Angle Identities}

\vspace{2mm}

$$\sin^2x=\frac{1-\cos 2x}{2}\hspace{20mm} \cos^2x=\frac{1+\cos 2x}{2}$$

\vspace{3mm}

\subsubsection*{Double-Angle Identities}

\hspace{5mm}$\cos 2x=2\cos^2x-1=1-2\sin^2x=\cos^2x-\sin^2x$\\

\hspace{5mm}$\sin 2x=2\sin x \cos x$

\vfill

Let's learn some other techniques for integrating trig functions.

\newpage

\subsection*{Tips for Integrating $\int\sin^m x\,dx$, $\int\cos^nx\,dx$, and $\int\sin^m x\cos^n x\,dx$}

If the power of sine $m$ is \textbf{odd}, separate out a factor of $\sin x$ and write the remaining $\sin^{m-1}x$ in terms of cosines using the Pythagorean identity. Then use $u$-substitution with $u=\cos x$.

\Example $\disp\int\sin^5 x\,dx$

\vfill

If the power of cosine $n$ is \textbf{odd}, separate out a factor of $\cos x$ and write the remaining $\cos^{n-1}x$ in terms of sines using the Pythagorean identity. Then use $u$-substitution with $u=\sin x$.

\Example $\disp\int\frac{\cos^5 x}{\sin^{3/2}x}\,dx$

\vfill

\newpage

If $m$ and $n$ are both \textbf{even}, use the half-angle identities.

\Example $\disp\int\sin^2x\cos^4x\,dx$

\newpage

\subsection*{Tips for Integrating $\int\tan^m x\sec^nx\,dx$}
%
If the power of tangent $m$ is \textbf{odd}, separate out a factor of $\sec x\tan x$ and write the remaining $\tan^{m-1}x$ in terms of secants using the Pythagorean identity. Then use $u$-substitution with $u=\sec x$.\\

If the power of secant $n$ is \textbf{even}, separate out a factor of $\sec^2x$ and write the remaining $\sec^{n-2}x$ in terms of tangents using the Pythagorean identity. Then use $u$-substitution with $u=\tan x$. \\

\Example $\disp\int\tan^5 x\sec^4x\,dx$

\newpage

If the power of tangent $m$ is \textbf{even} and the power of secant $n$ is \textbf{odd}, write $\tan^m x$ in terms of secants using the Pythagorean identity. This produces a polynomial in $\sec x$. 

\Example $\disp\int\tan^2x\sec x \,dx$

\vspace{40mm}

\subsection*{Tips for Integrating $\int\tan^m x$ and $\int\sec^n x\,dx$}

If the power of secant $n$ is \textbf{odd}, separate out a factor of $\sec^2 x$ and use integration by parts with $u=\sec^{n-2}x$ and $d\varv=\sec^2x\,dx$.

\ExampleCont Continue the example at the top of this page!


\vfill

If the power of secant $n$ is \textbf{even}, separate out a factor of $\sec^2x$ and write the remaining $\sec^{n-2}x$ in terms of tangents using the Pythagorean identity. Then use $u$-substitution with $u=\tan x$. 

\newpage

If the power of tangent $m$ is even or odd, separate out a factor of $\tan^2x$ and write it in terms of secants using the Pythagorean identity. Expand the integral to write it as a difference of integrals. Use $u$-substitution with $u=\tan x$ on the first integral and repeat the process if necessary on the second integral.

\Example $\disp\int\tan^5 x\,dx$

\vfill

\subsection*{Additional Tips for Integrating Trig Functions}

Integrating $\int\cot^m x\,dx$, $\int \csc^n x\,dx$, and $\int\cot^m x\csc^n x\,dx$ will use the same rules as those for powers and products of tangents and secants.\\

If the trig function you wish to integrate does not fall under the tips from this section, use the definitions of the trig functions to convert the integrand in terms of sines and cosines.
\end{document}