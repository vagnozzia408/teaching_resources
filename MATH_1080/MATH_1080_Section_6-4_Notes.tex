\documentclass[12pt]{article}
%%% DOCUMENT FORMATTING %%%
\usepackage[margin=1in]{geometry}
\usepackage{enumitem}
\setlength{\parindent}{0pt}
\newcommand{\disp}{\displaystyle}
\usepackage{multicol}

%%% HEADER %%%
\usepackage{fancyhdr}
\pagestyle{fancy}
\fancyhf{}
\lhead{MATH 1080}
\rhead{Vagnozzi}
\cfoot{\thepage}

%%% MATH NOTATION & SYMBOLS %%%
\usepackage{amssymb}
\usepackage{amsmath}
\newcommand{\R}{\mathbb{R}}
\newcommand{\N}{\mathbb{N}}
\newcommand{\Z}{\mathbb{Z}}
\newcommand{\lp}{\left(}
\newcommand{\rp}{\right)}
\newcommand{\ls}{\left[}
\newcommand{\rs}{\right]}
\newcommand{\lb}{\left\{}
\newcommand{\rb}{\right\}}
\newcommand{\arccot}{\text{arccot}}
\newcommand{\arccsc}{\text{arccsc}}
\newcommand{\arcsec}{\text{arcsec}}
\DeclareSymbolFont{matha}{OML}{txmi}{m}{it}% txfonts
\DeclareMathSymbol{\varv}{\mathord}{matha}{118} 

%%% TABLES %%%
\usepackage{colortbl}

%%% GRAPHS %%%
\usepackage{tikz}
\usepackage{pgfplots}
\pgfplotsset{compat=1.15}
\usepgfplotslibrary{fillbetween}
\usetikzlibrary{angles,quotes}

%%% ENVIRONMENTS %%%
\newcommand{\Example}{\paragraph{\Writinghand \hspace{0.1mm} Example.}}
\newcommand{\Examples}{\paragraph{\Writinghand \hspace{0.1mm} Examples.}}
\newcommand{\ExampleCont}{\paragraph{\Writinghand \hspace{0.1mm} Example (continued).}}
\newcommand{\ExamplesCont}{\paragraph{\Writinghand \hspace{0.1mm} Examples (continued).}}

\newcommand{\boxenv}[2]{
	\fbox{
	\begin{minipage}{0.97\textwidth}
	\vspace{2mm}	
	\paragraph{#1} #2
	\vspace{2mm}
	\end{minipage}
	}}

%%% FUN THINGS %%%
\newcommand*\tc[1]{\tikz[baseline=(char.base)]{
            \node[shape=circle,draw,inner sep=2pt] (char) {#1};}}
\usepackage{marvosym}

%%% MISC %%%
\usepackage{hyperref}


\setcounter{page}{23}

\begin{document}
\section*{6.4: Volume by Shells}

\boxenv{Learning Objectives.}{Upon successful completion of Section 6.4, you will be able to\dots
		
	\begin{itemize}[leftmargin=6mm]
		\item Answer conceptual questions involving the Shell Method.
		\item Use the shell method to find the volume of the solid of revolution about the $y$-axis.
		\item Use the shell method to find the volume of the solid of revolution about the $x$-axis.
		\item Use the shell method to find the volume of the solid of revolution about other horizontal and vertical lines (other than the $x$-axis and $y$-axis).
		\item Use both the shell method and washer method to find the volume of the solid of revolution about an indicated axis or line.
		\item Find the volume of a solid of revolution using any method.
		\item Solve applications involving the shell method.
	\end{itemize}
	\vspace{-4mm}
}

\vspace{5mm}

\subsection*{Volume by Cylindrical Shells}

\begin{minipage}{0.8\linewidth}
\paragraph{Motivation.} Suppose we want to find the volume of a solid generated when a region is rotated about the $y$-axis. If using the Disk or Washer Method, we would need to make slices \textit{perpendicular} to the axis of rotation (the $y$-axis) and integrate with respect to $y$. To do so, we would need to write $y=f(x)$ in terms of $y$, but this may be difficult for some functions.
\end{minipage}%

\subsubsection*{Shell Method}

Instead, we can create slices that are \textbf{parallel} to the axis of rotation. Rather than generating disk or washer cross sections, this will generate a series of \textbf{cylindrical shells}.

\newpage

\boxenv{Shell Method about a Vertical Line.} {Let $f$ be a continuous function $f(x)\geq 0$ on the interval $[a,b]$. If the region $R$ bounded by the graph of $f$, the $x$-axis, and the lines $x=a$ and $x=b$ is revolved about a vertical line (such as the $y$-axis), the volume of the resulting solid of revolution is

$$V=\int_a^b2\pi r(x)h(x)\,dx,$$

where $r(x)$ is the radius of the cylindrical shell and $h(x)$ is the height of the shell.}

\vspace{5mm}

\textbf{Note:} If a region between $y=c$ and $y=d$ is instead rotated about a horizontal line (such as the $x$-axis), the Shell Method can be used by creating horizontal slices and integrating with respect to $y$.
$$V=\int_c^d 2\pi r(y)h(y)\,dy$$

\Example Let $R$ be the region bounded by $y=e^{-x^2}$, $y=0$, $x=0$, and $x=1$. Use the Shell Method to find the volume of the solid generated when $R$ is rotated about the $y$-axis.

\vspace{5mm}

\begin{tikzpicture}[scale=0.9]
	\begin{axis}[grid=none,
		xtick=\empty,
		ytick=\empty,
       	axis x line=middle,
       	xmax=2.1, xmin=-2.1,
       	axis y line=center,
       	ymax=1.2, ymin=-.5,
       	xlabel=$x$,ylabel=$y$, axis on top,
       	axis line style = <->
    ]
		\addplot[name path=f,smooth,domain=-2:2,color=blue,samples=100,thick,<->] {e^(-x^2)};
   \end{axis}
\end{tikzpicture}

\newpage

\Example Let $R$ be the region bounded by $y=x^3$, $y=8$, and $x=0$. Use the Shell Method to find the volume of the solid generated when $R$ is rotated about the $x$-axis.

\newpage

\Example Let $R$ be the region bounded by $y=x^3$, $x=2$, and $y=0$. Use the Shell Method to set up the integral(s) needed to find the volume of the solid generated when $R$ is rotated about\dots

\begin{itemize}
\item[(a)] the line $x=3$.

\vspace{95mm}

\item[(b)] the line $y=-1$.
\end{itemize}

\end{document}