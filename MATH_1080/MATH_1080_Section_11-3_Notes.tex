\documentclass[12pt]{article}
%%% DOCUMENT FORMATTING %%%
\usepackage[margin=1in]{geometry}
\usepackage{enumitem}
\setlength{\parindent}{0pt}
\newcommand{\disp}{\displaystyle}

%%% HEADER %%%
\usepackage{fancyhdr}
\pagestyle{fancy}
\fancyhf{}
\lhead{MATH 1080}
\rhead{Vagnozzi}
\cfoot{\thepage}

%%% MATH NOTATION & SYMBOLS %%%
\usepackage{amssymb}
\usepackage{amsmath}
\newcommand{\R}{\mathbb{R}}
\newcommand{\N}{\mathbb{N}}
\newcommand{\Z}{\mathbb{Z}}
\newcommand{\lp}{\left(}
\newcommand{\rp}{\right)}
\newcommand{\ls}{\left[}
\newcommand{\rs}{\right]}
\newcommand{\lb}{\left\{}
\newcommand{\rb}{\right\}}
\newcommand{\arccot}{\text{arccot}}
\newcommand{\arccsc}{\text{arccsc}}
\newcommand{\arcsec}{\text{arcsec}} 

%%% TABLES %%%
\usepackage{colortbl}

%%% GRAPHS %%%
\usepackage{tikz}
\usepackage{pgfplots}
\pgfplotsset{compat=1.15}
\usepgfplotslibrary{fillbetween}
\usetikzlibrary{angles,quotes}

%%% ENVIRONMENTS %%%
\newcommand{\Example}{\paragraph{\Writinghand \hspace{0.1mm} Example.}}
\newcommand{\ExampleCont}{\paragraph{\Writinghand \hspace{0.1mm} Example (continued).}}
\newcommand{\boxenv}[2]{
	\fbox{
	\begin{minipage}{0.97\textwidth}
	\vspace{2mm}	
	\paragraph{#1} #2
	\vspace{2mm}
	\end{minipage}
	}}

%%% FUN THINGS %%%
\newcommand*\tc[1]{\tikz[baseline=(char.base)]{
            \node[shape=circle,draw,inner sep=2pt] (char) {#1};}}
\usepackage{marvosym}

%%% MISC %%%
\usepackage{hyperref}


\setcounter{page}{152}

\begin{document}
\section*{11.3: Taylor Series}

\boxenv{Learning Objectives.}{Upon successful completion of Section 11.3, you will be able to\dots
		
	\begin{itemize}[leftmargin=6mm]
		\item Answer conceptual questions involving Taylor series.
		\item Find the Taylor series and interval of convergence for functions centered at $a$.
		\item Find the Taylor series for a given function centered at $a$.
		\item Manipulate a given Taylor series to find the Taylor series for a function centered at $0$.
		\item Find remainders for Taylor series and show the remainder for certain Taylor series goes to $0$ as $n$ goes to infinity for all $x$ in the interval of convergence.
	\end{itemize}
	\vspace{-4mm}
}

\vspace{5mm}

\subsection*{Taylor and Maclaurin Series}

In the previous sections, we learned about power series and $n$th-order Taylor polynomials. We'll now combine these two ideas by formally defining \textbf{Taylor series}, which are power series where the coefficients have a specific form.

\vspace{3mm}

\boxenv{Definition.}{Suppose the function $f$ has derivatives of all orders on an interval containing the point $a$. The \textbf{Taylor series for \textit{f} centered at \textit{a}} is

\vspace{15mm}
}

\vspace{5mm}

\boxenv{Definition.}{A \textbf{Maclaurin series} for a function $f$ is the Taylor series for $f$ centered at the point $a=0$.

\vspace{15mm}
}

\paragraph{Notes about Taylor Series.}
\begin{itemize}
\item If $f$ has a power series representation at $a$, then that power series must be the Taylor series of $f$ centered at $a$.
\item It is possible for a function to not equal its Taylor series.
\item For Taylor series to be useful, we need to know\dots
\begin{itemize}
	\item The values of $x$ for which the Taylor series converges (the interval of convergence).
	\item The values of $x$ or which the Taylor series for $f$ \underline{equals} $f$.
\end{itemize}
\end{itemize}

\newpage


\subsection*{Finding the Interval of Convergence for a Taylor Series}
\Example Find the Maclaurin series for $f(x)=e^x$. Also find the interval of convergence.

\vfill

\Example Find the Taylor series for $f(x)=\cos x$ centered at $a=\pi$. Also find the interval of convergence.

\vfill

\newpage

\subsection*{Determining When a Function Equals Its Taylor Series}

Recall from Section 11.1 that the \textbf{remainder} when using a Taylor polynomial $p_n$ to approximate $f$ is
$$R_n(x)=f(x)-p_n(x).$$

We can use the remainder to to determine when a function $f$ is \underline{equal to} its Taylor series representation.

\vspace{4mm}

\boxenv{Theorem: Convergence of Taylor Series.}{Let $f$ have derivatives of all orders on an open interval containing the point $a$. The Taylor series for $f$ centered at $a$ \textbf{converges to $f$} for all $x$ in the interval if and only if

\vspace{15mm}

where $R_n(x)=\dfrac{f^{(n+1)}(c)}{(n+1)!}(x-a)^{n+1}$ is the remainder at $x$, with $c$ between $x$ and $a$.
}

\vspace{5mm}

\boxenv{Taylor's Inequality.}{If $\big|f^{(n+1)}(x)\big|\leq M$ for $|x-a|\leq d$, then

\vspace{2mm}

$$\big|R_n(x)\big|\leq\dfrac{M|x-a|^{n+1}}{(n+1)!}\text{ for }|x-a|<d.$$

}

\Example Prove that $e^x$ is equal to its Maclaurin series for all $x$.

\newpage

\Example Prove that $\cos x$ is equal to its Taylor series centered at $a=\pi$ for all $x$.

\vfill

\subsection*{Manipulating Taylor Series}

We can use the tools from the previous section to manipulate Taylor series like any other power series.

\Example Use the Taylor series $\arctan(x)=\disp\sum_{k=0}^\infty\dfrac{(-1)^k x^{2k+1}}{2k+1}$, for $|x|\leq 1$, to find the first four nonzero terms of the Taylor series for the function $x\arctan\left(x^2\right)$ centered at $0$.

\vfill

\end{document}