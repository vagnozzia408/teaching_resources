\documentclass[12pt]{article}
%%% DOCUMENT FORMATTING %%%
\usepackage[margin=1in]{geometry}
\usepackage{enumitem}
\setlength{\parindent}{0pt}
\newcommand{\disp}{\displaystyle}

%%% HEADER %%%
\usepackage{fancyhdr}
\pagestyle{fancy}
\fancyhf{}
\lhead{MATH 1080}
\rhead{Vagnozzi}
\cfoot{\thepage}

%%% MATH NOTATION & SYMBOLS %%%
\usepackage{amssymb}
\usepackage{amsmath}
\newcommand{\R}{\mathbb{R}}
\newcommand{\N}{\mathbb{N}}
\newcommand{\Z}{\mathbb{Z}}
\newcommand{\lp}{\left(}
\newcommand{\rp}{\right)}
\newcommand{\ls}{\left[}
\newcommand{\rs}{\right]}
\newcommand{\lb}{\left\{}
\newcommand{\rb}{\right\}}
\newcommand{\arccot}{\text{arccot}}
\newcommand{\arccsc}{\text{arccsc}}
\newcommand{\arcsec}{\text{arcsec}} 

%%% TABLES %%%
\usepackage{colortbl}

%%% GRAPHS %%%
\usepackage{tikz}
\usepackage{pgfplots}
\pgfplotsset{compat=1.15}
\usepgfplotslibrary{fillbetween}
\usetikzlibrary{angles,quotes}

%%% ENVIRONMENTS %%%
\newcommand{\Example}{\paragraph{\Writinghand \hspace{0.1mm} Example.}}
\newcommand{\ExampleCont}{\paragraph{\Writinghand \hspace{0.1mm} Example (continued).}}
\newcommand{\boxenv}[2]{
	\fbox{
	\begin{minipage}{0.97\textwidth}
	\vspace{2mm}	
	\paragraph{#1} #2
	\vspace{2mm}
	\end{minipage}
	}}

%%% FUN THINGS %%%
\newcommand*\tc[1]{\tikz[baseline=(char.base)]{
            \node[shape=circle,draw,inner sep=2pt] (char) {#1};}}
\usepackage{marvosym}

%%% MISC %%%
\usepackage{hyperref}


\setcounter{page}{108}

\begin{document}
\section*{10.4: Divergence, p-Series, and Integral Tests}

\boxenv{Learning Objectives.}{Upon successful completion of Section 10.4, you will be able to\dots
		
	\begin{itemize}[leftmargin=6mm]
		\item Answer conceptual questions involving Divergence or p-series Tests.
		\item Use the Divergence Test to determine whether series diverge.
		\item Use Divergence or p-series Tests to determine the convergence or divergence of a series.
		\item Determine if a series converges or diverges using the properties and tests introduced so far.
		\item Use the Integral Test to determine whether series converge or diverge.
		\item Use Divergence, Integral, or p-series Tests to determine the convergence or divergence of a series.
		\item Estimate the value of a series using Theorem 10.13.
		\item Determine if a series converges or diverges using the properties and tests introduced so far.
	\end{itemize}
	\vspace{-4mm}
}

\vspace{5mm}

\subsection*{The Divergence Test}

\boxenv{Divergence Test.}{Consider the series $\disp\sum_{n=1}^\infty a_n$. 

\vspace{25mm}}

\newpage

\Example Determine if the series $\disp\sum_{n=1}^\infty\arctan(n)$ converges or diverges.

\vfill 

\Example Determine if the series $\disp\sum_{n=1}^\infty\ln\left(\dfrac{3n^3+n}{n^3+4}\right)$ converges or diverges.

\vfill

\Example Consider the series $\disp\sum_{n=1}^\infty \dfrac{1}{n}$, called the \textbf{harmonic series}. How can we use what we know about $\disp\int_1^\infty\frac{1}{x}\,dx$ to show that $\disp\sum_{n=1}^\infty\dfrac{1}{n}$ diverges?

\vfill

\vfill

\newpage

\subsection*{The Integral Test}

\boxenv{Integral Test.}{Suppose for some positive integer $N\geq 1$, the function $f(x)$ is

\vspace{2mm}

\begin{itemize}
	\item[(1)] continuous for $x \geq N$,
	\item[(2)] positive for $x\geq N$, 
	\item[(3)] decreasing for $x\geq N$,
\end{itemize}

and let $a_k=f(k)$ for $k\geq N$, where $k$ is an integer. Then

\begin{align*}
\disp\sum_{k=N}^\infty a_k\text{ converges } & \text{ if } \int_N^\infty f(x)\,dx\text{ converges, and}\\
\disp\sum_{k=N}^\infty a_k\text{ diverges } & \text{ if }\int_N^\infty f(x)\,dx\text{ diverges.}
\end{align*}

\vspace{-2mm}}

\vspace{5mm}

\textbf{\underline{Note 1}:} The value of $\disp\int_{N}^\infty f(x)\,dx$ is \textbf{NOT} (in general) the value of the series sum.

\vspace{3mm}

\textbf{\underline{Note 2}:} If $\disp\sum_{k=N}^\infty a_k$ converges, then $\disp\sum_{k=1}^\infty a_k$ also converges, because convergence is not affected by a finite number of terms. (A similar idea holds for divergence.)

\newpage

\Example $\disp\sum_{n=1}^\infty \frac{1}{(2n+1)^3}$

\newpage

\Example $\disp\sum_{n=2}^\infty\frac{1}{n\ln n}$

\newpage

\subsection*{The p-Series Test}
\boxenv{p-Series Test.}{The $p$-series $\disp\sum_{n=1}^\infty\frac{1}{n^p}$ converges when $p>1$ and diverges when $p\leq 1$.}

\Example Determine if the following series converge or diverge.

\begin{multicols}{2}
\begin{itemize}
	\item[\tc{1}] $\disp\sum_{n=1}^\infty\frac{1}{n^4}$
	\item[\tc{2}] $\disp\sum_{n=1}^\infty\frac{1}{4^n}$
\end{itemize}
\end{multicols}

\vspace{30mm}

\subsection*{Estimating the Sum of a Series}

Suppose we know $\disp\sum_{n=1}^\infty a_n$ converges by the integral test (so $f(x)$, where $f(n)=a_n$, is positive, continuous, and decreasing for $x\geq 1$), and we want to find an approximation to the sum.

\vfill

\boxenv{Definition.}{The \textbf{remainder} $R_n$ is the error made when estimating a sum $S$ by the $n^{th}$ partial sum $S_n$.}

\vfill


\newpage


\subsection*{The Integral Estimation Theorem}

\vspace{50mm}

\Example Consider the series $\disp\sum_{n=1}^\infty\frac{1}{n^4}$.

(a) Find the partial sum $S_5$ of the series.
	
	\vspace{30mm}
	
(b) Estimate the error in using $S_5$ as an approximation to the sum of the series.
	
\vfill

(c) Use $n=5$ and $S_n+\disp\int_{n+1}^\infty f(x)\,dx\leq S\leq S_n+\int_n^\infty f(x)\,dx$ to improve the estimate of $S$. 

\vfill

\newpage

\Example Estimate $\disp\sum_{n=1}^\infty\frac{1}{(2n+1)^6}$ so that the error in estimation is less than $\disp\frac{1}{10^6}$.

\end{document}