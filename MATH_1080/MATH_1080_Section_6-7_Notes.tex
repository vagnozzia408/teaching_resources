\documentclass[12pt]{article}
%%% DOCUMENT FORMATTING %%%
\usepackage[margin=1in]{geometry}
\usepackage{enumitem}
\setlength{\parindent}{0pt}
\newcommand{\disp}{\displaystyle}
\usepackage{multicol}

%%% HEADER %%%
\usepackage{fancyhdr}
\pagestyle{fancy}
\fancyhf{}
\lhead{MATH 1080}
\rhead{Vagnozzi}
\cfoot{\thepage}

%%% MATH NOTATION & SYMBOLS %%%
\usepackage{amssymb}
\usepackage{amsmath}
\newcommand{\R}{\mathbb{R}}
\newcommand{\N}{\mathbb{N}}
\newcommand{\Z}{\mathbb{Z}}
\newcommand{\lp}{\left(}
\newcommand{\rp}{\right)}
\newcommand{\ls}{\left[}
\newcommand{\rs}{\right]}
\newcommand{\lb}{\left\{}
\newcommand{\rb}{\right\}}
\newcommand{\arccot}{\text{arccot}}
\newcommand{\arccsc}{\text{arccsc}}
\newcommand{\arcsec}{\text{arcsec}}
\DeclareSymbolFont{matha}{OML}{txmi}{m}{it}% txfonts
\DeclareMathSymbol{\varv}{\mathord}{matha}{118} 

%%% TABLES %%%
\usepackage{colortbl}

%%% GRAPHS %%%
\usepackage{tikz}
\usepackage{pgfplots}
\pgfplotsset{compat=1.15}
\usepgfplotslibrary{fillbetween}
\usetikzlibrary{angles,quotes}

%%% ENVIRONMENTS %%%
\newcommand{\Example}{\paragraph{\Writinghand \hspace{0.1mm} Example.}}
\newcommand{\Examples}{\paragraph{\Writinghand \hspace{0.1mm} Examples.}}
\newcommand{\ExampleCont}{\paragraph{\Writinghand \hspace{0.1mm} Example (continued).}}
\newcommand{\ExamplesCont}{\paragraph{\Writinghand \hspace{0.1mm} Examples (continued).}}

\newcommand{\boxenv}[2]{
	\fbox{
	\begin{minipage}{0.97\textwidth}
	\vspace{2mm}	
	\paragraph{#1} #2
	\vspace{2mm}
	\end{minipage}
	}}

%%% FUN THINGS %%%
\newcommand*\tc[1]{\tikz[baseline=(char.base)]{
            \node[shape=circle,draw,inner sep=2pt] (char) {#1};}}
\usepackage{marvosym}

%%% MISC %%%
\usepackage{hyperref}


\setcounter{page}{1}

\begin{document}
\section*{6.7: Physical Applications}

\boxenv{Learning Objectives.}{Upon successful completion of Section 6.7, you will be able to\dots
		
	\begin{itemize}[leftmargin=6mm]
		\item Find the mass of a thin bar with a given density function.
		\item Find the work done given constant force.
		\item Find the work done given a variable force function $f(x)$.
		\item Solve work problems involving springs and Hooke's law.
		\item Solve work problems involving lifting ropes/chains/cables.
		\item Solve work problems involving pumping water.
		\item Solve applications involving pressure.
		\item Solve applications involving hydrostatic force.
	\end{itemize}
	\vspace{-4mm}
}

\vspace{5mm}

\subsection*{Density and Mass}

\textbf{Density} is the concentration of mass in an object. Usually density indicates mass per volume (e.g., kg/m$^3$) and an object with uniform density satisfies the equation

$$\text{mass}=\text{density}\cdot\text{volume}.$$

\vspace{3mm}

If the density varies with the object, we need to use calculus. The problem of finding the mass of a 2- or 3-dimensional object given a density function that varies by position requires multi-variable calculus. So we will look at the problem of finding the mass of a 1-dimensional object using calculus. For 1-dimensional objects, we use \textit{linear density} (e.g., kg/m).

\vspace{5mm}

\textbf{Goal:} Suppose a 1-dimensional object, such as a thin bar or wire, is represented by the interval $a\leq x\leq b$. Find the mass of the object given that the object's linear density $\rho(x)$ varies along its length.

\newpage

\Example A thin bar is represented by the interval $0\leq x\leq\pi$. Find the mass of this bar if its density is given by $\rho(x)=1+\sin x$.

\vspace{100mm}

\subsection*{Work}

\textbf{Work} is an important concept for determining the amount of energy needed to perform various tasks. Work is done when a force moves an object.\\

\textbf{Examples of situations where we want to find the amount of work done.}
\begin{itemize}
\item the work needed to lift a heavy object
\item the work needed to wind up a heavy chain
\item the work needed to pump water up and out of a tank
\item the work needed to stretch or compress a spring
\end{itemize}



\end{document}