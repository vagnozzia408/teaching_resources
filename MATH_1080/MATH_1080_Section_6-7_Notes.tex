\documentclass[12pt]{article}
%%% DOCUMENT FORMATTING %%%
\usepackage[margin=1in]{geometry}
\usepackage{enumitem}
\setlength{\parindent}{0pt}
\newcommand{\disp}{\displaystyle}

%%% HEADER %%%
\usepackage{fancyhdr}
\pagestyle{fancy}
\fancyhf{}
\lhead{MATH 1080}
\rhead{Vagnozzi}
\cfoot{\thepage}

%%% MATH NOTATION & SYMBOLS %%%
\usepackage{amssymb}
\usepackage{amsmath}
\newcommand{\R}{\mathbb{R}}
\newcommand{\N}{\mathbb{N}}
\newcommand{\Z}{\mathbb{Z}}
\newcommand{\lp}{\left(}
\newcommand{\rp}{\right)}
\newcommand{\ls}{\left[}
\newcommand{\rs}{\right]}
\newcommand{\lb}{\left\{}
\newcommand{\rb}{\right\}}
\newcommand{\arccot}{\text{arccot}}
\newcommand{\arccsc}{\text{arccsc}}
\newcommand{\arcsec}{\text{arcsec}} 

%%% TABLES %%%
\usepackage{colortbl}

%%% GRAPHS %%%
\usepackage{tikz}
\usepackage{pgfplots}
\pgfplotsset{compat=1.15}
\usepgfplotslibrary{fillbetween}
\usetikzlibrary{angles,quotes}

%%% ENVIRONMENTS %%%
\newcommand{\Example}{\paragraph{\Writinghand \hspace{0.1mm} Example.}}
\newcommand{\ExampleCont}{\paragraph{\Writinghand \hspace{0.1mm} Example (continued).}}
\newcommand{\boxenv}[2]{
	\fbox{
	\begin{minipage}{0.97\textwidth}
	\vspace{2mm}	
	\paragraph{#1} #2
	\vspace{2mm}
	\end{minipage}
	}}

%%% FUN THINGS %%%
\newcommand*\tc[1]{\tikz[baseline=(char.base)]{
            \node[shape=circle,draw,inner sep=2pt] (char) {#1};}}
\usepackage{marvosym}

%%% MISC %%%
\usepackage{hyperref}


\setcounter{page}{34}

\begin{document}
\section*{6.7: Physical Applications}

\boxenv{Learning Objectives.}{Upon successful completion of Section 6.7, you will be able to\dots
		
	\begin{itemize}[leftmargin=6mm]
		\item Find the mass of a thin bar with a given density function.
		\item Find the work done given constant force.
		\item Find the work done given a variable force function $f(x)$.
		\item Solve work problems involving springs and Hooke's law.
		\item Solve work problems involving lifting ropes/chains/cables.
		\item Solve work problems involving pumping water.
		\item Solve applications involving pressure.
		\item Solve applications involving hydrostatic force.
	\end{itemize}
	\vspace{-4mm}
}

\vspace{5mm}

\subsection*{Density and Mass}

\textbf{Density} is the concentration of mass in an object. Usually density indicates mass per volume (e.g., kg/m$^3$) and an object with uniform density satisfies the equation

$$\text{mass}=\text{density}\cdot\text{volume}.$$

\vspace{3mm}

If the density varies with the object, we need to use calculus. The problem of finding the mass of a 2- or 3-dimensional object given a density function that varies by position requires multi-variable calculus. So we will look at the problem of finding the mass of a 1-dimensional object using calculus. For 1-dimensional objects, we use \textit{linear density} (e.g., kg/m).

\vspace{5mm}

\textbf{Goal:} Suppose a 1-dimensional object, such as a thin bar or wire, is represented by the interval $a\leq x\leq b$. Find the mass of the object given that the object's linear density $\rho(x)$ varies along its length.

\newpage

\Example A thin bar is represented by the interval $0\leq x\leq\pi$. Find the mass of this bar if its density is given by $\rho(x)=1+\sin x$.

\vfill

\subsection*{Work}

Work is an important concept for determining the amount of energy needed to perform various tasks. \textbf{Work} is done when a force moves an object.\\

Examples of situations where we want to find the amount of work done include\dots
\begin{itemize}
\item the work needed to lift a heavy object
\item the work needed to wind up a heavy chain
\item the work needed to pump water up and out of a tank
\item the work needed to stretch or compress a spring
\end{itemize}

\boxenv{Work Done by a Constant Force.}{If an object is moved a distance $d$ in the direction of an applied \underline{constant} force $F$, then the \textbf{work} done by the constant force is
$$W=F\cdot d.$$

\textbf{Note:} Force is mass times acceleration, $F=m\cdot a$.}

\newpage

\textbf{Note on Units.}
\begin{center}
\begin{tabular}{|c|c|}
\hline
 & \textbf{SI Metric System} \\
\hline
Displacement & meter (m) \\
\hline
Mass & kilogram (kg) \\
\hline
Force & newton (N $=$ kg $\cdot$ m/s$^2$)\\
\hline
Work & joule (J $=$ N $\cdot$ m)\\
\hline
\end{tabular}
\end{center}

\Example How much work is required to move an object from $x=0$ to $x=10$ (measured in meters) in the presence of a constant force of 3 N acting along the $x$-axis?

\vspace{40mm}

We know that $W=F\cdot d$ for a constant force. What if the force \textit{varies}?

\vspace{3mm}

Suppose an object moves along the $x$-axis in the positive direction from $x=a$ to $x=b$. At each point, a force $F(x)$ acts on the object. How do we find the work done?

\vfill

\boxenv{Work Done by a Variable Force.}{If an object is moved along a straight line by a continuously varying force $F(x)$, then the \textbf{work} done by the variable force as the object is moved from $x=a$ to $x=b$ is

\vspace{18mm}
}

\newpage

\subsubsection*{Work and Springs}

For problems involving springs, we will need the following law from physics.\\

\boxenv{Hooke's Law.}{The force required to maintain a spring stretched $x$ units beyond its natural length is proportional to $x$:
$$F(x)=kx\text{, where $k$ is a positive constant called the \textbf{spring constant}}.$$

\vspace{-3mm}}

\Example Suppose a force of 30 N is required to stretch and hold a spring 0.2 m from its equilibrium position. How much work is required to compress the spring 0.4 m from its equilibrium position? Assume Hooke's Law is obeyed.

\newpage

\Example A spring requires 100 J of work to be stretched 0.5 m from its equilibrium position. How much work is required to stretch the spring 1.25 m from its equilibrium position? Assume Hooke's Law is obeyed.

\vspace{100mm}

\subsubsection*{Work and Lifting Problems}

Now we will look at finding the work needed to lift an object (such as a rope, cable, or chain) when the motion is vertical and the force is due to gravity. We use that $F=mg$, where $g=9.8$ m/s$^2$ is the acceleration due to gravity. We'll need calculus to find the work done when lifting ropes, chains, and cables because different parts of the rope/chain/cable will be lifted different distances.\\

\textbf{Finding work done in lifting problems:} Suppose we have a chain of length $L$ meters with constant density $\rho$ kg/m hanging vertically from a scaffolding platform at a construction site. How can we find the work done in lifting the chain to the platform?

\newpage 

\Example A heavy rope 20 m long hangs over the edge of a building 40 m high. The rope has a linear density of 2 kg/m.

\begin{itemize}

\item[(a)] How much work is done in pulling the rope to the top of the building?

\vfill

\item[(b)] How much work is done in pulling half the rope to the top of the building?

\vfill 

\item[(c)] A 30 kg load is attached to the bottom of the rope and lifted to the top of the building. How much work is done?

\vfill 

\end{itemize}

\newpage

\subsubsection*{Work and Pumping Liquids Out of Tanks}

\textbf{Finding work done in pumping problems:} Suppose a fluid (such as water) with density $\rho$ is pumped out of a tank to a height $h$ above the bottom of the tank. How much work is required, assuming the tank is full of water?

\newpage 

\Example A circular swimming pool has a diameter of 8 meters, the sides are 2 meters high, and the depth of the water is 1.5 meters. The density of water is 1000 kg/m$^3$.

\begin{itemize}
\item[(a)] How much work is required to pump all of the water out over the side?

\vfill

\item[(b)] How much work is required to pump half the water out over the side?

\vfill
\end{itemize}

\newpage

\subsection*{Force and Pressure}

We want to determine the force exerted on a surface by a body of water. We will also need to know about pressure. \textbf{Pressure} is a force per unit area (e.g., N/m$^2$).\\

\textbf{Idea:} Say we have water putting pressure on a surface of area $A$ m$^2$ that is $h$ meters below the surface. Then the pressure on the surface is computed as

$$\text{pressure}=\frac{\text{force}}{A}=\frac{\text{volume}\cdot\text{density}\cdot g}{A}=\frac{Ah\rho g}{A}=\rho gh$$

and is called the \textbf{hydrostatic pressure} (pressure of water at rest). Note that hydrostatic pressure has the same magnitude in all directions, so the hydrostatic pressure on a vertical wall in a pool at depth $h$ is the same as the hydrostatic pressure on a horizontal surface at depth $h$.\\

How do we find the force on a vertical wall, such as the face of a dam, assuming the water completely covers the face of the dam and the water level is at the top of the dam?

\newpage

\Example A plate shaped like an isosceles triangle with a height of 1 m is placed on a vertical wall 1 m below the surface of a pool filled with water. Compute the force on the plate.
\end{document}