\documentclass[12pt]{article}
\input{MATH_1060_Preamble.tex}

\setcounter{page}{119}

\begin{document}
\section*{4.3: What Derivatives Tell Us}

\boxenv{Learning Objectives.}{Upon successful completion of Section 4.3, you will be able to\dots
		
	\begin{itemize}[leftmargin=6mm]
		\item Answer conceptual questions involving derivatives.
		\item Find the intervals on which a function is increasing or decreasing.
		\item Use the first derivative test to locate critical points and local and absolute extrema.
		\item Sketch the graph of a function given properties of the function.
		\item Determine the concavity on intervals and find inflection points.
		\item Determine if critical points correspond to local maxima/minima using the second derivative test.
		\item Compare the graphs of a function with the graphs of its first and second derivatives.
	\end{itemize}
	\vspace{-4mm}
}

\vspace{5mm}

\subsection*{First Derivatives}

First derivatives tell us where a function is increasing or decreasing.
\begin{itemize}
\item If the derivative is \textit{positive} on an interval, then the function is \textit{increasing}.
\item If the derivative is \textit{negative} on an interval, then the function is \textit{decreasing}.
\end{itemize}

Derivatives also allow us to classify critical points.

\vspace{5mm}

\boxenv{First Derivative Test.}{Suppose that $x=c$ is a critical point of a function $f$. If $f'$\dots
\begin{itemize}
\item changes from $+$ to $-$ around $x=c$, then $f(c)$ is a local maximum.
\item changes from $-$ to $+$ around $x=c$, then $f(c)$ is a local minimum.
\item does not change sign, then $f(c)$ is neither.
\end{itemize}
\vspace{-4mm}
}

\Example For the function $f(x)=x^3+x^2-x$, find the intervals on which $f$ increases and decreases and classify all critical points.

\newpage

\boxenv{Theorem.}{Suppose that $f$ is a continuous function with one and only one critical point. If that critical point is a local extremum, then it is also a global extremum.}

\Example For $f(x)=\disp\frac{e^x}{e^{2x}+1}$, find the intervals on which $f$ increases and decreases and classify all critical points.

\vspace{50mm}

\subsection*{Second Derivatives}

Second derivatives reveal information about the \textbf{curvature} or \textbf{concavity} of the graph of a function.

\vspace{5mm}

\boxenv{Theorem.}{Suppose that $f$ is twice differentiable on an open interval.
\begin{itemize}
\item If $f''(x)>0$ for all $x$ in the interval, then $f$ is concave up.
\item If $f''(x)<0$ for all $x$ in the interval, then $f$ is concave down.
\end{itemize}

\vspace{-3mm}
}

\vspace{5mm}

\boxenv{Definition.}{If $f$ is continuous at $x=c$ and $f$ changes concavity at that point (i.e.\ $f''$ changes sign), then the point $\lp c, f(c)\rp$ is called an \textbf{inflection point}.}

\vspace{2mm}

\begin{center}
            \begin{tikzpicture}[scale=1.1]
                \begin{axis}[
                	axis x line=middle,
                	xmax=4, xmin=-4,
                	axis y line=center,
                	ymax=3, ymin=-2.5,
                	axis line style=<->,
			ticks=none%,
                	%xlabel=$x$,ylabel=$y$
                    ]
                    %\addplot[name path=f,smooth,domain=-2.3:2.5,color=blue,samples=100,thick] {0.3*(x+2)*x*(x-1)*(x-2)};
                    \addplot[name path=f,smooth,domain=-0.604:1.104,color=blue,samples=100,thick] {0.3*(x+2)*x*(x-1)*(x-2)};
                    \addplot[name path=f,smooth,domain=1.104:2.5,color=green,samples=100,->,thick] {0.3*(x+2)*x*(x-1)*(x-2)};
                    \addplot[name path=f,smooth,domain=-2.3:-0.604,color=red,samples=100,<-,thick] {0.3*(x+2)*x*(x-1)*(x-2)};
                \end{axis}
            \end{tikzpicture}
        \end{center}

\newpage

To locate inflection points, we will\dots
\begin{enumerate}
	\item[\tc{1}] Identify the value(s) where $f''$ is zero or fails to exist.
	\item[\tc{2}] Identify whether $f''$ changes in sign around those value(s).
\end{enumerate}

\vspace{5mm}

\Example Consider the function $f(x)=x-2\arctan(x)$.
\begin{enumerate}
\item[\tc{1}] Find the intervals where $f$ is increasing/decreasing and classify all critical points.
\vspace{75mm}
\item[\tc{2}] Find the intervals of concavity and any inflection points.
\end{enumerate}

\newpage

If a function $f$ is twice differentiable, then the following test can also be applied to classify a critical point.

\vspace{5mm}

\boxenv{Second Derivative Test.}{Suppose that $f'(c)=0$ and that $f''(c)$ exists.
\begin{enumerate}
	\item If $f''(c)>0$ (i.e.\ $f$ is concave up), then $f(c)$ is a local minimum.
	\item If $f''(c)<0$ (i.e.\ $f$ is concave down), then $f(c)$ is a local maximum.
\end{enumerate}
\vspace{-5mm}
}

\vspace{5mm}

Note that this test does not apply if $f'(c)$ is undefined, because $f''(c)$ will also be undefined.

\Example Determine the critical points of $f(x)=x^3-x$ and use the second derivative test to classify the critical points.

\vfill

\begin{center}
\renewcommand{\arraystretch}{1.25}
\begin{tabular}{|c||c|c|}
\hline
\textbf{Function} & \textbf{First Derivative} & \textbf{Second Derivative} \\
$f$ & $f'$ & $f''$ \\
\hline
Increasing & Positive & --- \\
\hline
Decreasing & Negative & --- \\
\hline
Local Maximum & Zero or DNE, $+$ to $-$ & Negative \\
\hline
Local Minimum & Zero or DNE, $-$ to $+$ & Positive\\
\hline
Concave Up & Increasing & Positive \\
\hline
Concave Down & Decreasing & Negative \\
\hline
Inflection Point & Local Max/Min & Zero or DNE, changes sign \\
\hline
\end{tabular}
\end{center}

\end{document}