\documentclass[12pt]{article}
\input{MATH_1060_Preamble.tex}

\setcounter{page}{88}

\begin{document}
\section*{3.8: Implicit Differentiation}

\boxenv{Learning Objectives.}{Upon successful completion of Section 3.8, you will be able to\dots
		
	\begin{itemize}[leftmargin=6mm]
		\item Answer conceptual questions involving implicit differentiation.
		\item Find derivatives using implicit differentiation.
		\item Find second derivatives using implicit differentiation.
		\item Use implicit differentiation to find slopes of curves at given points.
		\item Use implicit differentiation along with the product, quotient, and chain rules to find derivatives.
		\item Solve applications using implicit differentiation.
		\item Answer questions involving tangent lines using implicit differentiation.
		\item Find normal lines using implicit differentiation.
	\end{itemize}
	\vspace{-4mm}
}

\vspace{5mm}

\subsection*{Implicit vs.\ Explicit}

Thus far, we have been working with functions in \textbf{explicit form}. In other words, these functions are solved for $y$ completely in terms of $x$.

$$y=x^2+2x+1\,\Longrightarrow\,\frac{dy}{dx}=2x+2$$

\vspace{3mm}

Now consider a function $y=f(x)$ presented in \textbf{implicit form}. 

$$x^2+y^2=xy\,\Longrightarrow\,\frac{dy}{dx}=\text{???}$$

\vspace{3mm}

In this section, we will introduce the technique of \textbf{implicit differentiation} to find $\disp\frac{dy}{dx}$ when $y$ is not given or impossible to state in terms of $x$.

\newpage

\subsection*{Implicit Differentiation}

Consider the equation $x^2+y^2=1$. There are two main steps to implicit differentiation.

\begin{enumerate}
	\item[\tc{1}] Differentiate both sides of the equation, remembering to apply the chain rule to each $y$ derivative.
	\vspace{30mm}
	\item[\tc{2}] Solve for $\disp\frac{dy}{dx}$.
	\vspace{30mm}
\end{enumerate}

\paragraph{Example.} Consider $y-x^2+2x=1$. Note that this can be expressed as $y=x^2-2x+1$, allowing us to find $\disp\frac{dy}{dx}$ in the usual manner.

$$\frac{dy}{dx}=2x-2$$

\vspace{3mm}

If we instead use implicit differentiation, we obtain the same result.

$$\frac{dy}{dx}-2x+2=0 \,\Longleftrightarrow\,\frac{dy}{dx}=2x-2$$

\vspace{3mm}

\Example Consider the equation $y^2+3x=2$. Find $\disp\frac{dy}{dx}\bigg|_{(-1,\sqrt{5})}$.

\newpage

\Example Find $\disp\frac{dy}{dx}$ given that $yx^2+y^3=\cos y$. 

\vspace{60mm}

\Example Find $\frac{dy}{dx}$ given that $x^2y^2+e^y=\tan y$.

\vspace{60mm}

\Example Find the equation of the normal line at $(1,1)$ for $x^3+x^2y+4y^2=6$.

\newpage

\Example Find $\disp\frac{dy}{dx}$ given that $1+x=\sin\lp xy^2\rp$.

\vspace{60mm}

\Example The curves $x^2-y^2=5$ and $4x^2+9y^2=72$ intersect at $(3,2)$. Show that this intersection occurs at a right angle.

\vspace{60mm}

\boxenv{Remark.}{Every implicit equation has an underlying assumption that a function \\ $y=f(x)$ exists that will satisfy the equation. If there is no such function $y$, then $\disp\frac{dy}{dx}$ will represent a function that does not exist.}

\vspace{3mm}

\paragraph{Summary.} Here are some notes to keep in mind when using implicit differentiation.
\begin{itemize}
	\item Implicit differentiation should be used whenever we wish to find $\disp\frac{dy}{dx}$, but $y$ is not stated explicitly in terms of $x$.
	\item This technique is a special case of the \textit{chain rule}.
	\item All derivative rules that we have learned can be applied when using this technique.
	\item Not every implicit equation is meaningful.   However, when asked to solve implicit equations in this course, you may assume that this is not the case.
\end{itemize}

\newpage 

Note that where implicit differentiation is required, there may be more than one way to solve a problem. For example, consider the following example, in which you could choose to find the derivative by combining implicit differentiation with either the \textit{quotient rule} or the \textit{product rule}.

\Example Find $\disp\frac{dy}{dx}$ for $\disp\frac{2x+y}{x^2}=y$.

\end{document}