\documentclass[12pt]{article}
%%% DOCUMENT FORMATTING %%%
\usepackage[margin=1in]{geometry}
\usepackage{enumitem}
\setlength{\parindent}{0pt}
\newcommand{\disp}{\displaystyle}

%%% HEADER %%%
\usepackage{fancyhdr}
\pagestyle{fancy}
\fancyhf{}
\lhead{MATH 1060}
\rhead{Vagnozzi}
\cfoot{\thepage}

%%% MATH NOTATION & SYMBOLS %%%
\usepackage{amssymb}
\usepackage{amsmath}
\newcommand{\R}{\mathbb{R}}
\newcommand{\N}{\mathbb{N}}
\newcommand{\Z}{\mathbb{Z}}
\newcommand{\lp}{\left(}
\newcommand{\rp}{\right)}
\newcommand{\ls}{\left[}
\newcommand{\rs}{\right]}
\newcommand{\lb}{\left\{}
\newcommand{\rb}{\right\}}
\newcommand{\arccot}{\text{arccot}}
\newcommand{\arccsc}{\text{arccsc}}
\newcommand{\arcsec}{\text{arcsec}} 

%%% TABLES %%%
\usepackage{colortbl}

%%% GRAPHS %%%
\usepackage{tikz}
\usepackage{pgfplots}
\pgfplotsset{compat=1.15}
\usepgfplotslibrary{fillbetween}
\usetikzlibrary{angles,quotes}

%%% ENVIRONMENTS %%%
\newcommand{\Example}{\paragraph{\Writinghand \hspace{0.1mm} Example.}}
\newcommand{\ExampleCont}{\paragraph{\Writinghand \hspace{0.1mm} Example (continued).}}
\newcommand{\boxenv}[2]{
	\fbox{
	\begin{minipage}{0.97\textwidth}
	\vspace{2mm}	
	\paragraph{#1} #2
	\vspace{2mm}
	\end{minipage}
	}}

%%% FUN THINGS %%%
\newcommand*\tc[1]{\tikz[baseline=(char.base)]{
            \node[shape=circle,draw,inner sep=2pt] (char) {#1};}}
\usepackage{marvosym}

%%% MISC %%%
\usepackage{hyperref}


\setcounter{page}{152}

\begin{document}
\section*{4.9: Antiderivatives}

\boxenv{Learning Objectives.}{Upon successful completion of Section 4.9, you will be able to\dots
		
	\begin{itemize}[leftmargin=6mm]
		\item Answer conceptual questions involving antiderivatives.
		\item Find all antiderivatives of a function.
		\item Determine the indefinite integral of a function.
		\item Given a function, find the antiderivative satisfying a given condition.
		\item Given the derivative of a function, find the function satisfying an initial value.
		\item Graph the solutions to a differential equation, then the particular solution given an initial value.
		\item Find a position function given a velocity function and an initial position or an acceleration function and an initial velocity and an \\ initial position.
		\item Solve applications involving derivatives.
	\end{itemize}
	\vspace{-4mm}
}

\vspace{5mm}

\subsection*{Introduction}

The two major concepts in calculus that we have learned thus far are the \textit{limit} and the \textit{derivative}. The third major concept, introduced in this section, is that of the \textit{antiderivative}.

\vspace{5mm}

\boxenv{Definition.}{The function $F$ is called an \textbf{antiderivative} of $f$ if $F'(x)=f(x)$. In other words, an antiderivative $F$ is a function whose derivative is a given function $f$.}

\vspace{5mm}

\Example Let's list some antiderivatives for the function $f(x)=2x$. 

\newpage

\subsection*{The General Antiderivative}

\boxenv{Theorem.}{Let $F$ be an antiderivative of $f$. Then all antiderivatives of $f$ have the form $F(x)+C$, where $C\in\R$ is an arbitrary constant.}

\vspace{5mm}

\Example Find the general antiderivative for $f(x)=2x$. 

\vspace{20mm}

\Example Find the antiderivatives of the following functions.
\begin{enumerate}
\item[\tc{1}] $y'=4x^3+3x^2+2x+1$

\vspace{20mm}

\item[\tc{2}] $f(x)=\disp\frac{2}{1+x^2}$

\vspace{20mm}

\item[\tc{3}] $f(x)=\cos x-\sin x+\sec^2 x$

\vspace{20mm}

\item[\tc{4}] $f(x)=3e^x+3^x\ln 3 + 3$

\vspace{20mm}
\end{enumerate}

\subsection*{Indefinite Integrals}

Recall that the differential operator $\disp\frac{d}{dx}$ instructs us to take the derivative of the function under consideration.


$$\frac{d}{dx}\lp \arctan 2x\rp =\frac{2}{1+4x^2}$$

\newpage

The operation of \textbf{antidifferentiation} is denoted by an \textbf{indefinite integral}.

$$\int f(x)\,dx=F(x)+C\text{, where } F'(x)=f(x)$$

\vspace{3mm}

The integral symbol $\disp\int$ must always be paired with the differential $dx$ (or a differential that corresponds to another variable, such as $dt$ or $d\theta$). We will explain what this differential represents in Chapter 5, but for now, know that this notation instructs us to find the general antiderivative for a function $f$.

\vspace{3mm}

$$\int f(x)\,dx \,\Longleftrightarrow\, \text{``find the general antiderivative of }f$$

\vspace{3mm}

We can thus write $\disp\int \frac{2}{1+4x^2}\,dx=\arctan 2x+C$.

\vspace{3mm}

\Example Evaluate the following indefinite integrals.

\begin{enumerate}
\item[\tc{1}] $\disp\int 2x\,dx$

\vspace{20mm}

\item[\tc{2}] $\disp\int\lp 3+e^x\rp dx$

\vspace{20mm}

\item[\tc{3}] $\disp\int \lp \sec\theta \tan \theta + \sec^2\theta\rp d\theta$
\end{enumerate}

\newpage

\subsection*{Antiderivative Rules}

We can find the antiderivative of a \textbf{constant} $k\in\R$.

$$\int k\,dx=kx+C$$

\vspace{5mm}

Antidifferentiation is a \textbf{linear operation}, i.e.\ for $a,b\in\R$,

$$\int \big( af(x)\pm bg(x)\big) dx = aF(x)\pm bG(x)+C.$$

\vspace{5mm}

To find the antiderivative of a \textbf{power term}, we reverse the power rule for derivatives.

$$\int x^n\,dx=\frac{x^{n+1}}{n+1}+C\text{, where }n\neq -1$$

\vspace{5mm}

\Example Evaluate the indefinite integrals.

\begin{enumerate}
\item[\tc{1}] $\disp\int \lp x^5+x^3+x\rp dx$

\vspace{30mm}

\item[\tc{2}] $\disp\int\lp \frac{3}{\sqrt{x}}+\frac{2}{\sqrt{x^3}}\rp dx$

\vspace{30mm}
\end{enumerate}

To handle powers of negative one, we use the following rule.
$$\int x^{-1}\,dx=\int\frac{1}{x}\,dx=\ln|x|+C$$

\vspace{3mm}

The necessity of the absolute value will become clear later on.

\newpage

\Example Evaluate the indefinite integral $\disp\int\lp\frac{1}{x^3}+\frac{2}{x^2}+\frac{3}{x}\rp dx$.

\vspace{30mm}

\paragraph{Common Antiderivative Mistakes.} Be mindful of the following errors when evaluating indefinite integrals.

\vspace{5mm}

One mistake is to attempt to apply the ``powers of negative one'' rule from the previous page to other negative powers.

$$\int x^{-3}\,dx=\int\frac{1}{x^3}\,dx\neq \ln\vert x^3\vert+C$$

\vspace{5mm}

Another temptation is to attempt to find the antiderivative of a product or quotient of functions in the manner below. Unfortunately, antiderivatives of products and quotients do not ``behave nicely.'' Unless the integrand can be simplified, we must use alternative methods to find the antiderivatives of such functions, which are taught in MATH 1080.

$$\int f(x)g(x)\,dx\neq F(x)G(x)+C$$

$$\int \frac{f(x)}{g(x)}\,dx\neq\frac{F(x)}{G(x)}+C$$

\vspace{5mm}

If presented with an indefinite integral of a product or quotient of functions in MATH 1060, you should \textbf{simplify} the integrand to find the antiderivative.

\Example Evaluate the indefinite integral. 

\vspace{5mm}

\hspace{10mm} $\disp\int t^2\lp 1-\frac{1}{t^2}+\frac{2}{t^3}\rp dt$

\newpage

\Example Evaluate the indefinite integral. 

\vspace{5mm}

\hspace{10mm} $\disp \int \frac{x^3-x^2-2x}{x^2-2x}\,dx$

\vspace{60mm}

\Example True or False: $\disp\int\ln x\,dx=x\ln x+x+C$

\vspace{50mm}

\Example Evaluate the indefinite integral.

\vspace{5mm}

\hspace{10mm} $\disp\int\frac{3+2\sqrt{1-x^2}}{\sqrt{1-x^2}}\,dx$

\newpage

\Example Evaluate the indefinite integral.

\vspace{5mm}

\hspace{10mm} $\disp\int\lp t^2+1\rp\lp 2t-5\rp\,dt$

\vspace{60mm}

\Example Evaluate the indefinite integral.

\vspace{5mm}

\hspace{10mm} $\disp\int\frac{\cos x - 1}{\sin^2 x}\,dx$

\vspace{60mm}

\subsection*{Introduction to Differential Equations}

Antiderivatives are a critical element of solving differential equations.

\vspace{3mm}

\boxenv{Definition.}{An equation involving an unknown function and one or more of its derivatives is called a \textbf{differential equation}.}

\vspace{5mm} For example, the following is a differential equation.

$$4x^2y''+12xy'+3y=0$$

We can see that the equation on the previous page involves both the first and second derivatives of some function $y$, but we don't know which function $y$ actually is. To solve a differential equation, we need to identify the {function} that satisfies such an equation. There are entire courses dedicated to differential equations, but we'll learn how to solve some basic ones in our course.

\vspace{5mm}

\Example Solve the differential equation $2y'-3=\cos t$.

\vspace{60mm}

\paragraph{Initial Value Problems.} If an initial condition is given (i.e.\ a particular value of the unknown function), it is possible to solve for the constant $C$. A differential equation with an initial condition is referred to as an \textbf{initial value problem}.

\vspace{5mm}

\Example Solve the initial value problem.
$$e^x(y'-1)=e^{2x}\hspace{25mm} y(0)=3$$

\newpage 

\paragraph{Motion Problems Revisited.} Recall that, given a position function $s=f(t)$, we can find velocity and acceleration by taking the first and second derivatives of $s$, respectively. With the concept of antiderivatives, we can now see the following relationships.

$$v(t)=\frac{ds}{dt}=s'(t)\,\Longrightarrow\, {s(t)=\int v(t)\,dt}$$

$$a(t)=\frac{d^2s}{dt^2}=s''(t)=v'(t)\,\Longrightarrow\, v(t)=\int a(t)\,dt$$

\vspace{3mm}

Given initial conditions, we can ``recover'' the position function from an acceleration function.

\vspace{5mm}

\Example If $a(t)=-32$ ft/s$^2$, find the velocity and position functions, $s$ and $v$, given that $v(0)=2$ ft/s and $s(0)=10$ feet.


\end{document}