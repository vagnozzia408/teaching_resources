\documentclass[12pt]{article}
\input{MATH_1060_Preamble.tex}

\setcounter{page}{58}

\begin{document}
\section*{3.2: The Derivative as a Function}

\boxenv{Learning Objectives.}{Upon successful completion of Section 3.2, you will be able to\dots
		
	\begin{itemize}[leftmargin=6mm]
		\item Answer conceptual questions involving the derivative as a function.
		\item Obtain the graphs of derivative functions from graphs of functions.
		\item Find points where functions are continuous and differentiable.
		\item Find derivatives of functions using limits.
		\item Solve applications involving derivatives as functions.
		\item Use graphs of functions to analyze slopes of tangent lines.
		\item Obtain graphs of functions from graphs of their derivative function.
		\item Find equations of normal lines.
		\item Find vertical tangent lines from graphs.
	\end{itemize}
	\vspace{-4mm}
}

\vspace{5mm}

\subsection*{Defining the Derivative Function}

In Section 3.1, we learned how to find the \textbf{derivative at a point} $x=a$.

$$f'(a)=\lim_{x\to a}\frac{f(x)-f(a)}{x-a}=\lim_{h\to 0}\frac{f(a+h)-f(a)}{h}$$

\vspace{2mm}

If we wanted the derivative at a new point, this would require an entirely new limit. Fortunately, there is a way of calculating derivatives without having to recompute a new limit each time. Because we know that limits, and hence derivatives, are \textit{functions}, we will work towards finding the \textbf{derivative function}. Once found, we can calculate $f'(a)$ for whatever value of $a$ we would like.

\vspace{5mm}

\boxenv{Definition.}{The \textbf{derivative} of $f$ is the function defined by

$$f'(x)=\lim_{h\to 0}\frac{f(x+h)-f(x)}{h},$$

provided the limit exists. The process of finding this limit, i.e.\ finding the derivative, is called \textbf{differentiation.}}

\vspace{5mm}

\boxenv{Remark.}{The domain of $f'$ is contained within the domain of $f$.}

\newpage

\paragraph{Derivative Notation.} The derivative (function) of $y=f(x)$ may be denoted in several different ways as follows.

$$f'(x)=y'=y'(x)=\frac{dy}{dx}=\frac{df}{dx}$$

\vspace{3mm}

The symbol $\disp\frac{d}{dx}$ is called the \textbf{differential operator} and it instructs us to take the derivative. 

$$\frac{d}{dx}(3x-5)\,\Longleftrightarrow\,\text{``take the derivative of }3x-5$$

\vspace{3mm}

The evaluation of a derivative at a point $x=a$ may be denoted as follows.

$$f'(a)=y'(a)=\frac{dy}{dx}\bigg|_{x=a}=\frac{df}{dx}\bigg|_{x=a}$$

\vspace{3mm}

\Example Find the derivative of $y=3x-5$.

\vspace{45mm}

\Example Find the derivative of $f(x)=x^2+3x$.

\vspace{50mm}

\Example Find the slope of the tangent line at three different $x$ values for the function $f$ in the previous example.

\newpage

\Example Evaluate the following expression.

\vspace{5mm}

\hspace{10mm} $\disp\frac{d}{dx}\lp\sqrt{x}\rp$

\vspace{50mm}

\Example For $y=\sqrt{x}$, at which $x$-value will the slope of the tangent line be one?

\vspace{50mm}

\Example Find $\disp\frac{dy}{dx}$ for $y=\disp\frac{1}{x+1}$.

\newpage

\subsection*{Normal Lines}

Recall that the derivative at a point represents the slope of a tangent line at that point. We have already used derivatives to find the equation of a tangent line. We can also use derivatives to identify the slope (and thus equation) of a \textit{normal line}.

\vspace{3mm}

\boxenv{Definition.}{Two lines with slopes $m_1$ and $m_2$, respectively, are \textbf{perpendicular} if the slopes multiply to negative one.

$$m_1 m_2 = -1\,\Longleftrightarrow\, m_1=-\frac{1}{m_2}$$

Hence, $m_1$ and $m_2$ are \textbf{negative reciprocals.}}

\vspace{3mm}

\boxenv{Definition.}{A \textbf{normal line} to the graph of $y=f(x)$ at a point $x=c$ is perpendicular to the tangent line at that point.

$$m_T m_N=-1\,\Longleftrightarrow\, m_N=-\frac{1}{m_T}=-\frac{1}{f'(c)}$$}

\vspace{5mm}

\Example The derivative of $f(x)=x^2+3x$ is $f'(x)=2x+3$. Find the equation of the normal line to $f(x)$ at $x=3$.

\vspace{40mm}

\subsection*{Derivative Examples}

\Example Find the derivative of $f(x)=\sqrt{2-2x}$.

\newpage

\Example Find the derivative of $f(x)=\disp\frac{x}{x+1}$.

\vspace{50mm}

\subsection*{Differentiability}

In Section 2.6, we learned about \textit{continuity}. This property of functions has a relationship with the differentiability of a function.

\vspace{5mm}

\boxenv{Theorem.}{If $f$ is differentiable at $x=c$, then it is continuous at $x=c$.}

\vspace{5mm}

Note that continuity does \underline{not} imply differentiability. Consider the absolute value function, $f(x)=\vert x \vert$. For the derivative to exist using its limit definition, the associated left- and right-hand limits must be equal.

$$\lim_{x\to 0^+}\frac{\vert x+0\vert -\vert 0 \vert}{x-0}=\lim_{x\to 0^+}\frac{x}{x}=1$$

$$\lim_{x\to 0^-}\frac{\vert x+0\vert - \vert 0 \vert}{x-0}=\lim_{x\to 0^-}-\frac{x}{x}=-1$$

\vspace{3mm}

Hence, $f'(0)$ does not exist although $f$ is continuous at $x=0$.

\vspace{3mm}

\paragraph{When Differentiability Fails.} A function will fail to be differentiable in the following situations.
\begin{itemize}
	\item Any \textbf{discontinuity} causes the derivative to not exist.
	\item The existence of a \textbf{corner} or \textbf{cusp} will cause the derivative to not exist. (See the absolute value function example above.)
	\item The existence of a \textbf{vertical tangent} will cause the derivative to not exist.
	\item The derivative will not exist at an \textbf{endpoint} because a two-sided limit is not possible.
\end{itemize}

\newpage

\Example Circle the locations within $\ls -2,2\rs$ where $f$ is not differentiable.

\vspace{5mm}

\begin{center}
            \begin{tikzpicture}[scale=1.25]
                \begin{axis}[
                	axis x line=middle,
                	xmax=2.5, xmin=-2.5,
                	axis y line=center,
                	ymax=1.75, ymin=0,
	              	xlabel=$x$,ylabel=$f$,
	              	x axis line style=<->
                    ]
                    \addplot[name path=f,smooth,domain=-2:0,color=blue,samples=100,thick] {abs(x+1)};
                    \addplot[name path=f,smooth,domain=0:2,color=blue,samples=100,thick] {-(x-1)^2 + 1};
 
		    \addplot[mark=*,color=blue] coordinates {(-2,1)};
		    \addplot[mark=*,color=blue] coordinates {(0,0)};
		    \addplot[mark=*,color=blue] coordinates {(2,0)};
		    \addplot[mark=*,color=blue] coordinates {(1,1.5)};
																						
		    \addplot[mark=*,color=blue,fill=white] coordinates {(1,1)};
		    \addplot[mark=*,color=blue,fill=white] coordinates {(0,1)};
		\draw [dashed] (5,0) -- (5,5);			
                \end{axis}
            \end{tikzpicture}
        \end{center}

\end{document}