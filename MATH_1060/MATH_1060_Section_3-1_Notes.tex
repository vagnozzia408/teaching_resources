\documentclass[12pt]{article}
\input{MATH_1060_Preamble.tex}

\setcounter{page}{54}

\begin{document}
\section*{3.1: Introduction to the Derivative}

\boxenv{Learning Objectives.}{Upon successful completion of Section 3.1, you will be able to\dots
		
	\begin{itemize}[leftmargin=6mm]
		\item Answer conceptual questions involving tangent lines and derivatives.
		\item Solve applications involving the use of limits to calculate derivatives.
		\item Use limit definitions to find equations of tangent lines.
		\item Use limit definitions to evaluate derivatives at given points.
		\item Compute average and instantaneous rates of change from graphs and tables.
		\item Determine functions given limits of difference quotients.
	\end{itemize}
	\vspace{-4mm}
}

\vspace{5mm}

\subsection*{The Idea Behind Derivatives}

One of the major questions in calculus is: How can we calculate the slope of a tangent line? We know that we can calculate an \textit{average} rate of change by finding the slope of a secant line. How can we calculate an \textit{instantaneous} rate of change?

\vspace{5mm}

\begin{center}
            \begin{tikzpicture}[scale=1.5]
                \begin{axis}[
                	axis x line=middle,
                	xmax=5.6, xmin=0,
                	axis y line=center,
                	ymax=150, ymin=60,
                	xlabel=$x$,ylabel=$f$,
                	xticklabels={},
                	yticklabels={}
                    ]
                    \addplot[name path=f,smooth,domain=0:5,color=blue,samples=100,->,thick] {-16*x^2 + 96*x};
          
          		    \draw[color=blue] (4.7,120) node[anchor=south] {$f(x)$};

                \end{axis}
            \end{tikzpicture}
            
            \vspace{15mm}
            
            View an interactive demo here: \url{https://www.desmos.com/calculator/2vmz7bdvgo}

        \end{center}

\newpage

\subsection*{The Derivative at a Point}

\boxenv{Definition.}{The line tangent to the curve $y=f(x)$ at $x=a$ has slope

$$m_{\text{tan}}=\lim_{x\to a}m_{\text{sec}}=\lim_{x\to a}\frac{f(x)-f(a)}{x-a},$$

provided the limit exists. This limit is called the \textbf{derivative} of $f$ at the point $a$ and is denoted by $f'(a)$, read as ``$f$ prime of $a$.''

\vspace{25mm}

This value, when it exists, is sometimes called the \textbf{instantaneous rate of change} or the \textbf{slope of the curve}.
}

\vspace{5mm}

Because derivatives are, by definition, limits, there is a useful property of limits that can help us understand derivatives.

\vspace{3mm}

\boxenv{Theorem.}{Suppose $\disp\lim_{x\to a}f(x)=L$ and $\disp\lim_{x\to a}f(x)=M$. Then $L=M$. In other words, if a limit exists, it is unique.}

\vspace{3mm}

Now consider the function $f(x)=\disp\frac{\sin x}{x}$. Note that $f$ is not continuous at $x=0$ because $f(0)$ is undefined. However, the following function \underline{is} continuous at $x=0$\dots

$$f'(x)=\lim_{t\to x}\frac{\sin t}{t}$$

\vspace{2mm}

By the uniqueness of limits, we can say that the derivative is a \textbf{function}. In other words, for every point $a$, there is at most one value of $f'(a)$.

\vspace{5mm}

\Example Find the equation of the line tangent to $f(x)=\sqrt{3x}$ at $x=3$.

\newpage

\Example Find the equation of the line tangent to $f(x)=\disp\frac{1}{x}$ at $x=2$.

\vspace{50mm}

\Example Find the equation of the line tangent to $f(x)=x^2+2$ at $x=0$.

\vspace{50mm}

\paragraph{Alternative Definition of the Derivative.} If we let $h$ be the distance between $x$ and $a$, i.e.\ $h=x-a$, then
$$x=a+h\text{ and } x\to a \,\Longleftrightarrow\, h\to 0,$$

leading us to an equivalent formulation of the limit definition of a derivative at a point.

\vspace{5mm}

\boxenv{Definition.}{The \textbf{derivative} of a function $f$ at $x=a$, denoted $f'(a)$, is

$$f'(a)=\lim_{h\to 0}\frac{f(a+h)-f(a)}{h},$$

provided the limit exists.}

\vspace{5mm}

\boxenv{Definition.}{If $f'(a)$ exists, we say that $f$ is \textbf{differentiable} at $x=a$.}

\newpage

\Example Find the equation of the line tangent to $f(x)=2x^2-3x$ at $x=1$.

\vspace{50mm}

\Example Suppose that $y=-\frac{1}{2}x+5$ is tangent to $f$ at $x=2$. Find $f'(2)$.


\end{document}