\documentclass[12pt]{article}
%%% DOCUMENT FORMATTING %%%
\usepackage[margin=1in]{geometry}
\usepackage{enumitem}
\setlength{\parindent}{0pt}
\newcommand{\disp}{\displaystyle}

%%% HEADER %%%
\usepackage{fancyhdr}
\pagestyle{fancy}
\fancyhf{}
\lhead{MATH 1060}
\rhead{Vagnozzi}
\cfoot{\thepage}

%%% MATH NOTATION & SYMBOLS %%%
\usepackage{amssymb}
\usepackage{amsmath}
\newcommand{\R}{\mathbb{R}}
\newcommand{\N}{\mathbb{N}}
\newcommand{\Z}{\mathbb{Z}}
\newcommand{\lp}{\left(}
\newcommand{\rp}{\right)}
\newcommand{\ls}{\left[}
\newcommand{\rs}{\right]}
\newcommand{\lb}{\left\{}
\newcommand{\rb}{\right\}}
\newcommand{\arccot}{\text{arccot}}
\newcommand{\arccsc}{\text{arccsc}}
\newcommand{\arcsec}{\text{arcsec}} 

%%% TABLES %%%
\usepackage{colortbl}

%%% GRAPHS %%%
\usepackage{tikz}
\usepackage{pgfplots}
\pgfplotsset{compat=1.15}
\usepgfplotslibrary{fillbetween}
\usetikzlibrary{angles,quotes}

%%% ENVIRONMENTS %%%
\newcommand{\Example}{\paragraph{\Writinghand \hspace{0.1mm} Example.}}
\newcommand{\ExampleCont}{\paragraph{\Writinghand \hspace{0.1mm} Example (continued).}}
\newcommand{\boxenv}[2]{
	\fbox{
	\begin{minipage}{0.97\textwidth}
	\vspace{2mm}	
	\paragraph{#1} #2
	\vspace{2mm}
	\end{minipage}
	}}

%%% FUN THINGS %%%
\newcommand*\tc[1]{\tikz[baseline=(char.base)]{
            \node[shape=circle,draw,inner sep=2pt] (char) {#1};}}
\usepackage{marvosym}

%%% MISC %%%
\usepackage{hyperref}


\setcounter{page}{33}

\begin{document}
\section*{2.4: Infinite Limits}

\boxenv{Learning Objectives.}{Upon successful completion of Section 2.4, you will be able to\dots
		
	\begin{itemize}[leftmargin=6mm]
		\item Answer conceptual questions involving infinite limits and vertical asymptotes.
		\item Find infinite limits numerically or graphically.
		\item Sketch graphs or functions involving infinite limits.
		\item Evaluate limits analytically.
		\item Find vertical asymptotes.
	\end{itemize}
	\vspace{-4mm}
}

\vspace{5mm}

\subsection*{Introduction to Infinite Limits}

\boxenv{Definition.}{In an \textbf{infinite limit}, the dependent variable ($y$-value) becomes ``boundless'' in the positive or negative direction as the dependent variable ($x$-value) approaches some finite value.}

\vspace{3mm}

For example, if $f(x)\to\infty$ as $x\to c$, we will write the following.

$$\lim_{x\to c}f(x)=\infty$$

\vspace{3mm}

\boxenv{Remark.}{Technically, such limits do not exist. However, this particular case of the limit not existing provides valuable information about the behavior of a function, so we make a minor exception in our notation to indicate this special case.}

\vspace{5mm}

Infinite limits will occur in cases when attempting direct substitution gives a zero in the \textit{denominator only}.

\vspace{3mm}

\textbf{Example.} Consider $f(x)=\disp\frac{1}{x}$. What happens as $x\to 0^+$?

\begin{center}
\begin{tabular}{|
>{\columncolor[HTML]{CBCEFB}}c |c|c|c|c|c|c|}
\hline
$x$   & 0.1 & 0.01 & 0.001 & 0.0001 & 0.00001 & 0.000001 \\ \hline
$1/x$ & 1  & 10   & 100   & 1,000  & 10,000  & 100,000  \\ \hline
\end{tabular}
\end{center}
In limit terms, we would say that $\disp\lim_{x\to 0^+}\frac{1}{x}=\infty$. 

\vspace{3mm}

Similarly, we could numerically show that $\disp\lim_{x\to 0^-}\frac{1}{x}=-\infty$.

\newpage

\begin{center}
            \begin{tikzpicture}[scale=1.25]
                \begin{axis}[
                	axis x line=middle,
                	xmax=4.5, xmin=-4.5,
                	axis y line=center,
                	ymax=4.5, ymin=-4.5,
                	xlabel=$x$,ylabel=$y$,
                	axis line style=<->
                    ]
                    \addplot[name path=f,smooth,domain=-4.2:-0.25,color=blue,samples=100,thick,<->] {x^(-1)};
                    \addplot[name path=f,smooth,domain=0.25:4.2,color=blue,samples=100,thick,<->] {x^(-1)};
                    \draw (2.5,2) node[anchor=south] {\color{blue} $f(x)=\frac{1}{x}$};
                \end{axis}
            \end{tikzpicture}
        \end{center}

Unfortunately, there is no ``good'' and ``compact'' way of expressing our way through such limits. In MATH 1060, we use the following notational conventions.

\vspace{3mm}

\boxenv{Remark.}{Let $c \in \R$ with $c > 0$. We use the following notation.

$$ \frac{c}{\text{small } +} = +\infty $$ 

$$ \frac{c}{\text{small } -} = -\infty $$

\vspace{-3mm}}

\vspace{5mm}

\textbf{Summary.} Direct substitution is always a reasonable first step when trying to evaluate a limit. So far, we know that if direct substitution\dots
\begin{itemize}
	\item gives a real number, then no further work is needed to evaluate a limit.
	\item gives the $\frac{0}{0}$ indeterminate form, the limit usually exists, and we can apply techniques from Section 2.3 to evaluate it.
	\item gives 0 in the denominator only, we have an infinite limit. This limit technically does not exist, but we will investigate it anyway.
\end{itemize}

\subsection*{Strategy for Evaluating Infinite Limits}

If we know we have an infinite limit, we can use the following general strategy to determine whether the $f(x)$ values approach $+\infty$ or $-\infty$. For a limit $\disp\lim_{x\to c}f(x)$\dots
\begin{enumerate}
	\item[\tc{1}] Choose a test point close to $c$ based on the direction from which $x$ is approaching. (Choose a value smaller than $c$ if $x\to c^-$ and a value larger than $c$ if $x\to c^+$.)
	\item[\tc{2}] Plug the test point into the denominator to determine if the denominator approaches zero ``through the positives'' (small $+$) or ``through the negatives'' (small $-$).
\end{enumerate}

\newpage

\Example Evaluate the following limits.

$$\lim_{x\to 2^-}\frac{x+3}{x-2}\text{\hspace{10mm}and\hspace{10mm}}\lim_{x\to 2^+}\frac{x+3}{x-2}$$

\vspace{50mm}

\Example Evaluate the following limit.

\vspace{5mm}

\hspace{10mm} $\disp\lim_{x\to 0^+}\frac{x^2-3x+2}{x^3-2x^2}$

\vspace{33mm}

\Example Evaluate the following limit.

\vspace{5mm}

\hspace{10mm} $\disp\lim_{x\to 0^+}\frac{1+2x}{x^2}$

\vspace{33mm}

\Example Evaluate the following limit.

\vspace{5mm}

\hspace{10mm} $\disp\lim_{j\to 0^+}\frac{8^j}{1-3^j}$

\newpage

\subsection*{Vertical Asymptotes}

Infinite limits provide information about vertical asymptotes.

\vspace{3mm}

\boxenv{Definition.}{The line $x=c$ is called a \textbf{vertical asymptote} (V.A.) of a function $f$ if any one of the limits as $x$ approaches $c$ is infinite.}

\vspace{3mm}

For example, we saw earlier in this section that $\disp\lim_{x\to 0^+}\frac{1}{x}=\infty$. We can thus say that $x=0$ is a vertical asymptote of $f(x)=\disp\frac{1}{x}$.

\Example Find (and verify) the vertical asymptotes of $f(x)=\disp\frac{x+1}{2x^2+x-3}$.

\vspace{60mm}

\Example Prove that $x=\disp\frac{\pi}{2}$ is a vertical asymptote for $y=\tan x$.

\newpage

\Example Evaluate each of the following limits.

\begin{enumerate}
	\item[\tc{1}] $\disp\lim_{x\to 0^-}\frac{x^2+1}{\sin x}$
	
	\vspace{40mm}
	
	\item[\tc{2}] $\disp\lim_{x\to 2^+}\frac{-x}{5^x-25}$
	
	\vspace{40mm}
	
	\item[\tc{3}] $\disp\lim_{t\to\frac{3}{2}^+}\frac{t^2+6t-7}{2t-3}$
	
	\vspace{40mm}
	
	\item[\tc{4}] $\disp\lim_{x\to e^-}\frac{-\ln x}{x-e}$
\end{enumerate}

\end{document}