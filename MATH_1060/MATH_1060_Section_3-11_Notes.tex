\documentclass[12pt]{article}
%%% DOCUMENT FORMATTING %%%
\usepackage[margin=1in]{geometry}
\usepackage{enumitem}
\setlength{\parindent}{0pt}
\newcommand{\disp}{\displaystyle}

%%% HEADER %%%
\usepackage{fancyhdr}
\pagestyle{fancy}
\fancyhf{}
\lhead{MATH 1060}
\rhead{Vagnozzi}
\cfoot{\thepage}

%%% MATH NOTATION & SYMBOLS %%%
\usepackage{amssymb}
\usepackage{amsmath}
\newcommand{\R}{\mathbb{R}}
\newcommand{\N}{\mathbb{N}}
\newcommand{\Z}{\mathbb{Z}}
\newcommand{\lp}{\left(}
\newcommand{\rp}{\right)}
\newcommand{\ls}{\left[}
\newcommand{\rs}{\right]}
\newcommand{\lb}{\left\{}
\newcommand{\rb}{\right\}}
\newcommand{\arccot}{\text{arccot}}
\newcommand{\arccsc}{\text{arccsc}}
\newcommand{\arcsec}{\text{arcsec}} 

%%% TABLES %%%
\usepackage{colortbl}

%%% GRAPHS %%%
\usepackage{tikz}
\usepackage{pgfplots}
\pgfplotsset{compat=1.15}
\usepgfplotslibrary{fillbetween}
\usetikzlibrary{angles,quotes}

%%% ENVIRONMENTS %%%
\newcommand{\Example}{\paragraph{\Writinghand \hspace{0.1mm} Example.}}
\newcommand{\ExampleCont}{\paragraph{\Writinghand \hspace{0.1mm} Example (continued).}}
\newcommand{\boxenv}[2]{
	\fbox{
	\begin{minipage}{0.97\textwidth}
	\vspace{2mm}	
	\paragraph{#1} #2
	\vspace{2mm}
	\end{minipage}
	}}

%%% FUN THINGS %%%
\newcommand*\tc[1]{\tikz[baseline=(char.base)]{
            \node[shape=circle,draw,inner sep=2pt] (char) {#1};}}
\usepackage{marvosym}

%%% MISC %%%
\usepackage{hyperref}


\setcounter{page}{104}

\begin{document}
\section*{3.11: Related Rates of Change}

\boxenv{Learning Objectives.}{Upon successful completion of Section 3.11, you will be able to\dots
		
	\begin{itemize}[leftmargin=6mm]
		\item Answer conceptual questions involving related rates.
		\item Solve related rates problems involving geometry.
		\item Solve applications involving related rates.
	\end{itemize}
	\vspace{-4mm}
}

\vspace{5mm}

\subsection*{Introduction to Related Rates}

In \textbf{related rates} problems, we compute the rate of change of one quantity in terms of the rate of change of another quantity that may be more easily measured.

\vspace{3mm}

The procedure hinges on finding an equation that relates the two quantities, then using the chain rule to differentiate both sides \textit{with respect to time}.

\vspace{3mm}

\Example Suppose the radius of a circle is variable with time. Find $\disp\frac{dA}{dt}$, i.e.\ the instantaneous rate of change of the area of the circle with respect to time.

\vspace{50mm}

\paragraph{Strategy.} To solve related rates problems, you can follow the general strategy below.

\begin{enumerate}
\item[\tc{1}] Draw a diagram, where relevant, and indicate the quantities that vary.
\item[\tc{2}] Describe all variables and indicate units.
\item[\tc{3}] Specify the rate of change you wish to find and record any given information.
\item[\tc{4}] Find an equation involving the variable whose rate of change is to be found.
\item[\tc{5}] Differentiate the equation with respect to time.
\item[\tc{6}] Substitute all known quantities/rates and solve for the desired rate. (Include units!)
\end{enumerate}

\newpage

\Example A rock is thrown into a pond and the impact creates ripples in the form of concentric circles. If the radius of the outer circle is increasing at a rate of 3 in/sec, at what rate is the area of the outer ripple changing when the radius is 12 in?

\vspace{100mm}

\Example A cylindrical tank with a radius of 5 meters is being filled with water at a rate of 3 m$^3$/min. How fast is the height of the water increasing?

\newpage 

\Example A particle orbits a circle of radius one. As it passes through $\disp\lp\frac{1}{2},\frac{\sqrt{3}}{2}\rp$, its $y$-coordinate is decreasing at a rate of 4 units per second. Find $\disp\frac{dx}{dt}$ at the same point.

\vspace{100mm}

\Example A spherical balloon is being inflated and the radius is increasing by 2~ft/min. At what rate is the surface area of the balloon changing when the radius is 3 feet? 

$\lp SA_{\text{SPHERE}}=4\pi r^2\rp$

\newpage

\Example The balloon leaves the ground 500 feet away from an observer and rises vertically at the rate of 140 feet per minute. At what rate is the balloon's angle of elevation from the observer increasing at the instant the balloon is 500 feet above the ground?

\vspace{100mm}

\Example Suppose that both the radius and height of a cylinder (in cm) depend on time. Find the rate of change of the volume of the cylinder with respect to time (in seconds). 

\newpage

\Example The surface area of a cone with radius $r$ and height $h$ is given by the formula $S=\pi r\sqrt{r^2+h^2}$. Find the rate of change of the surface area with respect to time if\dots

\begin{enumerate}
\item[\tc{a}] $r$ is held constant.

\vspace{50mm}

\item[\tc{b}] $h$ is held constant.

\vspace{50mm}

\item[\tc{c}] both $r$ and $h$ are time-dependent.

\end{enumerate}

\newpage 

\Example A conical paper cup 8 inches across the top and 6 inches deep is full of water. The cup springs a leak at the bottom and loses water at the rate of 2 in$^3$/min. How fast is the water level dropping when the water is exactly 3 inches deep? $\lp V_{\text{cone}}=\frac{1}{3}\pi r^2h\rp$

\vspace{90mm}

\Example A police cruiser, approaching a right-angled intersection from the north, is chasing a speeding car that turned the corner and is moving straight east. When the cruiser is 0.6 miles north of the intersection and the car is 0.8 miles to the east, the police determine with radar that the distance between them is increasing at 20 mph. If the cruiser is moving at 60 mph at the instant of measurement, what is the speed of the speeding car?

\end{document}