\documentclass[12pt]{article}
\input{MATH_1060_Preamble.tex}

\setcounter{page}{176}

\begin{document}
\section*{5.3: The Fundamental Theorems of Calculus}

\boxenv{Learning Objectives.}{Upon successful completion of Section 5.3, you will be able to\dots
		
	\begin{itemize}[leftmargin=6mm]
		\item Answer conceptual questions involving the Fundamental Theorem of Calculus.
		\item Given a graph and areas of designated regions, evaluate area functions.
		\item Find and verify area functions.
		\item Evaluate definite integrals using the Fundamental Theorem of Calculus.
		\item Find areas bounded by functions.
		\item Evaluate derivatives of definite integrals.
		\item Evaluate area functions.
		\item Maximize values of definite integrals and solve integral equations.
	\end{itemize}
	\vspace{-4mm}
}

\vspace{5mm}

\subsection*{The First Fundamental Theorem of Calculus}

\vspace{3mm}

Recall from Section 4.9 that a function $F$ is called an \textbf{antiderivative} of $f$ on an interval $[a,b]$ if 
$$F'(x)=f(x)\text{ for all }x\in[a,b].$$

\vspace{5mm}

\boxenv{First Fundamental Theorem of Calculus.}{If $f$ is continuous on $[a,b]$, then the function $F$ defined on $[a,b]$ by
$$F(x)=\int_a^x f(t)\,dt$$

\vspace{2mm}

is continuous on $[a,b]$, differentiable on $[a,b]$, and has derivative
$$F'(x)=\frac{d}{dx}\int_a^x f(t)\,dt=f(x)\text{ for all }x\in[a,b].$$

In other words, $F'(x)=f(x)\,\Longrightarrow \, F$ is an antiderivative of $f$.}

\vspace{5mm}

The First Fundamental Theorem of Calculus (FTC1) establishes that \textit{differentiation} and \textit{integration} are, in a sense, inverse operations, much like multiplication and division. For example\dots

\newpage

\Example Find the local extrema for $F(x)=\disp\int_0^x\cos(t)\,dt$ on the interval $[0,2\pi]$.

\vspace{50mm}

\Example Find the intervals of increase/decrease and the intervals of concavity for 
$$F(x)=\int_1^x\lp t^2-7t+10\rp\,dt.$$

\vspace{50mm}

\Example Find the equation of the line tangent to
$$F(x)=\int_2^x\lp 3t^2-t\rp\,dt\text{ at }x=2.$$

\newpage

\boxenv{Chain Rule for Integral Functions.}{Let $f$ be a continuous function defined on $[a,b]$ and let $u$ be a continuous function defined on $[a,b]$ so that $a\leq u(x)\leq b$ for all $x\in [a,b]$. Furthermore, let $F$ be the function defined as follows.

$$F(x)=\int_a^{u(x)} f(t)\,dt$$

\vspace{2mm}

Then $F'(x)=f\big( u(x)\big)\cdot u'(x)$ for all $x\in [a,b]$.}

\vspace{3mm}

\Example Find the derivative of $F(x)=\disp\int_0^{x^3}\lp 2z-z^2\rp\,dz$.

\vspace{40mm}

\Example Find the derivative of each of the following functions.

\begin{itemize}
	\item[\tc{1}] $F(x)=\disp\int_1^{\ln x}e^t\,dt$
	
	\vspace{25mm}
	
	\item[\tc{2}] $F(x)=\disp\int_{\tan x}^0\frac{1}{1+k^2}\,dk$
	
	\vspace{25mm}
	
\end{itemize}

Where does the First Fundamental Theorem lead us? We will begin to see a pattern in the following example.

\newpage

\Example Use the First Fundamental Theorem of Calculus to evaluate $\disp\int_2^5\lp e^t+2t\rp\,dt$.

\vspace{60mm}

\subsection*{The Second Fundamental Theorem of Calculus}

We now introduce the Second Fundamental Theorem of Calculus (FTC2), sometimes referred to as the \textit{Fundamental Theorem of Integral Calculus}.

\vspace{5mm}

\boxenv{Second Fundamental Theorem of Calculus.}{Let $f$ be continuous on $[a,b]$ and \\ suppose that $F$ is an antiderivative for $f$ on the same interval. Then
$$\int_a^b f(x)\,dx=F(b)-F(a).$$

\vspace{-3mm}}

\vspace{5mm}

Recall the definition of the definite integral as a limit of a Riemann sum.

$$\lim_{n\to\infty}\sum_{i=1}^n f(x_i^*)\Delta x=\int_a^b f(x)\,dx.$$

\vspace{3mm}

The Second Fundamental Theorem of Calculus now allows us to evaluate integrals without dealing with this clunky limit!

\vspace{5mm}

\paragraph{Notation for Evaluating Definite Integrals.}

\vspace{35mm}

Because we are subtracting two antiderivatives, the ``$+C$'' may be ignored, as $C-C=0$.

\newpage

\Example We previously found that the area under $f(x)=x^2+1$ on $[0,2]$ is $\disp\frac{14}{3}$ through the limit of a Riemann sum. Using FTC2, now find this area directly.

\vspace{30mm}

\Example Evaluate the following definite integrals.

\begin{itemize}
	\item[\tc{1}] $\disp\int_{-1}^3 (2+x)\,dx$
	
	\vspace{30mm}
	
	\item[\tc{2}] $\disp\int_{-2}^{-1}\frac{2}{3x}\,dx$
	
	\vspace{30mm}
	
	\item[\tc{3}] $\disp\int_{\frac{\pi}{3}}^{\frac{\pi}{4}}\sec^2(x)\,dx$
	
	\vspace{30mm}
	
	\item[\tc{4}] $\disp\int_0^{1000}dx$
\end{itemize}

\newpage

\ExampleCont Evaluate the following definite integrals.
\begin{itemize}
	\item[\tc{5}] $\disp\int_4^9\frac{\sqrt{u^3}-1}{\sqrt{u^3}}\,du$
	
	\vspace{30mm}
	
	\item[\tc{6}] $\disp\int_0^{\frac{\pi}{2}}(2-3\sin\theta)\,d\theta$
	
	\vspace{30mm}
	
	\item[\tc{7}] $\disp\int_1^{\sqrt{3}}\frac{4}{1+x^2}\,dx$
	
	\vspace{30mm}

\end{itemize}

\Example Solve the equation for $x$: $\disp\int_0^x\lp 4t-9\rp\,dt=-4.$

\newpage

\Example If $x>0$, solve for $x$: $\disp\int_x^{x^2}\frac{1}{t}\,dt=\pi$.

\vspace{60mm}

\Example Evaluate the definite integrals.
\begin{itemize}
	\item[\tc{1}] $\disp\int_1^3\lp x^2+\frac{1}{x^2}\rp\,dx$
	
	\vspace{30mm}
	
	\item[\tc{2}] $\disp\int_1^5 2\sqrt{x-1}\,dx$
	
	\vspace{30mm}
	
	\item[\tc{3}] $\disp\int_1^4 t\lp \sqrt{t}+t^{-2}\rp\,dt$
	
	\vspace{30mm}
	
\end{itemize}

\newpage

\ExampleCont Evaluate the definite integrals.

\begin{itemize}
	\item[\tc{4}] $\disp\int_{\frac{1}{2}}^{\frac{\sqrt{3}}{2}}\frac{2}{\sqrt{1-j^2}}\,dj$
	
	\vspace{35mm}
	
	\item[\tc{5}] $\disp\int_2^1\lp\frac{1}{x^4}-\frac{1}{\sqrt{x^5}}\rp\,dx$
	
	\vspace{35mm}
\end{itemize}

\Example Find a linear function $f$ passing through $(0,-3)$ with $\disp\int_0^1 f(x)\,dx=1$.

\newpage

\Example Find the maximum and minimum values of $\disp\int_0^x \cos(t)\,dt$ on the interval $(0,2\pi)$.

\vspace{60mm}

Recall that, for FTC2 to apply, the function $f$ must be continuous on the interval $[a,b]$. Because of this condition, note that we cannot use FTC2 to evaluate the following.

$$\int_{-1}^2\frac{1}{x}\,dx\neq\ln |x|\Big|_{-1}^2=\ln(2)-\ln(1)=\ln(2)$$

Why not?

\vspace{10mm}

\paragraph{Connection Back to Limits.} The two major thematic elements of calculus, the \textit{derivative} and the \textit{integral}, share a common element: both are limits. In fact, calculus does not exist without the limit!

$$f'(x)=\textcolor{blue}{\lim_{h\to 0}}\frac{f(x+h)-f(x)}{h}$$

\vspace{2mm}

$$\int_a^b f(x)\,dx=\textcolor{blue}{\lim_{n\to\infty}}\sum_{i=1}^n f(x_i^*)\Delta x$$

\vspace{2mm}

\newpage

\subsection*{Approximating Integrals}

Broadly speaking, the main challenge of integration is finding an antiderivative. Sometimes, if the antiderivative is difficult to find, we may approximate a definite integral using the following method.

\vspace{3mm}

\boxenv{Integral Approximation.}{If $f$ is continuous on $[a,b]$, then
$$\int_a^b f(x)\,dx\approx f(a)(b-a)\text{ so long as }b\text{ is ``close'' to }a.$$

\vspace{-4mm}
}

\vspace{40mm}

\Example Calculate an approximation to $\disp\int_{\frac{\pi}{4}}^{\frac{\pi}{3}}x^2\cos(x)\,dx.$

\vspace{110mm}

Note: This integral requires a double application of ``integration by parts'' (MATH 1080).

\end{document}