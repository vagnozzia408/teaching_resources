\documentclass[12pt]{article}
%%% DOCUMENT FORMATTING %%%
\usepackage[margin=1in]{geometry}
\usepackage{enumitem}
\setlength{\parindent}{0pt}
\newcommand{\disp}{\displaystyle}

%%% HEADER %%%
\usepackage{fancyhdr}
\pagestyle{fancy}
\fancyhf{}
\lhead{MATH 1060}
\rhead{Vagnozzi}
\cfoot{\thepage}

%%% MATH NOTATION & SYMBOLS %%%
\usepackage{amssymb}
\usepackage{amsmath}
\newcommand{\R}{\mathbb{R}}
\newcommand{\N}{\mathbb{N}}
\newcommand{\Z}{\mathbb{Z}}
\newcommand{\lp}{\left(}
\newcommand{\rp}{\right)}
\newcommand{\ls}{\left[}
\newcommand{\rs}{\right]}
\newcommand{\lb}{\left\{}
\newcommand{\rb}{\right\}}
\newcommand{\arccot}{\text{arccot}}
\newcommand{\arccsc}{\text{arccsc}}
\newcommand{\arcsec}{\text{arcsec}} 

%%% TABLES %%%
\usepackage{colortbl}

%%% GRAPHS %%%
\usepackage{tikz}
\usepackage{pgfplots}
\pgfplotsset{compat=1.15}
\usepgfplotslibrary{fillbetween}
\usetikzlibrary{angles,quotes}

%%% ENVIRONMENTS %%%
\newcommand{\Example}{\paragraph{\Writinghand \hspace{0.1mm} Example.}}
\newcommand{\ExampleCont}{\paragraph{\Writinghand \hspace{0.1mm} Example (continued).}}
\newcommand{\boxenv}[2]{
	\fbox{
	\begin{minipage}{0.97\textwidth}
	\vspace{2mm}	
	\paragraph{#1} #2
	\vspace{2mm}
	\end{minipage}
	}}

%%% FUN THINGS %%%
\newcommand*\tc[1]{\tikz[baseline=(char.base)]{
            \node[shape=circle,draw,inner sep=2pt] (char) {#1};}}
\usepackage{marvosym}

%%% MISC %%%
\usepackage{hyperref}


\setcounter{page}{131}

\begin{document}
\section*{4.5: Applied Optimization}

\boxenv{Learning Objectives.}{Upon successful completion of Section 4.5, you will be able to\dots
		
	\begin{itemize}[leftmargin=6mm]
		\item Answer conceptual questions involving optimization.
		\item Optimize the sum, product, or sum of squares of two number given constraints.
		\item Solve optimization problems involving geometry or algebraically defined curves.
		\item Solve applications involving optimization.
	\end{itemize}
	\vspace{-4mm}
}

\vspace{5mm}

\subsection*{Optimization Problems}

In \textbf{applied optimization} problems, the goal is to \textit{optimize} (i.e.\ maximize or minimize) a quantity, subject to one or more constraints.

\Example What is the maximum value of $x+y$ if we must have $y+x^2=4$?

\vfill

\boxenv{Definition.}{In applied optimization problems, the \textbf{objective function} is the function you wish to maximize or minimize.}

\vspace{3mm}

\boxenv{Definition.}{For an applied optimization problem in a single variable, an \textbf{optimizer} is a value of the variable at which the function is maximized or minimized.}

\newpage

\paragraph{Strategy.} To solve an optimization problem, we follow this general procedure.

\begin{enumerate}
\item[\tc{1}] Draw a diagram if applicable and define all variables with units.
\item[\tc{2}] Identify the objective function and any constraints that apply.
\item[\tc{3}] Use the constraint equation to convert the objective function from a multivariable equation to a \textit{single}-variable equation.
\item[\tc{4}] Find the domain of the single-variable objective function.
\item[\tc{5}] Find critical points and locate possible optimizers. Discard any critical points that do not make sense in context.
\item[\tc{6}] Verify that you have located the global extremum using the First or Second Derivative Test.
\item[\tc{7}] Solve for the remaining variable and summarize your results, paying close attention to the quantity for which the problem is asking.
\end{enumerate}

\vspace{3mm}

\Example Suppose $x$ and $y$ are positive numbers. What is the maximum value of $e^{x+y}$ if we must have that $x^2+y^2=1$?

\newpage

\Example A cylindrical tank must hold $4\pi$ m$^3$ of water. What are the dimensions of the tank that minimize its surface area? (The surface area of a cylinder is $S=2\pi r^2+2\pi rh$.)

\newpage

\Example A small paper container is constructed as follows: Squares of length $x$ are cut from the corners of a $12\times 12$ cm piece of cardstock. The remaining sides are then folded up and taped together. What are the dimensions of the square cutouts that maximize the volume of the container?

\newpage

\paragraph{The Distance Formula.} It is common in applied settings to want to maximize or minimize \textit{distance}. Recall how we can find the distance between two points $A(x_1,y_1)$ and $B(x_2,y_2)$.

\vspace{70mm}

\paragraph{Optimizing the Square.} In solving an optimization problem with an objective function~$f$, it is sometimes easier to work with $f^2$ versus $f$. The following theorem can help us.

\vspace{5mm}

\boxenv{Theorem.}{Suppose that $f$ is nonnegative, i.e.\ $f(x)\geq 0$ for all $x$ in the domain of $f$. Then we have the following equivalence.

$$x_0\text{ minimizes (maximizes) } f^2 \,\Longleftrightarrow \, x_0\text{ minimizes (maximizes) }f$$

\vspace{-3mm}
}

\vspace{5mm}

Optimizing the quantity $f^2$ instead of $f$ is a process called \textbf{optimizing the square}.

\vspace{5mm}

Note that the square root function is always nonnegative on its domain, i.e.
$$g(x)=\sqrt{f(x)}\geq 0 \text{ for all $x$ in the domain of }g.$$

Hence, to find the optimizer of $g$, we can equivalently find the optimizer of $g^2=\lp\sqrt{f}\rp^2=f$.

\newpage

\Example If you start at the point $(1,1)$, to what point on the line $y=2x$ can we draw a line segment with minimum distance?

\end{document}