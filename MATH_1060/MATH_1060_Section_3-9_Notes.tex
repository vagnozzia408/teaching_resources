\documentclass[12pt]{article}
%%% DOCUMENT FORMATTING %%%
\usepackage[margin=1in]{geometry}
\usepackage{enumitem}
\setlength{\parindent}{0pt}
\newcommand{\disp}{\displaystyle}

%%% HEADER %%%
\usepackage{fancyhdr}
\pagestyle{fancy}
\fancyhf{}
\lhead{MATH 1060}
\rhead{Vagnozzi}
\cfoot{\thepage}

%%% MATH NOTATION & SYMBOLS %%%
\usepackage{amssymb}
\usepackage{amsmath}
\newcommand{\R}{\mathbb{R}}
\newcommand{\N}{\mathbb{N}}
\newcommand{\Z}{\mathbb{Z}}
\newcommand{\lp}{\left(}
\newcommand{\rp}{\right)}
\newcommand{\ls}{\left[}
\newcommand{\rs}{\right]}
\newcommand{\lb}{\left\{}
\newcommand{\rb}{\right\}}
\newcommand{\arccot}{\text{arccot}}
\newcommand{\arccsc}{\text{arccsc}}
\newcommand{\arcsec}{\text{arcsec}} 

%%% TABLES %%%
\usepackage{colortbl}

%%% GRAPHS %%%
\usepackage{tikz}
\usepackage{pgfplots}
\pgfplotsset{compat=1.15}
\usepgfplotslibrary{fillbetween}
\usetikzlibrary{angles,quotes}

%%% ENVIRONMENTS %%%
\newcommand{\Example}{\paragraph{\Writinghand \hspace{0.1mm} Example.}}
\newcommand{\ExampleCont}{\paragraph{\Writinghand \hspace{0.1mm} Example (continued).}}
\newcommand{\boxenv}[2]{
	\fbox{
	\begin{minipage}{0.97\textwidth}
	\vspace{2mm}	
	\paragraph{#1} #2
	\vspace{2mm}
	\end{minipage}
	}}

%%% FUN THINGS %%%
\newcommand*\tc[1]{\tikz[baseline=(char.base)]{
            \node[shape=circle,draw,inner sep=2pt] (char) {#1};}}
\usepackage{marvosym}

%%% MISC %%%
\usepackage{hyperref}


\setcounter{page}{93}

\begin{document}
\section*{3.9: Derivatives of Log and Exponential Functions}

\boxenv{Learning Objectives.}{Upon successful completion of Section 3.9, you will be able to\dots
		
	\begin{itemize}[leftmargin=6mm]
		\item Answer conceptual questions involving derivatives of logarithmic and exponential \\ functions.
		\item Find derivatives involving logarithms and exponentials.
		\item Find equations of tangent lines for exponential, logarithmic, and power functions.
		\item find derivatives using logarithmic differentiation.
		\item Find higher order derivatives of functions involving logarithms and exponentials.
		\item Evaluate limits of logarithmic and exponential functions using the definition of the derivative.
	\end{itemize}
	\vspace{-4mm}
}

\vspace{5mm}

\subsection*{Logarithmic and Exponential Functions}

In this section, we will learn how to take derivatives of logarithmic and exponential functions.

\begin{center}
            \begin{tikzpicture}
                \begin{axis}[
                	axis x line=middle,
                	xmax=4.5, xmin=-3.5,
                	axis y line=center,
                	ymax=9, ymin=-5.5,
                	xlabel=$x$,ylabel=$y$,
                	axis line style=<->
                    ]
                    \addplot[name path=f,smooth,domain=-3:2,color=blue,samples=100,<->,thick] {e^x};
                    \addplot[name path=f,smooth,domain=0.01:4,color=red,samples=100,<->,thick] {ln(x)};
                \end{axis}
            \end{tikzpicture}
        \end{center}
        
\subsection*{Derivatives of Exponential Functions}

Let $y=b^x$ where $b>0$. The derivative of $y$ is\dots

$$\frac{d}{dx}\lp b^x\rp=b^x\ln b$$

\vspace{3mm}

In the case that $b=e$, the same rule applies.

\newpage

Remember when applying this derivative rule to be mindful of the \textit{chain rule}.

$$\frac{d}{dx}\lp b^{g(x)}\rp=b^{g(x)}\ln b\cdot g'(x)$$

\vspace{3mm}

\textit{Proof of $\disp\frac{d}{dx}\lp b^x\rp=b^x\ln b$.} Using the limit definition of the derivative, the proof is as follows.

\begin{align*}
	\frac{d}{dx}\big(b^x\big) &= \lim_{h\to 0}\frac{b^{x+h}-b^x}{h} \\ 
				      &= \lim_{h\to 0}\frac{b^xb^h - b^x}{h} \\ 
				      &= \lim_{h\to 0}\frac{b^x(b^h - 1)}{h} \\ 
				      &= b^x\underbrace{\lim_{h\to 0}\frac{b^h - 1}{h}}_{=\ln b} \\ 
				      &= b^x\ln b
\end{align*}
\begin{flushright}
	$\square$
\end{flushright}


Care must be taken to ensure that you do not conflate the derivative rule for \textit{power} functions with that for \textit{exponential} functions.

$$\frac{d}{dx}\lp b^x\rp \neq xb^{x-1}$$

\vspace{5mm}

A general rule of thumb to help determine which rule is appropriate is to note where the variable $x$ appears in a function with an exponent.
\begin{itemize}
	\item If $x$ is the \textbf{base} of the function, use the \textbf{power rule}.
	
	\vspace{15mm}
	
	\item If $x$ is the \textbf{exponent} of the function, use the derivative rule for exponential functions.
	
	\vspace{15mm}
	
\end{itemize}

\newpage

\subsection*{Derivatives of Logarithmic Functions}

Let $y=\log_b x$ where $b>0$. The derivative of $y$ is\dots

$$\frac{d}{dx}\lp\log_b x\rp=\frac{1}{x\ln b}$$

\vspace{3mm}

In the case that $b=e$, we have the following result.

\vspace{20mm}

Again, be careful to apply the chain rule when appropriate.

$$\frac{d}{dx}\lp\log_b g(x)\rp=\frac{1}{g(x)\ln b}\cdot g'(x)$$

\vspace{5mm}

\textit{Proof of $\disp\frac{d}{dx}\lp\log_b x\rp=\frac{1}{x\ln b}$.}

\vspace{50mm}

\Example Find $\disp\frac{dy}{dx}$ given that $y=x^2+2^x+\log_2 x-\ln x$.

\vspace{30mm}

\Example Find $y'$ given that $y=10^{x^3-3x}$.

\newpage

\Example Find the derivative of $y=\disp\frac{2^x}{3^x+4^x}$.

\vspace{60mm}

\Example Find $\disp\frac{dy}{dx}$ given that $y=\ln(kx)$ where $k\in\R$, $k\neq 0$.

\vspace{60mm}

\Example Find $f'$ given that $f(x)=\ln\lp xe^x-e^x\rp$.

\newpage

\paragraph{Applying Logarithm Laws.} We can often use properties of logarithms to simplify functions before finding the derivative.

$$\log_b\lp f\times g\rp=\log_b f+\log_b g$$

$$\log_b\lp f\div g\rp = \log_b f-\log_b g$$

\vspace{3mm}

\Example Find the derivative of $y=\ln\lp\sin x\cos x\rp$.

\vspace{45mm}

\Example Find the derivative of $f(x)=\ln\lp\disp\frac{x^2+1}{x^2-1}\rp$.

\vspace{45mm}

\Example Find $h'(\ln 2)$ given that $h(x)=\ln\lp e^{2x}+1\rp$.

\newpage

\Example Find the derivative of $y=\log_4\lp e^{2x}+3^{x^2}\rp$.

\vspace{50mm}

\subsection*{Logarithmic Differentiation}

Earlier, we said that we can determine whether to use the power rule or the derivative rule for exponential functions based on whether the variable $x$ appears in the base or the exponent of the function. What should we do if $x$ appears in the base \textit{and} the exponent?

$$\frac{d}{dx}\lp x^x\rp \neq x\cdot x^{x-1}$$

$$\frac{d}{dx}\lp x^x\rp x^x\cdot \ln x$$

\vspace{3mm}

For such functions, we will use a technique called \textbf{logarithmic differentiation}. This technique uses the logarithm law that
$$\ln\lp x^r\rp=r\ln (x).$$

\vspace{5mm}

\paragraph{Applying Logarithmic Differentiation.} The general process for applying this technique is to do the following\dots
\begin{enumerate}
	\item[\tc{1}] Set the function equal to $y$ and take the natural logarithm of both sides.
	\item[\tc{2}] Use the logarithm law above to manipulate the equation.
	\item[\tc{3}] Apply implicit differentiation.
\end{enumerate}

\newpage

\Example Find the derivative of $y=x^x$.

\vspace{100mm}

\Example Find the derivative of $y=x^{\sin\lp \pi x\rp}$.
\end{document}