\documentclass[12pt]{article}
\input{MATH_1060_Preamble.tex}

\setcounter{page}{191}

\begin{document}
\section*{5.5: The Substitution Method}

\boxenv{Learning Objectives.}{Upon successful completion of Section 5.5, you will be able to\dots
		
	\begin{itemize}[leftmargin=6mm]
		\item Answer conceptual questions involving the Substitution Rule.
		\item Evaluate indefinite integrals using substitution when the substitution is given.
		\item Evaluate indefinite integrals using substitution when the substitution is not given.
		\item Evaluate definite integrals using substitution.
		\item Find areas of regions using integration that requires substitution.
	\end{itemize}
	\vspace{-4mm}
}

\vspace{5mm}

\subsection*{U-Substitution}

When we differentiate a composite function, we do so using the \textbf{chain rule}.

$$f\big( g(x)\big)\,\Longrightarrow\, f'\big( g(x)\big)\cdot g'(x)$$

\vspace{5mm}

In integration, we often need to apply the chain rule in reverse.

$$f'\big( g(x)\big)\cdot g'(x)\,\Longrightarrow\, f\big( g(x)\big)+C$$

\vspace{5mm}

In general, applying the chain rule in reverse may not be straightforward. In such cases, the process is usually facilitated by making what we call a ``$u$-substitution.''

\vspace{5mm}

\paragraph{Strategy for Indefinite Integrals.} Suppose we have an indefinite integral of the form 
$$\disp\int f\big(\textcolor{blue}{g(x)}\big) \textcolor{red}{g'(x)}\,dx.$$
\begin{itemize}
\item[\tc{1}] Set $\textcolor{blue}{u=g(x)}$ so that $\textcolor{red}{du=g'(x)\,dx}$.
\item[\tc{2}] The integral may now be expressed as $\disp\int f(\textcolor{blue}{u})\,\textcolor{red}{du}$.
\item[\tc{3}] If $F$ is an antiderivative for $f$, then $\disp\int f(\textcolor{blue}{u})\,\textcolor{red}{du}=F(\textcolor{blue}{u})+C$.
\item[\tc{4}] We can now substitute $\textcolor{blue}{u=g(x)}$ into $F$ to obtain the final result.

$$\int f\big( \textcolor{blue}{g(x)}\big) \textcolor{red}{g'(x)\,dx}=F\big(\textcolor{blue}{g(x)}\big) + C$$
\end{itemize}

\newpage

\Example Apply the $u$-substitution method to evaluate $\disp\int \textcolor{red}{x}\sin(\textcolor{blue}{2x^2})\,\textcolor{red}{dx}$.

\vspace{50mm}

\Example Apply the $u$-substitution method to evaluate the indefinite integrals.
\begin{itemize}
	\item[\tc{1}] $\disp\int x\lp x^2+1\rp^4\,dx$
	
	\vspace{55mm}
	
	\item[\tc{2}] $\disp\int\frac{x}{\sqrt{4-9x^2}}\,dx$
	
\end{itemize}

\newpage
\ExampleCont Apply the $u$-substitution method to evaluate the indefinite integrals.
\begin{itemize}
	\item[\tc{3}] $\disp\int\tan x\,dx$
	
	\vspace{55mm}
	
	\item[\tc{4}] $\disp\int\sqrt{2x+1}\,dx$
	
	\vspace{55mm}
	
	\item[\tc{5}] $\disp\int \frac{1}{1+16x^2}\,dx$
\end{itemize}

\newpage

\Example Evaluate the integral $\disp\int\sec^3(x)\tan(x)\,dx$.

\vspace{55mm}

\paragraph{Strategy for Definite Integrals.} For definite integrals, the process of $u$-substitution is nearly the same, but we must modify the limits of integration to correspond with the change of variable.

\vspace{5mm}

In other words, if $\textcolor{blue}{u=g(x)}$ and $\textcolor{red}{du=g'(x)\,dx}$, then

$$\int_a^b f\big( \textcolor{blue}{g(x)}\big) \textcolor{red}{g'(x)}\,dx=\int_{u(a)}^{u(b)} f(\textcolor{blue}{u}\big)\,\textcolor{red}{du}.$$

\vspace{3mm}

\Example Evaluate the definite integral $\disp\int_2^e\frac{1}{x\lp\ln x\rp^2}\,dx$.

\newpage

\Example Evaluate the integral $\disp\int_{\ln\sqrt{3}}^{0}\frac{e^x}{1+e^{2x}}\,dx$.

Hint: $e^{2x}=\lp e^x\rp^2$.

\vspace{55mm}

\Example Evaluate the integral $\disp\int_1^2\frac{10x^2}{x^3+2}\,dx$.

\vspace{60mm}

\Example Evaluate the integral $\disp\int_2^4 x\sqrt{4-x}\,dx$.

\newpage

\Example Evaluate the integral $\disp\int_2^5\frac{x+1}{x^2+2x}\,dx$.

\vspace{60mm}

\Example Evaluate the integral $\disp\int_\frac{\pi}{4}^\frac{3\pi}{4}\frac{\sin x}{\sqrt{1-\cos^2x}}\,dx$ in two ways.

\vspace{2mm}

Hint: For the second way, use the Pythagorean trig identity $\sin^2 x+\cos^2 x=1$.

\newpage

\Example Evaluate the integral $\disp\int_0^{\ln 2}\frac{e^{2x}}{1+e^x}\,dx$.

\vfill

\paragraph{Looking Ahead.} We often view differentiation and integration as ``inverses'' of one another, but in reality, integration is often a much more involved process than differentiation. Integration requires a collection of tools and ``tricks,'' and the substitution method is only one of those. If you take MATH 1080, you'll learn more of these integration techniques!
\end{document}