%% In the documentclass line, replace "noanswers" with "answers" to view the key.

\documentclass[noanswers]{exam}
\usepackage[utf8]{inputenc}

\title{Chapter 1 Practice Problems}
\author{Sections 1.4--1.6}
\date{STAT 2300}

\usepackage[bottom=2.2cm, left=2.2cm, right=2.2cm, top=2.2cm]{geometry}
%\usepackage[paperheight=11in, paperwidth=17in, margin=1in]{geometry}
\usepackage{dsfont}
\usepackage{amsmath}
\usepackage{amssymb}
\usepackage{amsthm}
\usepackage{array}
\usepackage{stmaryrd}
\usepackage{pgfplots}
\pgfplotsset{width=10cm,compat=1.9}
\usepackage{multicol}
\setlength{\columnsep}{1in}
\usepackage{nicefrac}

\usepackage{multirow}
\usepackage{enumitem}[shortlabels]
\usepackage{tabu}
\definecolor{purp}{RGB}{102,0,204}
\usepackage{tabularx}
\newcolumntype{C}{>{\centering\arraybackslash $}X<{$}}
\usepackage{wrapfig}
\usepackage[export]{adjustbox}


\makeatletter
\pagestyle{headandfoot}
\firstpageheader{\@date}{\@title}{\@author}
\firstpageheadrule
\runningfootrule
\runningfooter{}{\thepage\ / \numpages}{\@title}
\makeatother

\newcommand{\abs}[1]{\left|#1\right|}
\newcommand{\mat}[4]{\left( \begin{tabular}{>{$}c<{$} >{$}c<{$}} #1&#2 \\ #3&#4 \end{tabular} \right)}
\newcommand{\msc}[1]{\mathds{#1}}
\newcommand{\Z}{\mathds{Z}}
\newcommand{\R}{\mathds{R}}
\newcommand{\N}{\mathds{N}}
\newcommand{\Q}{\mathds{Q}}
\newcommand{\C}{\mathds{C}}
\newcommand{\so}{\implies}
\newcommand{\set}[2]{\left\{ #1 \:|\: #2 \right\}}
\newcommand{\bso}{\Longleftarrow}
\newcommand{\ra}{\rightarrow}
\newcommand{\gen}[1]{\left\langle #1 \right\rangle}
\newcommand{\olin}[1]{\overline{#1}}
\newcommand{\Img}[1]{\text{Im}\left(#1\right)}
\newcommand{\llra}{\longleftrightarrow}
\newcommand{\lra}{\longrightarrow}
\newcommand{\xra}[1]{\xrightarrow{#1}}
\newcommand{\wo}{\setminus}
\newcommand{\mcal}[1]{\mathcal{#1}}
\newcommand{\Aut}[1]{\text{Aut}\left(#1\right)}
\newcommand{\Inn}[1]{\text{Inn}\left(#1\right)}
\newcommand{\syl}[2]{\text{Syl}_{#1}(#2)}
\newcommand{\norm}[1]{\left\|#1\right\|}
\newcommand{\infnorm}[1]{\left\|#1\right\|_{\infty}}
\newcommand{\xn}{\{x_n\}}
\newcommand{\sig}{\sigma}
\newcommand{\id}{\text{id}}
\newcommand{\ep}{\epsilon}
\newcommand{\st}{\text{ s.t. }}
\newcommand{\ran}[1]{\text{Ran}(#1)}
\newcommand{\nCr}[2]{\binom{#1}{#2}}
\newcommand{\Exr}[1]{\paragraph{Exercise #1:}}
\newcommand{\pg}{\paragraph{}}
\newcommand{\ulin}[1]{\underline{#1}}
\newcommand{\tc}[1]{\textcolor{purp}{#1}}

% Solution Specs
\unframedsolutions
\renewcommand{\solutiontitle}{}
\SolutionEmphasis{\color{purp}}
\CorrectChoiceEmphasis{\color{purp}\bfseries}

%\begin{solution}[\stretch{1}]
%	hurp durp flurp
%\end{solution}

%\pagestyle{empty}

\begin{document}

%\noindent\begin{tabular}{@{}p{.3in}p{3in}@{}}
%Name: & \hrulefill
%\end{tabular}


\begin{questions} 

	\question You've been assigned the task of studying people's opinions on the latest full-house showing of \textit{The Sound of Music} at your local theatre. Because you took Statistical Methods, you know that you should generate a random sample to be representative of the opinions of all 240 play attendees. Identify the \textbf{sampling method} used in each of the strategies described below.  
	
	\vspace{3mm}

	\begin{parts}
	
		\part The theatre has three equally sized seating sections: left, center, and right. You randomly choose one person from the left section, one person from the center section, and one person from the right section, then repeat until you have selected 24 people.
	
		\begin{solution}[\stretch{1}]
	
			\vspace{3mm}		
		
			Stratified sampling

			\vspace{3mm}		
			
		\end{solution}
	
		\part You place all of the play attendees' tickets into a box and shuffle them around, then draw 24 numbers from the hat at random.
	
		\begin{solution}[\stretch{1}]
		
			\vspace{3mm}		
				
			Simple random sampling

			\vspace{3mm}		
				
		\end{solution}
	
		\part You select the first 24 people to leave the theatre.
		
		\begin{solution}[\stretch{1}]

			\vspace{3mm}		
			
			Convenience sampling

			\vspace{3mm}		
			
		\end{solution}
	
		\part Starting with the first seat in the front row, you select every fifth person until you have selected 24 people.
	
		\begin{solution}[\stretch{1}]
	
			\vspace{3mm}
	
			Systematic sampling
	
			\vspace{3mm}
	
		\end{solution}
	
		\part The theatre has ten rows of 24 seats. You throw a ten-sided die and choose the people in the row indicated by the die to be in your sample.
	
		\begin{solution}[\stretch{1}]
		
			\vspace{3mm}

			Cluster sampling
		
			\vspace{3mm}	
	
		\end{solution}
		
		\part Of the above five sampling methods, which are appropriate statistical techniques for an inferential statistical study? Why?
	
		\begin{solution}[\stretch{1}]
		
			\vspace{3mm}

			All but convenience sampling are appropriate sampling methods. Convenience sampling is non-random and can introduce bias into the study.
		
			\vspace{3mm}	
	
		\end{solution}
	
	\end{parts}

	\question For each of the following, describe how the specified type of \textbf{bias} could be present.

\vspace{3mm}

	\begin{parts}
	
		\part A high school wishes to learn whether stress about school increases a student's likelihood of substance abuse. In each student's required meeting with their guidance counselor, they are asked whether stress over their schoolwork has caused them to use any illegal substances. How might \textbf{response bias} be present in this situation?
		
		\begin{solution}[\stretch{1}]
			\vspace{3mm}
					
			Students may be dishonest in their responses about using illegal substances for fear of repercussions.
			
			\vspace{3mm}		
		\end{solution}
	
\newpage	
	
		\part An airline wants to conduct a passenger satisfaction survey, focusing on how comfortable they find their flight. They use a systematic sample, starting with the third passenger and then selecting every sixth passenger for the sample. The result is a sample of people all sitting in an aisle seat. While this might make it easy to gather their responses, how could \textbf{sampling bias} be present in this example?
	
		\begin{solution}[\stretch{1}]		
			\vspace{3mm}		
				
			People sitting in aisle seats may perceive themselves as having more room and being more comfortable than passengers in middle seats, excluding an important group of the population.

			\vspace{3mm}		
		\end{solution}
	
		\part Google Reviews can be really helpful for the average person looking for a new restaurant, but tend to attract people with the most extreme experiences, either positive or negative. How is \textbf{non-response bias} at play in this scenario?
		
		\begin{solution}[\stretch{1}]
			\vspace{3mm}		
			
			Those who respond have extreme experiences that will skew the results, while non-respondents will have more moderate perspectives that likely reflect those of the average individual.

			\vspace{3mm}		
		\end{solution}
	
	\end{parts}

	
	\question To determine the effectiveness of this season's flu vaccine, medical researchers recruit 100 adult participants from South Carolina for a study. At the start of the flu season, the researchers randomly select 50 of the participants to receive the flu vaccine and 50 to receive a placebo. At the end of flu season, the proportion of participants in each treatment group who contracted the is flu compared.
	
	\vspace{3mm}
	
	\begin{parts}
	
		\part What is the \textbf{population} being studied?
		
		\begin{solution}[\stretch{1}]
			\vspace{3mm}		
			
			All adults in South Carolina

			\vspace{3mm}		
		\end{solution}
		
		\part What is the \textbf{factor} (explanatory variable)?
		
		\begin{solution}[\stretch{1}]
			\vspace{3mm}
					
			Whether or not the participant received the flu shot
			
			\vspace{3mm}		
		\end{solution}
	
		\part What are the \textbf{treatments}?
	
		\begin{solution}[\stretch{1}]		
			\vspace{3mm}		
				
			Placebo vaccine, flu vaccine

			\vspace{3mm}		
		\end{solution}
	
		\part What is the \textbf{response variable}?
		
		\begin{solution}[\stretch{1}]
			\vspace{3mm}		
			
			Whether or not the participant got the flu

			\vspace{3mm}		
		\end{solution}
		
		\part What is \textbf{control group} in this study?
		
		\begin{solution}[\stretch{1}]
			\vspace{3mm}		
			
			The placebo group

			\vspace{3mm}		
		\end{solution}
	
		\part Identify the \textbf{experimental units}.
		
		\begin{solution}[\stretch{1}]
			\vspace{3mm}		
			
			The 100 adults recruited for the study

			\vspace{3mm}		
		\end{solution}	
		
		\part Which is present in this study, \textbf{random assignment} or \textbf{random sampling}? What can this element of randomness tell us about the conclusions we can draw from this study? (Hint: See the table on page 11 of the Lecture Notes.)
		
		\begin{solution}[\stretch{1}]
			\vspace{3mm}		
			
			Random assignment to the two treatments is present, but random selection is not. We can make cause-and-effect conclusions, but only among subjects similar to those in the study.

			\vspace{3mm}		
		\end{solution}
	
	\end{parts}


\end{questions}

%-----------------------------------------------------------------------------%

\end{document}
