%% In the documentclass line, replace "noanswers" with "answers" to view the key.

\documentclass[noanswers]{exam}
\usepackage[utf8]{inputenc}

\title{Chapter 5 Practice Problems}
\author{Sections 5.2--5.3}
\date{STAT 2300}

\usepackage[bottom=2.2cm, left=2.2cm, right=2.2cm, top=2.2cm]{geometry}
%\usepackage[paperheight=11in, paperwidth=17in, margin=1in]{geometry}
\usepackage{dsfont}
\usepackage{amsmath}
\usepackage{amssymb}
\usepackage{amsthm}
\usepackage{array}
\usepackage{stmaryrd}
\usepackage{pgfplots}
\pgfplotsset{width=10cm,compat=1.9}
\usepackage{multicol}
\setlength{\columnsep}{1in}
\usepackage{nicefrac}

\usepackage{multirow}
\usepackage{enumitem}[shortlabels]
\usepackage{tabu}
\definecolor{purp}{RGB}{102,0,204}
\usepackage{tabularx}
\newcolumntype{C}{>{\centering\arraybackslash $}X<{$}}
\usepackage{wrapfig}
\usepackage[export]{adjustbox}


\makeatletter
\pagestyle{headandfoot}
\firstpageheader{\@date}{\@title}{\@author}
\firstpageheadrule
\runningfootrule
\runningfooter{}{\thepage\ / \numpages}{\@title}
\makeatother

\newcommand{\abs}[1]{\left|#1\right|}
\newcommand{\mat}[4]{\left( \begin{tabular}{>{$}c<{$} >{$}c<{$}} #1&#2 \\ #3&#4 \end{tabular} \right)}
\newcommand{\msc}[1]{\mathds{#1}}
\newcommand{\Z}{\mathds{Z}}
\newcommand{\R}{\mathds{R}}
\newcommand{\N}{\mathds{N}}
\newcommand{\Q}{\mathds{Q}}
\newcommand{\C}{\mathds{C}}
\newcommand{\so}{\implies}
\newcommand{\set}[2]{\left\{ #1 \:|\: #2 \right\}}
\newcommand{\bso}{\Longleftarrow}
\newcommand{\ra}{\rightarrow}
\newcommand{\gen}[1]{\left\langle #1 \right\rangle}
\newcommand{\olin}[1]{\overline{#1}}
\newcommand{\Img}[1]{\text{Im}\left(#1\right)}
\newcommand{\llra}{\longleftrightarrow}
\newcommand{\lra}{\longrightarrow}
\newcommand{\xra}[1]{\xrightarrow{#1}}
\newcommand{\wo}{\setminus}
\newcommand{\mcal}[1]{\mathcal{#1}}
\newcommand{\Aut}[1]{\text{Aut}\left(#1\right)}
\newcommand{\Inn}[1]{\text{Inn}\left(#1\right)}
\newcommand{\syl}[2]{\text{Syl}_{#1}(#2)}
\newcommand{\norm}[1]{\left\|#1\right\|}
\newcommand{\infnorm}[1]{\left\|#1\right\|_{\infty}}
\newcommand{\xn}{\{x_n\}}
\newcommand{\sig}{\sigma}
\newcommand{\id}{\text{id}}
\newcommand{\ep}{\epsilon}
\newcommand{\st}{\text{ s.t. }}
\newcommand{\ran}[1]{\text{Ran}(#1)}
\newcommand{\nCr}[2]{\binom{#1}{#2}}
\newcommand{\Exr}[1]{\paragraph{Exercise #1:}}
\newcommand{\pg}{\paragraph{}}
\newcommand{\ulin}[1]{\underline{#1}}
\newcommand{\tc}[1]{\textcolor{purp}{#1}}

% Solution Specs
\unframedsolutions
\renewcommand{\solutiontitle}{}
\SolutionEmphasis{\color{purp}}
\CorrectChoiceEmphasis{\color{purp}\bfseries}
\setlength\fillinlinelength{0in}

%\begin{solution}[\stretch{1}]
%	hurp durp flurp
%\end{solution}

%\pagestyle{empty}

\renewcommand{\arraystretch}{1.5}
\usepackage{cancel}

\begin{document}

\noindent\begin{tabular}{@{}p{.3in}p{3in}@{}}
Name: & \hrulefill
\end{tabular}

\vspace{3mm}

\begin{questions} 
	
	\vspace{3mm}
	
	\question Consider the following table of data collected from a sample of 36 students in STAT 3090 last fall. Students were asked whether they were a morning person or a night person, as well as what their hot beverage of choice is. Students could only choose one response for each variable.
	
	\begin{center}
\begin{tabular}{|c|c|c|c|c|}
\hline
 & \hspace{3mm} Coffee (C) \hspace{3mm} & \hspace{4mm} Tea (T) \hspace{4mm} & Hot Cocoa (H) & TOTAL\\
 \hline
 Morning Person (M) & 3 & 3 & 2 & 8 \\
 \hline
 Night Person (N) & 17 & 3 & 8 & 28 \\
\hline
TOTAL & 20 & 6 & 10 & 36 \\
\hline
\end{tabular}
\end{center}

For each of the following, use proper probability \textbf{notation}, write the associated \textbf{fraction}, and express your final answer as a \textbf{decimal} number rounded to four places. If you apply a probability rule (such as the Addition Rule or Complement Rule), \textbf{show the formula} in your work. 

\vspace{3mm}

Find the probability that a randomly selected student from the class\dots 

\begin{parts}
	
	\vspace{3mm}
	
	\part Prefers tea
	
	\begin{solution}[\stretch{1}]
	\vspace{3mm}
	$P(T)=\frac{6}{36}=0.1667$
	\vspace{3mm}
	\end{solution}
	
	\part Is a morning person \textbf{and} prefers coffee
	
	\begin{solution}[\stretch{1}]
	\vspace{3mm}
	$P(M\text{ and } C)=\frac{3}{36}=0.0833$
	\vspace{3mm}
	\end{solution}
	
	\part Is a night person \textbf{and} prefers hot cocoa
	
	\begin{solution}[\stretch{1}]
	\vspace{3mm}
	$P(N\text{ and } H)=\frac{8}{36}=0.2222$
	\vspace{3mm}
	\end{solution}
	
	\part Prefers tea \textbf{or} hot cocoa
	
	\begin{solution}[\stretch{1}]
	\vspace{3mm}
	$P(T\text{ or } H)=\frac{6}{36}+\frac{10}{36}-0=\frac{16}{36}=0.4444$
	\vspace{3mm}
	\end{solution}
	
	\part Prefers coffee \textbf{or} is a morning person
	
	\begin{solution}[\stretch{1}]
	\vspace{3mm}
	$P(C\text{ or } M)=\frac{20}{36}+\frac{8}{36}-\frac{3}{36}=\frac{25}{36}=0.6944$
	\vspace{3mm}
	\end{solution}
	
	\part Is \textbf{not} a night person
	
	\begin{solution}[\stretch{1}]
	\vspace{3mm}
	$P(N^C)=1-P(N)=1-\frac{28}{36}=\frac{8}{36}=0.2222$
	\vspace{3mm}
	\end{solution}
	
	
	\part Does \textbf{not} prefer coffee or tea
	
	\begin{solution}[\stretch{1}]
	\vspace{3mm}
	There are two ways to think about this problem. You could directly apply the Complement Rule: $P\left((C \text{ or } T)^C\right)=1-P(C\text{ or } T)=1-\left(\frac{20}{36}+\frac{6}{36}-0\right)=\frac{10}{36}=0.2778$
	
	\vspace{3mm}
	
	Or you could consider that the complement of $C$ or $T$ is $(C\text{ or } T)^C=H$:
	
	$P\left((C\text{ or } T)^C\right)=P(H)=\frac{10}{36}=0.2778$
	\vspace{3mm}
	\end{solution}
	
\end{parts}

\newpage

\question Identify two mutually exclusive events in this situation. What is the probability of their \textbf{intersection}, i.e., the probability that they both occur at the same time? Express your answer using probability notation.

\begin{solution}[\stretch{1}]

\vspace{3mm}

Example: ``Being a morning person and a night person are mutually exclusive. $P(M\cap N)=0$''

\vspace{3mm}
\end{solution} 

\question There were two variables used to record responses from students in the sample: whether a student is a morning or night person, and what their hot beverage of choice is. 

\vspace{3mm}

\begin{parts}
	\part What \textbf{type} of variables are these?
	
	\begin{solution}[\stretch{1}]

\vspace{3mm}

Qualitative

\vspace{3mm}
\end{solution} 
	
	\part What \textbf{level of measurement} do both of these variables have?
	\begin{solution}[\stretch{1}]

\vspace{3mm}

Nominal

\vspace{3mm}
\end{solution}

\end{parts}

\question General Leia Organa can plan a campaign to fight one major intergalactic battle or three small galactic battles. She believes she has a probability of 0.77 of winning the large battle ($L$) and a probability of 0.89 of winning each of the small battles ($S$). Victories or defeats in the small battles are independent. Leia must win either the large battle or all three small battles to win the campaign. Which strategy should she choose?

\vspace{3mm}

\begin{parts}
\part First find the probability of winning the large battle.

\begin{solution}[\stretch{1}]

\vspace{3mm}
$P(L)=0.77$
\vspace{3mm}

\end{solution}

\part Find the probability of winning all three small battles.

\begin{solution}[\stretch{1}]

\vspace{3mm}
$P(S \text{ and } S \text{ and } S) = P(S)P(S)P(S)=(0.89)^3=0.7050$
\vspace{3mm}

\end{solution}

\part Which strategy should she choose if she wants to win the campaign? Justify your answer.

\begin{solution}[\stretch{1}]

\vspace{3mm}
She should choose to fight the large intergalactic battle since there is a higher probability of winning it than winning three small battles.
\vspace{3mm}

\end{solution}

\end{parts}
 
\end{questions}
%-----------------------------------------------------------------------------%

\end{document}
