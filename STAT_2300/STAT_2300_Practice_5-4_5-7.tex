%% In the documentclass line, replace "noanswers" with "answers" to view the key.

\documentclass[noanswers]{exam}
\usepackage[utf8]{inputenc}

\title{Chapter 5 Practice Problems}
\author{Sections 5.4 \& 5.7}
\date{STAT 2300}

\usepackage[bottom=2.2cm, left=2.2cm, right=2.2cm, top=2.2cm]{geometry}
%\usepackage[paperheight=11in, paperwidth=17in, margin=1in]{geometry}
\usepackage{dsfont}
\usepackage{amsmath}
\usepackage{amssymb}
\usepackage{amsthm}
\usepackage{array}
\usepackage{stmaryrd}
\usepackage{pgfplots}
\pgfplotsset{width=10cm,compat=1.9}
\usepackage{multicol}
\setlength{\columnsep}{1in}
\usepackage{nicefrac}

\usepackage{multirow}
\usepackage{enumitem}[shortlabels]
\usepackage{tabu}
\definecolor{purp}{RGB}{102,0,204}
\usepackage{tabularx}
\newcolumntype{C}{>{\centering\arraybackslash $}X<{$}}
\usepackage{wrapfig}
\usepackage[export]{adjustbox}


\makeatletter
\pagestyle{headandfoot}
\firstpageheader{\@date}{\@title}{\@author}
\firstpageheadrule
\runningfootrule
\runningfooter{}{\thepage\ / \numpages}{\@title}
\makeatother

\newcommand{\abs}[1]{\left|#1\right|}
\newcommand{\mat}[4]{\left( \begin{tabular}{>{$}c<{$} >{$}c<{$}} #1&#2 \\ #3&#4 \end{tabular} \right)}
\newcommand{\msc}[1]{\mathds{#1}}
\newcommand{\Z}{\mathds{Z}}
\newcommand{\R}{\mathds{R}}
\newcommand{\N}{\mathds{N}}
\newcommand{\Q}{\mathds{Q}}
\newcommand{\C}{\mathds{C}}
\newcommand{\so}{\implies}
\newcommand{\set}[2]{\left\{ #1 \:|\: #2 \right\}}
\newcommand{\bso}{\Longleftarrow}
\newcommand{\ra}{\rightarrow}
\newcommand{\gen}[1]{\left\langle #1 \right\rangle}
\newcommand{\olin}[1]{\overline{#1}}
\newcommand{\Img}[1]{\text{Im}\left(#1\right)}
\newcommand{\llra}{\longleftrightarrow}
\newcommand{\lra}{\longrightarrow}
\newcommand{\xra}[1]{\xrightarrow{#1}}
\newcommand{\wo}{\setminus}
\newcommand{\mcal}[1]{\mathcal{#1}}
\newcommand{\Aut}[1]{\text{Aut}\left(#1\right)}
\newcommand{\Inn}[1]{\text{Inn}\left(#1\right)}
\newcommand{\syl}[2]{\text{Syl}_{#1}(#2)}
\newcommand{\norm}[1]{\left\|#1\right\|}
\newcommand{\infnorm}[1]{\left\|#1\right\|_{\infty}}
\newcommand{\xn}{\{x_n\}}
\newcommand{\sig}{\sigma}
\newcommand{\id}{\text{id}}
\newcommand{\ep}{\epsilon}
\newcommand{\st}{\text{ s.t. }}
\newcommand{\ran}[1]{\text{Ran}(#1)}
\newcommand{\nCr}[2]{\binom{#1}{#2}}
\newcommand{\Exr}[1]{\paragraph{Exercise #1:}}
\newcommand{\pg}{\paragraph{}}
\newcommand{\ulin}[1]{\underline{#1}}
\newcommand{\tc}[1]{\textcolor{purp}{#1}}

% Solution Specs
\unframedsolutions
\renewcommand{\solutiontitle}{}
\SolutionEmphasis{\color{purp}}
\CorrectChoiceEmphasis{\color{purp}\bfseries}
\setlength\fillinlinelength{0in}

%\begin{solution}[\stretch{1}]
%	hurp durp flurp
%\end{solution}

%\pagestyle{empty}

\renewcommand{\arraystretch}{1.5}
\usepackage{cancel}

\begin{document}

%\noindent\begin{tabular}{@{}p{.3in}p{3in}@{}}
%Name: & \hrulefill
%\end{tabular}
%
%\vspace{3mm}

\begin{questions} 
		
	\question A survey of high school students indicated that 33\% are in a relationship, 25\% are involved in sports, and 11\% are involved in both. Use the information to answer the following questions.

\vspace{3mm}

\begin{parts}
		
	\part What is the probability that a student is involved in a relationship \textbf{given} that they're involved in sports? 
	
	\begin{solution}[\stretch{1}]
	\vspace{3mm}
	$P(R|S) = \frac{P(R \cap S)}{P(S)} = \frac{0.11}{0.25} = 0.44$
	\vspace{3mm}
	\end{solution}
	
	\part Is being in a relationship \textbf{independent} of being involved in sports? Justify your answer using probability (regardless of your own personal theories!).
	
	\begin{solution}[\stretch{1}]
	\vspace{3mm}
	$P(R|S)=0.44 \neq P(R)=0.33 \; \Rightarrow$ These are \textbf{dependent} events.
	\vspace{3mm}
	\end{solution}

\end{parts}

\question A company has two suppliers for electrical components. China ships 73\% of the electrical components used by the supplier. The probability that the component will be defective \textbf{given} that it was shipped from China is 0.06. What is the probability that a randomly selected component received by the supplier will ship from China \textbf{and} be defective?

\begin{solution}[\stretch{1}]
\vspace{1mm}
$P(C)=0.73$, $P(D|C)=0.06$

\vspace{3mm}
Using the Multiplication Rule: $P(C \cap D) = P(D|C)P(C) = 0.06(0.73) = 0.0438$
\vspace{1mm}
\end{solution}

\question You have a standard deck of 52 cards. Recall that a deck of cards has four suites (hearts, diamonds, spades, clubs), each with thirteen values (2-10, J, Q, K, A). Find the probability that you draw two aces \textbf{in a row} without replacing the first ace. 
	
	\begin{solution}[\stretch{1}]
	\vspace{1mm}
	$P(\text{1st A and 2nd A})=P(\text{1st A})\times P(\text{2nd A }|\text{ 1st A})=\frac{4}{52}\times\frac{3}{51}=0.0045$
	\vspace{1mm}
	\end{solution}
	
\question Your Pie, a great pizza place in Clemson, has 10 vegetable and 8 meat toppings to choose from. You have a coupon for a free pizza with five toppings. Use this information to answer the following questions.

\vspace{3mm}

\begin{parts}

\part How many ways could you choose five \textbf{different toppings}? (Hint: Use the Combinations Rule.)

\begin{solution}[\stretch{1}]
	\vspace{3mm}
	$\displaystyle\phantom{!}_{18}C_5=\frac{18!}{5!(18-5)!}=8,568$ ways to choose five different toppings
	\vspace{1mm}
	\end{solution}
	
	\vspace{3mm}
	
\part How many ways could you choose five  \textbf{different vegetable toppings}?

\begin{solution}[\stretch{1}]
	\vspace{3mm}
	$\displaystyle\phantom{!}_{10}C_5=\frac{10!}{5!(10-5)!}=252$ ways to choose five different vegetable  toppings
	\vspace{1mm}
	\end{solution}
	
	\vspace{3mm}
	
\part If you randomly select toppings, what is the \textbf{probability} that you choose five different vegetable \mbox{toppings?} Round your answer to four decimal places. (Hint: Use your answers from parts a and b.)

\begin{solution}[\stretch{1}]
	\vspace{3mm}
	$P(\text{5 different veggie toppings})=\displaystyle\frac{\phantom{!}_{10}C_5}{\phantom{!}_{18}C_5}=\frac{252}{8,568}=0.0294$
	\vspace{3mm}
	\end{solution}


\end{parts}
	
	


 
\end{questions}
%-----------------------------------------------------------------------------%

\end{document}
