%% In the documentclass line, replace "noanswers" with "answers" to view the key.

\documentclass[noanswers]{exam}
\usepackage[utf8]{inputenc}

\title{Chapter 5 Practice Problems}
\author{Sections 5.1 \& 5.5}
\date{STAT 2300}

\usepackage[bottom=2.2cm, left=2.2cm, right=2.2cm, top=2.2cm]{geometry}
%\usepackage[paperheight=11in, paperwidth=17in, margin=1in]{geometry}
\usepackage{dsfont}
\usepackage{amsmath}
\usepackage{amssymb}
\usepackage{amsthm}
\usepackage{array}
\usepackage{stmaryrd}
\usepackage{pgfplots}
\pgfplotsset{width=10cm,compat=1.9}
\usepackage{multicol}
\setlength{\columnsep}{1in}
\usepackage{nicefrac}

\usepackage{multirow}
\usepackage{enumitem}[shortlabels]
\usepackage{tabu}
\definecolor{purp}{RGB}{102,0,204}
\usepackage{tabularx}
\newcolumntype{C}{>{\centering\arraybackslash $}X<{$}}
\usepackage{wrapfig}
\usepackage[export]{adjustbox}


\makeatletter
\pagestyle{headandfoot}
\firstpageheader{\@date}{\@title}{\@author}
\firstpageheadrule
\runningfootrule
\runningfooter{}{\thepage\ / \numpages}{\@title}
\makeatother

\newcommand{\abs}[1]{\left|#1\right|}
\newcommand{\mat}[4]{\left( \begin{tabular}{>{$}c<{$} >{$}c<{$}} #1&#2 \\ #3&#4 \end{tabular} \right)}
\newcommand{\msc}[1]{\mathds{#1}}
\newcommand{\Z}{\mathds{Z}}
\newcommand{\R}{\mathds{R}}
\newcommand{\N}{\mathds{N}}
\newcommand{\Q}{\mathds{Q}}
\newcommand{\C}{\mathds{C}}
\newcommand{\so}{\implies}
\newcommand{\set}[2]{\left\{ #1 \:|\: #2 \right\}}
\newcommand{\bso}{\Longleftarrow}
\newcommand{\ra}{\rightarrow}
\newcommand{\gen}[1]{\left\langle #1 \right\rangle}
\newcommand{\olin}[1]{\overline{#1}}
\newcommand{\Img}[1]{\text{Im}\left(#1\right)}
\newcommand{\llra}{\longleftrightarrow}
\newcommand{\lra}{\longrightarrow}
\newcommand{\xra}[1]{\xrightarrow{#1}}
\newcommand{\wo}{\setminus}
\newcommand{\mcal}[1]{\mathcal{#1}}
\newcommand{\Aut}[1]{\text{Aut}\left(#1\right)}
\newcommand{\Inn}[1]{\text{Inn}\left(#1\right)}
\newcommand{\syl}[2]{\text{Syl}_{#1}(#2)}
\newcommand{\norm}[1]{\left\|#1\right\|}
\newcommand{\infnorm}[1]{\left\|#1\right\|_{\infty}}
\newcommand{\xn}{\{x_n\}}
\newcommand{\sig}{\sigma}
\newcommand{\id}{\text{id}}
\newcommand{\ep}{\epsilon}
\newcommand{\st}{\text{ s.t. }}
\newcommand{\ran}[1]{\text{Ran}(#1)}
\newcommand{\nCr}[2]{\binom{#1}{#2}}
\newcommand{\Exr}[1]{\paragraph{Exercise #1:}}
\newcommand{\pg}{\paragraph{}}
\newcommand{\ulin}[1]{\underline{#1}}
\newcommand{\tc}[1]{\textcolor{purp}{#1}}

% Solution Specs
\unframedsolutions
\renewcommand{\solutiontitle}{}
\SolutionEmphasis{\color{purp}}
\CorrectChoiceEmphasis{\color{purp}\bfseries}
\setlength\fillinlinelength{0in}

%\begin{solution}[\stretch{1}]
%	hurp durp flurp
%\end{solution}

%\pagestyle{empty}

\renewcommand{\arraystretch}{1.5}
\usepackage{cancel}

\begin{document}

\noindent\begin{tabular}{@{}p{.3in}p{3in}@{}}
Name: & \hrulefill
\end{tabular}

\vspace{3mm}

\begin{questions} 
	
	\question Your favorite fast food joint has a specialty sandwich and side combo. For the sandwich you can order a burger (B) or a chicken sandwich (C), and for the side you can order fries (F), onion rings (R), or a \mbox{salad (S).}
	
	\vspace{3mm}
	
	\begin{parts} 
	
	\part Use the \textbf{Multiplication Rule of Counting} to determine the number of possible sandwich and side combinations you could order.
		
	\begin{solution}[\stretch{1}]
	
			\vspace{3mm}		
		
			2 sandwiches $\times$ 3 sides $= 6$ possible outcomes
			
			\vspace{3mm}		
			
		\end{solution}	
	
	\part Draw a \textbf{tree diagram} to determine the possible sandwich and side combinations you could order. Write the \textbf{sample space} of possible outcomes $S$ below it.
	
		\begin{solution}[\stretch{3.5}]
	
			\vspace{3mm}		
			
			\begin{center}
			\begin{tikzpicture}
				\node[] at (-2,0) {\underline{Sandwich}};
				\node[] at (0,0) {\underline{Side}};
				\node[] at (2,0) {\underline{Outcome}};
				
				\node[] at (-2,-1.25) {B};
				\node[] at (-2,-3) {C};
				\node[] at (0,-.75) {F};
				\node[] at (0,-1.25) {R};
				\node[] at (0,-1.75) {S};
				\node[] at (0,-2.5) {F};
				\node[] at (0,-3) {R};
				\node[] at (0,-3.5) {S};
				
				\draw[-] (-1.8,-1.25) -- (-.2,-.75) {};
				\draw[-] (-1.8,-1.25) -- (-.2,-1.25) {};
				\draw[-] (-1.8,-1.25) -- (-.2,-1.75) {};
				
				\draw[-] (-1.8,-3) -- (-.2,-2.5) {};
				\draw[-] (-1.8,-3) -- (-.2,-3) {};
				\draw[-] (-1.8,-3) -- (-.2,-3.5) {};
				
				\node[] at (2,-.75) {BF};
				\node[] at (2,-1.25) {BR};
				\node[] at (2,-1.75) {BS};
				
				\node[] at (2,-2.5) {CF};
				\node[] at (2,-3) {CR};
				\node[] at (2,-3.5) {CS};
				
				\draw[-] (.2,-.75) -- (1.6,-.75) {};
				\draw[-] (.2,-1.25) -- (1.6,-1.25) {};
				\draw[-] (.2,-1.75) -- (1.6,-1.75) {};
				\draw[-] (.2,-2.5) -- (1.6,-2.5) {};
				\draw[-] (.2,-3) -- (1.6,-3) {};
				\draw[-] (.2,-3.5) -- (1.6,-3.5) {};
			\end{tikzpicture}
			\end{center}
			
			\vspace{3mm}
		
			$S=\left\{\text{BF, BR, BS, CF, CR, CS}\right\}$ or $S=\left\{\text{(B,F), (B,R), (B,S), (C,F), (C,R), (C,S)}\right\}$
			
			\vspace{3mm}		
			
		\end{solution}	
		
		\part The cashier is a little bit bored and decides to try and guess what sandwich and side combo customers will order. If all of the order combinations are equally likely, what is the probability the cashier guesses an order \textbf{correctly}?
		
		\begin{solution}[\stretch{1}]
	
			\vspace{1mm}		
		
			P(correct) $= \displaystyle \frac{1}{6}$
			
			\vspace{1mm}		
			
		\end{solution}
		
		\part What is the probability that a randomly selected customer getting a combo orders \textbf{onion rings} as \mbox{their side?}
		
		\begin{solution}[\stretch{1}]
	
			\vspace{1mm}		
		
			P(onion rings) or P(R) $= \displaystyle \frac{2}{6}=\frac{1}{3}$
			
			\vspace{1mm}		
			
		\end{solution}

	\end{parts}
	
	\question Consider a probability experiment in which you toss a coin three times.
	
	\vspace{2mm}
	
	\begin{parts}
		\part Write out the \textbf{sample space} for the experiment. Let H denote heads and T denote tails.
		
		\begin{solution}[\stretch{1}]
		
		\vspace{2mm}
		
		$S=\left\{\text{HHH, HHT, HTH, THH, HTT, TTH, THT, TTT}\right\}$
		
		\vspace{2mm}
		
		\end{solution}
		
		\part What is the probability that \textbf{two out of three} tosses result in heads?
		
		\begin{solution}[\stretch{1}]
		
		\vspace{2mm}
		Three outcomes with 2 heads $\Rightarrow$ P(2 heads) $=\displaystyle\frac{3}{8}$
		\vspace{2mm}
		
		\end{solution}
		
		\part What is the probability that you get \textbf{at least one} tails?
		
		\begin{solution}[\stretch{1}]
		
		\vspace{2mm}
		Seven outcomes with at least one tail $\Rightarrow$ P(at least one tails) $=\displaystyle \frac{7}{8}$
		\end{solution}
		
	\end{parts}
	
\newpage
	
	\question For each of the following experiments, consider (1) whether it is done with or without replacement and (2)~whether the order of selection matters. Use these two facts to determine which counting rule you should use to answer the question.
	
	\vspace{3mm}
		
	\begin{parts}
		\part You're planning a stellar international vacation for when COVID-19 travel restrictions are lifted. You have five destinations to choose from and you'd like to visit three of them. \textbf{How many possible ways} could you plan your trip itinerary?
				
		\begin{solution}[\stretch{1}]
			\vspace{3mm}		
			Permutation (without replacement, order matters) with $n=5$ and $r=3$.
			
			\vspace{3mm}
			
			$n\cdot (n-1)\cdots (n-r+1)=5(4)(3)=60$ possible ways
						
			\vspace{3mm}
		\end{solution}
		
		\part The threat of COVID-19 has passed and you're packing your bags! You have 10 different snacks in your cabinet to take with you on your flight, but only have room left in your backpack for four of them. If you only have one type of each snack, \textbf{how many possible snack combinations} do you have to choose from?
		
		\begin{solution}[\stretch{1}]
			\vspace{3mm}	
			
			Combination (without replacement, order does not matter) with $n=10$ and $r=4$.
			\vspace{3mm}
			
			$\phantom{!}_{10} \text{C}_4=\displaystyle\frac{10!}{4!(10-4)!}=\frac{10\cdot 9\cdot 8\cdot 7\cdot \cancel{6}\cdot \cancel{5}\cdot \cancel{4}\cdot \cancel{3}\cdot \cancel{2}\cdot \cancel{1}}{4\cdot 3\cdot 2\cdot 1\cdot \cancel{6}\cdot \cancel{5}\cdot \cancel{4}\cdot \cancel{3}\cdot \cancel{2}\cdot \cancel{1}}=\frac{5040}{24}=210$ possible snack combinations
				
			\vspace{3mm}
		\end{solution}
		
		\part Even though you have your own snacks, you're not one to turn down the snacks provided on the plane. The flight attendant passes through with the refreshment cart 3 times during your flight, and each time you have the option of getting pretzels, peanuts, or a cookie.  \textbf{How many refreshment combinations} could you have throughout your flight?
		\begin{solution}[\stretch{1}]
			\vspace{3mm}		
			Use the Multiplication Rule for Counting (with replacement) with $n=3$ and $r=3$.
			
			\vspace{3mm}
			
			$n^r=3^3=3\cdot 3\cdot 3=27$ possible refreshment combinations			
			\vspace{3mm}
		\end{solution}
		
	\end{parts}
	
	\vspace{3mm}
	
\end{questions}

%-----------------------------------------------------------------------------%

\end{document}
