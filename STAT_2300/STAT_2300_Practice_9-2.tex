%% In the documentclass line, replace "noanswers" with "answers" to view the key.

\documentclass[noanswers]{exam}
\usepackage[utf8]{inputenc}

\title{Chapter 9 Practice Problems}
\author{Section 9.2}
\date{STAT 2300}

\usepackage[bottom=2.2cm, left=2.2cm, right=2.2cm, top=2.2cm]{geometry}
%\usepackage[paperheight=11in, paperwidth=17in, margin=1in]{geometry}
\usepackage{dsfont}
\usepackage{amsmath}
\usepackage{amssymb}
\usepackage{amsthm}
\usepackage{array}
\usepackage{stmaryrd}
\usepackage{pgfplots}
\pgfplotsset{width=10cm,compat=1.9}
\usepackage{multicol}
\setlength{\columnsep}{1in}
\usepackage{nicefrac}

\usepackage{multirow}
\usepackage{enumitem}[shortlabels]
\usepackage{tabu}
\definecolor{purp}{RGB}{102,0,204}
\usepackage{tabularx}
\newcolumntype{C}{>{\centering\arraybackslash $}X<{$}}
\usepackage{wrapfig}
\usepackage[export]{adjustbox}


\makeatletter
\pagestyle{headandfoot}
\firstpageheader{\@date}{\@title}{\@author}
\firstpageheadrule
\runningfootrule
\runningfooter{}{\thepage\ / \numpages}{\@title}
\makeatother

\newcommand{\abs}[1]{\left|#1\right|}
\newcommand{\mat}[4]{\left( \begin{tabular}{>{$}c<{$} >{$}c<{$}} #1&#2 \\ #3&#4 \end{tabular} \right)}
\newcommand{\msc}[1]{\mathds{#1}}
\newcommand{\Z}{\mathds{Z}}
\newcommand{\R}{\mathds{R}}
\newcommand{\N}{\mathds{N}}
\newcommand{\Q}{\mathds{Q}}
\newcommand{\C}{\mathds{C}}
\newcommand{\so}{\implies}
\newcommand{\set}[2]{\left\{ #1 \:|\: #2 \right\}}
\newcommand{\bso}{\Longleftarrow}
\newcommand{\ra}{\rightarrow}
\newcommand{\gen}[1]{\left\langle #1 \right\rangle}
\newcommand{\olin}[1]{\overline{#1}}
\newcommand{\Img}[1]{\text{Im}\left(#1\right)}
\newcommand{\llra}{\longleftrightarrow}
\newcommand{\lra}{\longrightarrow}
\newcommand{\xra}[1]{\xrightarrow{#1}}
\newcommand{\wo}{\setminus}
\newcommand{\mcal}[1]{\mathcal{#1}}
\newcommand{\Aut}[1]{\text{Aut}\left(#1\right)}
\newcommand{\Inn}[1]{\text{Inn}\left(#1\right)}
\newcommand{\syl}[2]{\text{Syl}_{#1}(#2)}
\newcommand{\norm}[1]{\left\|#1\right\|}
\newcommand{\infnorm}[1]{\left\|#1\right\|_{\infty}}
\newcommand{\xn}{\{x_n\}}
\newcommand{\sig}{\sigma}
\newcommand{\id}{\text{id}}
\newcommand{\ep}{\epsilon}
\newcommand{\st}{\text{ s.t. }}
\newcommand{\ran}[1]{\text{Ran}(#1)}
\newcommand{\nCr}[2]{\binom{#1}{#2}}
\newcommand{\Exr}[1]{\paragraph{Exercise #1:}}
\newcommand{\pg}{\paragraph{}}
\newcommand{\ulin}[1]{\underline{#1}}
\newcommand{\tc}[1]{\textcolor{purp}{#1}}

% Solution Specs
\unframedsolutions
\renewcommand{\solutiontitle}{}
\SolutionEmphasis{\color{purp}}
\CorrectChoiceEmphasis{\color{purp}\bfseries}
\setlength\fillinlinelength{1in}

%\begin{solution}[\stretch{1}]
%	hurp durp flurp
%\end{solution}

%\pagestyle{empty}

\renewcommand{\arraystretch}{1.5}
\usepackage{cancel}

\begin{document}

\noindent\begin{tabular}{@{}p{.3in}p{3in}@{}}
Name: & \hrulefill
\end{tabular}

\vspace{4mm}

\noindent One frequently used test to measure the reading ability of children is the DRP, or Degree of Reading Power. It is known that the distribution of DRP scores is normally distributed. A researcher suspects that the mean score $\mu$ of all 500 third-graders in Henrico County Schools is different from the national mean, which is 32 points. To test her suspicion, she administers the DRP to a random sample of 22 Henrico County third-grade students. Their scores are recorded in the following table:
    \begin{center}
    \begin{tabular}{|c|c|c|c|c|c|c|c|c|c|c|}
    \hline
        40 & 26 & 39 & 17 & 42 & 18 & 24 & 43 & 46 & 27 & 19 \\
        \hline
        47 & 19 & 26 & 37 & 34 & 15 & 45 & 41 & 39 & 31 & 46 \\
        \hline
    \end{tabular}
    \end{center}

\begin{questions} 

\vspace{1mm}

\question Find the \textit{t} critical value for a 90\% confidence interval for the true \textbf{mean}, $\mu$, of the population.

\begin{solution}[\stretch{1}]
\vspace{1mm}

$\alpha=1-0.9=0.1\;\Rightarrow\; \alpha/2=0.05$ \hspace{10mm} $n=22 \;\Rightarrow\; df=21$ \hspace{10mm} $t_{\alpha/2}=t_{.05}=1.72$

\begin{center}
\begin{tikzpicture}
        \def\normaltwo{\x,{2.5*1/exp(((\x-3)^2)/2)}}
        \def\y{4.5}
        \def\z{1.5}
        \def\mu{3}
        \def\fy{2.5*1/exp(((\y-3)^2)/2)}
        \def\fz{2.5*1/exp(((\z-3)^2)/2)}
        \fill [fill=purp!30] (\y,-.1) -- plot[domain=\y:6.5] (\normaltwo) -- (6.5,-.1) -- cycle;

        \draw[domain=-.5:6.5,samples=100] plot (\normaltwo) node[right] {};
        \draw[dashed] ({\y},{\fy}) -- ({\y},-.1) node[below] {$t_{.05}$};
        \node[] at ({\y},-.8) {1.72};
        \draw[] ({\mu},{0}) -- ({\mu},-.1) node[below] {\small{$0$}};
        \draw[-] (-.7,-.1) -- (6.7,-.1) node[right] {};
        \node[] at (1, 1.5) {0.95};
        \draw[-] (1,1.3) -- (1.5,.5);
        \node[] at (5.5, 1) {0.05};
        \draw[-] (5.5,0.8) -- (5.1,.15);

        \node[] at (7.0,-.1) {$t$};
    \end{tikzpicture}
    \end{center}
\end{solution}

\vspace{-5mm}

\question State and verify the conditions for the confidence interval.

\begin{solution}[\stretch{1}]

\vspace{1mm}

\tc{(1) Sample must be randomly selected from the population --- stated in the problem.}
        
        \vspace{3mm}
        
        (2) Sample size is small relative to the population --- $n=22<0.05(500)=25$.
        
        \vspace{3mm}
        
        \tc{(3) $\overline{X}$ must be approximately normal --- true since $X$ is said to be normal.}

\end{solution}

\question Compute the 90\% confidence interval for the mean DRP score in Henrico County Schools. (You can find your point estimate and standard deviation in your calculator using the \textbf{1-Var Stats} function. Compare your answers with a classmate to make sure you entered things in correctly!)

\begin{solution}[\stretch{1}]
$\overline{x}=32.77,\; s=10.86$
        
        \vspace{2mm}
        
        $\overline{x}\pm t_{.05}\left(\frac{s}{\sqrt{n}}\right) \;\Rightarrow\; 32.77\pm 1.72\left(\frac{10.86}{\sqrt{22}}\right) \;\Rightarrow\; (28.79,36.75)$

\end{solution}

\question Write a sentence of interpretation for the confidence interval you computed in the previous question.

\begin{solution}[\stretch{1}]

We are 90\% confident that the true mean DRP score of Henrico County third-graders is between 28.79 and 36.75 points.

\end{solution}

\question Use the confidence interval you constructed in to comment on whether you agree with the researcher's suspicion. Explain your reasoning clearly.

\begin{solution}[\stretch{1}]


I would disagree with the researcher's suspicion. Since 32 is in the confidence interval, we don't have evidence that the true mean DRP score for Henrico County third-graders is different than 32.
\end{solution}


\end{questions}
%-----------------------------------------------------------------------------%

\end{document}
