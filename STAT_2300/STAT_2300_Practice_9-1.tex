%% In the documentclass line, replace "noanswers" with "answers" to view the key.

\documentclass[noanswers]{exam}
\usepackage[utf8]{inputenc}

\title{Chapter 9 Practice Problems}
\author{Section 9.1}
\date{STAT 2300}

\usepackage[bottom=2.2cm, left=2.2cm, right=2.2cm, top=2.2cm]{geometry}
%\usepackage[paperheight=11in, paperwidth=17in, margin=1in]{geometry}
\usepackage{dsfont}
\usepackage{amsmath}
\usepackage{amssymb}
\usepackage{amsthm}
\usepackage{array}
\usepackage{stmaryrd}
\usepackage{pgfplots}
\pgfplotsset{width=10cm,compat=1.9}
\usepackage{multicol}
\setlength{\columnsep}{1in}
\usepackage{nicefrac}

\usepackage{multirow}
\usepackage{enumitem}[shortlabels]
\usepackage{tabu}
\definecolor{purp}{RGB}{102,0,204}
\usepackage{tabularx}
\newcolumntype{C}{>{\centering\arraybackslash $}X<{$}}
\usepackage{wrapfig}
\usepackage[export]{adjustbox}


\makeatletter
\pagestyle{headandfoot}
\firstpageheader{\@date}{\@title}{\@author}
\firstpageheadrule
\runningfootrule
\runningfooter{}{\thepage\ / \numpages}{\@title}
\makeatother

\newcommand{\abs}[1]{\left|#1\right|}
\newcommand{\mat}[4]{\left( \begin{tabular}{>{$}c<{$} >{$}c<{$}} #1&#2 \\ #3&#4 \end{tabular} \right)}
\newcommand{\msc}[1]{\mathds{#1}}
\newcommand{\Z}{\mathds{Z}}
\newcommand{\R}{\mathds{R}}
\newcommand{\N}{\mathds{N}}
\newcommand{\Q}{\mathds{Q}}
\newcommand{\C}{\mathds{C}}
\newcommand{\so}{\implies}
\newcommand{\set}[2]{\left\{ #1 \:|\: #2 \right\}}
\newcommand{\bso}{\Longleftarrow}
\newcommand{\ra}{\rightarrow}
\newcommand{\gen}[1]{\left\langle #1 \right\rangle}
\newcommand{\olin}[1]{\overline{#1}}
\newcommand{\Img}[1]{\text{Im}\left(#1\right)}
\newcommand{\llra}{\longleftrightarrow}
\newcommand{\lra}{\longrightarrow}
\newcommand{\xra}[1]{\xrightarrow{#1}}
\newcommand{\wo}{\setminus}
\newcommand{\mcal}[1]{\mathcal{#1}}
\newcommand{\Aut}[1]{\text{Aut}\left(#1\right)}
\newcommand{\Inn}[1]{\text{Inn}\left(#1\right)}
\newcommand{\syl}[2]{\text{Syl}_{#1}(#2)}
\newcommand{\norm}[1]{\left\|#1\right\|}
\newcommand{\infnorm}[1]{\left\|#1\right\|_{\infty}}
\newcommand{\xn}{\{x_n\}}
\newcommand{\sig}{\sigma}
\newcommand{\id}{\text{id}}
\newcommand{\ep}{\epsilon}
\newcommand{\st}{\text{ s.t. }}
\newcommand{\ran}[1]{\text{Ran}(#1)}
\newcommand{\nCr}[2]{\binom{#1}{#2}}
\newcommand{\Exr}[1]{\paragraph{Exercise #1:}}
\newcommand{\pg}{\paragraph{}}
\newcommand{\ulin}[1]{\underline{#1}}
\newcommand{\tc}[1]{\textcolor{purp}{#1}}

% Solution Specs
\unframedsolutions
\renewcommand{\solutiontitle}{}
\SolutionEmphasis{\color{purp}}
\CorrectChoiceEmphasis{\color{purp}\bfseries}
\setlength\fillinlinelength{1in}

%\begin{solution}[\stretch{1}]
%	hurp durp flurp
%\end{solution}

%\pagestyle{empty}

\renewcommand{\arraystretch}{1.5}
\usepackage{cancel}

\begin{document}

\noindent\begin{tabular}{@{}p{.3in}p{3in}@{}}
Name: & \hrulefill
\end{tabular}

\vspace{4mm}

\noindent Taylor wants to estimate the true proportion of undergraduate students at Clemson University who watch \textit{Game of Thrones}. She randomly selects 150 Clemson undergraduate students and finds that 54 of them are dedicated \textit{Game of Thrones} fans.
\begin{questions} 

\question What is the \textbf{sample proportion} of fans $\hat{p}$ from the sample of Clemson students?

\begin{solution}[\stretch{1}]
\vspace{1mm}

$\hat{p}=\frac{54}{150}=0.36$

\vspace{1mm}
\end{solution}

\question Check the \textbf{three conditions} required in order for us to construct a valid confidence interval. (There are approximately 20,000 undergraduate students at Clemson.)

\begin{solution}[\stretch{1}]

\vspace{1mm}

(1) It states in the problem that the students are randomly selected.

\vspace{2mm}

(2) $n=150<0.05(20,000)=1000 \Rightarrow$ The sample size is small relative to the population.

\vspace{2mm}

(3) $n\hat{p}(1-\hat{p})=150(0.36)(0.64)=34.56\geq 10 \Rightarrow \hat{p}$ is normally distributed.

\vspace{1mm}
\end{solution}

\question Find the \textbf{critical value} needed to compute a 93\% confidence interval. Include a sketch.

\begin{solution}[\stretch{1}]
\vspace{1mm}
$\alpha=1-0.93=0.07$ \hspace{10mm} $\nicefrac{\alpha}{2}=0.035$

$z_{.035}$ is the z-score with an area of 0.035 on the \textit{right} $\Rightarrow$ From the table: $z_{.035}=1.81$

\vspace{2mm}

\begin{tikzpicture}
        \def\normaltwo{\x,{2.5*1/exp(((\x-3)^2)/2)}}
        \def\y{4.5}
        \def\z{1.5}
        \def\mu{3}
        \def\fy{2.5*1/exp(((\y-3)^2)/2)}
        \def\fz{2.5*1/exp(((\z-3)^2)/2)}
        \fill [fill=purp!30] (\y,0) -- plot[domain=\y:6.5] (\normaltwo) -- (6.5,0) -- cycle;

        \draw[domain=-.5:6.5,samples=100] plot (\normaltwo) node[right] {};
        \draw[dashed] ({\y},{\fy}) -- ({\y},0) node[below] {$z_{.035}$};
        \draw[] ({\mu},{.1}) -- ({\mu},0) node[below] {\small{$0$}};
        \draw[-] (-.7,0) -- (6.7,0) node[right] {};
        \node[] at (1, 1.5) {0.9650};
        \draw[-] (1,1.3) -- (1.5,.5);
        \node[] at (5.5, 1) {0.0350};
        \draw[-] (5.5,0.8) -- (5.1,.15);

        \node[] at (7.0,0) {$Z$};
    \end{tikzpicture}

\vspace{1mm}
\end{solution}

\question Find the estimate for the \textbf{standard error} $\sigma_{\hat{p}}$. Round to four decimal places.

\begin{solution}[\stretch{1}]

\vspace{1mm}

$\sqrt{\frac{\hat{p}(1-\hat{p}}{n}}=\sqrt{\frac{0.36(0.64)}{150}}=0.0392$

\vspace{1mm}
\end{solution}

\question What is the \textbf{margin of error} for a 93\% confidence interval? (Use your answers from \#3 and \#4.)

\begin{solution}[\stretch{1}]
\vspace{1mm}
MOE $=z_{.035}\sqrt{\frac{\hat{p}(1-\hat{p}}{n}}=(1.81)(0.0392)=0.0710$
\vspace{1mm}
\end{solution}

\question Find the 93\% \textbf{confidence interval} for the population proportion $p$.

\begin{solution}[\stretch{1}]
\vspace{1mm}
$\hat{p}\pm \text{MOE}=0.36\pm 0.0710=(0.289,0.431)$
\vspace{1mm}
\end{solution}

\question Complete the following sentence: We are \fillin[93\%] confident that the true \fillin[proportion] of Clemson undergraduate students who are \textit{Game of Thrones} fans is between \fillin[0.289] and \fillin[0.431].

\newpage

\question In a July 2001 research note, the U.S.\ Department of Transportation reported the results of the National Occupant Protection Use Survey. One focus of the survey was to determine the level of cell phone use by drivers while they are operating a motor passenger vehicle. Data collected by observers at randomly selected intersections across the country revealed that, in a sample of 1165 drivers, 35 were using their cell phones.

\vspace{3mm}

\begin{parts}

\part What is the point estimate for $p$, the true driver cell phone rate?

\begin{solution}[\stretch{1}]
\vspace{3mm}

$\hat{p}=\frac{35}{1165}\approx 0.03$
\vspace{3mm}
\end{solution}

\part Check and verify the three conditions for a 98\% confidence interval. (Suppose there are approximately 227.5 million drivers in the U.S.)

\begin{solution}[\stretch{1}]
\vspace{3mm}

(1) The problem states that the sample was randomly selected from the population.

\vspace{3mm}

(2) $1165 < 0.05(227,500,000)=11,375,000$

\vspace{3mm}

(3) $n\hat{p}(1-\hat{p})=1165(0.03)(0.97)=33.9015\geq10$

\vspace{3mm}
\end{solution}

\part Compute a 98\% confidence interval for the proportion of drivers that are using their cell phones.

\begin{solution}[\stretch{1}]
\vspace{3mm}
$\frac{\alpha}{2}=\frac{1-0.98}{2}=0.01\Rightarrow$ Critical Value: $z_{.01}\approx 2.33$
\vspace{3mm}

$\hat{p}\pm z_{.01}\sqrt{\frac{\hat{p}(1-\hat{p}}{n}}=0.03\pm 2.33\sqrt{\frac{0.03(0.97)}{1165}}=(0.018,0.042)$

\vspace{3mm}
\end{solution}

\part Interpret the confidence \textbf{interval} that you found in Part (c).

\begin{solution}[\stretch{1}]
\vspace{3mm}
We are 98\% confident that the true proportion of drivers using their cell phones while driving is between 0.018 and 0.042.
\vspace{3mm}
\end{solution}

\part Interpret the confidence \textbf{level} you used to construct the confidence interval.
\begin{solution}[\stretch{1}]
\vspace{3mm}
If we found many random samples of size 1165 and constructed confidence intervals for $p$ for each, we would expect approximately 98\% of the intervals to contain the true population proportion $p$.
\vspace{3mm}
\end{solution}
\end{parts}

\question A magazine company is planning to survey customers to determine the proportion who will renew their subscription for the upcoming year. 
The magazine wants to estimate the population proportion with 95\% confidence and a margin of error equal to 0.03.

\vspace{3mm}
\begin{parts}

\part Based on the information provided, what is the minimum sample size required?

\begin{solution}[\stretch{1}]
\vspace{3mm}

$n=\hat{p}(1-\hat{p})\left(\frac{z_{.025}}{E}\right)^2=(0.5)(1-0.5)\left(\frac{1.96}{0.03}\right)^2=1067.11\approx 1068$ customers
\vspace{3mm}
\end{solution}

\part Suppose that the magazine company looked back through randomly selected records for their current customers and found that 72\% of customers had renewed their subscription from the previous year. If the company uses this as an estimate for $\hat{p}$, what is the sample size they will now need for the same confidence level and margin of error?

\begin{solution}[\stretch{1}]

\vspace{3mm}
$n=\hat{p}(1-\hat{p})\left(\frac{z_{.025}}{E}\right)^2=(0.72)(1-0.72)\left(\frac{1.96}{0.03}\right)^2=860.52\approx 861$ customers
\end{solution}

\end{parts}
\end{questions}
%-----------------------------------------------------------------------------%

\end{document}
