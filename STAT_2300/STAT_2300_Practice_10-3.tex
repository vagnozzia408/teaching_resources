%% In the documentclass line, replace "noanswers" with "answers" to view the key.

\documentclass[noanswers]{exam}
\usepackage[utf8]{inputenc}

\title{Chapter 10 Practice Problems}
\author{Section 10.3}
\date{STAT 2300}

\usepackage[bottom=2.2cm, left=2.2cm, right=2.2cm, top=2.2cm]{geometry}
%\usepackage[paperheight=11in, paperwidth=17in, margin=1in]{geometry}
\usepackage{dsfont}
\usepackage{amsmath}
\usepackage{amssymb}
\usepackage{amsthm}
\usepackage{array}
\usepackage{stmaryrd}
\usepackage{pgfplots}
\pgfplotsset{width=10cm,compat=1.9}
\usepackage{multicol}
\setlength{\columnsep}{1in}
\usepackage{nicefrac}

\usepackage{multirow}
\usepackage{enumitem}[shortlabels]
\usepackage{tabu}
\definecolor{purp}{RGB}{102,0,204}
\usepackage{tabularx}
\newcolumntype{C}{>{\centering\arraybackslash $}X<{$}}
\usepackage{wrapfig}
\usepackage[export]{adjustbox}


\makeatletter
\pagestyle{headandfoot}
\firstpageheader{\@date}{\@title}{\@author}
\firstpageheadrule
\runningfootrule
\runningfooter{}{\thepage\ / \numpages}{\@title}
\makeatother

\newcommand{\abs}[1]{\left|#1\right|}
\newcommand{\mat}[4]{\left( \begin{tabular}{>{$}c<{$} >{$}c<{$}} #1&#2 \\ #3&#4 \end{tabular} \right)}
\newcommand{\msc}[1]{\mathds{#1}}
\newcommand{\Z}{\mathds{Z}}
\newcommand{\R}{\mathds{R}}
\newcommand{\N}{\mathds{N}}
\newcommand{\Q}{\mathds{Q}}
\newcommand{\C}{\mathds{C}}
\newcommand{\so}{\implies}
\newcommand{\set}[2]{\left\{ #1 \:|\: #2 \right\}}
\newcommand{\bso}{\Longleftarrow}
\newcommand{\ra}{\rightarrow}
\newcommand{\gen}[1]{\left\langle #1 \right\rangle}
\newcommand{\olin}[1]{\overline{#1}}
\newcommand{\Img}[1]{\text{Im}\left(#1\right)}
\newcommand{\llra}{\longleftrightarrow}
\newcommand{\lra}{\longrightarrow}
\newcommand{\xra}[1]{\xrightarrow{#1}}
\newcommand{\wo}{\setminus}
\newcommand{\mcal}[1]{\mathcal{#1}}
\newcommand{\Aut}[1]{\text{Aut}\left(#1\right)}
\newcommand{\Inn}[1]{\text{Inn}\left(#1\right)}
\newcommand{\syl}[2]{\text{Syl}_{#1}(#2)}
\newcommand{\norm}[1]{\left\|#1\right\|}
\newcommand{\infnorm}[1]{\left\|#1\right\|_{\infty}}
\newcommand{\xn}{\{x_n\}}
\newcommand{\sig}{\sigma}
\newcommand{\id}{\text{id}}
\newcommand{\ep}{\epsilon}
\newcommand{\st}{\text{ s.t. }}
\newcommand{\ran}[1]{\text{Ran}(#1)}
\newcommand{\nCr}[2]{\binom{#1}{#2}}
\newcommand{\Exr}[1]{\paragraph{Exercise #1:}}
\newcommand{\pg}{\paragraph{}}
\newcommand{\ulin}[1]{\underline{#1}}
\newcommand{\tc}[1]{\textcolor{purp}{#1}}

% Solution Specs
\unframedsolutions
\renewcommand{\solutiontitle}{}
\SolutionEmphasis{\color{purp}}
\CorrectChoiceEmphasis{\color{purp}\bfseries}
\setlength\fillinlinelength{1in}

%\begin{solution}[\stretch{1}]
%	hurp durp flurp
%\end{solution}

%\pagestyle{empty}

\renewcommand{\arraystretch}{1.5}
\usepackage{cancel}

\begin{document}

%\noindent\begin{tabular}{@{}p{.3in}p{3in}@{}}
%Name: & \hrulefill
%\end{tabular}
%
%\vspace{4mm}

\begin{questions} 

\question Tony Stark is developing a new model of the Iron Man suit. His previous suit model uses an average of 222.4 kilowatts (kW), and he wishes to see if his new model is more efficient (in other words, if the mean energy consumption in kW is less than the previous model). He takes a sample of five randomly selected suits made under the new model and finds that the mean energy consumption is 210.2 kW with a standard deviation of 9.9 kW. He knows that the energy consumption for a given suit follows a normal distribution based on his previous tests. Test Tony's hypothesis at the $\alpha=0.01$ significance level.

\vspace{3mm}

\begin{parts}

\part Define the \textbf{parameter} of interest and state the \textbf{hypotheses}.

\begin{solution}[\stretch{1}]
\vspace{3mm}

Let $\mu=$ true mean energy consumption in kW for the new Iron Man suit model.

\vspace{3mm}

$H_0:\mu=222.4$ kW

$H_1:\mu<222.4$ kW

\vspace{3mm}
\end{solution}

\part Verify that the necessary \textbf{conditions} hold to conduct a hypothesis test for mean. (Suppose that he has already made 100 suits using the new model.)

\begin{solution}[\stretch{1}]
\vspace{3mm}

(1) Must have a random sample selected from the population --- the problem states that the five suits were randomly selected.

\vspace{3mm}

(2) $5 \leq 0.05(100)$, so the sample size is small enough relative to the population size.

\vspace{3mm}

(3) The population (energy consumption) is stated to be normally distributed.

\vspace{3mm}

\end{solution}

\part Find the appropriate \textbf{test statistic}.

\begin{solution}[\stretch{1}]

\vspace{3mm}

$t_0=\displaystyle\frac{\overline{x}-\mu_0}{\nicefrac{s}{\sqrt{n}}}=\frac{210.2-222.4}{\nicefrac{9.9}{\sqrt{5}}}=-2.76$

\vspace{3mm}
\end{solution}

\part Find the \textbf{p-value} for the hypothesis test.

\begin{solution}[\stretch{1}]

\vspace{3mm}

$df=n-1=4$

\vspace{3mm}

$p$-value $=P(t<-2.76)=P(t>2.76) \; \Rightarrow \; 0.025 < p\text{-value} < 0.05$

\vspace{3mm}
\end{solution}

\part State your \textbf{conclusion} in context of the problem.
\begin{solution}[\stretch{1}]

\vspace{3mm}

We fail to reject $H_0$ because the $p$-value is greater than $\alpha=0.01$. At the 0.01 (1\%) significance level, there is insufficient evidence to conclude that the true mean energy consumption for the new Iron Man suit is less than 222.4 kW (i.e.\ that the mean energy consumption for the new suit is lower than the previous model). 

\vspace{3mm}

\end{solution}
\end{parts}

\end{questions}
%-----------------------------------------------------------------------------%

\end{document}
