%% In the documentclass line, replace "noanswers" with "answers" to view the key.

\documentclass[noanswers]{exam}
\usepackage[utf8]{inputenc}

\title{Chapter 1 Practice Problems}
\author{Sections 1.1--1.3}
\date{STAT 2300}

\usepackage[bottom=2.2cm, left=2.2cm, right=2.2cm, top=2.2cm]{geometry}
%\usepackage[paperheight=11in, paperwidth=17in, margin=1in]{geometry}
\usepackage{dsfont}
\usepackage{amsmath}
\usepackage{amssymb}
\usepackage{amsthm}
\usepackage{array}
\usepackage{stmaryrd}
\usepackage{pgfplots}
\pgfplotsset{width=10cm,compat=1.9}
\usepackage{multicol}
\setlength{\columnsep}{1in}
\usepackage{nicefrac}

\usepackage{multirow}
\usepackage{enumitem}[shortlabels]
\usepackage{tabu}
\definecolor{purp}{RGB}{102,0,204}
\usepackage{tabularx}
\newcolumntype{C}{>{\centering\arraybackslash $}X<{$}}
\usepackage{wrapfig}
\usepackage[export]{adjustbox}


\makeatletter
\pagestyle{headandfoot}
\firstpageheader{\@date}{\@title}{\@author}
\firstpageheadrule
\runningfootrule
\runningfooter{}{\thepage\ / \numpages}{\@title}
\makeatother

\newcommand{\abs}[1]{\left|#1\right|}
\newcommand{\mat}[4]{\left( \begin{tabular}{>{$}c<{$} >{$}c<{$}} #1&#2 \\ #3&#4 \end{tabular} \right)}
\newcommand{\msc}[1]{\mathds{#1}}
\newcommand{\Z}{\mathds{Z}}
\newcommand{\R}{\mathds{R}}
\newcommand{\N}{\mathds{N}}
\newcommand{\Q}{\mathds{Q}}
\newcommand{\C}{\mathds{C}}
\newcommand{\so}{\implies}
\newcommand{\set}[2]{\left\{ #1 \:|\: #2 \right\}}
\newcommand{\bso}{\Longleftarrow}
\newcommand{\ra}{\rightarrow}
\newcommand{\gen}[1]{\left\langle #1 \right\rangle}
\newcommand{\olin}[1]{\overline{#1}}
\newcommand{\Img}[1]{\text{Im}\left(#1\right)}
\newcommand{\llra}{\longleftrightarrow}
\newcommand{\lra}{\longrightarrow}
\newcommand{\xra}[1]{\xrightarrow{#1}}
\newcommand{\wo}{\setminus}
\newcommand{\mcal}[1]{\mathcal{#1}}
\newcommand{\Aut}[1]{\text{Aut}\left(#1\right)}
\newcommand{\Inn}[1]{\text{Inn}\left(#1\right)}
\newcommand{\syl}[2]{\text{Syl}_{#1}(#2)}
\newcommand{\norm}[1]{\left\|#1\right\|}
\newcommand{\infnorm}[1]{\left\|#1\right\|_{\infty}}
\newcommand{\xn}{\{x_n\}}
\newcommand{\sig}{\sigma}
\newcommand{\id}{\text{id}}
\newcommand{\ep}{\epsilon}
\newcommand{\st}{\text{ s.t. }}
\newcommand{\ran}[1]{\text{Ran}(#1)}
\newcommand{\nCr}[2]{\binom{#1}{#2}}
\newcommand{\Exr}[1]{\paragraph{Exercise #1:}}
\newcommand{\pg}{\paragraph{}}
\newcommand{\ulin}[1]{\underline{#1}}
\newcommand{\tc}[1]{\textcolor{purp}{#1}}

% Solution Specs
\unframedsolutions
\renewcommand{\solutiontitle}{}
\SolutionEmphasis{\color{purp}}
\CorrectChoiceEmphasis{\color{purp}\bfseries}

%\begin{solution}[\stretch{1}]
%	hurp durp flurp
%\end{solution}

%\pagestyle{empty}

\begin{document}

%\noindent\begin{tabular}{@{}p{.3in}p{3in}@{}}
%Name: & \hrulefill
%\end{tabular}

\begin{questions} 

	\question A student engagement committee at Clemson decides to conduct a survey to learn about how students are involved in campus clubs and organizations. The committee randomly selects 125 students and finds that 73\% report being involved in at least one club or organization on campus.
	
	\vspace{5mm}

	\begin{parts}
	
		\part Who are the \textbf{individuals} in this example?
	
		\begin{solution}[\stretch{1}]
	
			\vspace{5mm}		
		
			Clemson students

			\vspace{5mm}		
			
		\end{solution}
	
		\part Identify the \textbf{sample} in this situation.
	
		\begin{solution}[\stretch{1}]
		
			\vspace{5mm}		
				
			125 Clemson students

			\vspace{5mm}		
				
		\end{solution}
	
		\part Describe the \textbf{population of interest} in this situation.
		
		\begin{solution}[\stretch{1}]

			\vspace{5mm}		
			
			All Clemson students

			\vspace{5mm}		
			
		\end{solution}
	
		\part Identify the \textbf{statistic} in this study.
	
		\begin{solution}[\stretch{1}]
	
			\vspace{5mm}
	
			The 73\% of students in the sample who are involved in at least one club or organization on campus
	
			\vspace{5mm}
	
		\end{solution}
	
		\part Describe the \textbf{parameter} in this study.
	
		\begin{solution}[\stretch{1}]
		
			\vspace{5mm}

			The true percentage of all students at Clemson who are involved in at least one club or organization on campus
		
			\vspace{5mm}	
	
		\end{solution}
	
	\end{parts}

	\question Identify whether the following are examples of \textbf{descriptive} or \textbf{inferential} statistics. Describe \textbf{why} you give your answer.

\vspace{5mm}

	\begin{parts}
	
		\part A researcher wants to study the exercise habits of residents in Pickens County. She randomly selects a sample of residents using phone book listings and conducts short phone interviews with 203 residents who answer the phone, asking them about their weekly exercise routine. Using her sample, she estimates that 62\% of residents in Pickens County exercise 2-3 times per week.
		
		\begin{solution}[\stretch{1}]
			\vspace{5mm}
					
			Inferential statistics --- the researcher is drawing a generalization about the population of all residents in Pickens County using information obtained from the sample.
			
			\vspace{5mm}		
		\end{solution}
	
		\part The owner of a small start-up business with 19 employees asks each of them to submit a timesheet logging their hours worked each day. Summarizing the data he has collected, he finds that, in the last week, his employees worked an average of 9.5 hours per day.
	
		\begin{solution}[\stretch{1}]		
			\vspace{5mm}		
				
			Descriptive --- the owner is summarizing data gathered from all of his employees and is not generalizing to a larger population.

			\vspace{5mm}		
		\end{solution}

	\end{parts}
	
	\newpage	
	
	\question Identify whether the described scenario is ab \textbf{observational study} or an \textbf{experiment} and explain how you know.
	
	\begin{parts}
	
		\part A marine biologist wishes to study the effectiveness to two different dolphin training methods. She randomly selects three dolphins at the aquarium to be trained under the first method and three dolphins to be trained under the second. She works with each dolphin using the assigned method and records the length of time required for the dolphin to learn to toss a ball through a hoop on signal.
		
		\begin{solution}[\stretch{1}]
			\vspace{3mm}
					
			Experiment --- the researcher is imposing a particular training method (treatment) on each dolphin.
			
			\vspace{3mm}		
		\end{solution}
	
		\part 4-H is a youth development program that equips young people with life skills they need to be successful. A university wishes to study the effects of being involved in 4-H growing up on college preparedness. Researchers identify 54 college students who are 4-H alumni and 52 students who are not and ask them each whether they felt they had the necessary skills to succeed in college when they began as a freshman.
	
		\begin{solution}[\stretch{1}]		
			\vspace{3mm}		
				
			Observational study --- researchers are not influencing the subjects and are only recording characteristics about them that are independent of the researcher.

			\vspace{3mm}		
		\end{solution}

	\end{parts}
	
	\question Identify each of the following variables as \textbf{qualitative or quantitative} and specify its \textbf{level of measurement} (nominal, ordinal, interval, or ratio).
	
	\begin{parts}
	
		\part A stressed-out college student records the \textbf{amount of sleep} she gets per night during a semester.
		
		\begin{solution}[\stretch{1}]
			\vspace{3mm}
					
			Quantitative, ratio
			
			\vspace{3mm}		
		\end{solution}
	
		\part A chemist records the \textbf{temperature} (in $^\circ$C) of a heated solution every 30 seconds as it cools.
		
		\begin{solution}[\stretch{1}]
			\vspace{3mm}
					
			Quantitative, interval
			
			\vspace{3mm}		
		\end{solution}
		
		\part A researcher records whether a sample of Clemson students purchases their textbooks \textbf{online or from the campus bookstore}.
		
		\begin{solution}[\stretch{1}]
			\vspace{3mm}
					
			Qualitative, nominal
			
			\vspace{3mm}		
		\end{solution}
		
		\part A coffee shop owner records the \textbf{number of caramel macchiatos} that are ordered each day by customers.
		
		\begin{solution}[\stretch{1}]
			\vspace{3mm}
					
			Quantitative, ratio
			
			\vspace{3mm}		
		\end{solution}
		
		\part An instructor records the responses to a \textbf{satisfaction} evaluation (Very Satisfied, Satisfied, Neutral, Slightly Dissatisfied, Very Dissatisfied).
		
		\begin{solution}[\stretch{1}]
			\vspace{3mm}
					
			Qualitative, ordinal
			
			\vspace{3mm}		
		\end{solution}
		
		\part The county elections commission divides the county up by \textbf{zip codes} to identify potential polling places.
		
		\begin{solution}[\stretch{1}]
			\vspace{3mm}
					
			Qualitative, nominal
			
			\vspace{3mm}		
		\end{solution}

	\end{parts}
	
	\question Random sampling is powerful because it increases our likelihood of obtaining a representative sample of our population. Using your own words, describe what it means to have a \textbf{representative sample}.

	\begin{solution}[\stretch{1}]
			\vspace{3mm}
					
			A \underline{representative sample} is a sample that reflects the characteristics of the larger population on a small scale. In other words, we see the same trends observed in the population reflected in the smaller sample.
			
		\end{solution}

\end{questions}

%-----------------------------------------------------------------------------%

\end{document}
