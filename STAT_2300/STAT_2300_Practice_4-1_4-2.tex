%% In the documentclass line, replace "noanswers" with "answers" to view the key.

\documentclass[noanswers]{exam}
\usepackage[utf8]{inputenc}

\title{Chapter 4 Practice Problems}
\author{Sections 4.1--4.2}
\date{STAT 2300}

\usepackage[bottom=2.2cm, left=2.2cm, right=2.2cm, top=2.2cm]{geometry}
%\usepackage[paperheight=11in, paperwidth=17in, margin=1in]{geometry}
\usepackage{dsfont}
\usepackage{amsmath}
\usepackage{amssymb}
\usepackage{amsthm}
\usepackage{array}
\usepackage{stmaryrd}
\usepackage{pgfplots}
\pgfplotsset{width=10cm,compat=1.9}
\usepackage{multicol}
\setlength{\columnsep}{1in}
\usepackage{nicefrac}

\usepackage{multirow}
\usepackage{enumitem}[shortlabels]
\usepackage{tabu}
\definecolor{purp}{RGB}{102,0,204}
\usepackage{tabularx}
\newcolumntype{C}{>{\centering\arraybackslash $}X<{$}}
\usepackage{wrapfig}
\usepackage[export]{adjustbox}


\makeatletter
\pagestyle{headandfoot}
\firstpageheader{\@date}{\@title}{\@author}
\firstpageheadrule
\runningfootrule
\runningfooter{}{\thepage\ / \numpages}{\@title}
\makeatother

\newcommand{\abs}[1]{\left|#1\right|}
\newcommand{\mat}[4]{\left( \begin{tabular}{>{$}c<{$} >{$}c<{$}} #1&#2 \\ #3&#4 \end{tabular} \right)}
\newcommand{\msc}[1]{\mathds{#1}}
\newcommand{\Z}{\mathds{Z}}
\newcommand{\R}{\mathds{R}}
\newcommand{\N}{\mathds{N}}
\newcommand{\Q}{\mathds{Q}}
\newcommand{\C}{\mathds{C}}
\newcommand{\so}{\implies}
\newcommand{\set}[2]{\left\{ #1 \:|\: #2 \right\}}
\newcommand{\bso}{\Longleftarrow}
\newcommand{\ra}{\rightarrow}
\newcommand{\gen}[1]{\left\langle #1 \right\rangle}
\newcommand{\olin}[1]{\overline{#1}}
\newcommand{\Img}[1]{\text{Im}\left(#1\right)}
\newcommand{\llra}{\longleftrightarrow}
\newcommand{\lra}{\longrightarrow}
\newcommand{\xra}[1]{\xrightarrow{#1}}
\newcommand{\wo}{\setminus}
\newcommand{\mcal}[1]{\mathcal{#1}}
\newcommand{\Aut}[1]{\text{Aut}\left(#1\right)}
\newcommand{\Inn}[1]{\text{Inn}\left(#1\right)}
\newcommand{\syl}[2]{\text{Syl}_{#1}(#2)}
\newcommand{\norm}[1]{\left\|#1\right\|}
\newcommand{\infnorm}[1]{\left\|#1\right\|_{\infty}}
\newcommand{\xn}{\{x_n\}}
\newcommand{\sig}{\sigma}
\newcommand{\id}{\text{id}}
\newcommand{\ep}{\epsilon}
\newcommand{\st}{\text{ s.t. }}
\newcommand{\ran}[1]{\text{Ran}(#1)}
\newcommand{\nCr}[2]{\binom{#1}{#2}}
\newcommand{\Exr}[1]{\paragraph{Exercise #1:}}
\newcommand{\pg}{\paragraph{}}
\newcommand{\ulin}[1]{\underline{#1}}
\newcommand{\tc}[1]{\textcolor{purp}{#1}}

% Solution Specs
\unframedsolutions
\renewcommand{\solutiontitle}{}
\SolutionEmphasis{\color{purp}}
\CorrectChoiceEmphasis{\color{purp}\bfseries}
\setlength\fillinlinelength{0in}

%\begin{solution}[\stretch{1}]
%	hurp durp flurp
%\end{solution}

%\pagestyle{empty}

\renewcommand{\arraystretch}{1.25}
\usepackage{cancel}

\usepackage{scalerel}

\begin{document}

%\noindent\begin{tabular}{@{}p{.4in}p{3.5in}@{}}
%Name: & \hrulefill
%\end{tabular}
%
%\vspace{2mm}

\noindent Use JMP to answer the following questions. Instructions for how to find the relevant values in JMP are included in Chapter 4 of your Lecture Guide.

\begin{questions} 

\question A company wishes to study the relationship between sales volume and the amount of money spent on advertisements. The following data are collected.

\begin{center}
\begin{tabular}{|c|c|}
\hline
\textbf{Advertising} & \textbf{Sales Volume} \\
(thousand dollars) & (thousand dollars) \\
\hline
100 & 400 \\
\hline
200 & 550 \\
\hline
300 & 800 \\
\hline
400 & 1200 \\
\hline
\end{tabular}
\end{center}

\vspace{2mm}

\begin{parts}

\part Identify the \textbf{explanatory} and \textbf{response} variables.

\begin{solution}[\stretch{1}]

\vspace{3mm}

\underline{Explanatory}: Amount spent on advertising (in thous.\ dollars), \underline{Response}: Sales volume (in thous.\ dollars)

\vspace{3mm}

\end{solution}

\part Find and interpret the \textbf{correlation coefficient} in context of the problem.

\begin{solution}[\stretch{1}]

\vspace{3mm}

$r=0.9783$. There is a strong, positive linear relationship between amount spent on ads and  \mbox{sales volume.}

\vspace{3mm}

\end{solution}

\part Using \textbf{Table II} in Chapter 4 of your Lecture Guide, determine whether there is a linear relationship between the two variables.

\begin{solution}[\stretch{1}]

\vspace{3mm}

Critical value for $n=4$: 0.950

$|r|=|0.9782|=0.9783>0.950 \; \Rightarrow$ There is a linear relationship.

\vspace{3mm}

\end{solution}

\part What is the least-squares \textbf{regression equation}? 

\begin{solution}[\stretch{1}]

\vspace{3mm}

$\hat{y}=75+2.65x$

\vspace{3mm}

\end{solution}

\part Interpret the \textbf{slope} of the regression equation.

\begin{solution}[\stretch{1}]

\vspace{3mm}

For every additional one thousand dollars (\$1,000) spent on advertising, sales volume is predicted to increase on average by 2.65 thousand dollars (\$2,650).

\vspace{3mm}

\end{solution}

\part Does the \textbf{y-intercept} of the regression equation have a valid interpretation in this context? If so, interpret it. If not, explain why.

\begin{solution}[\stretch{1}]

\vspace{3mm}

The y-intercept of $b_0=75$ does not have a valid interpretation in this context because zero thousand dollars is outside the range of observed advertising expense amounts.

\vspace{3mm}

\end{solution}

\part Find the \textbf{predicted} sales volume when \$400,000 are spent on advertising.

\begin{solution}[\stretch{1}]

\vspace{3mm}

$\hat{y}=75+2.65(400)=1135$ thousand dollars (\$1,135,000)

\vspace{3mm}

\end{solution}

\part Find the \textbf{residual} for when \$400,000 are spent on advertising.

\begin{solution}[\stretch{1}]

\vspace{3mm}

$y-\hat{y}=1200-1135=65$ thousand dollars
\vspace{3mm}

\end{solution}

\end{parts}

\newpage

\question An economist wants to determine the relationship between one's FICO score (a measure of credit score) and the interest rate of a 36-month auto loan. The data in the table below represent the interest rate (in percent) that a bank might offer on a 36-month auto loan for a sample of various FICO score.

\begin{center}
\begin{tabular}{|c|c|c|c|c|c|c|}
\hline
\textbf{FICO Score} & 545 & 595 & 640 & 675 & 705 & 750 \\
\hline
\textbf{Interest Rate (\%)} & 18.982 & 17.967 & 12.218 & 8.612 & 6.680 & 5.150 \\
\hline
\end{tabular}
\end{center}

\vspace{2mm}

\begin{parts}

\part Identify the \textbf{explanatory} and \textbf{response} variables.

\begin{solution}[\stretch{1}]

\vspace{3mm}

\underline{Explanatory}: FICO score, \underline{Response}: Interest rate on a 36-month auto loan (in percent)

\vspace{3mm}

\end{solution}

\part Find and interpret the \textbf{correlation coefficient} in context of the problem.

\begin{solution}[\stretch{1}]

\vspace{3mm}

$r=-0.9759$. There is a strong, negative linear relationship between FICO score and interest rate.

\vspace{3mm}

\end{solution}

\part Using \textbf{Table II} in Chapter 4 of your Lecture Guide, determine whether there is a linear relationship between the two variables.

\begin{solution}[\stretch{1}]

\vspace{3mm}

Critical value for $n=6$: 0.811

$|r|=|-0.9759|=0.9759>0.811 \; \Rightarrow$ There is a linear relationship.

\vspace{3mm}

\end{solution}

\part What is the least-squares \textbf{regression equation}? Round values to three decimal places.

\begin{solution}[\stretch{1}]

\vspace{3mm}

$\hat{y}=61.369-0.076x$

\vspace{3mm}

\end{solution}

\part Interpret the \textbf{slope} of the regression equation.

\begin{solution}[\stretch{1}]

\vspace{3mm}

For every additional point on someone's FICO score, the interest rate on a 36-month auto loan is predicted to decrease on average by 0.076\%.

\vspace{3mm}

\end{solution}

\part Does the \textbf{y-intercept} of the regression equation have a valid interpretation in this context? If so, interpret it. If not, explain why.

\begin{solution}[\stretch{1}]

\vspace{3mm}

The y-intercept of $b_0=61.369$ does not have a valid interpretation because a FICO score of zero is not within the range of observed credit scores.

\vspace{3mm}

\end{solution}

\part Find the \textbf{predicted} sales volume for someone with a FICO score of 640. 

\begin{solution}[\stretch{1}]

\vspace{3mm}

$\hat{y}=61.369-0.076(640)=12.729$\%

\vspace{3mm}

\end{solution}

\part Find the \textbf{residual} for someone with a FICO score of 640.

\begin{solution}[\stretch{1}]

\vspace{3mm}

$y-\hat{y}=12.218-12.729=-0.511$\%

\vspace{3mm}

The interest rate for someone with a FICO score of 640 is actually 0.511\% less than the regression equation predicts.
\vspace{3mm}

\end{solution}

\end{parts}

\end{questions}
%-----------------------------------------------------------------------------%

\end{document}
