%% In the documentclass line, replace "noanswers" with "answers" to view the key.

\documentclass[noanswers]{exam}
\usepackage[utf8]{inputenc}

\title{Chapter 10 Practice Problems}
\author{Section 10.2}
\date{STAT 2300}

\usepackage[bottom=2.2cm, left=2.2cm, right=2.2cm, top=2.2cm]{geometry}
%\usepackage[paperheight=11in, paperwidth=17in, margin=1in]{geometry}
\usepackage{dsfont}
\usepackage{amsmath}
\usepackage{amssymb}
\usepackage{amsthm}
\usepackage{array}
\usepackage{stmaryrd}
\usepackage{pgfplots}
\pgfplotsset{width=10cm,compat=1.9}
\usepackage{multicol}
\setlength{\columnsep}{1in}
\usepackage{nicefrac}

\usepackage{multirow}
\usepackage{enumitem}[shortlabels]
\usepackage{tabu}
\definecolor{purp}{RGB}{102,0,204}
\usepackage{tabularx}
\newcolumntype{C}{>{\centering\arraybackslash $}X<{$}}
\usepackage{wrapfig}
\usepackage[export]{adjustbox}


\makeatletter
\pagestyle{headandfoot}
\firstpageheader{\@date}{\@title}{\@author}
\firstpageheadrule
\runningfootrule
\runningfooter{}{\thepage\ / \numpages}{\@title}
\makeatother

\newcommand{\abs}[1]{\left|#1\right|}
\newcommand{\mat}[4]{\left( \begin{tabular}{>{$}c<{$} >{$}c<{$}} #1&#2 \\ #3&#4 \end{tabular} \right)}
\newcommand{\msc}[1]{\mathds{#1}}
\newcommand{\Z}{\mathds{Z}}
\newcommand{\R}{\mathds{R}}
\newcommand{\N}{\mathds{N}}
\newcommand{\Q}{\mathds{Q}}
\newcommand{\C}{\mathds{C}}
\newcommand{\so}{\implies}
\newcommand{\set}[2]{\left\{ #1 \:|\: #2 \right\}}
\newcommand{\bso}{\Longleftarrow}
\newcommand{\ra}{\rightarrow}
\newcommand{\gen}[1]{\left\langle #1 \right\rangle}
\newcommand{\olin}[1]{\overline{#1}}
\newcommand{\Img}[1]{\text{Im}\left(#1\right)}
\newcommand{\llra}{\longleftrightarrow}
\newcommand{\lra}{\longrightarrow}
\newcommand{\xra}[1]{\xrightarrow{#1}}
\newcommand{\wo}{\setminus}
\newcommand{\mcal}[1]{\mathcal{#1}}
\newcommand{\Aut}[1]{\text{Aut}\left(#1\right)}
\newcommand{\Inn}[1]{\text{Inn}\left(#1\right)}
\newcommand{\syl}[2]{\text{Syl}_{#1}(#2)}
\newcommand{\norm}[1]{\left\|#1\right\|}
\newcommand{\infnorm}[1]{\left\|#1\right\|_{\infty}}
\newcommand{\xn}{\{x_n\}}
\newcommand{\sig}{\sigma}
\newcommand{\id}{\text{id}}
\newcommand{\ep}{\epsilon}
\newcommand{\st}{\text{ s.t. }}
\newcommand{\ran}[1]{\text{Ran}(#1)}
\newcommand{\nCr}[2]{\binom{#1}{#2}}
\newcommand{\Exr}[1]{\paragraph{Exercise #1:}}
\newcommand{\pg}{\paragraph{}}
\newcommand{\ulin}[1]{\underline{#1}}
\newcommand{\tc}[1]{\textcolor{purp}{#1}}

% Solution Specs
\unframedsolutions
\renewcommand{\solutiontitle}{}
\SolutionEmphasis{\color{purp}}
\CorrectChoiceEmphasis{\color{purp}\bfseries}
\setlength\fillinlinelength{1in}

%\begin{solution}[\stretch{1}]
%	hurp durp flurp
%\end{solution}

%\pagestyle{empty}

\renewcommand{\arraystretch}{1.5}
\usepackage{cancel}

\begin{document}

%\noindent\begin{tabular}{@{}p{.3in}p{3in}@{}}
%Name: & \hrulefill
%\end{tabular}
%
%\vspace{4mm}

\noindent According to a study by Tufts University, 48.3\% of U.S.\ college students voted in the 2016 General Election. A political science researcher wonders if the college student voter turnout is higher this year. She takes a random sample of 455 college students after the 2020 election and finds that 247 voted. Test the researcher's hypothesis at the $\alpha=0.01$ level.

\begin{questions} 

\vspace{1mm}

\question Define the population parameter and state the null and alternative hypotheses.

\begin{solution}[\stretch{1}]

\vspace{1mm}

Let $p=$ the true proportion of college students who voted in the 2020 Election.

\vspace{3mm}

$H_0:p=0.483$
$H_1:p>0.483$

\vspace{1mm}

\end{solution}

\question State and verify the necessary \textbf{conditions} for the hypothesis test.

\begin{solution}[\stretch{1}]

\vspace{1mm}

(1) The data come from a random sample --- stated in the problem.

(2) It can be reasonably assumed that 455 is less than 5\% of all U.S.\ college students.

(3) $np_0(1-p_0)=455(0.483)(1-0.483)=113.62\geq 10$

\vspace{1mm}

\end{solution}

\question Compute the \textbf{test statistic}.

\begin{solution}[\stretch{1}]

\vspace{1mm}

$z_0=\displaystyle\frac{\frac{247}{455}-0.483}{\sqrt{\frac{(.483)(1-.483)}{455}}}=2.56$
\vspace{1mm}

\end{solution}

\question Find the \textbf{p-value}.

\begin{solution}[\stretch{1}]

\vspace{1mm}

p-value $=P(Z>2.56)=1-P(Z<2.56)=1-0.9948=0.0052$

\vspace{1mm}

\end{solution}

\question State and justify your \textbf{decision} about the null hypothesis and write your \textbf{conclusion} in terms of the alternative hypothesis.

\begin{solution}[\stretch{1}]

\vspace{1mm}

Reject $H_0$ since p-value $=0.0052<\alpha=0.01$. At the $\alpha=0.01$ significance level, we have sufficient evidence that the true proportion of students who voted in the 2020 election is higher than it was in the 2016 election. 

\vspace{1mm}

\end{solution}


\end{questions}
%-----------------------------------------------------------------------------%

\end{document}
